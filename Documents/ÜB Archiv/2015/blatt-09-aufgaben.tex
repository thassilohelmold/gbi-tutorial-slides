\documentclass[12pt]{article}

\input{preamble-aufgaben}

\usetikzlibrary{positioning}

\tikzstyle{every node} = [circle, draw, fill = white, inner sep = 3pt, minimum width = 3pt, font = \scriptsize] % small
\tikzstyle{every path} = [-, shorten >= 0pt, thick]% [->, > = stealth, shorten >= 0pt, thick]

% =======================================================
%\newcounter{blattnr}
\setcounter{blattnr}{9}
\newcommand{\ausgabetermin}{23.~Dezember 2015}
\newcommand{\abgabetermin}{15.~Januar 2015}
%\newcommand{\punkteblatt}{13} % Blatt 1
%\newcommand{\punkteblattphysik}{13} % Blatt 1
%\newcommand{\punkteblatt}{17} % Blatt 2
%\newcommand{\punkteblattphysik}{14} % Blatt 2
%\newcommand{\punkteblatt}{18} % Blatt 3
%\newcommand{\punkteblattphysik}{18} % Blatt 3
%\newcommand{\punkteblatt}{18} % Blatt 4
%\newcommand{\punkteblattphysik}{18} % Blatt 4
%\newcommand{\punkteblatt}{18} % Blatt 5
%\newcommand{\punkteblattphysik}{18} % Blatt 5
%\newcommand{\punkteblatt}{20} % Blatt 6
%\newcommand{\punkteblattphysik}{20} % Blatt 6
%\newcommand{\punkteblatt}{20} % Blatt 7
%\newcommand{\punkteblattphysik}{0} % Blatt 7
%\newcommand{\punkteblatt}{18} % Blatt 8
%\newcommand{\punkteblattphysik}{18} % Blatt 8
\newcommand{\punkteblatt}{17} % Blatt 9
\newcommand{\punktetotal}{159} %
\newcommand{\punkteblattphysik}{17} % Blatt 9
\newcommand{\punktetotalphysik}{136} %
% =======================================================

\begin{document}

%\noindent
%Mit \textbf{[nicht Physik]} gekennzeichnete Aufgaben müssen von
%Studenten der Physik nicht bearbeitet werden.\\

% -----------------------------------------------------------------------------

\begin{aufgabe}[2 + 2 + 2 + 2 + 1 + 2  = 11]
  Für jede positive ganze Zahl $n \in \N_+$ sei $G_n = (V_n, E_n)$ der gerichtete Graph mit der Knotenmenge $V_n = \{0, 1\}^n$ und der Kantenmenge
  \begin{equation*}
%    E_n = \{ \{x, y\} \in 2^{V_n} \mid \exists i \in \Z_n \colon (x_i \neq y_i \land \forall k \in \Z_n \setminus \{i\} \colon x_k = y_k) \}.
    E_n = \{ (x, y) \in V_n \times V_n \mid \exists i \in \Z_n \colon (x_i \neq y_i \land \forall k \in \Z_n \smallsetminus \{i\} \colon x_k = y_k) \}.
  \end{equation*}
  \begin{enumerate}
    \item Zeichnen Sie $G_1$, $G_2$ und $G_3$ jeweils in ein kartesisches Koordinatensystem der entsprechenden Dimension.
    \item Geben Sie einen geschlossenen arithmetischen Ausdruck für $\abs{E_n}$ an. Dabei bedeutet \emph{geschlossen}, dass in dem Ausdruck weder das Summenzeichen $\sum$ noch das Produktzeichen $\prod$ vorkommt.
    \item Geben Sie für jede positive ganze Zahl $n \in \N_+$ eine Einbettung $f_n$ von $G_n$ in $G_{n + 1}$ an, das heißt, eine injektive Abbildung $f_n \colon V_n \to V_{n + 1}$ derart, dass
          \begin{equation*}
%            \forall x \in V_n\; \forall y \in V_n \colon \left( \{x, y\} \in E_n \rightarrow \{f_n(x), f_n(y)\} \in E_{n + 1} \right).
            \forall x \in V_n\; \forall y \in V_n \colon \left( (x, y) \in E_n \rightarrow (f_n(x), f_n(y)) \in E_{n + 1} \right).
          \end{equation*}
    \item Geben Sie einen Pfad $p = (v_0, v_1, v_2, v_3)$ von $(0,0,0)$ nach $(1,1,1)$ in $G_3$ an. Geben Sie außerdem einen Pfad $q$ von $(0,0,0,0)$ nach $(1,1,1,1)$ in $G_4$ an, der den Pfad $(f_3(v_0), f_3(v_1), f_3(v_2), f_3(v_3))$ als Teilpfad enthält, wobei $f_3$ die Einbettung von $G_3$ in $G_4$ aus der vorangegangenen Teilaufgabe sei.
    \item Geben Sie für jede positive ganze Zahl $n \in \N_+$ einen geschlossenen arithmetischen Ausdruck für
          \begin{equation*}
            \gamma_n = \min\{ |p| \mid \text{$p$ ist Pfad in $G_n$ von $(0,0,\dotsc,0)$ nach $(1,1,\dotsc,1)$} \}
          \end{equation*}
          an.
    \item Geben Sie für jede positive ganze Zahl $n \in \N_+$ einen Graph"=Isomorphismus $\varphi_n$ von $G_n$ nach $G_n$ an, der nicht die identische Abbildung ist. %, das heißt, eine bijektive Abbildung $\varphi_n \colon V_n \to V_n$ derart, dass
%          \begin{equation*}
%            \forall x \in V_n\; \forall y \in V_n \colon \left( \{x, y\} \in E_n \leftrightarrow \{\varphi_n(x), \varphi_n(y)\} \in E_n \right),
%          \end{equation*}
%          und derart, dass ein Knoten $x \in V_n$ existiert, für den $\varphi_n(x) \neq x$ gilt.
    % \item Geben Sie für jede positive ganze Zahl $n \in \N_+$ einen geschlossenen arithmetischen Ausdruck für
    %       \begin{equation*}
    %         \xi_n = \min\{ |U| \mid U \subseteq E_n \land (V_n, E_n \setminus U) \text{ unzusammenhängend} \}
    %       \end{equation*}
    %       an.
          % Der Graph $G_n$ is zusammenhängend. %Was ist die kleinste Anzahl an Kanten die aus $G_n$ entfernt werden müssen, damit $G_n$ % Wie viele Kanten müssen mindestens aus $G_n$ entfernt werden, damit der entstehende Graph unzusammenhängend ist?
    % \item Geben Sie für jede positive ganze Zahl $n \in \N_+$ ein Teilmenge $F_n$ von $E_n$ so an, dass der ungerichtete Graph $H_n = (V_n, F_n)$ ein Baum ist.

    %       \emph{Hinweis:} Bei der Definition von $F_n$ können die Abbildungen
    %       \begin{align*}
    %         \sigma_n \colon E_n &\to     \N_0,\\
    %                           x &\mapsto \sum_{i \in \Z_n} x_i,
    %       \end{align*}
    %       und
    %       \begin{align*}
    %         \tau_n \colon E_n &\to     \Z,\\
    %                         x &\mapsto \begin{dcases*}
    %                                      -1, &falls $\forall i \in \Z_n \colon x_i = 0$,\\
    %                                      \min\{ i \in \Z_n \mid x_i = 1 \}, &sonst,
    %                                    \end{dcases*}
    %       \end{align*}
    %       von Nutzen sein.
  \end{enumerate}
\end{aufgabe}

\begin{loesung}
  \begin{enumerate}
    \item \quad\begin{tikzpicture}[auto, baseline = {([yshift=-10pt]current bounding box.north)},>={Latex}]
            \node (0) {$0$};
            \node (1) [right = 1cm of 0] {$1$};

            \path (0) edge[bend left=8,->] (1);
            \path (1) edge[bend left=8,->] (0);
          \end{tikzpicture}
          \qquad\qquad\qquad
          \quad\begin{tikzpicture}[auto, baseline = {([yshift=-10pt]current bounding box.north)},>={Latex}]
            \node (00) {$(0, 0)$};
            \node (10) [right = 1cm of 00] {$(1, 0)$};
            \node (01) [above = 1cm of 00] {$(0, 1)$};
            \node (11) [right = 1cm of 01] {$(1, 1)$};

            \path (00) edge[bend left=8,->] (10)
                  (00) edge[bend left=8,->] (01)
                  (11) edge[bend left=8,->] (01)
                  (11) edge[bend left=8,->] (10);
            \path (10) edge[bend left=8,->] (00)
                  (01) edge[bend left=8,->] (00)
                  (01) edge[bend left=8,->] (11)
                  (10) edge[bend left=8,->] (11);
          \end{tikzpicture}

          \quad\begin{tikzpicture}[auto, baseline = {([yshift=-10pt]current bounding box.north)},>={Latex}]
            \node (000) {$(0, 0, 0)$};
            \node (001) [above right = 0.7cm of 000] {$(0, 0, 1)$};
            \node (010) [above = 1.4cm of 000] {$(0, 1, 0)$};
            \node (011) [above right = 0.7cm of 010] {$(0, 1, 1)$};
            \node (100) [right = 1.4cm of 000] {$(1, 0, 0)$};
            \node (101) [above right = 0.7cm of 100] {$(1, 0, 1)$};
            \node (110) [above = 1.4cm of 100] {$(1, 1, 0)$};
            \node (111) [above right = 0.7cm of 110] {$(1, 1, 1)$};

            \path (000) edge[bend left=8,->] (001)
                  (000) edge[bend left=8,->] (010)
                  (000) edge[bend left=8,->] (100)
                  (111) edge[bend left=8,->] (110)
                  (111) edge[bend left=8,->] (101)
                  (111) edge[bend left=8,->] (011)
                  (001) edge[bend left=8,->] (011)
                  (001) edge[bend left=8,->] (101)
                  (010) edge[bend left=8,->] (110)
                  (010) edge[bend left=8,->] (011)
                  (100) edge[bend left=8,->] (101)
                  (100) edge[bend left=8,->] (110);
            \path (001) edge[bend left=8,->] (000)
                  (010) edge[bend left=8,->] (000)
                  (100) edge[bend left=8,->] (000)
                  (110) edge[bend left=8,->] (111)
                  (101) edge[bend left=8,->] (111)
                  (011) edge[bend left=8,->] (111)
                  (011) edge[bend left=8,->] (001)
                  (101) edge[bend left=8,->] (001)
                  (110) edge[bend left=8,->] (010)
                  (011) edge[bend left=8,->] (010)
                  (101) edge[bend left=8,->] (100)
                  (110) edge[bend left=8,->] (100);
          \end{tikzpicture}

          \begin{korrektur}
            Jeweils 0.5 Punkte für $G_1$ und $G_2$. Und 1 Punkt für $G_3$.

            Das mit dem "`kartesischen Koordinatensystem"' war gedacht
            als Mittel, um einheitlichere Abgaben zu erreichen. Keine
            Abzüge für andere Bilder.

            Wenn die Knoten anonym sind, \emph{insgesamt} 0.5 Punkte Abzug.

            Für ungerichtete Graphen \emph{insgesamt} 0.5 Punkte Abzug.
          \end{korrektur}
    \item $\abs{E_n} = 2^{n} \cdot n$

      \emph{Erklärung (nicht verlangt):} Es sei $n \in \N_+$. Der Graph $G_n$ hat genau $2^n$ Knoten. Jeder dieser Knoten hat Ausgangsgradgrad $n$, das heißt, genau $n$ wegführende Kanten. Und so wird jede Kante genau einmal gezählt.
      % Die Kantenzahl beträgt somit
      %     \begin{equation*}
      %       \frac{\sum_{x \in V_n} n}{2} = \frac{2^n \cdot n}{2} = 2^{n - 1} \cdot n.
      %     \end{equation*}
      %     Wir müssen $\sum_{x \in V_n} n$ durch $2$ dividieren, da wir im Ausdruck $\sum_{x \in V_n} n$ jede Kante doppelt zählen, einmal je inzidenten Knoten (und jede Kante, die keine Schlinge ist, ist zu genau zwei verschiedenen Knoten inzident).
    \item Für jedes $n \in \N_+$ ist
          \begin{align*}
            f_n \colon V_n &\to     V_{n + 1},\\
                         x &\mapsto (x, 0),
          \end{align*}
          eine mögliche Einbettung von $G_n$ in $G_{n + 1}$.

          $x\mapsto (x,1)$ und $x\mapsto (0,x)$ sind andere und es gibt noch viel mehr Möglichkeiten.
          \begin{korrektur}
            Bitte darauf achten, dass die Bilder verbundener Knoten wieder verbunden sind.
          \end{korrektur}
    \item Ein möglicher Pfad $p$ ist $((0,0,0), (1,0,0), (1,1,0), (1,1,1))$. 

      Ein möglicher Pfad $q$ ist
      $((0,0,0,0), (1,0,0,0), (1,1,0,0), (1,1,1,0), (1,1,1,1))$.

      \begin{korrektur}
        Je 0.5 Punkte auf jeden Pfad und 1 Punkt, wenn das Bild von
        $p$ in $q$ vorkommt.
      \end{korrektur}
    \item $\gamma_n = n$

          Um von $(0,0,\dotsc,0)$ nach $(1,1,\dotsc,1)$ in $G_n$ zu kommen müssen genau $n$ bits von $0$ auf $1$ kippen.
          \begin{korrektur}
            \textbf{Achtung:} $|p|$ wurde in der Vorlesung nicht
            präzise definiert. Wir akzeptieren auch $n+1$, aber nicht $n-1$.
          \end{korrektur}
    \item Für jedes $n \in \N_+$ ist \zB
          \begin{align*}
            \varphi_n \colon V_n &\to     V_n,\\
                          (x, 0) &\mapsto (x, 1),\\
                          (x, 1) &\mapsto (x, 0),
          \end{align*}
          ein Isomorphismus von $G_n$ nach $G_n$. Für $n = 1$ degeneriert $(x, 0)$ zu $0$ und $(x, 1)$ zu $1$.

          \begin{korrektur}
            Wir haben Probleme, vernünftig mit $n$-Tupeln umzugehen.
            Deshalb seien Sie bitte bei der Korrektur auch ein bisschen großzügig.
          \end{korrektur}

    % \item $\xi_n = n$

    %       Tatsächlich genügt es $n$ Kanten aus $G_n$ zu entfernen, damit der entstehende Graph unzusammenhängend wird: Man wähle einfach einen Knoten und entferne alle zu diesem Knoten inzidenten Kanten.

    %       Das man mit einer kleineren Anzahl an Kanten nicht auskommt, ist schwerer einzusehen.

    % \item Die Kantenmenge
    %       \begin{equation*}
    %         F_n = \{ \{x, y\} \in E_n \mid \sigma_n(x) < \sigma_n(y) \land \tau_n(x) \leq \tau_n(y) \}
    %       \end{equation*}
    %       leistest das Gewünschte.
  \end{enumerate}
\end{loesung}

% -----------------------------------------------------------------------------
\begin{aufgabe}[1 + 1 + 2 + 2 = 6]
  \emph{Hinweis:} Benutzen Sie in dieser Aufgabe die Definition von
  "`Zyklus"' aus dem aktualisierten Skript: Ein Zyklus ist ein
  geschlossener Pfad, dessen Länge größer als oder gleich $1$ ist.

  Ein sogenannter DAG (engl.~\emph{directed acyclic graph}) ist ein
  gerichteter Graph, der keine Zyklen enthält.

  \begin{enumerate}
  \item Geben Sie einen DAG mit $4$ Knoten an, der
    \begin{itemize}
    \item kein Baum ist, und
    \item einen Teilgraphen mit $4$ Knoten enthält, der ein Baum ist.
    \end{itemize}
  \item Geben Sie einen DAG mit $6$ Knoten und $9$ Kanten an, der
    keinen Pfad der Länge $2$ enthält.
  \item Begründen Sie, warum jeder Baum ein DAG ist.
  \item Es sei $G=(V,E)$ ein DAG und es seien $x,y\in V$ zwei Knoten
    von $G$ mit der Eigenschaft: $(x,y)\in E^*$ und $(y,x)\in E^*$.
    %
    Beweisen Sie: $x=y$.
  \end{enumerate}
\end{aufgabe}

\begin{loesung}
  \begin{enumerate}
  \item
    \begin{tikzpicture}[auto, baseline = {([yshift=-10pt]current
        bounding box.north)},>={Latex}]
      \node (A) {};
      \node (B) [right = 1cm of A] {};
      \node (C) [below = 1cm of A] {};
      \node (D) [right = 1cm of C] {};

      \draw[->] (A) -- (B);
      \draw[->] (B) -- (D);
      \draw[->] (A) -- (C);
      \draw[->] (C) -- (D);
    \end{tikzpicture} \\[3mm]
  \item
    \begin{tikzpicture}[auto, baseline = {([yshift=-10pt]current
        bounding box.north)},>={Latex}]
      \node (A1) {};
      \node (B1) [right = 1cm of A1] {};
      \node (C1) [right = 1cm of B1] {};
      \node (A2) [below = 1cm of A1] {};
      \node (B2) [right = 1cm of A2] {};
      \node (C2) [right = 1cm of B2] {};

      \draw[->] (A1) -- (A2);
      \draw[->] (A1) -- (B2);
      \draw[->] (A1) -- (C2);
      \draw[->] (B1) -- (A2);
      \draw[->] (B1) -- (B2);
      \draw[->] (B1) -- (C2);
      \draw[->] (C1) -- (A2);
      \draw[->] (C1) -- (B2);
      \draw[->] (C1) -- (C2);
    \end{tikzpicture} \\[3mm]
  \item Es ist zu zeigen, dass ein Baum keine Zyklen enthält.

    Angenommen ein Graph $G=(V,E)$ ist ein Baum mit Wurzel $r\in V$
    und er enthält einen Zyklus $p=(v_0,\dots,v_n)$, also $n\geq 1$
    und $v_0=v_n$.

    Da $G$ ein Baum ist, gibt es einen Pfad von $q$ von $r$ zu $v_0$.
    Wenn man diesen Pfad um die Folge $v_1, \dots,v_n$ verlängert,
    erhält man wiederum einen Pfad von $r$ zu $v_0$, der aber länger
    als also verschieden von $q$ ist.

    Also gibt es mindestens zwei Pfade von $r$ nach $v_0$ im
    Widerspruch zur Annahme, dass $G$ ein Baum ist.
  \item Wäre $x\not= y$, dann gäbe es wegen $(x,y)\in E^*$ einen Pfad
    $(v_0,\dots,v_n)$ mit $x=v_0$, $y=v_n$ und $n\geq 1$, wegen
    $(y,x)\in E^*$ gäbe es einen Pfad $(v'_0,\dots,v'_m)$ mit $y=v'_0$,
    $x=v'_m$ und $m\geq 1$.

    Dann wäre aber $(v_0,\dots, v_n, v'_1, \dots, v'_m)$ ein Pfad von
    $x$ nach $x$ einer Länge $\geq 1$ im Widerspruch zu der Tatsache,
    dass $G$ ein DAG ist.
  \end{enumerate}
\end{loesung}

% -----------------------------------------------------------------------------

\end{document}
%%%
%%% Local Variables:
%%% fill-column: 70
%%% mode: latex
%%% TeX-command-default: "XPDFLaTeX"
%%% TeX-master: "korrektur.tex"
%%% End:
