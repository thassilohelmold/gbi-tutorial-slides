%beamer

% This video for Turing:
% https://www.youtube.com/watch?v=VxvY4rI15sM

% Comment/uncomment this line to toggle handout mode
% \newcommand{\handout}{}

%% Beamer-Klasse im korrekten Modus
\ifdefined \handout
\documentclass[handout]{beamer} % Handout mode
\else
\documentclass{beamer}
\fi

%% UTF-8-Encoding
\usepackage[utf8]{inputenc}

% % \bigtimes abgeschrieben von http://tex.stackexchange.com/questions/14386/importing-a-single-symbol-from-a-different-font
% \DeclareFontFamily{U}{mathx}{\hyphenchar\font45}
% \DeclareFontShape{U}{mathx}{m}{n}{
%       <5> <6> <7> <8> <9> <10> gen * mathx
%       <10.95> mathx10 <12> <14.4> <17.28> <20.74> <24.88> mathx12
%       }{}
% \DeclareSymbolFont{mathx}{U}{mathx}{m}{n}
% \DeclareMathSymbol{\bigtimes}{\mathop}{mathx}{161}

\RequirePackage{xcolor}

\def\9{\square}
%\def\9{\blank}

% f"ur Aussagenlogik
\colorlet{alcolor}{blue}
\RequirePackage{tikz}
\usetikzlibrary{arrows.meta}
\newcommand{\alimpl}{\mathrel{\tikz[x={(0.1ex,0ex)},y={(0ex,0.1ex)},>={Classical TikZ Rightarrow[]}]{\draw[alcolor,->,line width=0.7pt,line cap=round] (0,0) -- (15,0);\path (0,-6);}}}
\newcommand{\aleqv}{\mathrel{\tikz[x={(0.1ex,0ex)},y={(0ex,0.1ex)},>={Classical TikZ Rightarrow[]}]{\draw[alcolor,<->,line width=0.7pt,line cap=round] (0,0) -- (18,0);\path (0,-6);}}}
\newcommand{\aland}{\mathbin{\raisebox{-0.6pt}{\rotatebox{90}{\texttt{\color{alcolor}\char62}}}}}
\newcommand{\alor}{\mathbin{\raisebox{-0.8pt}{\rotatebox{90}{\texttt{\color{alcolor}\char60}}}}}
%\newcommand{\ali}[1]{_{\mathtt{\color{alcolor}#1}}}
\newcommand{\alv}[1]{\mathtt{\color{alcolor}#1}}
\newcommand{\alnot}{\mathop{\tikz[x={(0.1ex,0ex)},y={(0ex,0.1ex)}]{\draw[alcolor,line width=0.7pt,line cap=round,line join=round] (0,0) -- (10,0) -- (10,-4);\path (0,-8) ;}}}
\newcommand{\alP}{\alv{P}} %ali{#1}}
%\newcommand{\alka}{\negthinspace\hbox{\texttt{\color{alcolor}(}}}
\newcommand{\alka}{\negthinspace\text{\texttt{\color{alcolor}(}}}
%\newcommand{\alkz}{\texttt{\color{alcolor})}}\negthinspace}
\newcommand{\alkz}{\text{\texttt{\color{alcolor})}}\negthinspace}
\newcommand{\AAL}{A_{AL}}
\newcommand{\LAL}{\hbox{\textit{For}}_{AL}}
\newcommand{\AxAL}{\hbox{\textit{Ax}}_{AL}}
\newcommand{\AxEq}{\hbox{\textit{Ax}}_{Eq}}
\newcommand{\AxPL}{\hbox{\textit{Ax}}_{PL}}
\newcommand{\AALV}{\hbox{\textit{Var}}_{AL}}
\newcommand{\MP}{\hbox{\textit{MP}}}
\newcommand{\GEN}{\hbox{\textit{GEN}}}
\newcommand{\W}{\ensuremath{\hbox{\textbf{w}}}\xspace}
\newcommand{\F}{\ensuremath{\hbox{\textbf{f}}}\xspace}
\newcommand{\WF}{\ensuremath{\{\W,\F\}}\xspace}
\newcommand{\val}{\hbox{\textit{val}}}
\newcommand{\valDIb}{\val_{D,I,\beta}}

\newcommand*{\from}{\colon}

% die nachfolgenden Sachen angepasst an cmtt
\newlength{\ttquantwd}
\setlength{\ttquantwd}{1ex}
\newlength{\ttquantht}
\setlength{\ttquantht}{6.75pt}
\def\plall{%
  \tikz[line width=0.67pt,line cap=round,line join=round,baseline=(B),alcolor] {
    \draw (-0.5\ttquantwd,\ttquantht) -- node[coordinate,pos=0.4] (lll){} (-0.25pt,-0.0pt) -- (0.25pt,-0.0pt) -- node[coordinate,pos=0.6] (rrr){} (0.5\ttquantwd,\ttquantht);
    \draw (lll) -- (rrr);
    \coordinate (B) at (0,-0.35pt);
  }%
}
\def\plexist{%
  \tikz[line width=0.67pt,line cap=round,line join=round,baseline=(B),alcolor] {
    \draw (-0.9\ttquantwd,\ttquantht) -- (0,\ttquantht) -- node[coordinate,pos=0.5] (mmm){} (0,0) --  (-0.9\ttquantwd,0);
    \draw (mmm) -- ++(-0.75\ttquantwd,0);
    \coordinate (B) at (0,-0.35pt);
  }\ensuremath{\,}%
}
\let\plexists=\plexist
\newcommand{\NT}[1]{\ensuremath{\langle\mathrm{#1} \rangle}}

\newcommand{\CPL}{\text{\itshape Const}_{PL}}
\newcommand{\FPL}{\text{\itshape Fun}_{PL}}
\newcommand{\RPL}{\text{\itshape Rel}_{PL}}
\newcommand{\VPL}{\text{\itshape Var}_{PL}}
\newcommand{\ATer}{A_{\text{\itshape Ter}}}
\newcommand{\ARel}{A_{\text{\itshape Rel}}}
\newcommand{\AFor}{A_{\text{\itshape For}}}
\newcommand{\LTer}{L_{\text{\itshape Ter}}}
\newcommand{\LRel}{L_{\text{\itshape Rel}}}
\newcommand{\LFor}{L_{\text{\itshape For}}}
\newcommand{\NTer}{N_{\text{\itshape Ter}}}
\newcommand{\NRel}{N_{\text{\itshape Rel}}}
\newcommand{\NFor}{N_{\text{\itshape For}}}
\newcommand{\PTer}{P_{\text{\itshape Ter}}}
\newcommand{\PRel}{P_{\text{\itshape Rel}}}
\newcommand{\PFor}{P_{\text{\itshape For}}}

\newcommand{\plka}{\alka}
\newcommand{\plkz}{\alkz}
%\newcommand{\plka}{\plfoo{(}}
%\newcommand{\plkz}{\plfoo{)}}
\newcommand{\plcomma}{\hbox{\texttt{\color{alcolor},}}}
\newcommand{\pleq}{{\color{alcolor}\,\dot=\,}}

% MODIFIED (DJ)
% previously: \newcommand{\plfoo}[1]{\mathtt{\color{alcolor}#1}}
\newcommand{\plfoo}[1]{\texttt{\color{alcolor}#1}}

\newcommand{\plc}{\plfoo{c}}
\newcommand{\pld}{\plfoo{d}}
\newcommand{\plf}{\plfoo{f}}
\newcommand{\plg}{\plfoo{g}}
\newcommand{\plh}{\plfoo{h}}
\newcommand{\plx}{\plfoo{x}}
\newcommand{\ply}{\plfoo{y}}
\newcommand{\plz}{\plfoo{z}}
\newcommand{\plR}{\plfoo{R}}
\newcommand{\plS}{\plfoo{S}}

\newcommand{\bv}{\mathrm{bv}}
\newcommand{\fv}{\mathrm{fv}}

%\newcommand{\AxAL}{\hbox{\textit{Ax}}_{AL}}
%\newcommand{\AALV}{\hbox{\textit{Var}}_{AL}}

%\renewcommand{\#}[1]{\literal{#1}}
\newcommand{\A}{\mathcal{A}}
\newcommand{\Adr}{\text{Adr}}
\newcommand{\ar}{\mathrm{ar}}
\newcommand{\ascii}[1]{\literal{\char#1}}
%\newcommand{\assert}[1]{\text{/\!\!/\ } #1}
\newcommand{\assert}[1]{\colorbox{black!7!white}{\ensuremath{\{\;#1\;\}}}}
\newcommand{\Assert}[1]{$\langle$\textit{#1}$\rangle$}
\newcommand{\B}{\mathcal{B}}
\newcommand{\bfmod}{\mathbin{\kw{ mod }}}
\newcommand{\bb}{{\text{bb}}}
\def\bottom{\hbox{\small$\pmb{\bot}$}}
\newcommand{\card}[1]{|#1|}
%\newcommand{\cod}{\mathop{\text{cod}}}  % ist in thwmathabbrevs
\newcommand{\Conf}{\mathcal{C}}
\newcommand{\define}[1]{\emph{#1}}
%\renewcommand{\dh}{d.\,h.\@\xspace}
%\newcommand{\Dh}{D.\,h.\@\xspace}
%\newcommand{\engl}[1]{engl.\xspace\emph{#1}}
\newcommand{\eps}{\varepsilon}
%\newcommand{\evtl}{evtl.\@\xspace}
\newcommand{\fbin}{\text{bin}}
\newcommand{\finv}{\text{inv}}
\newcommand{\fnum}{\text{num}}
\newcommand{\fNum}{{\text{Num}}}
\newcommand{\frepr}{\text{repr}}
\newcommand{\fRepr}{\text{Repr}}
\newcommand{\fZkpl}{\text{Zkpl}}
\newcommand{\fLen}{\text{Len}}
\newcommand{\fsem}{\text{sem}}
\providecommand{\fspace}{\mathord{\text{space}}}
\providecommand{\fSpace}{\mathord{\text{Space}}}
\providecommand{\ftime}{\mathord{\text{time}}}
\providecommand{\fTime}{\mathord{\text{Time}}}
\newcommand{\fTrans}{\text{Trans}}
\newcommand{\fVal}{\text{Val}}

% MODIFIED (DJ)
\newcommand{\Val}{\text{Val}}

%\def\G{\mathbb{Z}}
\newcommand{\HT}[1]{\normalfont\textsc{HT-#1}}
\newcommand{\htr}[3]{\{#1\}\;#2\; \{#3\}}
\newcommand{\Id}{\text{I}}
%\newcommand{\ie}{i.\,e.\@\xspace}
\newcommand{\instr}[2]{\texttt{#1}\ \textit{#2}}
\newcommand{\Instr}[2]{\texttt{#1}\ \textrm{#2}}
\newcommand{\instrr}[3]{\texttt{#1}\ \textit{#2}\texttt{(#3)}}
\newcommand{\Instrr}[3]{\texttt{#1}\ \textrm{#2}\texttt{(#3)}}
\newcommand{\io}{\!\mid\!}
\usepackage{KITcolors}
\newcommand{\literal}[1]{\hbox{\textcolor{blue!95!white}{\textup{\texttt{\scalebox{1.11}{#1}}}}}}
%\newcommand{\literal}[1]{\hbox{\textcolor{KITblue!80!black}{\textup{\texttt{#1}}}}}
\def\kasten#1{\leavevmode\literal{\setlength{\fboxsep}{1pt}\fbox{\vrule  width 0pt height 1.5ex depth 0.5ex #1}}}
\newcommand{\kw}[1]{\ensuremath{\mathbf{#1}}}
\newcommand{\lang}[1]{\ensuremath{\langle#1\rangle}}
%\newcommand{\maw}{m.\,a.\,w.\@\xspace}
%\newcommand{\MaW}{M.\,a.\,w.\@\xspace}
\newcommand{\mdefine}[2][FOOBAR]{\define{#2}\def\foobar{FOOBAR}\def\optarg{#1}\ifx\foobar\optarg\def\optarg{#2}\fi\graffito{\optarg}}
\newcommand{\meins}{\rotatebox[origin=c]{180}{1}}
\newcommand{\Mem}{\text{Mem}}
\newcommand{\memread}{\text{memread}}
\newcommand{\memwrite}{\text{memwrite}}
\providecommand{\meta}[1]{\ensuremath{\langle}\textit{#1}\ensuremath{\rangle}}
%\newcommand{\N}{\mathbb{N}}
\newcommand{\NP}{\mathbf{NP}}
\newcommand{\Nadd}{N_{\text{add}}}
\newcommand{\Nmult}{N_{\text{mult}}}
% MODIFIED (DJ): added \!, mathcal{O}
\newcommand{\Oh}[1]{\mathcal{O}\!\left(#1\right)}
\newcommand{\Om}[1]{\Omega\!\left(#1\right)}
\newcommand{\personname}[1]{\textsc{#1}}
\newcommand{\regname}[1]{\texttt{#1}}
\newcommand{\mima}{\textsc{Mima}\xspace}
\newcommand{\mimax}{\textsc{Mima-X}\xspace}

\def\Pclass{\text{\bfseries P}}
\def\PSPACE{\text{\bfseries PSPACE}}

\newcommand{\SPush}{\text{push}}
\newcommand{\SPop}{\text{pop}}
\newcommand{\SPeek}{\text{peek}}
\newcommand{\STop}{\text{top}}
\newcommand{\STos}{\text{\itshape tos}}
\newcommand{\SBos}{\text{\itshape bos}}

%\newcommand{\R}{\mathbb{R}}
\newcommand{\Rnullplus}{\R_0^{+}}
\newcommand{\Rplus}{\R_{+}}
\newcommand{\resp}{resp.\@\xspace}
\newcommand{\Sem}{\text{Sem}}
\newcommand{\sgn}{\mathop{\text{sgn}}}
\newcommand{\sqbox}{\mathop{\raisebox{-6.2pt}{\hbox{\hbox to 0pt{$^{^{\sqcap}}$\hss}$^{^{\sqcup}}$}}}}
\newcommand{\sqleq}{\sqsubseteq}
\newcommand{\sqgeq}{\sqsupseteq}
% MODIFIED (DJ): added \!
\newcommand{\Th}[1]{\Theta\!\left(#1\right)}
%\newcommand{\usw}{usw.\@\xspace}
\newcommand{\V}[1]{\hbox{\textit{#1}}}
\newcommand{\x}{\times}
\newcommand{\ZK}{\mathbb{K}}
%\newcommand{\Z}{\mathbb{Z}}
\newcommand{\zB}{z.\,B.\@\xspace}
\newcommand{\ZB}{Z.\,B.\@\xspace}
% \newcommand{\bb}{{\text{bb}}}
% \def\##1{\hbox{\textcolor{darkblue}{\texttt{#1}}}}
% \def\A{\mathcal{A}}
% \newcommand{\0}{\#0}
% \newcommand{\1}{\#1}
% \newcommand{\Obj}{\text{Obj}}
% \newcommand{\start}{\mathop{\text{start}}}
% \newcommand{\compactlist}{\addtolength{\itemsep}{-\parskip}}
% \newcommand{\fval}{\text{val}}
% \newcommand{\lang}[1]{\ensuremath{\langle#1\rangle}}
% \newcommand{\io}{\!\mid\!}
% \def\sqbox{\mathop{\raisebox{-6.2pt}{\hbox{\hbox to 0pt{$^{^{\sqcap}}$\hss}$^{^{\sqcup}}$}}}}
% \def\sqleq{\sqsubseteq}
% \def\sqgeq{\sqsupseteq}
\def\Td{T_{\overline{d}}}
% \newcommand{\csym}[1]{\ensuremath{\#{c}_{\#{\hbox{\scriptsize #1}}}}}
% \newcommand{\F}{\ensuremath{\mathcal{F}}}
% \newcommand{\fsym}[2]{\ensuremath{\#{f}^{\#{\hbox{\scriptsize #1}}}_{\#{\hbox{\scriptsize #2}}}}}
% \newcommand{\rsym}[2]{\ensuremath{\#{R}^{\#{\hbox{\scriptsize #1}}}_{\#{\hbox{\scriptsize #2}}}}}
% \newcommand{\xsym}[1]{\ensuremath{\#{x}_{\#{\hbox{\scriptsize #1}}}}}
% \newcommand{\I}{\mathcal{I}}
% ********************************************************************

\usepackage[blue]{../framework/thwregex}
\usepackage{environ}
\usepackage{bm}
\usepackage{calc}
\usepackage{varwidth}
\usepackage{wasysym}
\usepackage{mathtools}


% Das ist der KIT-Stil
%\usepackage{../TutTexbib/beamerthemekit}
\usepackage[deutsch,titlepage0]{../framework/KIT/beamerthemeKITmod}
\TitleImage[width=\titleimagewd]{../figures/titlepage.jpg}
%\usetheme[deutsch,titlepage0]{KIT}

% Include PDFs
\usepackage{pdfpages}

% Libertine font (Original GBI font)
\usepackage{libertine}
%\renewcommand*\familydefault{\sfdefault}  %% Only if the base font of the document is to be sans serif

% Nicer math symbols
\usepackage{eulervm}
%\usepackage{mathpazo}
\renewcommand\ttdefault{cmtt} % Computer Modern typewriter font, see lecture slides.

\usepackage{csquotes}

%%%%%%

%% Schönere Schriften
\usepackage[TS1,T1]{fontenc}

%% Bibliothek für Graphiken
\usepackage{graphicx}

%% der wird sowieso in jeder Datei gesetzt
\graphicspath{{../figures/}}

%% Anzeigetiefe für Inhaltsverzeichnis: 1 Stufe
\setcounter{tocdepth}{1}

%% Hyperlinks
\usepackage{hyperref}
% I don't know why, but this works and only includes sections and NOT subsections in the pdf-bookmarks.
\hypersetup{bookmarksdepth=subsection} 

%\usepackage{lmodern}
\usepackage{colortbl}
\usepackage[absolute,overlay]{textpos}
\usepackage{listings}
\usepackage{forloop}
%\usepackage{algorithmic} % PseudoCode package 

\usepackage{tikz}
\usetikzlibrary{matrix}
\usetikzlibrary{arrows.meta}
\usetikzlibrary{automata}
\usetikzlibrary{tikzmark}

% Needed for gbi-macros
\usepackage{xspace}

%%%%%%

%% Verbatim
\usepackage{moreverb}

%%%%%%%%%%%%%%%%%%%%%%%%%%%%%%%%%%%% Copy end

%% Tabellen
\usepackage{array}
\usepackage{multicol}
\usepackage{hhline}

%% Bibliotheken für viele mathematische Symbole
\usepackage{amsmath, amsfonts, amssymb}

%% Deutsche Silbentrennung und Beschriftungen
\usepackage[ngerman]{babel}

\usepackage{kbordermatrix}

% kbordermatrix settings
\renewcommand{\kbldelim}{(} % Left delimiter
\renewcommand{\kbrdelim}{)} % Right delimiter

\input{../config.tex}



% define custom \handout command flag if handout mode is toggled  #DirtyAsHellButWell...
\only<beamer:0>{\def\handout{}} %beamer:0 == handout mode

\newcommand{\R}{\mathbb{R}}
\newcommand{\N}{\mathbb{N}}
\newcommand{\Z}{\mathbb{Z}}
\newcommand{\Q}{\mathbb{Q}}
\newcommand{\BB}{\mathbb{B}}
\newcommand{\C}{\mathbb{C}}
\newcommand{\K}{\mathbb{K}}
\newcommand{\G}{\mathbb{G}}
\newcommand{\nullel}{\mathcal{O}}
\newcommand{\einsel}{\mathds{1}}
\newcommand{\Pot}{\mathcal{P}}
\renewcommand{\O}{\text{O}}

\def\word#1{\hbox{\textcolor{blue}{\texttt{#1}}}}
\let\literal\word
\def\mword#1{\hbox{\textcolor{blue}{$\mathtt{#1}$}}}  % math word
\def\sp{\scalebox{1}[.5]{\textvisiblespace}}
\def\wordsp{\word{\sp}}

%\newcommand{\literal}[1]{\textcolor{blue}{\texttt{#1}}}
\newcommand{\realTilde}{\textasciitilde \ }
\newcommand{\setsize}[1]{\ensuremath{\left\lvert #1 \right\rvert}}
\newcommand{\size}[1]{\setsize{#1}}  % Shame on you, TeXStudio...
\newcommand{\set}[1]{\left\{#1\right\}}
\newcommand{\tuple}[1]{\left(#1\right)}
\newcommand{\normalvar}[1]{\text{$#1$}}

% Modified by DJ
\let\oldemptyset\emptyset
\let\emptyset\varnothing % proper emptyset

\newcommand{\boder}{\ensuremath{\mathbin{\textcolor{blue}{\vee}}}\xspace}
\newcommand{\bund}{\ensuremath{\mathbin{\textcolor{blue}{\wedge}}}\xspace}
\newcommand{\bimp}{\ensuremath{\mathrel{\textcolor{blue}{\to}}}\xspace}
\newcommand{\bgdw}{\ensuremath{\mathrel{\textcolor{blue}{\leftrightarrow}}}\xspace}
\newcommand{\bnot}{\ensuremath{\textcolor{blue}{\neg}}\xspace}
\newcommand{\bone}{\ensuremath{\textcolor{blue}{1}}\text{}}
\newcommand{\bzero}{\ensuremath{\textcolor{blue}{0}}\text{}}
\newcommand{\bleftBr}{\ensuremath{\textcolor{blue}{\texttt{(}}}\text{}}
\newcommand{\brightBr}{\ensuremath{\textcolor{blue}{\texttt{)}}}\text{}}

% Fix of \b... commands:

\renewcommand{\boder}{\alor}
\renewcommand{\bund}{\aland}
\renewcommand{\bimp}{\alimpl}
\renewcommand{\bgdw}{\aleqv}
\renewcommand{\bnot}{\alnot}
\renewcommand{\bleftBr}{\alka}
\renewcommand{\brightBr}{\alkz}
\newcommand{\alA}{\word A}
\newcommand{\alB}{\word B}
\newcommand{\alC}{\word C}

\newcommand{\plB}{\plfoo{B}}
\newcommand{\plE}{\plfoo{E}}

\newcommand{\summe}[2]{\sum\limits_{#1}^{#2}}
\newcommand{\limes}[1]{\lim\limits_{#1}}

%\newcommand{\numpp}{\advance \value{weeknum} by -2 \theweeknum \advance \value{weeknum} by 2}
%\newcommand{\nump}{\advance \value{weeknum} by -1 \theweeknum \advance \value{weeknum} by 1}

\newcommand{\mycomment}[1]{}
\newcommand{\Comment}[1]{}

%% DISCLAIMER START 
% It is INSANELY IMPORTANT NOT TO DO THIS OUTSIDE BEAMER CLASS! IN ARTCILE DOCUMENTS, THIS IS VERY LIKELY TO BUG AROUND!
\makeatletter%
\@ifclassloaded{beamer}%
{
	% TODO 
	% no time...
	% redefine section to ignore multiple \section calls with the same title
}%
{
	\errmessage{ERROR: section command redefinition outside of beamer class document! Please contact the author of this code.}
}%
\makeatother%
%% DISCLAIMER END

\newcounter{abc}
\newenvironment{alist}{
  \begin{list}{(\alph{abc})}{
      \usecounter{abc}\setlength{\leftmargin}{8mm}\setlength{\labelsep}{2mm}
    }
}{\end{list}}


\newcommand{\stdarraystretch}{1.20}
\renewcommand{\arraystretch}{\stdarraystretch}  % for proper row spacing in tables

\newcommand{\morescalingdelimiters}{   % for proper \left( \right) typography
	\delimitershortfall=-1pt  
	\delimiterfactor=1
}

\newcommand{\centered}[1]{\vspace{-\baselineskip}\begin{center}#1\end{center}\vspace{-\baselineskip}}

% for \implitem and \item[bla] stuff to look right:
\setbeamercolor*{itemize item}{fg=black}
\setbeamercolor*{itemize subitem}{fg=black}
\setbeamercolor*{itemize subsubitem}{fg=black}

\setbeamercolor*{description item}{fg=black}
\setbeamercolor*{description subitem}{fg=black}
\setbeamercolor*{description subsubitem}{fg=black}

\renewcommand{\qedsymbol}{\textcolor{black}{\openbox}}

\renewcommand{\mod}{\mathop{\textbf{mod}}}
\renewcommand{\div}{\mathop{\textbf{div}}}

\newcommand{\ceil}[1]{\left\lceil#1\right\rceil}
\newcommand{\floor}[1]{\left\lfloor#1\right\rfloor}
\newcommand{\abs}[1]{\left\lvert #1 \right\rvert}
\newcommand{\Matrix}[1]{\begin{pmatrix} #1 \end{pmatrix}}
\newcommand{\braced}[1]{\left\lbrace #1 \right\rbrace}

% "something" placeholder. Useful for repairing spacing of operator sections, like `\sth = 42`.
\def\sth{\vphantom{.}}

\def\fract#1/#2 {\frac{#1}{#2}} % ! Trailing space is crucial!
\def\dfract#1/#2 {\dfrac{#1}{#2}} % ! Trailing space is crucial!

\newcommand{\Mid}{\;\middle|\;}

\let\after\circ



\def\·{\cdot}
\def\*{\cdot}
\def\?>{\ensuremath{\rightsquigarrow}}  % Fuck you, Latex
\def\~~>{\ensuremath{\rightsquigarrow}}  

\newcommand{\tight}[1]{{\renewcommand{\arraystretch}{0.76} #1}}
\newcommand{\stackedtight}[1]{\renewcommand{\arraystretch}{0.76} \begin{matrix} #1 \end{matrix} }
\newcommand{\stacked}[1]{\begin{matrix} #1 \end{matrix} }
\newcommand{\casesl}[1]{\delimitershortfall=0pt  \left\lbrace\hspace{-.3\baselineskip}\begin{array}{ll} #1 \end{array}\right.}
\newcommand{\casesr}[1]{\delimitershortfall=0pt  \left.\begin{array}{ll} #1 \end{array}\hspace{-.3\baselineskip}\right\rbrace}
\newcommand{\caseslr}[1]{\delimitershortfall=0pt  \left\lbrace\hspace{-.3\baselineskip}\begin{array}{ll} #1 \end{array}\hspace{-.3\baselineskip}\right\rbrace}

\def\q#1uad{\ifnum#1=0\relax\else\quad\q{\the\numexpr#1-1\relax}uad\fi}
% e.g. \q1uad = \quad, \q2uad = \qquad etc.

\newcommand{\qqquad}{\q3uad}
\newcommand{\minusquad}{\hspace{-1em}}

%% Placeholder utils
% \§{#1}   Saves #1 as placeholder and prints it
% \.       Prints an \hphantom with the size of the recalled placeholder.
\def\indentstring{}
\def\§#1{\def\indentstring{#1}#1}
\def\.{{$\hphantom{\text{\indentstring}}$}}
%% Placeholder utils end

\newcommand{\impl}{\ifmmode\ensuremath{\mskip\thinmuskip\Rightarrow\mskip\thinmuskip}\else$\Rightarrow$\fi\xspace}
\newcommand{\Impl}{\ifmmode\implies\else$\Longrightarrow$\fi\xspace}

\newcommand{\derives}{\Rightarrow}

\newcommand{\gdw}{\ifmmode\mskip\thickmuskip\Leftrightarrow\mskip\thickmuskip\else$\Leftrightarrow$\fi\xspace}
\newcommand{\Gdw}{\ifmmode\iff\else$\Longleftrightarrow$\fi\xspace}

% Legacy code from the algo tutorial slides. Perhaps useful. Try with care.
\mycomment{
	\newcommand{\impl}{\ifmmode\ensuremath{\mskip\thinmuskip\Rightarrow\mskip\thinmuskip}\else$\Rightarrow$\xspace\fi}  
	\newcommand{\Impl}{\ifmmode\implies\else$\Longrightarrow$\xspace\fi}
	
	\newcommand{\gdw}{\ifmmode\mskip\thickmuskip\Leftrightarrow\mskip\thickmuskip\else$\Leftrightarrow$\xspace\fi}
	\newcommand{\Gdw}{\ifmmode\iff\else$\Longleftrightarrow$\xspace\fi}
}
	
\newcommand{\gdwdef}{\ifmmode\mskip\thickmuskip:\Leftrightarrow\mskip\thickmuskip\else:$\Leftrightarrow$\xspace\fi}
\newcommand{\Gdwdef}{\ifmmode\mskip\thickmuskip:\Longleftrightarrow\mskip\thickmuskip\else:$\Longleftrightarrow$\xspace\fi}

\newcommand{\symbitemnegoffset}{\hspace{-.5\baselineskip}}
\newcommand{\implitem}{\item[\impl\symbitemnegoffset]}
\newcommand{\Implitem}{\item[\Impl\symbitemnegoffset]}


\newcommand{\forcenewline}{\mbox{}\\}

\newcommand{\bfalert}[1]{\textbf{\alert{#1}}}
\let\elem\in   % I'm a Haskell freak. Don't judge me. :P


\def\|#1|{\text{\normalfont #1}}  % | steht für senkrecht (anstatt kursiv wie sonst im math mode)


% proper math typography
\newcommand{\functionto}{\longrightarrow}
\renewcommand{\geq}{\geqslant}
\renewcommand{\leq}{\leqslant}
\let\oldsubset\subset
\renewcommand{\subset}{\subseteq} % for all idiots out there using subset

\newenvironment{threealign}{%
	\[
	\begin{array}{r@{\ }c@{\ }l}
}{%
	\end{array}	
	\]
}

\newcommand{\concludes}{ \\ \hline  }
\newcommand{\deduction}[1]{
	\begin{varwidth}{.8\linewidth}
		\begin{tabular}{>{$}c<{$}}
			#1
		\end{tabular}
	\end{varwidth}	
}

\definecolor{hoareorange}{rgb}{1,.85,.6}
\newcommand{\hoareassert}[1]{\setlength{\fboxsep}{1pt}\setlength{\fboxrule}{-1.4pt}\fcolorbox{white}{hoareorange}{\ensuremath{\{\;#1\;\}}}\setlength\fboxrule{\defaultfboxrule}\setlength{\fboxsep}{3pt}}

\newcommand{\mailto}[1]{\href{mailto:#1}{{\textcolor{blue}{\underline{#1}}}}}
\newcommand{\urlnamed}[2]{\href{#2}{\textcolor{blue}{\underline{#1}}}}
\renewcommand{\url}[1]{\urlnamed{#1}{#1}}

\newcommand{\hanging}{\hangindent=0.7cm}
\newcommand{\indented}{\hanging}


% \hstretchto prints #2 left-aligned into a box of the width of #1
\def\hstretchto#1#2{%
	\mbox{}\vphantom{#2}\rlap{#2}\hphantom{#1}%
}

\def\vstretchto#1#2{%
	\mbox{}\hphantom{#2}\smash{#2}\vphantom{#1}%
}


%requires \thisyear to be defined (s. config.tex)!
\edef\nextyear{\the\numexpr\thisyear+1\relax}


% --- \frameheight constant ---
\newlength\fullframeheight
\newlength\framewithtitleheight
\setlength\fullframeheight{.92\textheight}
\setlength\framewithtitleheight{.86\textheight}

\newlength\frameheight
\setlength\frameheight{\fullframeheight}

\let\frametitleentry\relax
\let\oldframetitle\frametitle
\def\newframetitle#1{\global\def\frametitleentry{#1}\if\relax\frametitleentry\relax\else\setlength\frameheight{\framewithtitleheight}\fi\oldframetitle{#1}}
\let\frametitle\newframetitle

\def\newframetitleoff{\let\frametitle\oldframetitle}
\def\newframetitleon{\let\frametitle\newframetitle}
% --- \frameheight constant end ---

\newcommand{\fakeframetitle}[1]{%
	\vspace{-2.05\baselineskip}%
	{\Large \textbf{#1}} \\%
	\smallskip
}



\newenvironment{headframe}{\Huge THIS IS AN ERROR. PLEASE CONTACT THE ADMIN OF THIS TEX CODE. (headframe env def failed)}{}
\RenewEnviron{headframe}[1][]{
	\begin{frame}\frametitle{\ }
		\centering
		\Huge\textbf{\textsc{\BODY} \\
		}
		\Large {#1}
		\frametitle{\ }
	\end{frame}
}


\makeatletter
% Provides color if undefined.
\newcommand{\colorprovide}[2]{%
	\@ifundefinedcolor{#1}{\colorlet{#1}{#2}}{}}
\makeatother


\colorprovide{lightred}{red!30}
\colorprovide{lightgreen}{green!40}
\colorprovide{lightyellow}{yellow!50}
\colorprovide{lightblue}{blue!10}
\colorprovide{beamerlightred}{lightred}
\colorprovide{beamerlightgreen}{lightgreen}
\colorprovide{beamerlightyellow}{lightyellow}
\colorprovide{beamerlightblue}{lightblue}
\colorprovide{fullred}{red!60}
\colorprovide{fullgreen}{green}
\definecolor{darkred}{RGB}{115,48,38}
\definecolor{darkgreen}{RGB}{48,115,38}
\definecolor{darkyellow}{RGB}{100,100,0}

\only<handout:0>{\colorlet{adaptinglightred}{beamerlightred}}
\only<handout:0>{\colorlet{adaptinglightgreen}{beamerlightgreen}}
\only<handout:0>{\colorlet{adaptinglightyellow}{beamerlightyellow}}
\only<handout:0>{\colorlet{adaptinglightblue}{beamerlightblue}}
\only<beamer:0>{\colorlet{adaptinglightred}{lightred}}
\only<beamer:0>{\colorlet{adaptinglightgreen}{lightgreen}}
\only<beamer:0>{\colorlet{adaptinglightyellow}{lightyellow}}
\only<beamer:0>{\colorlet{adaptinglightblue}{lightblue}}
\only<handout:0>{\colorlet{adaptingred}{lightred}}
\only<beamer:0>{\colorlet{adaptingred}{fullred}}
\only<handout:0>{\colorlet{adaptinggreen}{lightgreen}}
\only<beamer:0>{\colorlet{adaptinggreen}{fullgreen}}



\newcommand{\TrueQuestion}[1]{
	\TrueQuestionE{#1}{}
}

\newcommand{\YesQuestion}[1]{
	\YesQuestionE{#1}{}
}

\newcommand{\FalseQuestion}[1]{
	\FalseQuestionE{#1}{}
}

\newcommand{\NoQuestion}[1]{
	\NoQuestionE{#1}{}
}

\newcommand{\DependsQuestion}[1]{
	\DependsQuestionE{#1}{}
}

\newcommand{\QuestionVspace}{\vspace{4pt}}
\newcommand{\QuestionParbox}[1]{\begin{varwidth}{.85\linewidth}#1\end{varwidth}}
\newcommand{\ExplanationParbox}[1]{\begin{varwidth}{.97\linewidth}#1\end{varwidth}}
\colorlet{questionlightgray}{gray!23}
\let\defaultfboxrule\fboxrule

% #1: bg color
% #2: fg color short answer
% #3: short answer text
% #4: question
% #5: explanation
\newcommand{\GenericQuestion}[5]{
	\setlength\fboxrule{2pt}
	\only<+|handout:0>{\hspace{-2pt}\fcolorbox{white}{questionlightgray}{\QuestionParbox{#4} \quad\textbf{?}}}
	\visible<+->{\hspace{-2pt}\fcolorbox{white}{#1}{\QuestionParbox{#4} \quad\textbf{\textcolor{#2}{#3}}} \if\relax#5\relax\else\ExplanationParbox{#5}\fi} \\
	\setlength\fboxrule{\defaultfboxrule}
}

% #1: Q text
% #2: Explanation
\newcommand{\TrueQuestionE}[2]{
	\GenericQuestion{adaptinglightgreen}{darkgreen}{Wahr.}{#1}{#2}
}

% #1: Q text
% #2: Explanation
\newcommand{\YesQuestionE}[2]{
	\GenericQuestion{adaptinglightgreen}{darkgreen}{Ja.}{#1}{#2}
}

% #1: Q text
% #2: Explanation
\newcommand{\FalseQuestionE}[2]{
	\GenericQuestion{adaptinglightred}{darkred}{Falsch.}{#1}{#2}
}

% #1: Q text
% #2: Explanation
\newcommand{\NoQuestionE}[2]{
	\GenericQuestion{adaptinglightred}{darkred}{Nein.}{#1}{#2}
}

% #1: Q text
% #2: Explanation
\newcommand{\DependsQuestionE}[2]{
	\GenericQuestion{adaptinglightyellow}{darkyellow}{Je nachdem!}{#1}{#2}
}

% #1: Q text
% #2: Answer
\newcommand{\ContentQuestion}[2]{
	\GenericQuestion{adaptinglightblue}{black}{\minusquad}{#1}{#2}
}

\ifnum\thisyear=2018 \else \errmessage{Old ILIAS link inside preamble. Please update.} \fi

\newcommand{\ILIAS}{\urlnamed{ILIAS}{https://ilias.studium.kit.edu/ilias.php?ref\_id=855240\&cmdClass=ilrepositorygui\&cmdNode=5r\&baseClass=ilrepositorygui}\xspace}

\newcommand{\Socrative}{\ifdefined\mysocrativeroom \only<handout:0>{socrative.com $\quad \~~> \quad $ Student login \\ Raumname:  \mysocrativeroom\\ \medskip}\else\fi}

\newcommand{\thasse}[1]{
	\ifdefined\ThassesTut #1\xspace \else\fi
}
\newcommand{\daniel}[1]{
	\ifdefined\DanielsTut #1\xspace \else\fi
}
\newcommand{\thassedaniel}[2]{\ifdefined\ThassesTut #1\else\ifdefined\DanielsTut #2\fi\fi\xspace}

\ifdefined\ThassesTut \ifdefined\DanielsTut \errmessage{ERROR: Both ThassesTut and DanielsTut flags are set. This is most likely an error. Please check your config.tex file.} \else \fi \else \ifdefined\DanielsTut \else \errmessage{ERROR: Neither ThassesTut  nor DanielsTut flags are set. This is most likely an error. Please check your config.tex file.} \fi\fi

%\newcommand{\sgn}{\text{sgn}}

%%%%%%%%%%%% INHALT %%%%%%%%%%%%%%%%

%% Wochennummer
\newcounter{weeknum}

%% Titelinformationen
\title[GBI-Tutorium \mytutnumber, Woche \theweeknum]{Grundbegriffe der Informatik \\ Tutorium \mytutnumber}

\subtitle{Woche \theweeknum \ | \mydate{\theweeknum} \\ \myname \ \  \normalfont (\mailto{\mymail})}
\author[\myname]{\myname}
\institute{KIT -- Karlsruher Institut für Technologie}
\date{\mydate{\theweeknum}\ }

% Modified, DJ (better safe than sorry)
\AuthorTitleSep{ – }

%% Titel einfügen
\newcommand{\titleframe}{\frame{\titlepage}}

%% Alles starten mit \starttut{X}
\newcommand{\starttut}[1]{\setcounter{weeknum}{#1}\pdfinfo{
		/Author (\myname)
		/Title  (GBI-Tutorium \mytutnumber, Woche \theweeknum)
	}\titleframe\frame{\frametitle{Inhalt}\tableofcontents} \AtBeginSection[]{%
		\begin{frame}{Wo sind wir gerade?}
		\tableofcontents[currentsection]
	\end{frame}\addtocounter{framenumber}{-1}}}


\newcommand{\framePrevEpisode}{
\begin{headframe}
	\mylasttimestext
\end{headframe}
}

\newcommand{\lastframetitled}[6]{
	\frame{\frametitle{#6}
		\vspace{-#2\baselineskip}
		\begin{figure}[H]
			\centering
			\LARGE \textbf{\textsc{#5}} \\
			\vspace{.2\baselineskip}
			\includegraphics[#1]{#3}
			\vspace{-6pt}
			\begin{center}
				\small \url{#4} 
			\end{center}
		\end{figure} 
	}
}

% #1 number
% #2 title 
% #3 vspace (positive) without unit (\baselineskip)
\newcommand{\xkcdframe}[3]{
	\lastframetitled{width=.96\textwidth}{#3}{xkcd/#1}{http://xkcd.com/#1}{}{#2}
}

\newcommand{\xkcdframevert}[3]
{
	\lastframetitled{height=.96\frameheight}{#3}{xkcd/#1}{http://xkcd.com/#1}{}{#2}
}

% #1 number
% #2 title 
% #3 vspace (positive) without unit (\baselineskip)
% #4 \includegraphics[] optional parameters
\newcommand{\xkcdframecustom}[4]
{
	\lastframetitled{#4}{#3}{xkcd/#1}{http://xkcd.com/#1}{}{#2}
}

\newcommand{\slideThanks}{
	\begin{frame}
	\frametitle{Credits}
	\begin{block}{}
		An der Erstellung des Foliensatzes haben mitgewirkt:\\[1em]
		Daniel Jungkind \\
		Thassilo Helmold \\
		Philipp Basler \\
		Nils Braun \\
		Dominik Doerner \\
		Ou Yue \\
	\end{block}
\end{frame}
}

%% Wörter DEPRECATED! DO NOT USE
\newcommand{\code}[1]{$\mathbf{#1}$}

\morescalingdelimiters

\begin{document}
\starttut{14}

%\thasse{\lastframe{0.55}{20}{xkcd/turing_test.png}{https://www.xkcd.com/}}

\mycomment{
\begin{frame}{Zu Blatt \#6}
	Durchschnitt: \quad jeder, der regelmäßig abgab, \textbf{hat den Schein}! :D
	\begin{itemize}
		\item \textbf{A6.1}: $\fract \cdots/b $ \impl was, wenn $b = 0$? \lightning \\
		\impl Umformungen waren nicht nötig, einfach Axiome benutzen und fertig.
	\end{itemize}
\end{frame}
}

\framePrevEpisode

\begin{frame}{Rückblick: Endliche Automaten}
	\begin{itemize}[<+->]
		\item Mealy- und Moore-Automaten
		\item Formale Definition und graphische Repräsentation
		\item $f, f_*, f_{**}$
		\item $g, g_*, g_{**}$
		\item Endliche Akzeptoren
		%\item Reguläre Ausdrücke
	\end{itemize}
\end{frame}

\begin{frame}[t]{Wahr oder Falsch?}
	\Socrative
	\begin{figure}
		\begin{tikzpicture}[->,>=stealth,shorten >=1pt,auto,node distance=2cm,
		semithick,initial text={}]
		\tikzstyle{every state}=[]
		
		\node[state,accepting] (B)   {$Z_1$};
		\node[state]		 (M)  [right of=B]		{$Z_2$};
		
		\path
		(B) edge [loop above]  node {\word 1} (B) 
		(B) edge 			  node {\word 0} (M) 
		(M) edge [loop above] node {\word 0, \word 1} (M);
	\end{tikzpicture}
	\end{figure}
	
	\DependsQuestionE{Dieser Automat erkennt die Sprache $ \{\word 1\}^*$.} {Die Angabe des Startzustands fehlt. Startet man in $Z_2$, akzeptiert der Automat gar nichts.}
\end{frame}


\section{Rechtslineare Grammatiken}
\begin{frame}{Rechtslineare Grammatiken}
	\begin{Definition}
		Eine Grammatik $G = (N, T, S, P)$ nennt man \textbf{rechtslinear}, wenn bei jeder Produktion auf der rechten Seite \textbf{höchstens ein} Nichtterminalsymbol und dieses nur \textbf{als letztes} Symbol steht.\\
		D. h., alle Produktionen folgen dem Schema $$X \to w \quad \text{oder} \quad X \to wY$$ mit $w \in T^*, \; X,Y \in N$.
	\end{Definition}
\end{frame}

\begin{frame}{Reguläre Sprachen}
	\begin{Satz}
		Für jede formale Sprache $L$ sind die folgenden drei Aussagen äquivalent:
		\begin{itemize}
			\item $L$ kann von einem endlichen Akzeptor erkannt werden.
			\item $L$ kann durch einen regulären Ausdruck beschrieben werden.
			\item $L$ kann von einer rechtslinearen Grammatik erzeugt werden.
		\end{itemize}
	\end{Satz}
	
	Eine solche Sprache nennen wir \textbf{regulär}.
\end{frame}

\begin{frame}{Beispiele für Umwandlungen}
	Siehe Übung 13, WS 15/16
\end{frame}

\begin{frame}{Beispiele}
	$G = (\{X\}, \{\word a, \word b\}, X, P )$ mit 
	$$P = \{X \to \word aX \mid \word{ba}X \mid \word b \mid \varepsilon \}$$
	% Das da ist FALSCH:
	%$$P = \{X \to aX \mid bY \mid \varepsilon, Y \to aX \mid bZ \mid \varepsilon, Z \to aZ \mid bZ\}$$ 
	ist eine rechtslineare Grammatik. Die dadurch erzeugte Sprache ist \pause $$L(G) = \set{ w \mid \forall v_1, v_2 \in \{\word a,\word b\}^\ast: w \neq v_1 \word{bb} v_2 },$$ der sie beschreibende reguläre Ausdruck ist \pause $$R =  \rx{(a|ba)*(b|O*)}.  $$ Der Automat dazu sieht so aus:
\end{frame}

\begin{frame}
	\begin{figure}[H]
		\centering
		\includegraphics[width=\linewidth]{regulaer/L1.pdf}
	\end{figure}
\end{frame}

% TODO Nicht rechtslin. Grammar, deren Sprache regulär ist!
\begin{frame}{Achtung!}
	Eine Grammatik kann nicht rechtslinear sein und \textbf{trotzdem} eine reguläre Sprache erzeugen! \\
	\bigskip
	$G = (\set{A}, \set{\word a}, A, \set{A \to \alert{AA} \mid \word a \mid \eps})$, \quad (\alert{nicht} rechtslinear) \\ \pause
	$G' = (\set{A}, \set{\word a}, A, \set{A \to \word aA \mid \eps})$; \quad (rechtslinear) \\ 
	\medskip
	\[\text{und } \quad L(G) = L(G') = \lang{\rx{a*}} \quad \textbf{(regulär!)}\]
\end{frame}

\begin{frame}{Noch mehr Beispiele}
	\begin{itemize}
		\item $G = (\{X \}, \{\word a,\word  b\}, X , \{X \to \word{ab}X \mid \word{bba}X \mid \varepsilon \}$, \\
			$L(G) = \lang{\visible<2-|handout:2>{\rx{(ab|bba)*}}} $
		\item $G = (\{X , Y\}, \{\word a, \word b\}, X , \{X \to \word aX \mid \word bX \mid \word{ababb}Y , Y \to \word aY \mid \word bY \mid \varepsilon \}$, \\
			$L(G) = \lang{\visible<3-|handout:2>{\rx{(a|b)*ababb(a|b)*}}} $
	\end{itemize}
\end{frame}

% Sorry, die Zeit hab ich nicht...
\mycomment{
	\begin{frame}{Aufgabe}
		Gegeben ist im folgenden jeweils eine Beschreibung einer formalen Sprache $L$ und ein dazugehöriges Alphabet. Schreiben Sie jeweils den regulären Ausdruck $R$ auf, für den $L(R) = L $ gilt und stellen Sie eine rechtslineare Grammatik $G$ auf, für die $L(G) = L $ gilt:
		\begin{itemize}
			\item Die Menge aller Worte über dem Alphabet $A=\{a,b,c\}$, die genau ein c enthalten. \\
			\visible<2-|handout:2>{
				\emph{Lösung}: $(a|b)*c(a|b)*$
			}
			\item Die Menge aller Worte über dem Alphabet $A=\{a,b\}$, bei denen die Anzahl der $b$ durch 3 teilbar ist. \\
			\visible<3-|handout:2>{
				\emph{Lösung}: $a*(ba*ba*ba*)*$
			}
		\end{itemize}
	\end{frame}
	
	\begin{frame}{Aufgabe}
		\textit{Gegeben sei die rechtslineare Grammatik } $$ G= (\{S\},\{a,b\},S,P) \qquad P = \{S\to baaS | baS | aaS | \varepsilon \} $$
		\begin{itemize}
			\only<1-3|handout:1,2>{
				\item Geben Sie einen endlichen Akzeptor $A$ an, so dass $L(A) = L(G)$ gilt
				\only<3|handout:2>{
					\begin{figure}[H]
						\includegraphics[scale=0.9]{regulaer/L2.pdf}
					\end{figure}
				}
			}
			\only<1,4-5|handout:1,3>{
				\item Geben Sie einen regulären Ausdruck $R$ an, so dass $ \langle R \rangle = L(G) $ gilt
				\only<5|handout:3>{
					$$ (baa|ba|aa)\ast$$
				}
			}
			\only<1,6-7|handout:1,3>{
				\item Geben Sie einen regulären Ausdruck $R$ an, der nicht das Zeichen $|$ enthält, und für den $\langle R \rangle = L(G) $ gilt.
				\only<7|handout:3>{
					$$ (aa)\ast (baa\ast)\ast $$
				}
			}
		\end{itemize}
	\end{frame}
	
}



%\begin{frame}
%	\frametitle{Was wir können:}
%	Von..
%	\begin{description}
%		\item[..rechtslinearen Gammatiken..] zu
%		\begin{itemize}
%			\item den Akzeptoren: \pause (mind.) jedes Nichtterminalsymbol ein Zustand, $|$ ist Verzweigung, Akzeptierende Zustände wählen \pause
%			\item den regulären Ausdrücken: \pause Schwierig!
%		\end{itemize}
%		\item[..endlichen Akzeptoren..]  zu
%		\begin{itemize}
%			\item den Grammatiken: \pause Zustandsübergang ist eine Produktion\pause
%			\item den regulären Ausdrücken: \pause Einzelne Wege abgehen
%		\end{itemize}
%		\item[..regulären Ausdrücken..] zu
%		\begin{itemize}
%			\item den Akzeptoren: \pause in Abschnitte teilen, $\ast$ ist Schleife, $|$ ist Verzweigung \pause
%			\item den rechtslinearen Grammatiken: \pause genauso wie Akzeptor
%		\end{itemize}
%	\end{description}
%\end{frame}

% TODO: Fix dirty hack in thwregex.sty, where I changed line 42 to print star in math-mode
% 		because otherwise the star was always raised in my config.
%	COmment(Daniel): Reverted your change. No problems whatsoever.
%		Use \rx everywhere

% Fuck you, Latex. In my algo tutorial slides, this isn't necessary. Why here then!?
\begin{frame}[t]
	\fakeframetitle{\only<2->{Grenzen endlicher Akzeptoren}}
	Gibt es einen endlichen Akzeptor $A$ mit $$L(A) = \set{ \word a^k\word b^k \Mid k\in \N_0 }?$$
	\pause
	Nein! Warum nicht? \visible<3->{Endliche Automaten können nicht unendlich weit „zählen“!}\\
	
	Gibt es einen endlichen Akzeptor, der alle gültigen Klammerausdrücke erkennt?\\ \pause
	Nein, aus dem selben Grund.
	\begin{figure}[H]
		\includegraphics[scale=0.5]{xkcd/tags_1144}
		\caption{ \texttt{\url{https://www.xkcd.com/1144/}} }
	\end{figure}
	\pause
	Kontextfreie Grammatiken \enquote{können also mehr} als endliche Akzeptoren.\\
	Wir wollen nun ein \enquote{gleichmächtiges} Konzept zu Akzeptoren.
\end{frame}

\section{Reguläre Ausdrücke}
\subsection{Definition}

\begin{frame}{Disclaimer!}
	\begin{center}
		\Large
			\bfalert{\Huge ACHTUNG!} \\ \medskip
		
		Gemeint sind \textbf{NICHT} sog. \emph{Regular Expressions}, die ihr vllt. aus Programmiersprachen kennt! \\ \bigskip
		
		{\normalsize (Die sind ähnlich, aber eben nicht das gleiche.)}
	\end{center}
\end{frame}

\begin{frame}{Reguläre Ausdrücke}
	Wir können uns reguläre Ausdrücke zusammenbauen aus
	\begin{itemize}
		\item den einzelnen Symbolen $x$ aus $A$ \pause
		\item zwei regulären Ausdrücken $R_1$ und $R_2$ mit $$\rx(R_1 R_2\rx) \qquad \text{ oder } \qquad \rx(R_1\rx|R_2\rx)$$ \pause
		\item einem Stern $R\rx*$ \pause
		\item oder dem leeren Ausdruck $\rx O$ \pause
	\end{itemize} 
	Klammern dürfen nach den Klammerregeln weggelassen werden:\\
	Stern vor „Punkt“ {\small (in diesem Fall unsichtbar)} vor Strich.
\end{frame}

\begin{frame}{Reguläre Ausdrücke}
	\begin{Beispiel}
		Sei $ A = \{ \word a, \word b, \word c\}$. Dann sind gültige reguläre Ausdrücke über $A$:\\
		$\rx{abc}$\\
		$\rx{a|b|c}$\\
		$\rx{(ab)*}$\\
		$\rx{O*}$
	\end{Beispiel}
\end{frame}


\begin{frame}{Sprache eines Ausdruckes}
	Die durch $R$ beschriebene Sprache $\lang{R}$ ist wie folgt definiert:
	\begin{itemize}
		\item $\lang{\rx{O}} = \emptyset$
		\item $\lang{x}=\{x\} \quad (\text{für }x\in A)$
		\item $\lang{R_1 \rx| R_2} = \lang{R_1} \cup \lang{R_2}$
		\item $\lang{R_1 R_2} = \lang{R_1} \cdot \lang{R_2}$
		\item $\lang{R\rx*} = \lang{R}^*$
	\end{itemize} 

	\bigskip
	Eine Sprache, für die es einen beschreibenden regulären Ausdruck gibt, nennt man \textbf{regulär}.
\end{frame}

\begin{frame}{Sprache eines Ausdruckes}
	\begin{Beispiel}
		\begin{itemize}
			\item $\lang{\rx{a}} = \{\word a\}$. \pause
			\item $\lang{\rx{ab}} = \lang{\rx{a}} \cdot \lang{\rx{b}} = \{\word a\word b\}$. \pause
			\item $\lang{\rx{a|b}} = \lang{\rx{a}}\cup\lang{\rx{b}} = \{\word a,\word b\}$. \pause
			\item $\lang{\rx{(a|b)*}} = \lang{\rx{a|b}}^* = \{\word a,\word b\}^*$. \pause
			\item $\lang{\rx{(a*b*)*}} = \lang{\rx{a*b*}}^* = \left(\lang{\rx{a*}}\lang{\rx{b*}}\right)^* 
			= \left(\lang{\word a}^*\lang{\word b}^*\right)^* = \left(\{\word a\}^* \· \{\word b\}^*\right)^*$\\
			$= \{\word a,\word b\}^*$.  
		\end{itemize}
	\end{Beispiel}
	\pause
	\begin{block}{Aufgabe}
		Gebt einen regulären Ausdruck $R$ an mit $\lang{R} = $  
		\begin{itemize}
			\item die Sprache aller binären Zweierpotenzen \\ \visible<+->{}
				  \visible<+-|handout:2>{\impl $R = \rx{0*10*}$}
			\item die Sprache aller geraden Binärzahlen \\
				  \visible<+-|handout:2>{\impl $R = \rx{(0|1)*0}$}
		\end{itemize}
	\end{block}
\end{frame}

\begin{frame}{Aufgabe: Reguläre Ausdrücke}
	In dieser Aufgabe geht es um die formalen Sprachen
	$$L_1 = \set{\word a^k \word b^m \Mid k, m \in \N_0 }, \qquad L_2 = \set{\word b^k \word a^m \Mid k, m \in \N_0 }.$$
	Gebt für jede der folgenden formalen Sprachen $L$ je einen regulären Ausdruck $R$ an mit $ \langle R \rangle = L$.
	\begin{itemize}
		\item $L = L_1 \cup L_2$ \\
			\visible<2-|handout:2>{$ \rx{a*b*|b*a*}$}
		\item $L = L_1 \cap L_2$ \\
			\visible<3-|handout:2>{$ \rx{a*|b*}$}
		\item $L = L_1\cdot L_2$ \\
			\visible<4-|handout:2>{$ \rx{a*b*b*a*}$ oder $\rx{a*b*a*}$}
		\item $L = L_1^*$ \\
			\visible<5-|handout:2>{$ \rx{(a*b*)*}$ oder $\rx{(a|b)*}$}
	\end{itemize}
\end{frame}

\begin{frame}{Aufgabe: Sprachen regulärer Ausdrücke}
	\begin{itemize}
		\item $\lang{\rx{(a|b)*abb(a|b)*}} = \visible<2-|handout:2>{\{\word a, \word b\}^* \cdot \{\word a\word b\word b\} \cdot \{\word a, \word b\}^*}$
		\item $\lang{\rx{a**}} = \visible<3-|handout:2>{\{\word a\}^*}$
		\item $\lang{\visible<4-|handout:2>{R\rx{(}R\rx{)*}}} = \lang{R}^+$ \quad (für bel. reg. Ausdruck $R$)
		\item $\lang{\visible<5-|handout:2>{\rx{O*}}} = \{\eps\}$
		\item $\lang{\visible<6-|handout:2>{\rx{a*ba*ba*b(a|b)*}}} = \set{ w \in \{\word a, \word b\}^* \Mid \size{w}_{\word b} > 2 } $
		\item $\lang{\visible<7-|handout:2>{\rx{b*a*}}} =$ Sprache aller Wörter über $\set{\word a, \word b}$, in denen das Teilwort \word{ab} nicht vorkommt.
	\end{itemize}
\end{frame}


% Layout: Regexe vertikal zentriert neben die Automaten geht nicht... Gnarf.
\begin{frame}{Übung: Akzeptoren und Reguläre Ausdrücke}
	Gebt graphisch einen endlichen Akzeptor sowie einen regulären Ausdruck über dem Alphabet $X=\{\word a,\word b\}$ an, der folgende Sprache akzeptiert:
	\begin{itemize}
		\item \begin{tabular}{lr}
			Die leere Menge: \visible<2-|handout:2>{\ $\lang{\rx{O}}$} \qquad \mbox{} & Die Menge des leeren Wortes: \visible<3-|handout:2>{\	$\lang{\rx{O*}}$ } \\
			\visible<2-|handout:2>{
				\scalebox{.8}{
					\begin{tikzpicture}[->,>=stealth,shorten >=1pt,auto,semithick,node distance=2cm,initial text={}]
					\tikzstyle{every state}=[]
					
					\node[initial,state] (M)                    {$m$};
					
					\path (M) edge [loop right] node {\word a, \word b} (M);
					\end{tikzpicture}
				}
				
			}
			&
			\visible<3-|handout:2>{
				\scalebox{.8}{
					\begin{tikzpicture}[->,>=stealth,shorten >=1pt,auto,node distance=2cm,
					semithick,initial text={}]
					\tikzstyle{every state}=[]
					
					\node[initial,state,accepting] (A)                    {$a$};
					\node[state]		 		   (M)  [right of=A]		{$m$};
					
					\path
					(A) edge 			  node {\word a, \word b} (M) 
					(M) edge [loop right] node {\word a, \word b} (M);
					\end{tikzpicture}
				}
			}
		\end{tabular}  
		\item Die Menge aller Worte, die genau ein \word b enthalten:
		\visible<4-|handout:2>{\ $\lang{\rx{a*ba*}}$ \\
			\scalebox{.8}{
				\begin{tikzpicture}[->,>=stealth,shorten >=1pt,auto,node distance=2cm,
				semithick,initial text={}]
				\tikzstyle{every state}=[]
				
				\node[initial,state] (A)                    {$a$};
				\node[state,accepting] (B)  [right of=A]     {$b$};
				\node[state]		 (M)  [right of=B]		{$m$};
				
				\path
				(A) edge [loop above]  node {\word a} (A) 
				(A) edge 			  node {\word b} (B) 
				(B) edge [loop above]  node {\word a} (B) 
				(B) edge 			  node {\word b} (M) 
				(M) edge [loop above] node {\word a, \word b} (M);
				\end{tikzpicture}
			}
			
		}
		\item Die Menge aller Worte, bei denen die Anzahl der $\word b$ durch 3 teilbar ist: 
		\visible<5-|handout:2>{ 
			\begin{minipage}[t]{.5\linewidth}
				\vspace{-\ht\strutbox} \scalebox{.8}{\begin{tikzpicture}[->,>=stealth,shorten >=1pt,auto,node distance=2cm,
					semithick,initial text={}]
					\tikzstyle{every state}=[]
					
					\node[initial,state,accepting] (0)                    {$0$};
					\node[state] (1)  [right of=A]     {$1$};
					\node[state]		 (2)  [right of=B]		{$2$};
					
					\path
					(0) edge [loop above]  node {\word a} (0) 
					(0) edge 			  node {\word b} (1) 
					(1) edge [loop above]  node {\word a} (1) 
					(1) edge 			  node {\word b} (2) 
					(2) edge [loop right] node {\word a} (2)
					(2) edge [bend left]  node {\word b} (0) ;
					\end{tikzpicture}}	
				
			\end{minipage} 
			$\lang{\rx{(a*ba*ba*b)*a*}}$ 
		}
	\end{itemize}
\end{frame}

\begin{frame}[t]{Wahr oder Falsch?}
	\FalseQuestionE{Jede kontextfreie Grammatik lässt sich als regulärer Ausdruck \\ darstellen.}{Die Grammatik muss dafür rechtslinear sein, kontextfrei ist \enquote{zu mächtig}.}
	\TrueQuestion{Jeder regulären Ausdruck lässt sich durch eine kontextfreie Grammatik darstellen.}
	\TrueQuestion{Für jede Sprache $L$ und jedes Wort $w \in L$ gilt: Es existiert ein endlicher Automat, der $w$ erkennt.}
	\FalseQuestion{Die Sprache der gültigen Klammerausdrücke ist regulär.}
	\TrueQuestionE{Die Sprache der gültigen Klammerausdrücke ist kontextfrei.}{}
\end{frame}



\def\abbrsize{\footnotesize}
\begin{frame}	
	\begin{block}{Was ihr nun wissen solltet}
		\begin{itemize}
			\item Turingmaschinen
			\item Komplexität
			\item Entscheidbarkeit -- Wir können alles. Außer Halteproblem. Und so Zeug.
		\end{itemize}
	\end{block}
	
	\begin{block}{Und so geht es weiter...}
		\vspace{-.3\baselineskip}
		\begin{itemize}
			\item Algorithmen I -- Mehr zu Algorithmen, Laufzeiten, Datenstrukturen, Graphen
			\item \textbf{T}{\abbrsize echnische} \textbf{I}{\abbrsize nformatik} -- Realisierung von Schaltungen, Prozessoren (MIMA, ...)
			\item \textbf{T}{\abbrsize heoretische} \textbf{G}{\abbrsize rundlagen der} \textbf{I}{\abbrsize nformatik} -- Mehr zu Grammatiken, Komplexität, Entscheidbarkeit, Turingmaschinen
		\end{itemize}
	\end{block}
\end{frame}

\thassedaniel{
	\begin{frame}{Falls ihr mehr wollt...}
		\begin{block}{Persönliche Empfehlungen}
			\begin{itemize}
				\item Design and Analysis of Algorithms (für Algorithmen I)
				\item CS50x
				\item From Nand to Tetris
				\item ICPC-Basispraktikum
			\end{itemize}
		\end{block}
		
		\begin{itemize}
			\item EDX (edx.org)
			\item Coursera (coursera.org)
		\end{itemize}
	\end{frame}
}{
\begin{frame}{Falls ihr mehr wollt...} % S. o.
	\begin{block}{Persönliche Empfehlungen}
		\begin{itemize}
			\item ICPC-Basispraktikum
		\end{itemize}
	\end{block}
\end{frame}
}


\begin{frame}{Das war GBI}
	\begin{columns}
		\pause
		\column{0.4\linewidth}
		\begin{figure}[H]
			\vspace{-20pt}
			\includegraphics[scale=0.45]{xkcd/heaven}
		\end{figure}
	
		\pause
		\column{0.5\linewidth}
		\begin{figure}[H]
			\vspace{-20pt}
			\includegraphics[scale=0.45]{xkcd/hell}
		\end{figure}
	\end{columns}
\end{frame}

\begin{headframe}[ Viel Erfolg  bei \\ euren Klausuren! \smiley]
	--- The End ---
\end{headframe}

% Scheint leider kein vernünftiges Abschieds-XKCD zu geben
\thasse{
	\xkcdframevert{287}{}{4.0}
	%\xkcdframe{5.8}{30}{xkcd/np_complete.png}{http://www.xkcd.com}{}
}

\slideThanks

\end{document}