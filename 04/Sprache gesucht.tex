\documentclass[a4paper]{article}
\usepackage[latin1]{inputenc}
\usepackage{amsmath}
\usepackage{amsfonts}
\usepackage{amssymb}
\usepackage{graphicx}
\usepackage{palatino}
\begin{document}

	\Large
	
	% S1

	$$ \{\circ\} \cdot \{\triangle, \square\}^* $$
	\vspace{15mm}
	
	$$ \{w \in \Sigma^* \mid w = \circ \cdot v, v \in \{\triangle, \square\}^* \} $$
	\vspace{15mm}
	
	Die Sprache der W�rter, die mit einem Kreis beginnen und danach keinen Kreis mehr enthalten.
	\vspace{15mm}
	
	% S2

	$$ \{\triangle, \square, \circ\}^* \cdot \triangle \cdot \{\triangle, \square, \circ\} $$
	\vspace{15mm}
	
	$$ \{w \in \Sigma^* \mid w = v \cdot \triangle \cdot \mu, v \in \Sigma^*, \mu \in \Sigma \} $$
	\vspace{15mm}
	
	Die Sprache der W�rter, deren vorletztes Zeichen ein Dreieck ist.
	\pagebreak
	
	
	% S3
	
	$$ (\{\square, \circ\}^* \cdot \{\triangle\triangle\}^* )^* $$
	\vspace{15mm}
	
	Die Sprache der W�rter, in denen nirgends eine ungerade Anzahl an Dreiecken nebeneinander steht.
	\vspace{15mm}
	
	% S4
	
	$$ \{\triangle, \circ\}^* \cdot ( \{\square\} \cdot \{\triangle, \circ\}^* \cdot \{\square\} \cdot \{\triangle, \circ\}^* )^* $$
	\vspace{15mm}
	
	Die Sprache der W�rter, in denen eine gerade Anzahl an Vierecken vorkommt.
	\vspace{15mm}
	
	% S3
	
	$$ \{\square, \triangle\}^* \cdot ( \{\circ\} \cdot \{\square, \triangle\}^+ )^* $$
	\vspace{15mm}
	
	Die Sprache der W�rter, in denen nirgends zwei Kreise aufeinander folgen.
	\vspace{15mm}
	
\end{document}