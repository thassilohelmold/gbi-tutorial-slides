%beamer

% Comment/uncomment this line to toggle handout mode
%\newcommand{\handout}{}

%% Beamer-Klasse im korrekten Modus
\ifdefined \handout
\documentclass[handout]{beamer} % Handout mode
\else
\documentclass{beamer}
\fi

%% UTF-8-Encoding
\usepackage[utf8]{inputenc}

% % \bigtimes abgeschrieben von http://tex.stackexchange.com/questions/14386/importing-a-single-symbol-from-a-different-font
% \DeclareFontFamily{U}{mathx}{\hyphenchar\font45}
% \DeclareFontShape{U}{mathx}{m}{n}{
%       <5> <6> <7> <8> <9> <10> gen * mathx
%       <10.95> mathx10 <12> <14.4> <17.28> <20.74> <24.88> mathx12
%       }{}
% \DeclareSymbolFont{mathx}{U}{mathx}{m}{n}
% \DeclareMathSymbol{\bigtimes}{\mathop}{mathx}{161}

\RequirePackage{xcolor}

\def\9{\square}
%\def\9{\blank}

% f"ur Aussagenlogik
\colorlet{alcolor}{blue}
\RequirePackage{tikz}
\usetikzlibrary{arrows.meta}
\newcommand{\alimpl}{\mathrel{\tikz[x={(0.1ex,0ex)},y={(0ex,0.1ex)},>={Classical TikZ Rightarrow[]}]{\draw[alcolor,->,line width=0.7pt,line cap=round] (0,0) -- (15,0);\path (0,-6);}}}
\newcommand{\aleqv}{\mathrel{\tikz[x={(0.1ex,0ex)},y={(0ex,0.1ex)},>={Classical TikZ Rightarrow[]}]{\draw[alcolor,<->,line width=0.7pt,line cap=round] (0,0) -- (18,0);\path (0,-6);}}}
\newcommand{\aland}{\mathbin{\raisebox{-0.6pt}{\rotatebox{90}{\texttt{\color{alcolor}\char62}}}}}
\newcommand{\alor}{\mathbin{\raisebox{-0.8pt}{\rotatebox{90}{\texttt{\color{alcolor}\char60}}}}}
%\newcommand{\ali}[1]{_{\mathtt{\color{alcolor}#1}}}
\newcommand{\alv}[1]{\mathtt{\color{alcolor}#1}}
\newcommand{\alnot}{\mathop{\tikz[x={(0.1ex,0ex)},y={(0ex,0.1ex)}]{\draw[alcolor,line width=0.7pt,line cap=round,line join=round] (0,0) -- (10,0) -- (10,-4);\path (0,-8) ;}}}
\newcommand{\alP}{\alv{P}} %ali{#1}}
%\newcommand{\alka}{\negthinspace\hbox{\texttt{\color{alcolor}(}}}
\newcommand{\alka}{\negthinspace\text{\texttt{\color{alcolor}(}}}
%\newcommand{\alkz}{\texttt{\color{alcolor})}}\negthinspace}
\newcommand{\alkz}{\text{\texttt{\color{alcolor})}}\negthinspace}
\newcommand{\AAL}{A_{AL}}
\newcommand{\LAL}{\hbox{\textit{For}}_{AL}}
\newcommand{\AxAL}{\hbox{\textit{Ax}}_{AL}}
\newcommand{\AxEq}{\hbox{\textit{Ax}}_{Eq}}
\newcommand{\AxPL}{\hbox{\textit{Ax}}_{PL}}
\newcommand{\AALV}{\hbox{\textit{Var}}_{AL}}
\newcommand{\MP}{\hbox{\textit{MP}}}
\newcommand{\GEN}{\hbox{\textit{GEN}}}
\newcommand{\W}{\ensuremath{\hbox{\textbf{w}}}\xspace}
\newcommand{\F}{\ensuremath{\hbox{\textbf{f}}}\xspace}
\newcommand{\WF}{\ensuremath{\{\W,\F\}}\xspace}
\newcommand{\val}{\hbox{\textit{val}}}
\newcommand{\valDIb}{\val_{D,I,\beta}}

\newcommand*{\from}{\colon}

% die nachfolgenden Sachen angepasst an cmtt
\newlength{\ttquantwd}
\setlength{\ttquantwd}{1ex}
\newlength{\ttquantht}
\setlength{\ttquantht}{6.75pt}
\def\plall{%
  \tikz[line width=0.67pt,line cap=round,line join=round,baseline=(B),alcolor] {
    \draw (-0.5\ttquantwd,\ttquantht) -- node[coordinate,pos=0.4] (lll){} (-0.25pt,-0.0pt) -- (0.25pt,-0.0pt) -- node[coordinate,pos=0.6] (rrr){} (0.5\ttquantwd,\ttquantht);
    \draw (lll) -- (rrr);
    \coordinate (B) at (0,-0.35pt);
  }%
}
\def\plexist{%
  \tikz[line width=0.67pt,line cap=round,line join=round,baseline=(B),alcolor] {
    \draw (-0.9\ttquantwd,\ttquantht) -- (0,\ttquantht) -- node[coordinate,pos=0.5] (mmm){} (0,0) --  (-0.9\ttquantwd,0);
    \draw (mmm) -- ++(-0.75\ttquantwd,0);
    \coordinate (B) at (0,-0.35pt);
  }\ensuremath{\,}%
}
\let\plexists=\plexist
\newcommand{\NT}[1]{\ensuremath{\langle\mathrm{#1} \rangle}}

\newcommand{\CPL}{\text{\itshape Const}_{PL}}
\newcommand{\FPL}{\text{\itshape Fun}_{PL}}
\newcommand{\RPL}{\text{\itshape Rel}_{PL}}
\newcommand{\VPL}{\text{\itshape Var}_{PL}}
\newcommand{\ATer}{A_{\text{\itshape Ter}}}
\newcommand{\ARel}{A_{\text{\itshape Rel}}}
\newcommand{\AFor}{A_{\text{\itshape For}}}
\newcommand{\LTer}{L_{\text{\itshape Ter}}}
\newcommand{\LRel}{L_{\text{\itshape Rel}}}
\newcommand{\LFor}{L_{\text{\itshape For}}}
\newcommand{\NTer}{N_{\text{\itshape Ter}}}
\newcommand{\NRel}{N_{\text{\itshape Rel}}}
\newcommand{\NFor}{N_{\text{\itshape For}}}
\newcommand{\PTer}{P_{\text{\itshape Ter}}}
\newcommand{\PRel}{P_{\text{\itshape Rel}}}
\newcommand{\PFor}{P_{\text{\itshape For}}}

\newcommand{\plka}{\alka}
\newcommand{\plkz}{\alkz}
%\newcommand{\plka}{\plfoo{(}}
%\newcommand{\plkz}{\plfoo{)}}
\newcommand{\plcomma}{\hbox{\texttt{\color{alcolor},}}}
\newcommand{\pleq}{{\color{alcolor}\,\dot=\,}}

% MODIFIED (DJ)
% previously: \newcommand{\plfoo}[1]{\mathtt{\color{alcolor}#1}}
\newcommand{\plfoo}[1]{\texttt{\color{alcolor}#1}}

\newcommand{\plc}{\plfoo{c}}
\newcommand{\pld}{\plfoo{d}}
\newcommand{\plf}{\plfoo{f}}
\newcommand{\plg}{\plfoo{g}}
\newcommand{\plh}{\plfoo{h}}
\newcommand{\plx}{\plfoo{x}}
\newcommand{\ply}{\plfoo{y}}
\newcommand{\plz}{\plfoo{z}}
\newcommand{\plR}{\plfoo{R}}
\newcommand{\plS}{\plfoo{S}}

\newcommand{\bv}{\mathrm{bv}}
\newcommand{\fv}{\mathrm{fv}}

%\newcommand{\AxAL}{\hbox{\textit{Ax}}_{AL}}
%\newcommand{\AALV}{\hbox{\textit{Var}}_{AL}}

%\renewcommand{\#}[1]{\literal{#1}}
\newcommand{\A}{\mathcal{A}}
\newcommand{\Adr}{\text{Adr}}
\newcommand{\ar}{\mathrm{ar}}
\newcommand{\ascii}[1]{\literal{\char#1}}
%\newcommand{\assert}[1]{\text{/\!\!/\ } #1}
\newcommand{\assert}[1]{\colorbox{black!7!white}{\ensuremath{\{\;#1\;\}}}}
\newcommand{\Assert}[1]{$\langle$\textit{#1}$\rangle$}
\newcommand{\B}{\mathcal{B}}
\newcommand{\bfmod}{\mathbin{\kw{ mod }}}
\newcommand{\bb}{{\text{bb}}}
\def\bottom{\hbox{\small$\pmb{\bot}$}}
\newcommand{\card}[1]{|#1|}
%\newcommand{\cod}{\mathop{\text{cod}}}  % ist in thwmathabbrevs
\newcommand{\Conf}{\mathcal{C}}
\newcommand{\define}[1]{\emph{#1}}
%\renewcommand{\dh}{d.\,h.\@\xspace}
%\newcommand{\Dh}{D.\,h.\@\xspace}
%\newcommand{\engl}[1]{engl.\xspace\emph{#1}}
\newcommand{\eps}{\varepsilon}
%\newcommand{\evtl}{evtl.\@\xspace}
\newcommand{\fbin}{\text{bin}}
\newcommand{\finv}{\text{inv}}
\newcommand{\fnum}{\text{num}}
\newcommand{\fNum}{{\text{Num}}}
\newcommand{\frepr}{\text{repr}}
\newcommand{\fRepr}{\text{Repr}}
\newcommand{\fZkpl}{\text{Zkpl}}
\newcommand{\fLen}{\text{Len}}
\newcommand{\fsem}{\text{sem}}
\providecommand{\fspace}{\mathord{\text{space}}}
\providecommand{\fSpace}{\mathord{\text{Space}}}
\providecommand{\ftime}{\mathord{\text{time}}}
\providecommand{\fTime}{\mathord{\text{Time}}}
\newcommand{\fTrans}{\text{Trans}}
\newcommand{\fVal}{\text{Val}}

% MODIFIED (DJ)
\newcommand{\Val}{\text{Val}}

%\def\G{\mathbb{Z}}
\newcommand{\HT}[1]{\normalfont\textsc{HT-#1}}
\newcommand{\htr}[3]{\{#1\}\;#2\; \{#3\}}
\newcommand{\Id}{\text{I}}
%\newcommand{\ie}{i.\,e.\@\xspace}
\newcommand{\instr}[2]{\texttt{#1}\ \textit{#2}}
\newcommand{\Instr}[2]{\texttt{#1}\ \textrm{#2}}
\newcommand{\instrr}[3]{\texttt{#1}\ \textit{#2}\texttt{(#3)}}
\newcommand{\Instrr}[3]{\texttt{#1}\ \textrm{#2}\texttt{(#3)}}
\newcommand{\io}{\!\mid\!}
\usepackage{KITcolors}
\newcommand{\literal}[1]{\hbox{\textcolor{blue!95!white}{\textup{\texttt{\scalebox{1.11}{#1}}}}}}
%\newcommand{\literal}[1]{\hbox{\textcolor{KITblue!80!black}{\textup{\texttt{#1}}}}}
\def\kasten#1{\leavevmode\literal{\setlength{\fboxsep}{1pt}\fbox{\vrule  width 0pt height 1.5ex depth 0.5ex #1}}}
\newcommand{\kw}[1]{\ensuremath{\mathbf{#1}}}
\newcommand{\lang}[1]{\ensuremath{\langle#1\rangle}}
%\newcommand{\maw}{m.\,a.\,w.\@\xspace}
%\newcommand{\MaW}{M.\,a.\,w.\@\xspace}
\newcommand{\mdefine}[2][FOOBAR]{\define{#2}\def\foobar{FOOBAR}\def\optarg{#1}\ifx\foobar\optarg\def\optarg{#2}\fi\graffito{\optarg}}
\newcommand{\meins}{\rotatebox[origin=c]{180}{1}}
\newcommand{\Mem}{\text{Mem}}
\newcommand{\memread}{\text{memread}}
\newcommand{\memwrite}{\text{memwrite}}
\providecommand{\meta}[1]{\ensuremath{\langle}\textit{#1}\ensuremath{\rangle}}
%\newcommand{\N}{\mathbb{N}}
\newcommand{\NP}{\mathbf{NP}}
\newcommand{\Nadd}{N_{\text{add}}}
\newcommand{\Nmult}{N_{\text{mult}}}
% MODIFIED (DJ): added \!, mathcal{O}
\newcommand{\Oh}[1]{\mathcal{O}\!\left(#1\right)}
\newcommand{\Om}[1]{\Omega\!\left(#1\right)}
\newcommand{\personname}[1]{\textsc{#1}}
\newcommand{\regname}[1]{\texttt{#1}}
\newcommand{\mima}{\textsc{Mima}\xspace}
\newcommand{\mimax}{\textsc{Mima-X}\xspace}

\def\Pclass{\text{\bfseries P}}
\def\PSPACE{\text{\bfseries PSPACE}}

\newcommand{\SPush}{\text{push}}
\newcommand{\SPop}{\text{pop}}
\newcommand{\SPeek}{\text{peek}}
\newcommand{\STop}{\text{top}}
\newcommand{\STos}{\text{\itshape tos}}
\newcommand{\SBos}{\text{\itshape bos}}

%\newcommand{\R}{\mathbb{R}}
\newcommand{\Rnullplus}{\R_0^{+}}
\newcommand{\Rplus}{\R_{+}}
\newcommand{\resp}{resp.\@\xspace}
\newcommand{\Sem}{\text{Sem}}
\newcommand{\sgn}{\mathop{\text{sgn}}}
\newcommand{\sqbox}{\mathop{\raisebox{-6.2pt}{\hbox{\hbox to 0pt{$^{^{\sqcap}}$\hss}$^{^{\sqcup}}$}}}}
\newcommand{\sqleq}{\sqsubseteq}
\newcommand{\sqgeq}{\sqsupseteq}
% MODIFIED (DJ): added \!
\newcommand{\Th}[1]{\Theta\!\left(#1\right)}
%\newcommand{\usw}{usw.\@\xspace}
\newcommand{\V}[1]{\hbox{\textit{#1}}}
\newcommand{\x}{\times}
\newcommand{\ZK}{\mathbb{K}}
%\newcommand{\Z}{\mathbb{Z}}
\newcommand{\zB}{z.\,B.\@\xspace}
\newcommand{\ZB}{Z.\,B.\@\xspace}
% \newcommand{\bb}{{\text{bb}}}
% \def\##1{\hbox{\textcolor{darkblue}{\texttt{#1}}}}
% \def\A{\mathcal{A}}
% \newcommand{\0}{\#0}
% \newcommand{\1}{\#1}
% \newcommand{\Obj}{\text{Obj}}
% \newcommand{\start}{\mathop{\text{start}}}
% \newcommand{\compactlist}{\addtolength{\itemsep}{-\parskip}}
% \newcommand{\fval}{\text{val}}
% \newcommand{\lang}[1]{\ensuremath{\langle#1\rangle}}
% \newcommand{\io}{\!\mid\!}
% \def\sqbox{\mathop{\raisebox{-6.2pt}{\hbox{\hbox to 0pt{$^{^{\sqcap}}$\hss}$^{^{\sqcup}}$}}}}
% \def\sqleq{\sqsubseteq}
% \def\sqgeq{\sqsupseteq}
\def\Td{T_{\overline{d}}}
% \newcommand{\csym}[1]{\ensuremath{\#{c}_{\#{\hbox{\scriptsize #1}}}}}
% \newcommand{\F}{\ensuremath{\mathcal{F}}}
% \newcommand{\fsym}[2]{\ensuremath{\#{f}^{\#{\hbox{\scriptsize #1}}}_{\#{\hbox{\scriptsize #2}}}}}
% \newcommand{\rsym}[2]{\ensuremath{\#{R}^{\#{\hbox{\scriptsize #1}}}_{\#{\hbox{\scriptsize #2}}}}}
% \newcommand{\xsym}[1]{\ensuremath{\#{x}_{\#{\hbox{\scriptsize #1}}}}}
% \newcommand{\I}{\mathcal{I}}
% ********************************************************************

\usepackage[blue]{../framework/thwregex}
\usepackage{environ}
\usepackage{bm}
\usepackage{calc}
\usepackage{varwidth}
\usepackage{wasysym}
\usepackage{mathtools}


% Das ist der KIT-Stil
%\usepackage{../TutTexbib/beamerthemekit}
\usepackage[deutsch,titlepage0]{../framework/KIT/beamerthemeKITmod}
\TitleImage[width=\titleimagewd]{../figures/titlepage.jpg}
%\usetheme[deutsch,titlepage0]{KIT}

% Include PDFs
\usepackage{pdfpages}

% Libertine font (Original GBI font)
\usepackage{libertine}
%\renewcommand*\familydefault{\sfdefault}  %% Only if the base font of the document is to be sans serif

% Nicer math symbols
\usepackage{eulervm}
%\usepackage{mathpazo}
\renewcommand\ttdefault{cmtt} % Computer Modern typewriter font, see lecture slides.

\usepackage{csquotes}

%%%%%%

%% Schönere Schriften
\usepackage[TS1,T1]{fontenc}

%% Bibliothek für Graphiken
\usepackage{graphicx}

%% der wird sowieso in jeder Datei gesetzt
\graphicspath{{../figures/}}

%% Anzeigetiefe für Inhaltsverzeichnis: 1 Stufe
\setcounter{tocdepth}{1}

%% Hyperlinks
\usepackage{hyperref}
% I don't know why, but this works and only includes sections and NOT subsections in the pdf-bookmarks.
\hypersetup{bookmarksdepth=subsection} 

%\usepackage{lmodern}
\usepackage{colortbl}
\usepackage[absolute,overlay]{textpos}
\usepackage{listings}
\usepackage{forloop}
%\usepackage{algorithmic} % PseudoCode package 

\usepackage{tikz}
\usetikzlibrary{matrix}
\usetikzlibrary{arrows.meta}
\usetikzlibrary{automata}
\usetikzlibrary{tikzmark}

% Needed for gbi-macros
\usepackage{xspace}

%%%%%%

%% Verbatim
\usepackage{moreverb}

%%%%%%%%%%%%%%%%%%%%%%%%%%%%%%%%%%%% Copy end

%% Tabellen
\usepackage{array}
\usepackage{multicol}
\usepackage{hhline}

%% Bibliotheken für viele mathematische Symbole
\usepackage{amsmath, amsfonts, amssymb}

%% Deutsche Silbentrennung und Beschriftungen
\usepackage[ngerman]{babel}

\usepackage{kbordermatrix}

% kbordermatrix settings
\renewcommand{\kbldelim}{(} % Left delimiter
\renewcommand{\kbrdelim}{)} % Right delimiter

\input{../config.tex}



% define custom \handout command flag if handout mode is toggled  #DirtyAsHellButWell...
\only<beamer:0>{\def\handout{}} %beamer:0 == handout mode

\newcommand{\R}{\mathbb{R}}
\newcommand{\N}{\mathbb{N}}
\newcommand{\Z}{\mathbb{Z}}
\newcommand{\Q}{\mathbb{Q}}
\newcommand{\BB}{\mathbb{B}}
\newcommand{\C}{\mathbb{C}}
\newcommand{\K}{\mathbb{K}}
\newcommand{\G}{\mathbb{G}}
\newcommand{\nullel}{\mathcal{O}}
\newcommand{\einsel}{\mathds{1}}
\newcommand{\Pot}{\mathcal{P}}
\renewcommand{\O}{\text{O}}

\def\word#1{\hbox{\textcolor{blue}{\texttt{#1}}}}
\let\literal\word
\def\mword#1{\hbox{\textcolor{blue}{$\mathtt{#1}$}}}  % math word
\def\sp{\scalebox{1}[.5]{\textvisiblespace}}
\def\wordsp{\word{\sp}}

%\newcommand{\literal}[1]{\textcolor{blue}{\texttt{#1}}}
\newcommand{\realTilde}{\textasciitilde \ }
\newcommand{\setsize}[1]{\ensuremath{\left\lvert #1 \right\rvert}}
\newcommand{\size}[1]{\setsize{#1}}  % Shame on you, TeXStudio...
\newcommand{\set}[1]{\left\{#1\right\}}
\newcommand{\tuple}[1]{\left(#1\right)}
\newcommand{\normalvar}[1]{\text{$#1$}}

% Modified by DJ
\let\oldemptyset\emptyset
\let\emptyset\varnothing % proper emptyset

\newcommand{\boder}{\ensuremath{\mathbin{\textcolor{blue}{\vee}}}\xspace}
\newcommand{\bund}{\ensuremath{\mathbin{\textcolor{blue}{\wedge}}}\xspace}
\newcommand{\bimp}{\ensuremath{\mathrel{\textcolor{blue}{\to}}}\xspace}
\newcommand{\bgdw}{\ensuremath{\mathrel{\textcolor{blue}{\leftrightarrow}}}\xspace}
\newcommand{\bnot}{\ensuremath{\textcolor{blue}{\neg}}\xspace}
\newcommand{\bone}{\ensuremath{\textcolor{blue}{1}}\text{}}
\newcommand{\bzero}{\ensuremath{\textcolor{blue}{0}}\text{}}
\newcommand{\bleftBr}{\ensuremath{\textcolor{blue}{\texttt{(}}}\text{}}
\newcommand{\brightBr}{\ensuremath{\textcolor{blue}{\texttt{)}}}\text{}}

% Fix of \b... commands:

\renewcommand{\boder}{\alor}
\renewcommand{\bund}{\aland}
\renewcommand{\bimp}{\alimpl}
\renewcommand{\bgdw}{\aleqv}
\renewcommand{\bnot}{\alnot}
\renewcommand{\bleftBr}{\alka}
\renewcommand{\brightBr}{\alkz}
\newcommand{\alA}{\word A}
\newcommand{\alB}{\word B}
\newcommand{\alC}{\word C}

\newcommand{\plB}{\plfoo{B}}
\newcommand{\plE}{\plfoo{E}}

\newcommand{\summe}[2]{\sum\limits_{#1}^{#2}}
\newcommand{\limes}[1]{\lim\limits_{#1}}

%\newcommand{\numpp}{\advance \value{weeknum} by -2 \theweeknum \advance \value{weeknum} by 2}
%\newcommand{\nump}{\advance \value{weeknum} by -1 \theweeknum \advance \value{weeknum} by 1}

\newcommand{\mycomment}[1]{}
\newcommand{\Comment}[1]{}

%% DISCLAIMER START 
% It is INSANELY IMPORTANT NOT TO DO THIS OUTSIDE BEAMER CLASS! IN ARTCILE DOCUMENTS, THIS IS VERY LIKELY TO BUG AROUND!
\makeatletter%
\@ifclassloaded{beamer}%
{
	% TODO 
	% no time...
	% redefine section to ignore multiple \section calls with the same title
}%
{
	\errmessage{ERROR: section command redefinition outside of beamer class document! Please contact the author of this code.}
}%
\makeatother%
%% DISCLAIMER END

\newcounter{abc}
\newenvironment{alist}{
  \begin{list}{(\alph{abc})}{
      \usecounter{abc}\setlength{\leftmargin}{8mm}\setlength{\labelsep}{2mm}
    }
}{\end{list}}


\newcommand{\stdarraystretch}{1.20}
\renewcommand{\arraystretch}{\stdarraystretch}  % for proper row spacing in tables

\newcommand{\morescalingdelimiters}{   % for proper \left( \right) typography
	\delimitershortfall=-1pt  
	\delimiterfactor=1
}

\newcommand{\centered}[1]{\vspace{-\baselineskip}\begin{center}#1\end{center}\vspace{-\baselineskip}}

% for \implitem and \item[bla] stuff to look right:
\setbeamercolor*{itemize item}{fg=black}
\setbeamercolor*{itemize subitem}{fg=black}
\setbeamercolor*{itemize subsubitem}{fg=black}

\setbeamercolor*{description item}{fg=black}
\setbeamercolor*{description subitem}{fg=black}
\setbeamercolor*{description subsubitem}{fg=black}

\renewcommand{\qedsymbol}{\textcolor{black}{\openbox}}

\renewcommand{\mod}{\mathop{\textbf{mod}}}
\renewcommand{\div}{\mathop{\textbf{div}}}

\newcommand{\ceil}[1]{\left\lceil#1\right\rceil}
\newcommand{\floor}[1]{\left\lfloor#1\right\rfloor}
\newcommand{\abs}[1]{\left\lvert #1 \right\rvert}
\newcommand{\Matrix}[1]{\begin{pmatrix} #1 \end{pmatrix}}
\newcommand{\braced}[1]{\left\lbrace #1 \right\rbrace}

% "something" placeholder. Useful for repairing spacing of operator sections, like `\sth = 42`.
\def\sth{\vphantom{.}}

\def\fract#1/#2 {\frac{#1}{#2}} % ! Trailing space is crucial!
\def\dfract#1/#2 {\dfrac{#1}{#2}} % ! Trailing space is crucial!

\newcommand{\Mid}{\;\middle|\;}

\let\after\circ



\def\·{\cdot}
\def\*{\cdot}
\def\?>{\ensuremath{\rightsquigarrow}}  % Fuck you, Latex
\def\~~>{\ensuremath{\rightsquigarrow}}  

\newcommand{\tight}[1]{{\renewcommand{\arraystretch}{0.76} #1}}
\newcommand{\stackedtight}[1]{\renewcommand{\arraystretch}{0.76} \begin{matrix} #1 \end{matrix} }
\newcommand{\stacked}[1]{\begin{matrix} #1 \end{matrix} }
\newcommand{\casesl}[1]{\delimitershortfall=0pt  \left\lbrace\hspace{-.3\baselineskip}\begin{array}{ll} #1 \end{array}\right.}
\newcommand{\casesr}[1]{\delimitershortfall=0pt  \left.\begin{array}{ll} #1 \end{array}\hspace{-.3\baselineskip}\right\rbrace}
\newcommand{\caseslr}[1]{\delimitershortfall=0pt  \left\lbrace\hspace{-.3\baselineskip}\begin{array}{ll} #1 \end{array}\hspace{-.3\baselineskip}\right\rbrace}

\def\q#1uad{\ifnum#1=0\relax\else\quad\q{\the\numexpr#1-1\relax}uad\fi}
% e.g. \q1uad = \quad, \q2uad = \qquad etc.

\newcommand{\qqquad}{\q3uad}
\newcommand{\minusquad}{\hspace{-1em}}

%% Placeholder utils
% \§{#1}   Saves #1 as placeholder and prints it
% \.       Prints an \hphantom with the size of the recalled placeholder.
\def\indentstring{}
\def\§#1{\def\indentstring{#1}#1}
\def\.{{$\hphantom{\text{\indentstring}}$}}
%% Placeholder utils end

\newcommand{\impl}{\ifmmode\ensuremath{\mskip\thinmuskip\Rightarrow\mskip\thinmuskip}\else$\Rightarrow$\fi\xspace}
\newcommand{\Impl}{\ifmmode\implies\else$\Longrightarrow$\fi\xspace}

\newcommand{\derives}{\Rightarrow}

\newcommand{\gdw}{\ifmmode\mskip\thickmuskip\Leftrightarrow\mskip\thickmuskip\else$\Leftrightarrow$\fi\xspace}
\newcommand{\Gdw}{\ifmmode\iff\else$\Longleftrightarrow$\fi\xspace}

% Legacy code from the algo tutorial slides. Perhaps useful. Try with care.
\mycomment{
	\newcommand{\impl}{\ifmmode\ensuremath{\mskip\thinmuskip\Rightarrow\mskip\thinmuskip}\else$\Rightarrow$\xspace\fi}  
	\newcommand{\Impl}{\ifmmode\implies\else$\Longrightarrow$\xspace\fi}
	
	\newcommand{\gdw}{\ifmmode\mskip\thickmuskip\Leftrightarrow\mskip\thickmuskip\else$\Leftrightarrow$\xspace\fi}
	\newcommand{\Gdw}{\ifmmode\iff\else$\Longleftrightarrow$\xspace\fi}
}
	
\newcommand{\gdwdef}{\ifmmode\mskip\thickmuskip:\Leftrightarrow\mskip\thickmuskip\else:$\Leftrightarrow$\xspace\fi}
\newcommand{\Gdwdef}{\ifmmode\mskip\thickmuskip:\Longleftrightarrow\mskip\thickmuskip\else:$\Longleftrightarrow$\xspace\fi}

\newcommand{\symbitemnegoffset}{\hspace{-.5\baselineskip}}
\newcommand{\implitem}{\item[\impl\symbitemnegoffset]}
\newcommand{\Implitem}{\item[\Impl\symbitemnegoffset]}


\newcommand{\forcenewline}{\mbox{}\\}

\newcommand{\bfalert}[1]{\textbf{\alert{#1}}}
\let\elem\in   % I'm a Haskell freak. Don't judge me. :P


\def\|#1|{\text{\normalfont #1}}  % | steht für senkrecht (anstatt kursiv wie sonst im math mode)


% proper math typography
\newcommand{\functionto}{\longrightarrow}
\renewcommand{\geq}{\geqslant}
\renewcommand{\leq}{\leqslant}
\let\oldsubset\subset
\renewcommand{\subset}{\subseteq} % for all idiots out there using subset

\newenvironment{threealign}{%
	\[
	\begin{array}{r@{\ }c@{\ }l}
}{%
	\end{array}	
	\]
}

\newcommand{\concludes}{ \\ \hline  }
\newcommand{\deduction}[1]{
	\begin{varwidth}{.8\linewidth}
		\begin{tabular}{>{$}c<{$}}
			#1
		\end{tabular}
	\end{varwidth}	
}

\definecolor{hoareorange}{rgb}{1,.85,.6}
\newcommand{\hoareassert}[1]{\setlength{\fboxsep}{1pt}\setlength{\fboxrule}{-1.4pt}\fcolorbox{white}{hoareorange}{\ensuremath{\{\;#1\;\}}}\setlength\fboxrule{\defaultfboxrule}\setlength{\fboxsep}{3pt}}

\newcommand{\mailto}[1]{\href{mailto:#1}{{\textcolor{blue}{\underline{#1}}}}}
\newcommand{\urlnamed}[2]{\href{#2}{\textcolor{blue}{\underline{#1}}}}
\renewcommand{\url}[1]{\urlnamed{#1}{#1}}

\newcommand{\hanging}{\hangindent=0.7cm}
\newcommand{\indented}{\hanging}


% \hstretchto prints #2 left-aligned into a box of the width of #1
\def\hstretchto#1#2{%
	\mbox{}\vphantom{#2}\rlap{#2}\hphantom{#1}%
}

\def\vstretchto#1#2{%
	\mbox{}\hphantom{#2}\smash{#2}\vphantom{#1}%
}


%requires \thisyear to be defined (s. config.tex)!
\edef\nextyear{\the\numexpr\thisyear+1\relax}


% --- \frameheight constant ---
\newlength\fullframeheight
\newlength\framewithtitleheight
\setlength\fullframeheight{.92\textheight}
\setlength\framewithtitleheight{.86\textheight}

\newlength\frameheight
\setlength\frameheight{\fullframeheight}

\let\frametitleentry\relax
\let\oldframetitle\frametitle
\def\newframetitle#1{\global\def\frametitleentry{#1}\if\relax\frametitleentry\relax\else\setlength\frameheight{\framewithtitleheight}\fi\oldframetitle{#1}}
\let\frametitle\newframetitle

\def\newframetitleoff{\let\frametitle\oldframetitle}
\def\newframetitleon{\let\frametitle\newframetitle}
% --- \frameheight constant end ---

\newcommand{\fakeframetitle}[1]{%
	\vspace{-2.05\baselineskip}%
	{\Large \textbf{#1}} \\%
	\smallskip
}



\newenvironment{headframe}{\Huge THIS IS AN ERROR. PLEASE CONTACT THE ADMIN OF THIS TEX CODE. (headframe env def failed)}{}
\RenewEnviron{headframe}[1][]{
	\begin{frame}\frametitle{\ }
		\centering
		\Huge\textbf{\textsc{\BODY} \\
		}
		\Large {#1}
		\frametitle{\ }
	\end{frame}
}


\makeatletter
% Provides color if undefined.
\newcommand{\colorprovide}[2]{%
	\@ifundefinedcolor{#1}{\colorlet{#1}{#2}}{}}
\makeatother


\colorprovide{lightred}{red!30}
\colorprovide{lightgreen}{green!40}
\colorprovide{lightyellow}{yellow!50}
\colorprovide{lightblue}{blue!10}
\colorprovide{beamerlightred}{lightred}
\colorprovide{beamerlightgreen}{lightgreen}
\colorprovide{beamerlightyellow}{lightyellow}
\colorprovide{beamerlightblue}{lightblue}
\colorprovide{fullred}{red!60}
\colorprovide{fullgreen}{green}
\definecolor{darkred}{RGB}{115,48,38}
\definecolor{darkgreen}{RGB}{48,115,38}
\definecolor{darkyellow}{RGB}{100,100,0}

\only<handout:0>{\colorlet{adaptinglightred}{beamerlightred}}
\only<handout:0>{\colorlet{adaptinglightgreen}{beamerlightgreen}}
\only<handout:0>{\colorlet{adaptinglightyellow}{beamerlightyellow}}
\only<handout:0>{\colorlet{adaptinglightblue}{beamerlightblue}}
\only<beamer:0>{\colorlet{adaptinglightred}{lightred}}
\only<beamer:0>{\colorlet{adaptinglightgreen}{lightgreen}}
\only<beamer:0>{\colorlet{adaptinglightyellow}{lightyellow}}
\only<beamer:0>{\colorlet{adaptinglightblue}{lightblue}}
\only<handout:0>{\colorlet{adaptingred}{lightred}}
\only<beamer:0>{\colorlet{adaptingred}{fullred}}
\only<handout:0>{\colorlet{adaptinggreen}{lightgreen}}
\only<beamer:0>{\colorlet{adaptinggreen}{fullgreen}}



\newcommand{\TrueQuestion}[1]{
	\TrueQuestionE{#1}{}
}

\newcommand{\YesQuestion}[1]{
	\YesQuestionE{#1}{}
}

\newcommand{\FalseQuestion}[1]{
	\FalseQuestionE{#1}{}
}

\newcommand{\NoQuestion}[1]{
	\NoQuestionE{#1}{}
}

\newcommand{\DependsQuestion}[1]{
	\DependsQuestionE{#1}{}
}

\newcommand{\QuestionVspace}{\vspace{4pt}}
\newcommand{\QuestionParbox}[1]{\begin{varwidth}{.85\linewidth}#1\end{varwidth}}
\newcommand{\ExplanationParbox}[1]{\begin{varwidth}{.97\linewidth}#1\end{varwidth}}
\colorlet{questionlightgray}{gray!23}
\let\defaultfboxrule\fboxrule

% #1: bg color
% #2: fg color short answer
% #3: short answer text
% #4: question
% #5: explanation
\newcommand{\GenericQuestion}[5]{
	\setlength\fboxrule{2pt}
	\only<+|handout:0>{\hspace{-2pt}\fcolorbox{white}{questionlightgray}{\QuestionParbox{#4} \quad\textbf{?}}}
	\visible<+->{\hspace{-2pt}\fcolorbox{white}{#1}{\QuestionParbox{#4} \quad\textbf{\textcolor{#2}{#3}}} \if\relax#5\relax\else\ExplanationParbox{#5}\fi} \\
	\setlength\fboxrule{\defaultfboxrule}
}

% #1: Q text
% #2: Explanation
\newcommand{\TrueQuestionE}[2]{
	\GenericQuestion{adaptinglightgreen}{darkgreen}{Wahr.}{#1}{#2}
}

% #1: Q text
% #2: Explanation
\newcommand{\YesQuestionE}[2]{
	\GenericQuestion{adaptinglightgreen}{darkgreen}{Ja.}{#1}{#2}
}

% #1: Q text
% #2: Explanation
\newcommand{\FalseQuestionE}[2]{
	\GenericQuestion{adaptinglightred}{darkred}{Falsch.}{#1}{#2}
}

% #1: Q text
% #2: Explanation
\newcommand{\NoQuestionE}[2]{
	\GenericQuestion{adaptinglightred}{darkred}{Nein.}{#1}{#2}
}

% #1: Q text
% #2: Explanation
\newcommand{\DependsQuestionE}[2]{
	\GenericQuestion{adaptinglightyellow}{darkyellow}{Je nachdem!}{#1}{#2}
}

% #1: Q text
% #2: Answer
\newcommand{\ContentQuestion}[2]{
	\GenericQuestion{adaptinglightblue}{black}{\minusquad}{#1}{#2}
}

\ifnum\thisyear=2018 \else \errmessage{Old ILIAS link inside preamble. Please update.} \fi

\newcommand{\ILIAS}{\urlnamed{ILIAS}{https://ilias.studium.kit.edu/ilias.php?ref\_id=855240\&cmdClass=ilrepositorygui\&cmdNode=5r\&baseClass=ilrepositorygui}\xspace}

\newcommand{\Socrative}{\ifdefined\mysocrativeroom \only<handout:0>{socrative.com $\quad \~~> \quad $ Student login \\ Raumname:  \mysocrativeroom\\ \medskip}\else\fi}

\newcommand{\thasse}[1]{
	\ifdefined\ThassesTut #1\xspace \else\fi
}
\newcommand{\daniel}[1]{
	\ifdefined\DanielsTut #1\xspace \else\fi
}
\newcommand{\thassedaniel}[2]{\ifdefined\ThassesTut #1\else\ifdefined\DanielsTut #2\fi\fi\xspace}

\ifdefined\ThassesTut \ifdefined\DanielsTut \errmessage{ERROR: Both ThassesTut and DanielsTut flags are set. This is most likely an error. Please check your config.tex file.} \else \fi \else \ifdefined\DanielsTut \else \errmessage{ERROR: Neither ThassesTut  nor DanielsTut flags are set. This is most likely an error. Please check your config.tex file.} \fi\fi

%\newcommand{\sgn}{\text{sgn}}

%%%%%%%%%%%% INHALT %%%%%%%%%%%%%%%%

%% Wochennummer
\newcounter{weeknum}

%% Titelinformationen
\title[GBI-Tutorium \mytutnumber, Woche \theweeknum]{Grundbegriffe der Informatik \\ Tutorium \mytutnumber}

\subtitle{Woche \theweeknum \ | \mydate{\theweeknum} \\ \myname \ \  \normalfont (\mailto{\mymail})}
\author[\myname]{\myname}
\institute{KIT -- Karlsruher Institut für Technologie}
\date{\mydate{\theweeknum}\ }

% Modified, DJ (better safe than sorry)
\AuthorTitleSep{ – }

%% Titel einfügen
\newcommand{\titleframe}{\frame{\titlepage}}

%% Alles starten mit \starttut{X}
\newcommand{\starttut}[1]{\setcounter{weeknum}{#1}\pdfinfo{
		/Author (\myname)
		/Title  (GBI-Tutorium \mytutnumber, Woche \theweeknum)
	}\titleframe\frame{\frametitle{Inhalt}\tableofcontents} \AtBeginSection[]{%
		\begin{frame}{Wo sind wir gerade?}
		\tableofcontents[currentsection]
	\end{frame}\addtocounter{framenumber}{-1}}}


\newcommand{\framePrevEpisode}{
\begin{headframe}
	\mylasttimestext
\end{headframe}
}

\newcommand{\lastframetitled}[6]{
	\frame{\frametitle{#6}
		\vspace{-#2\baselineskip}
		\begin{figure}[H]
			\centering
			\LARGE \textbf{\textsc{#5}} \\
			\vspace{.2\baselineskip}
			\includegraphics[#1]{#3}
			\vspace{-6pt}
			\begin{center}
				\small \url{#4} 
			\end{center}
		\end{figure} 
	}
}

% #1 number
% #2 title 
% #3 vspace (positive) without unit (\baselineskip)
\newcommand{\xkcdframe}[3]{
	\lastframetitled{width=.96\textwidth}{#3}{xkcd/#1}{http://xkcd.com/#1}{}{#2}
}

\newcommand{\xkcdframevert}[3]
{
	\lastframetitled{height=.96\frameheight}{#3}{xkcd/#1}{http://xkcd.com/#1}{}{#2}
}

% #1 number
% #2 title 
% #3 vspace (positive) without unit (\baselineskip)
% #4 \includegraphics[] optional parameters
\newcommand{\xkcdframecustom}[4]
{
	\lastframetitled{#4}{#3}{xkcd/#1}{http://xkcd.com/#1}{}{#2}
}

\newcommand{\slideThanks}{
	\begin{frame}
	\frametitle{Credits}
	\begin{block}{}
		An der Erstellung des Foliensatzes haben mitgewirkt:\\[1em]
		Daniel Jungkind \\
		Thassilo Helmold \\
		Philipp Basler \\
		Nils Braun \\
		Dominik Doerner \\
		Ou Yue \\
	\end{block}
\end{frame}
}

%% Wörter DEPRECATED! DO NOT USE
\newcommand{\code}[1]{$\mathbf{#1}$}

\morescalingdelimiters

\begin{document}
\starttut{4}



\framePrevEpisode

\begin{frame}{Rückblick}
	\begin{itemize}
		\item \textbf{Aussagen} sind Sätze, die wahr oder falsch sind
		\item Wir können Aussagen mit \textbf{Konnektiven} zusammenbauen: \\
		$\bund, \boder, \bnot, \bimp$
		\item \textbf{Aussagevariablen} helfen dabei, konkrete Inhalte zu ignorieren 
		\item \textbf{Interpretationen} liefern Wahrheitswerte zu Variablen
		\item $val_I(\*)$ liefert Wahrheitswert für ganze Formel (rekursiv)
	\end{itemize}
\end{frame}


\section{Aussagenlogik}

\begin{frame}{Modelle}
	\begin{Definition}
		Sei $G$ eine aussagenlogische Formel. \\
		Eine Interpretation $I$ heißt \textbf{Modell} von $G$, wenn gilt: \quad $val_I(G) = \W$. \\
		\pause
		\medskip
		Sei $\Gamma$\thasse{\footnote{Gamma}} eine Formelmenge.
		Eine Interpretation $I$ heißt \textbf{Modell} von $\Gamma$, wenn für alle Formeln $G \in  \Gamma$ gilt: \quad $val_I(G) = \W$.
	\end{Definition}
	\bigskip
	\pause
	\begin{block}{Schreibweisen}
		$\Gamma \models G$: jedes Modell von $\Gamma$ auch Modell von $G$ \\
		Beispiel: \quad $\set{A,B,C} \models G$ \quad Jedes Modell von $A, B$ \textbf{und} $C$ ist auch Modell von $G$ \\
		\medskip
		Schreibe $H \models G$ statt $\set{H} \models G$ \\
		\medskip
		Schreibe $\models G$ statt $\set{} \models G$ \\
		\impl G ist \emph{Tautologie} oder \emph{allgemeingültig} 
	\end{block}
\end{frame}

\begin{frame}{Modelle}
	\begin{Definition}
		Eine Formel $G$ heißt \textbf{erfüllbar}, wenn für mindestens ein $I$ wahr. \\
		
	\end{Definition}
	\pause
	\begin{Beispiel}
		Alle Modelle von \mword{(C \boder \bnot C) \bimp \left(\bnot(B \bimp A)\right)}? \\
		\pause
		\impl $I_1, I_2 \from \set{\word A, \word B, \word C} \functionto \BB, \; 
		I_1(v) = 
		\caseslr{\F, & v = \word A \\
				\W, & v = \word B \\
				\W, & v = \word C}, \;
		I_2(v) = 
		\caseslr{\F, & v = \word A \\
				\W, & v = \word B \\
				\F, & v = \word C}$. \\
		\impl Formel \emph{erfüllbar}. \\
		\pause
		\medskip
		Alle Modelle von \mword{\bnot(C \bimp C)}? \\
		\pause
		\impl Gibt keine, Formel \emph{unerfüllbar}.
	\end{Beispiel}
\end{frame}

\begin{frame}{Tautologie, Äquivalenz}
	\begin{Definition}
		Eine Formel $G$ heißt \textbf{Tautologie}, wenn für alle möglichen $I$ wahr. \\
		\pause
		\medskip
		Kurzschreibweise: Für Formeln $G$ und $H$ ist \\
		$$\bleftBr G \bgdw H \brightBr :\equiv \bleftBr\bleftBr G \bimp H \brightBr \bund \bleftBr H \bimp G\brightBr\brightBr$$ \\
		\medskip
		\alert{\textbf{Bitte aufpassen mit Pfeilen: \quad $\bgdw$ vs. $\gdw$ \quad $\bimp$ vs. $\impl$}}
	\end{Definition}
	\pause
	\begin{block}{Lemma}
		Für Formeln $G$ und $H$ gilt \\
		\[ G \equiv H \quad \text{ genau dann, wenn } \quad G \bgdw H \text{ Tautologie ist.} \]
	\end{block}
	\pause
	\begin{Beispiel}
		\centered{$\bnot(\bnot G) \bgdw G$ ist Tautologie, also $\bnot(\bnot G) \equiv G$}
	\end{Beispiel}
\end{frame}

\begin{frame}{Aussagenkalkül}
	\begin{block}{Was ist ein Kalkül?}
		\begin{itemize}
			\item Ein \emph{Rechensystem}: \; \textbf{Dinge} und was man mit ihnen \textbf{anstellen} darf. \\
			Bsp.: \quad $\R$ und $+, -, \*, /$ \qquad Schachbrett, Figuren und Zugregeln
		\end{itemize}
	\end{block}
	\pause
	\begin{block}{Aussagenkalkül}
		Haben
		\begin{itemize}
			\item Syntaktisch korrekte Formeln $For_{AL}$ \\
			\impl können erfüllbar sein oder nicht 
			\item Davon nennen wir einige bestimmte \textbf{Axiome} \\
			\impl setzen wir als Tautologien voraus
			\item Eine \emph{Schlussregel}: \textbf{Modus Ponens} \\
			...um neue wahre Aussagen zu konstruieren
		\end{itemize}
	\end{block}
\end{frame}

\begin{frame}{Axiome, Modus Ponens}
	\begin{block}{Axiome}
		\includegraphics[width=.90\linewidth]{../figures/Axiome} \\
		Wir bestimmen: Das sind unsere „Basis-Tautologien“.
	\end{block}
	\pause
	\begin{block}{Modus Ponens (MP)}
		
		\begin{columns}[T] 
			\begin{column}[T]{.45\textwidth} 
				\vspace{-.6\baselineskip}
				\begin{itemize}
					\item<2-> Wenn $G$ gilt
					\item<2-> und $G \bimp H$ gilt \\ \mbox{}
					\implitem<2-> dann gilt auch $H$.
					\item<4-> Schreibweise: \deduction{G \qquad G \bimp H \concludes H} 
				\end{itemize}
			\end{column}
			\hspace{-2\baselineskip}
			\begin{column}[T]{.55\textwidth} 
				\vspace{-.6\baselineskip}
				\begin{itemize}
					\item<3-> Wir wissen: „Es regnet.“
					\item<3-> Wir erinnern uns: \\ „Wenn es regnet, ist die Straße nass.“
					\implitem<3-> Also wissen wir: „Die Straße ist nass“.
				\end{itemize}
				\hspace{.6\baselineskip} \only<5->{\fbox{\parbox{.9\linewidth}{Mit MP können wir aus bekannten \\ Wahrheiten neue konstruieren!}}}
			\end{column}
		\end{columns}
		
	\end{block}
\end{frame}

\begin{frame}{Ableitungen}
	Haben Formelsammlung $\Gamma$ („Hypothesen“/„Prämissen“), \\
	wollen eine Formel $G$ daraus ableiten
	\pause
	\begin{block}{Ableitung von $G$ aus $\Gamma$}
		Eine „Abfolge“ von Formeln, die in $G$ mündet \quad (Schreibweise: \; $\Gamma \vdash G$) \\
		Was dürfen wir machen?
		\pause
		\begin{itemize}
			\item<+-> aus syntaktisch korrekten Formeln \emph{Axiome} bilden und hinschreiben
			\item<.-> \emph{Prämissen} aus $\Gamma$ hinschreiben
			\item<.-> aus zwei vorherigen Formeln mit \emph{Modus Ponens} eine neue konstruieren
			\implitem<+-> das machen wir solange, bis wir $G$ konstruiert haben
		\end{itemize}
	\end{block}
\end{frame}

\begin{frame}{Beweisbarkeit}
	\begin{block}{Beweis von $G$}
		\impl Ableitung von $G$ aus $\Gamma  = \emptyset$ \\
		\impl Wir verwenden nur Axiome und MP! \\
		Schreibweise: \quad $\vdash G$ \qquad „$G$ ist beweisbar“ \\
		Ein solches beweisbares $G$ nennen wir \textbf{Theorem} des Kalküls.
	\end{block}
	\pause 
	\begin{block}{Lemma}
		Eine AL-Formel $G$ ist genau dann Tautologie, wenn $G$ ein Theorem des AL-Kalküls ($=$~im Kalkül beweisbar) ist. \\
		\smallskip
		\centered{--- bzw. ---} 
		\smallskip
		Für jede AL-Formel $G$ gilt: \qquad $\models G \; \Gdw \; \vdash G$.\\
	\end{block}
	(Achtung: Es gibt Kalküle / Logiken, für die so etwas nicht gilt!)
	\pause
	\begin{block}{Lemma}
		Für Formeln $G$, $H$ gilt $G \vdash H$ genau dann, wenn $\vdash \bleftBr G \bimp H \brightBr$.
	\end{block}
\end{frame}


\section{Vollständige Induktion}

\morescalingdelimiters

% Induktion Vorstellung
\begin{frame}{Vollständige Induktion}
	Wir haben: Aussage $A_n$ für alle $n \in \N_0$ (z.B. $A_n$: „$\size{\word{a}^n} = n$“) \\
	Wir wollen beweisen: $A_n$ ist für alle $n \in \N_0$ wahr \\[0.5em]
	\pause
	Zeige dazu: \centered{	
		$A_n$ ist für $n = 0$ wahr  \\
		\textbf{und} \\
		Wenn $A_n$ für $n$ wahr ist, dann ist $A_n$ auch für $n+1$ wahr 
	}
\end{frame}

\begin{frame}{Vorgehen}
	Behauptung: $\forall n \in \N_0: (n^3 - n) \text{ ist durch 3 teilbar (tb)}$.
	\pause
	\begin{block}{Induktionsanfang (IA)}
		Beweise die Aussage für die erste Zahl (Basisfall):\\
		$n = 0 \impl (0^3 - 0) = 0$ ist durch 3 tb. \; \textbf{\checked}
	\end{block}
	\pause
	\begin{block}{Induktionsvoraussetzung (IV)}
		Wir nehmen ein $n$, von dem wir schon gezeigt haben, dass die Aussage gilt:\\
		Für \textbf{ein beliebiges aber festes} $n \in \N_0$ gelte: $(n^3 - n)$ ist durch 3 tb. 
	\end{block}
\end{frame}

\begin{frame}{Vorgehen}
	\begin{block}{Induktionsschritt (IS)}
		Zeige die Aussage für $n+1$, verwende dabei die IV.\\
		\pause
		\medskip
		Wir formen erst mal um:
		\begin{align*}
			(n+1)^3 - (n+1) &= n^3 + 3n^2 + 3n + 1 - n - 1 \\
			&= (n^3 - n) + (3n^2 + 3n) \\
			&= \underbrace{(n^3-n)}_{\shortstack{\footnotesize nach IV \\ \footnotesize durch 3 tb.}} + \underbrace{3 \* (n^2 + n)}_{\shortstack{\footnotesize offensichtlich \\ \footnotesize durch 3 tb.}}. \qed
		\end{align*}
	\end{block}

\end{frame}

\begin{frame}{Induktionsvoraussetzung}
	\Huge \centering
	\alert{
		„Für \textbf{ein} $n \in \N_0$ gelte ...“ \\
		\bigskip
		{ \LARGE
		Nicht: „Für alle...“,\\
		das wollen wir mit der Induktion ja erst zeigen!
		}
	}
\end{frame}

\begin{frame}[t]{Vollständige Induktion}
	\begin{itemize}
		\item Der Schraubenzieher im Beweis-Werkzeugkasten des Informatikers
		\item \textbf{Einfaches} Prinzip (Verstehen: Reines Auswendiglernen des Schemas kann schief gehen!), vielfältige Anwendungsmöglichkeiten
		\item \textbf{Variationen} möglich: Induktionsanfang bei $1, 42, ...$
	\end{itemize}
	
	\FalseQuestionE{Ich benutze für jeden Beweis Induktion.}{Mit einem Schraubenzieher bekommt man keinen Nagel in die Wand.}
\end{frame}


\begin{frame}{Zum Aufwärmen: Vogelfarben}
	Wir zeigen nun:\\[2em]
	{\LARGE
	Alle Vögel haben die gleiche Farbe!}\\
	\bigskip
	\only<beamer:0>{Achtung: Dieser Beweis ist natürlich \textbf{kaputt}!}
\end{frame}

\begin{frame}[t]{Zum Aufwärmen: Vogelfarben}
	\only<1|handout:1>{
		Das machen wir mit \textbf{vollständiger Induktion} und zeigen die folgende, äquivalente Aussage: \\
	}
	
	\only<1-3|handout:1-2>{
		\[
			\begin{array}{r@{\ }l}
			\forall n \in \N_+ : &\text{ In jeder Menge, die genau } n \text{ Vögel enthält,} \\
								 &\text{ haben alle Vögel die gleiche Farbe.}
			\end{array}
		\]
		
	}
	
	\only<2-3|handout:2>{	
		\begin{block}{Induktionsanfang}
			$n = 1$: Wenn eine Menge genau einen Vogel enthält, dann haben
			offensichtlich alle Vögel die gleiche Farbe. \textbf{\checked}
		\end{block}
	}
	\only<3|handout:2>{
		\begin{block}{Induktionsvoraussetzung}
			Für ein beliebiges aber festes $n$ gelte: In jeder
			Menge, die genau $n$ Vögel enthält, haben alle Vögel die gleiche Farbe.
		\end{block}
	}

	\only<4-5|handout:3-4> {
	\begin{block}{Induktionsschluss}
		\only<4|handout:3>{
			Wir zeigen die Aussage für $n+1$: Sei also $M$ eine Menge,
			die genau $n+1$ Vögel enthalte. Wir stellen uns vor, dass die Vögel
			alle nebeneinander sitzen:
		}
	
		\begin{figure}
			\includegraphics[scale=0.4]{induktion_voegel}
			\centering
		\end{figure}
	
		\only<5|handout:4>{
			Die Vögel $1, 2, ..., n$ bilden eine Menge mit genau $n$ Vögeln. Also haben sie nach IV alle die gleiche Farbe.\\ 
			Die Vögel $2, 3, ..., n + 1$ bilden auch eine Menge mit genau $n$ Vögeln. Also haben nach IV auch diese alle die gleiche Farbe.\\
			Damit haben auch die Vögel $1$ und $n + 1$ die gleiche Farbe, also haben alle Vögel die gleiche Farbe. $\qed$
		}
	\end{block}
	}
\end{frame}

\begin{frame}[t]{Vogelfarben: Auflösung}
	\begin{block}{Was geht schief?}
		\begin{figure}
			\includegraphics[scale=0.4]{induktion_voegel}
			\centering
		\end{figure}
		\pause
		Hübsches Bild. Scheint sauber. Ist bloß für $n=2$ völlig kaputt, die beiden Teilmengen \textbf{überlappen sich} dann nämlich \textbf{nicht}. \\
		\impl Wir können nicht sagen, dass der erste und letzte Vogel immer die gleiche Farbe haben. (Und wenn schon zwei Vögel nicht immer die gleiche Farbe haben, dann drei etc. auch nicht.) \\
		\impl Ganze Induktion \textbf{kaputt}. \frownie
	\end{block}	
	%Das Bild ist zwar außerordentlich hübsch, suggeriert aber leider etwas, was nicht immer stimmt: Für $n = 2$ überlappen sich die Teilmengen „ohne den ersten“ und „ohne den letzten“ Vogel nicht. Es ist also nicht erzwungen, dass beide Vögel die gleiche Farbe haben. (Und das macht „alles weitere“ auch kaputt: Wenn nicht immer 2 Vögel die gleiche Farbe haben, dann auch nicht immer 3 Vögel, usw.)
\end{frame}

% Induktion Übung
% -------------------------

\begin{frame}{Und jetzt ihr}
	Behauptung: \[\forall n \in \N_+ : \sum_{k=0}^{n}{\frac{1}{2^k}} = 2 \* (1 - \frac{1}{2^{n+1}})\]
	\begin{block}{Induktionsanfang}
		$n = 1$: $\sum_{k=0}^{1}{\frac{1}{2^k}} = \frac{3}{2} = 2 \* \frac{3}{4}$. \; \textbf{\checked}
	\end{block}
	\begin{block}{Induktionsvoraussetzung}
		Für ein $n \in \N_0$ gelte: $\sum_{k=0}^{n}{\frac{1}{2^k}} = 2 \* (1 - \frac{1}{2^{n+1}})$.
	\end{block}
\end{frame}

\begin{frame}{Und jetzt ihr}

	\begin{block}{Induktionsschritt}
		Zeige die Aussage für $n+1$:\\
		\begin{align*}
			\sum_{k=0}^{n+1}{\frac{1}{2^k}}
				&= \underbrace{\sum_{k=0}^{n}{\frac{1}{2^k}}}_{\stackrel{\text{IV}}{=} 2 \* (1 - \frac{1}{2^{n+1}})} + \frac{1}{2^{n+1}}\\
				&= 2 \* (1 - \frac{1}{2^{n+1}}) + \frac{1}{2^{n+1}}\\
				&= 2 \* (1 - \frac{2}{2^{n+2}} + \frac{1}{2^{n+2}})\\
				&= 2 \* (1 - \frac{1}{2^{(n+1)+1}}). \qed
		\end{align*}
	\end{block}
\end{frame}
% -------------------------

% Induktion Wörter Länge
\begin{frame}{Und jetzt mit Wörtern}
	\begin{block}{Behauptung}
		Seien $A, B$ zwei beliebige Alphabete. Definiere die Funktion $f \from A^* \functionto A^*$,
		\begin{align*}
			f(\eps) &:= \eps \\
			\text{Für } w \in A^*, \mu \in A: \quad f(\mu \· w) &:= 
			\begin{cases}
				\mu \· f(w), &\mu \in B \\
				f(w), &\text{sonst}
			\end{cases}\\
		\end{align*}
	
	Dann gilt $\forall w \in A^*: \size{f(w)} \le \size w$.
	\end{block}
\end{frame}

\begin{frame}{Und jetzt mit Wörtern}
	Induktion über die Wortlänge ($n = \size w$):\\[0.5em]
	\pause
	\begin{block}{Induktionsanfang}
		$n = 0$: Nur das leere Wort hat Länge 0. Also $w = \eps$.\\
		$f(\eps) = \eps \impl \size{f(w)} = \size w = 0$. \; \textbf{\checked}
	\end{block}
	\pause
	\begin{block}{Induktionsvoraussetzung}
		Für \textbf{ein} $n \in \N_0$ gelte: Für alle Wörter der Länge $n$ über $A$ (also $w \in A^n$) ist $\size{f(w)} \le \size w$.
	\end{block}
\end{frame}

\begin{frame}{Und jetzt mit Wörtern}
	\begin{block}{Induktionsschritt}
		Zeige die Aussage für $n+1$:\\
		Sei $w \in A^{n+1}$ ein Wort der Länge $n+1$.\\
		\pause
		Dann teilen wir es auf in $w = \mu \· v$, wobei $\mu \in A$ und $v \in A^n$.\\
		Nach IV gilt: $\size{f(v)} \le \size v$.\\
		\pause
		\smallskip
		\textbf{Fall 1}: \\
			\quad $\mu \in B$: $f(w) = \mu \· f(v)$ \\
			\quad \impl $\size{f(w)} = 1 + \size{f(v)} \le 1 + \size v = 1+n = \size w$.\\
		\pause
		\smallskip
		\textbf{Fall 2}: \\
			\quad $\mu \notin B$: $f(w) = f(v)$ \\
			\quad \impl $\size{f(w)} = \size{f(v)} \le \size v = n \le n+1 = \size w$.\\
		\pause
		\smallskip
		Also gilt: $\size{f(w)} \le \size w. \qed$
	\end{block}
\end{frame}

\section{Sprachen: Aufwärmen}


\begin{frame}{Rückblick}
	\begin{itemize}
		\item \textbf{Alphabet} $A$ mit Zeichen, aus denen wir Wörter zusammenbauen
		\item Nicht immer haben all diese Wörter einen Sinn
		\item Wir definieren selbst, welche Wörter wir als korrekt ansehen und akzeptieren wollen.
		\item Eine solche Teilmenge aller möglichen Wörter nennen wir \textbf{formale Sprache}
	\end{itemize}
\end{frame}

\begin{frame}{Rückblick}
	Auf formale Sprachen können wir \textbf{ähnliche} Operationen anwenden wie auf Wörtern:
	\begin{itemize}
		\item $L_1 \cdot L_2 = \{w_1 w_2 \mid w_1 \in L_1 \text{ und } w_2 \in L_2 \}$\\
		Jeweils ein Wort aus $L_1$ konkateniert mit einem Wort aus $L_2$.
		\pause
		\item $L^0 = \{\varepsilon \}, \qquad L^{i+1} = L^i \cdot L$\\
		Alle Wörter, die aus $i = 0,1,2...$ Wörtern der Sprache zusammengesetzt wurden
		\pause
		\item $L^+ = \bigcup \limits_{i=1}^\infty L^i \qquad L^* = L^+ \cup L^0$\\
		Alle Wörter, die sich aus den Wörtern der Sprache bilden lassen \\ 
		(ohne/mit zusätzlichem $\eps$ als Würze).
	\end{itemize}
	Ein Alphabet selbst ist \textbf{auch} ne formale Sprache, nämlich mit Wörtern der Länge 1.
\end{frame}

\begin{frame}[t]{Wahr oder falsch?}
	\FalseQuestionE{Jede Sprache enthält Wörter.}{ $\emptyset$ ist auch eine gültige Sprache.}
	\FalseQuestionE{$\word{01}^* = \{\eps, \word{01}, \word{0101}, ...\}$}{$\word{01}^*$ gibt es nicht, denn $\word{01} \neq \{\word{01}\}$.}
	\TrueQuestionE{Es gibt Sprachen $L$, für die gilt $\eps \in L^+$.}{\ZB $L = \set{\eps, \word{aaa}}$.}
	\FalseQuestionE{$L^+ = L^* \setminus L^0$.}{Gilt nicht, wenn $\varepsilon \in L$.}
	\TrueQuestionE{$\{\}^* \neq \{\} $.}{ $\{\}^* = \{\varepsilon\}$.}	
\end{frame}

% TODO: Im letzten Jahr hat die Zeit nicht gereicht,
% daher diesen Inhalt hier eher streichen.
\def\mycircle{\raisebox{1pt}{\Circle}}

\morescalingdelimiters

\begin{frame}{Zum Aufwärmen: Sprache gesucht!} 
	Haben Alphabet $A = \{ \mword \triangle, \mword \square, \mword \mycircle \}$.\\
\end{frame}

%\begin{frame}
%	\frametitle{Zum Aufwärmen: Sprache gesucht!}
%	
%	Ihr erhaltet eine Karte mit einer Beschreibung einer formalen Sprache über dem Alphabet $\Sigma = \{ \triangle, \square, \circ \}$.\\
%	Ziel ist es, jeweils alle Beschreibungen von gleichen Sprachen zu sammeln.\\
%	Dazu überlegt ihr euch zunächst in 4er-Gruppen, ob ihr Beschreibungen von gleichen Sprachen habt oder was andere Beschreibungen wären. \\
%	Dann tauschen sich jeweils 3 4er-Gruppen aus und sammeln die Beschreibungen.
%\end{frame}

\begin{frame}{Zum Aufwärmen: Sprache gesucht!}
	
	Die Sprache der Wörter, die mit einem Kreis beginnen und danach keinen Kreis mehr enthalten.
	\bigskip
	\pause
	
	$$ \{\mword \mycircle\} \cdot \{\mword \triangle, \mword \square\}^* $$
	\bigskip
	\pause
	
	$$ \set{w \in A^* \mid w = \mword \Circle \cdot v, v \in \{\mword \triangle, \mword \square\}^* } $$

\end{frame}

\begin{frame}{Zum Aufwärmen: Sprache gesucht!}
	
	Die Sprache der Wörter, deren vorletztes Zeichen ein Dreieck ist.
	\bigskip
	\pause
	
	$$ \{\mword \triangle, \mword \square, \mword \mycircle\}^* \cdot \set{\mword \triangle} \cdot \{\mword \triangle, \mword \square, \mword \mycircle\} $$
	\bigskip
	\pause

	$$ \{w \in A^* \mid w = v \cdot \mword \triangle \cdot z, v \in A^*, z \in A \} $$

	
\end{frame}

\begin{frame}{Zum Aufwärmen: Sprache gesucht!}
	
	$$ \left(\{\mword \square, \mword \mycircle\}^* \cdot \{\mword \triangle\mword \triangle\}^* \right)^* $$
	\bigskip
	\pause

	Die Sprache der Wörter, in denen nirgends eine ungerade Anzahl an Dreiecken nebeneinander steht.
\end{frame}

\begin{frame}{Zum Aufwärmen: Sprache gesucht!}
	
	$$ \{\mword \triangle, \mword \mycircle\}^* \cdot \left( \{\mword \square\} \cdot \{\mword \triangle, \mword \mycircle\}^* \cdot \{\mword \square\} \cdot \{\mword \triangle, \mword \mycircle\}^* \right)^* $$
	\bigskip
	\pause
	
	Die Sprache der Wörter, in denen eine gerade Anzahl an Vierecken vorkommt.

\end{frame}

\begin{frame}{Zum Aufwärmen: Sprache gesucht!}
	
	$$ \{\mword \square, \mword \triangle\}^* \cdot \left( \{\mword \mycircle\} \cdot \{\mword \square, \mword \triangle\}^+ \right)^* \cdot \{\mword \mycircle, \varepsilon\} $$
	\bigskip
	\pause

	Die Sprache der Wörter, in denen nirgends zwei Kreise aufeinander folgen.
\end{frame}

\section{Formale Sprachen}

\begin{frame}{Beispiel}
	Sei $A = \set{\word a,\word b}$ ein Alphabet. Mit $L$ wollen wir alle Wörter beschreiben, die genau ein $\word b$ enthalten. \\ \pause
	$$ L = \set{w_1 \word b w_2 \mid w_1, w_2 \in \{\word a\}^\ast } = \{\word a\}^\ast \cdot \{\word b \} \cdot \{\word a\}^\ast$$
	
	Was ist $L^3$? Was enthält $L^i$? \pause
	Zum Beispiel ist $$\word{aaababaaaabaa} = \word{aaaba} \cdot \word{baa} \cdot \word{aabaa} \in L_3$$ \pause
	$L^i$ enthält alle Wörter, die genau $i$-mal ein \word b enthalten! \\[1em]
	
	Was enthält $$L^i \setminus \{\word b\}^\ast ?$$ \pause
	Alle Wörter, die aus $i$ \word bs bestehen, aber auch noch mindestens ein \word a enthalten. \\
\end{frame}

\begin{frame}{Aufgabe}
	Welche Eigenschaft muss eine formale Sprache $L$ über einem Alphabet
	$A$ erfüllen, damit gilt: $$ L^0 \subseteq L^1 \subseteq L^2 \subseteq L^3 \subseteq ... $$
	
	\pause
	\begin{block}{Lösung}
		Das gilt, wenn $$ \varepsilon \in L $$
	\end{block}
	
\end{frame}

\subsubsection{A1}
\begin{frame}{Aufgabe (Klausur) \stars{4}}
		\begin{itemize}
			\item[(1)] Widerlegt: Für alle formalen Sprachen $L_1 , L_2$ gilt: 
			$$L_1^\ast \cup L_2^\ast = (L_1 \cup L_2 )^\ast$$
			
			\item[(2)] Zeigt: Für alle formalen Sprachen $L$ gilt: 
				$$L^\ast \cdot L = L^+ $$ 
	\end{itemize}

	Tipp zu (2) (nicht in der Klausur gegeben): Hier handelt es sich um eine Mengengleichheit, also brav mit \enquote{$\subseteq$} und \enquote{$\supseteq$} beweisen! \smiley 
\end{frame}

\begin{frame}[t]{Lösung}
	\textit{Für alle formalen Sprachen $L_1 , L_2$ gilt: 
		$$L_1^\ast \cup L_2^\ast = (L_1 \cup L_2 )^\ast$$ } \\[2em] \pause
	Diese Aussage ist falsch: Sei $L_1 = \{\word a\}$ und $L_2 = \{\word b\}$. Dann liegt \word {ab} in $(L_1 \cup L_2 )^\ast = \{\word a, \word b\}^\ast$ aber nicht in $L_1^\ast \cup L_2^\ast = \{\word a\}^\ast \cup \{\word b\}^\ast$.
\end{frame}

\begin{frame}[t]{Lösung}
	\textit{Für alle formalen Sprachen $L$ gilt: 
		$$L^\ast \cdot L = L^+ $$ } \\[1em] \pause
	Diese Aussage ist wahr! 
	\begin{block}{1. Schritt: $L^\ast \cdot L \subseteq L^+$:} \pause
	Wenn $w \in L^\ast \cdot L$ liegt, dann lässt es sich in Teilwörter auftrennen $$ w = w_1 \cdot w_2$$ mit $w_1 \in L^\ast$ und $w_2 \in L$. Für $w_1$ existiert ein $i \in \N_0$ mit $w_1 \in L^i$. Also $$w = w_1 w_2 \in L^i \cdot L = L^{i+1} \subset L^+.$$
	\end{block}
\end{frame}

\begin{frame}[t]{Lösung}
	\textit{Für alle formalen Sprachen $L$ gilt: 
		$$L^\ast \cdot L = L^+ $$ } \\[1em] 
	Diese Aussage ist wahr! 
	\begin{block}{2. Schritt: $L^\ast \cdot L \supseteq L^+$:} \pause
	Wähle nun $w \in L^+$. Dann existiert ein $i \in \N_+$ mit $w \in L^i$. Da $i > 0$ lässt es sich schreiben als $i = j + 1$ für ein $j \in \N_0$. Also ist $$w \in L^{j+1} = L^j \cdot L \subset L^\ast \cdot L. \qed$$
	\end{block}
\end{frame}

\subsubsection{A2}
\begin{frame}{Aufgabe (WS 2008) \stars{3}}
	Es sei $A = \{\word a, \word b\}$. Die Sprache $L \subset A^\ast$ sei definiert durch $$L = \left(\{\word a\}^\ast \cdot \{\word b\} \cdot \{\word a\}^\ast \right)^\ast$$
	Zeigt, dass jedes Wort $w$ aus $\{\word a, \word b\}^\ast$, das mindestens einmal das Zeichen
	\word b enthält, in $L$ liegt. (Hinweis: Macht eine Induktion über die Anzahl der
	Vorkommen des Zeichens \word b in $w$.)
\end{frame}

\begin{frame}{Lösung}
	$$L = \left(\{\word a\}^\ast \cdot \{\word b\} \cdot \{\word a\}^\ast \right)^\ast$$
	Sei $k$ die Anzahl der Vorkommen von \word b in einem Wort $w \in \{\word a, \word b\}^\ast$.
	\begin{block}{Induktionsanfang}  \pause
		Für $k = 1$: In diesem Fall lässt sich das Wort $w$ aufteilen in $$w = w_1 \cdot \word b \cdot w_2$$ wobei $w_1$ und $w_2$ keine \word b enthalten und somit in $\{\word a\}^\ast$ liegen. Damit gilt $w \in \{\word a\}^\ast \cdot \{\word b\} \cdot \{\word a\}^\ast$ und somit auch $$w \in \left(\{\word a\}^\ast \cdot \{\word b\} \cdot \{\word a\}^\ast \right)^\ast = L$$
	\end{block}
\end{frame}

\begin{frame}{Lösung}
	\begin{block}{Induktionsvoraussetzung}  \pause
		Für ein festes $k \in \N$ gilt, dass alle Wörter über $\{\word a, \word b\}^\ast$, die genau $k$-mal das Zeichen \word b enthalten, in $L$ liegen.
	\end{block} \pause
	\begin{block}{Induktionsschritt}  \pause
		Wir betrachten ein Wort $w$, das genau $k + 1$ mal das Zeichen \word b enthält. Dann kann man $w$ zerlegen in $w = w_1 \cdot w_2$, wobei $w_1$ genau einmal das Zeichen \word b enthält und $w_2$ genau $k$-mal das Zeichen \word b. \pause Nach Induktionsanfang liegt $w_1$ in $\{\word a\}^\ast \{\word b\}\{\word a\}^\ast$. Nach Induktionsvoraussetzung liegt $w_2$ in $(\{\word a\}^\ast \{\word b\}\{\word a\}^\ast )^\ast$, was bedeutet, dass $w = w_1 \cdot w_2$ in $$\left(\{\word a\}^\ast \{\word b\}\{\word a\}^\ast \right) \· \left(\{\word a\}^\ast \{\word b\}\{\word a\}^\ast \right)^\ast \subseteq \left(\{\word a\}^\ast \{\word b\}\{\word a\}^\ast \right)^\ast = L$$ liegt und die Behauptung ist gezeigt. $\qed$
	\end{block}
\end{frame}

%\subsubsection{A3}
%\begin{frame}
%	\frametitle{Noch mehr Aufgaben}
%	Begründen oder widerlegen Sie:
%	\begin{itemize}
%		\item Für alle formalen Sprachen $L$ gilt: 
%		$$(L_1^\ast \cdot L_2^\ast)^\ast = (L_1 \cdot L_2)^\ast$$ 
%		
%		\item Für alle formalen Sprachen $L_1 , L_2$ gilt: 
%		$$(L_1^\ast \cup L_2^\ast)^\ast = (L_1 \cup L_2 )^\ast$$
%	\end{itemize}
%\end{frame}
%
%\begin{frame}
%	\frametitle{Lösung}
%	\textit{Für alle formalen Sprachen $L_1 , L_2$ gilt: 
%		$$(L_1^\ast \cdot L_2^\ast )^\ast = (L_1 \cdot L_2 )^\ast$$ } \\[2em] \pause
%	Diese Aussage ist falsch: Sei $L_1 = \{a\}$ und $L_2 = \{b\}$. Dann liegt $$\mathbf{aa} = \mathbf{aa} \cdot \varepsilon$$ in $(L_1^\ast \cdot L_2^\ast ) = (L_1^\ast \cdot L_2^\ast )^1 \subset (L_1^\ast \cdot L_2^\ast )^\ast$, aber nicht in $(L_1 \cdot L_2 )^\ast = \{ab\}^\ast$.
%	
%\end{frame}
%
%\begin{frame}
%	\frametitle{Lösung}
%	\textit{Für alle formalen Sprachen $L_1 , L_2$ gilt: 
%		$$(L_1^\ast \cup L_2^\ast )^\ast = (L_1 \cup L_2 )^\ast$$ } \\[2em] \pause
%	Die Aussage ist korrekt: Sei $w$ ein Wort aus $(L_1^\ast \cup L_2^\ast )^\ast$. Dieses lässt sich in Teilwörter $w_1 , \cdots , w_k$ unterteilen, so dass für $1 \leq i \leq k$ gilt: 
%	$$w_i \in (L_1^\ast \cup L_2^\ast ) \implies w_i \in L_1^\ast \text{ oder } w_i \in L_2^\ast$$
%	Diese Teilwörter $w_i$ lassen sich wieder in Teilwörter $w_{i_1}, \dots w_{i_s}$ zerlegen, die entweder aus $L_1$ kommen, wenn $w_i \in L_1^\ast$ liegt, oder in $L_2$ liegen, wenn $w_i \in L_2^\ast$ liegt. Damit lässt sich $w$ in Teilwörter $w_{i_j}$ aus $L_1 \cup L_2$ unterteilen und es folgt $w \in (L_1 \cup L_2 )^\ast$. 
%\end{frame}
%
%\begin{frame}
%	\frametitle{Lösung}
%	\textit{Für alle formalen Sprachen $L_1 , L_2$ gilt: 
%		$$(L_1^\ast \cup L_2^\ast )^\ast = (L_1 \cup L_2 )^\ast$$ } \\[2em]
%	Sei umgekehrt ein Wort $w$ aus $(L_1 \cup L_2 )^\ast$. Dieses lässt sich dann in Teilwörter $w_1, \dots, w_k$ unterteilen, so dass für $1 \leq i \leq k$ gilt 
%	$$w_i \in L_1 \cup L_2 \implies w_i \in L_1 \subset L_1^\ast \text{ oder } w_i \in L_2 \subset L_2^\ast$$
%	Somit lässt sich $w$ in Teilwörter aus $L_1^\ast \cup L_2^\ast$ unterteilen, und es folgt $w \in (L_1^\ast \cup L_2^\ast )^\ast$.
%\end{frame}

\begin{frame}{Ausblick: Klammerausdrücke}
	
	Was ist mit der Sprache aller gültigen Klammerausdrücke? Können wir die auch mit $\set{}$, $\*$, ${}^*$ und ${}^+$ angeben? \only<beamer:0>{\\ \emph{Spoiler: Nein, das geht nicht!}}\\[1em]
	\pause
	
	\begin{block}{}
		\Large
		\centering
		COMING SOON... \\[1em]
	\end{block}

	\begin{figure}[H]
		\centering
		\includegraphics[scale=0.7]{xkcd/(.png}
		\vspace{-7pt}
		\caption{ \texttt{\url{https://xkcd.com/859/}} }
	\end{figure}
\end{frame}

%\section{Von der Darstellung zur Zahl}

\subsection{Definitionen}
\begin{frame}{Numerischer Wert}
	\begin{Definition}
		Zu einer Zahlenbasis $b$ definiere 
		$$ \text{Num}_b(\varepsilon) = 0  $$  
		$$ \text{Num}_b(wx) = b\cdot \text{Num}_b(w) + \text{num}_b(x) \text{ für alle } w\in Z_b^\ast, x\in Z_b $$ 
	\end{Definition}

	\pause
	Beachte: $Num_b : Z_b^* \to \Z$ ist Abbildung, die einem Wort (Zahlendarstellung) eine Zahl (Wert) zuordnet. Wir müssen diesen Wert aber natürlich wieder in eine  Darstellung umwandeln, um ihn aufschreiben zu können.
\end{frame}
\begin{frame}{Aufgabe}
	Berechnet die Zahlenwerte von $ 11_2, 321_4, B2_{16}$.
	\begin{align*} 
	\visible<1->{\text{Num}_2(11) &= 2\cdot \text{Num}_2(1) + \text{num}_2(1) \\
	&= 2\cdot 1 + 1 \\
	&= 3  \\}
	\visible<3->{\text{Num}_4(321) &=} \visible<4->{ 4\cdot \text{Num}_4(32) + \text{num}_4(1) \\
	&= 4\cdot \left( 4\cdot \text{Num}_4(3) + \text{num}_4(2) \right) + \text{num}_4(1) \\
	&= 4^2\cdot \text{num}_4(3) + 4 \cdot \text{num}_4(2) + \text{num}_4(1) \\
	&= 57 \\}
	\visible<5->{\text{Num}_{16}(B2) &=} \visible<6->{ 16 \cdot \text{Num}_{16}(B) + \text{num}_{16}(2) \\
	&= 16\cdot 11 + 2 \\
	&= 178}
	\end{align*}

\end{frame}

\begin{frame}{Wohldefiniertheit}
	\emph{Behauptung}: Die Definition 
		$$ \text{Num}_b(\varepsilon) = 0  $$  
		$$ \text{Num}_b(wx) = b\cdot \text{Num}_b(w) + \text{num}_b(x) \text{ für alle } w\in Z_b^\ast, x\in Z_b $$ 
		ist wohldefiniert und weist jedem Wort eine eindeutige Bedeutung zu, die dem Zahlenwert entspricht.
\end{frame}
\begin{frame}{Beweis}
	\begin{block}{Beweis durch vollständige Induktion über $n=\vert w \vert $}
	\begin{itemize}
		\only<1-2>{\item<1->[\emph{IA}] $n = 0 = \vert w \vert \implies w = \varepsilon $. \\
		Für $w = \varepsilon $ ist $\text{Num}_b$ wohldefiniert und sinnvoll (nämlich $Num_b(\varepsilon) = 0$).
		\item<2->[\emph{IV}] Für ein beliebig aber festes $n\in\N_0$ sei $Num_b(w) für alle w mit \setsize{w} = n wohldefiniert und entspreche dem Zahlenwert.$ }
		\only<3->{\item<3->[\emph{IS}] Wähle $w'$ mit $\vert w' \vert = n+1 $, dann gibt es ein $w\in Z_b^n, x\in Z_b$, so dass $ w' = wx $ \\
		Mit der Definition gilt nun $$ \text{Num}_b(w') = b\cdot \underbrace{\text{Num}_b(w)}_{IV} + \text{num}_b(x) $$
		Die Summe ist laut $IV$ wohldefiniert. Auch ist laut $IV$ $\text{Num}_b(w)$ der Zahlenwert von $w$ und damit auch $\text{Num}_b(w')$.}
	\end{itemize}
	\end{block}

\end{frame}

%\subsection{Aufgabe}
%\begin{frame}{Aufgabe. WS 2010 }
%Es bezeichne $\Z$ die Menge der ganzen Zahlen. Gegeben sei eine Ziffernmenge $Z_{-2} = \{N, E\}$ mit der Festlegung $num_2 (N) = 0$ und $num_2 (E) = 1$. Wir definieren eine Abbildung $Num_{-2} : Z_{-2}^\ast \to \Z$ wie folgt:
%	$$Num_{-2} (\varepsilon) = 0$$
%	$$\forall \ w \in Z_{-2}^\ast \ \forall \ x \in Z_{-2} : Num_{-2} (wx) = -2 \cdot Num_{-2} (w) + num_2 ( x )$$
%
%	\begin{itemize}	
%		\item Geben Sie für $w \in \{E, EN, EE, ENE, EEN, EEE\}$ jeweils $Num_{-2} (w)$ an.
%		\item Für welche Zahlen $x \in \Z$ gibt es ein $w \in Z_{-2}^\ast$ mit $Num_{-2} (w) = x$?
%	\end{itemize}
%\end{frame}
%
%\begin{frame}{Lösung}
%\textit{Geben Sie für $w \in \{E, EN, EE, ENE, EEN, EEE\}$ jeweils $Num_{-2} (w)$ an.} \pause
%	\begin{table}[h!]	
%		\begin{tabular}{>{$}l<{$}>{$}l<{$}}
%			Num_{-2} (E)\pause & = 1 \\ \pause 
%			Num_{-2} (EN)\pause & = -2 \\ \pause
%			Num_{-2} (EE)\pause & = -1 \\ \pause
%			Num_{-2} (ENE)\pause & = 5 \\ \pause
%			Num_{-2} (EEN)\pause & = 2 \\ \pause
%			Num_{-2} (EEE)\pause & = 3
%	\end{tabular}
%	\end{table}
%	\pause
%	\textit{Für welche Zahlen $x \in \Z$ gibt es ein $w \in Z_{-2}^\ast$ mit $Num_{-2} (w) = x$?} \\[1em]\pause
%	Für alle!
%\end{frame}

\section{Von der Zahl zur Darstellung}
\begin{frame}{Division und Modulo}
	\begin{block}{Definition}
		$ x \div y$ ist die ganzzahlige Division von x durch y.\\
		$ x \mod y$ liefert den Rest dieser Division
	\end{block} 
	\pause
	
	\begin{block}{Beobachtung}
		$ x\div y \in \N_0 \qquad x\mod y \in \{0,\dots, y-1\} $
	\end{block}
	\pause
	
	\begin{block}{Lemma}
		$$ x = y \cdot (x \div y ) + \left( x \mod y \right)$$ 
	\end{block}
	
\end{frame}

\begin{frame}
	\begin{block}{Beispiel}
		\begin{tabular}{ccc}
			& $x\div y$ & $x\mod y$ \\
			$x=2,y=3$ \pause &  0 & 2 \\  \pause
			$x=5 ,y=2$ \pause & 2 & 1 \\	\pause
			$x=8,y=2$ \pause & 4 & 0 \\	
		\end{tabular}
	\end{block}
\end{frame}

\begin{frame}{Beispiel}
	\begin{table}[h!]
		\centering
		\begin{tabular}{c|cccccccccccc}
			$x$ & 0 & 1 & 2 & 3 & 4 & 5 & 6 & 7 & 8 & 9 & 10 & 11 \\ \hline
			$x\div 4 $ & \only<2->{0 & 0 & 0 & 0 & 1 & 1 & 1 & 1 & 2 & 2 & 2 & 2 } \only<1|handout:0>{&&&&&&&&&&&} \\
			$4\left( x\div 4\right) $ & \only<3->{0&0&0&0&4&4&4&4&8&8&8&8} \only<1-2|handout:0>{&&&&&&&&&&}  \\
			$x\mod 4$ & \only<4>{0&1&2&3&0&1&2&3&0&1&2&3} \only<1-3|handout:0>{&} 
		\end{tabular}
	\end{table}
\end{frame}

\begin{frame}{Repräsentation}
	\begin{block}{Definition}
		$Repr_k(n)$ ist das kürzeste Wort $w\in Z_k^\ast$ mit $Num_k(w)=n$, also 
		$$ Num_k\left( Repr_k(n)\right) = n $$ 
	\end{block}
	\pause
	\emph{Anmerkung}:
	Im Allgemeinen $$ Repr_k\left(Num_k(w)\right) \neq w $$ da überflüssige Nullen wegfallen. 
\end{frame}

\begin{frame}{Repräsentation}
	Wir definieren
	\begin{align*}
		Repr_k : \; &\N_0 \to Z_k  \\
		n&\mapsto \begin{cases} repr_k(n) & n<k \\ Repr_k\left( n\div k \right) \cdot repr_k\left( n \mod k \right) & n\geq k 
		\end{cases} 
	\end{align*}
	
	\begin{block}{Aufgabe}
		Berechne folgende Darstellungen:\\
		$Repr_2(42) = \pause 101010$ \\
		$Repr_4(42) = \pause 222$ \\
		$Repr_8(42) = \pause 52$ \\
		$Repr_{16}(42) = \pause 2A$
	\end{block}
\end{frame}

\begin{frame}{Beispiel: Lösung}
	\begin{align*}
		Repr_8(42) &= Repr_8(42 \div 8) \cdot repr_8(42 \mod 8) \\
		&= Repr_8(5) \cdot repr_8(2)\\
		&= repr_8(5) \cdot 2\\
		&= 5 \cdot 2\\
		&= 52_8
	\end{align*}
	
\end{frame}


\begin{frame}	
	\begin{block}{Was ihr nun wissen solltet}
		\begin{itemize}
			\item Wie man wahre Aussagen konstruiert \qquad (\#NoFakeNews!) 
			\item Wie vollständige Induktion geht
			\item Wie man formale Sprachen angeben kann
			\item Wie man Beweise mit formalen Sprachen führt
		\end{itemize}
	\end{block}
	
	\begin{block}{Was nächstes Mal kommt}
		\begin{itemize}
			\item Wie man von Zahlendarstellungen zu Zahlen kommt...
			\item[] ... und wieder zurück
			\item Nicht immer so positiv: Negative Zahlen
			%\item Komprimierung: Huffmann-Codierungen
		\end{itemize}
	\end{block}
\end{frame}

\thassedaniel{
	\xkcdframevert{953}{Danke für eure Aufmerksamkeit! \smiley}{2.5}
}{
	\xkcdframe{1516}{Win by Induction}{2}
}
\slideThanks

\end{document}
