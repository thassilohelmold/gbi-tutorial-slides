%beamer

% Comment/uncomment this line to toggle handout mode
% \newcommand{\handout}{}

%% Beamer-Klasse im korrekten Modus
\ifdefined \handout
\documentclass[handout]{beamer} % Handout mode
\else
\documentclass{beamer}
\fi

%% UTF-8-Encoding
\usepackage[utf8]{inputenc}

% % \bigtimes abgeschrieben von http://tex.stackexchange.com/questions/14386/importing-a-single-symbol-from-a-different-font
% \DeclareFontFamily{U}{mathx}{\hyphenchar\font45}
% \DeclareFontShape{U}{mathx}{m}{n}{
%       <5> <6> <7> <8> <9> <10> gen * mathx
%       <10.95> mathx10 <12> <14.4> <17.28> <20.74> <24.88> mathx12
%       }{}
% \DeclareSymbolFont{mathx}{U}{mathx}{m}{n}
% \DeclareMathSymbol{\bigtimes}{\mathop}{mathx}{161}

\RequirePackage{xcolor}

\def\9{\square}
%\def\9{\blank}

% f"ur Aussagenlogik
\colorlet{alcolor}{blue}
\RequirePackage{tikz}
\usetikzlibrary{arrows.meta}
\newcommand{\alimpl}{\mathrel{\tikz[x={(0.1ex,0ex)},y={(0ex,0.1ex)},>={Classical TikZ Rightarrow[]}]{\draw[alcolor,->,line width=0.7pt,line cap=round] (0,0) -- (15,0);\path (0,-6);}}}
\newcommand{\aleqv}{\mathrel{\tikz[x={(0.1ex,0ex)},y={(0ex,0.1ex)},>={Classical TikZ Rightarrow[]}]{\draw[alcolor,<->,line width=0.7pt,line cap=round] (0,0) -- (18,0);\path (0,-6);}}}
\newcommand{\aland}{\mathbin{\raisebox{-0.6pt}{\rotatebox{90}{\texttt{\color{alcolor}\char62}}}}}
\newcommand{\alor}{\mathbin{\raisebox{-0.8pt}{\rotatebox{90}{\texttt{\color{alcolor}\char60}}}}}
%\newcommand{\ali}[1]{_{\mathtt{\color{alcolor}#1}}}
\newcommand{\alv}[1]{\mathtt{\color{alcolor}#1}}
\newcommand{\alnot}{\mathop{\tikz[x={(0.1ex,0ex)},y={(0ex,0.1ex)}]{\draw[alcolor,line width=0.7pt,line cap=round,line join=round] (0,0) -- (10,0) -- (10,-4);\path (0,-8) ;}}}
\newcommand{\alP}{\alv{P}} %ali{#1}}
%\newcommand{\alka}{\negthinspace\hbox{\texttt{\color{alcolor}(}}}
\newcommand{\alka}{\negthinspace\text{\texttt{\color{alcolor}(}}}
%\newcommand{\alkz}{\texttt{\color{alcolor})}}\negthinspace}
\newcommand{\alkz}{\text{\texttt{\color{alcolor})}}\negthinspace}
\newcommand{\AAL}{A_{AL}}
\newcommand{\LAL}{\hbox{\textit{For}}_{AL}}
\newcommand{\AxAL}{\hbox{\textit{Ax}}_{AL}}
\newcommand{\AxEq}{\hbox{\textit{Ax}}_{Eq}}
\newcommand{\AxPL}{\hbox{\textit{Ax}}_{PL}}
\newcommand{\AALV}{\hbox{\textit{Var}}_{AL}}
\newcommand{\MP}{\hbox{\textit{MP}}}
\newcommand{\GEN}{\hbox{\textit{GEN}}}
\newcommand{\W}{\ensuremath{\hbox{\textbf{w}}}\xspace}
\newcommand{\F}{\ensuremath{\hbox{\textbf{f}}}\xspace}
\newcommand{\WF}{\ensuremath{\{\W,\F\}}\xspace}
\newcommand{\val}{\hbox{\textit{val}}}
\newcommand{\valDIb}{\val_{D,I,\beta}}

\newcommand*{\from}{\colon}

% die nachfolgenden Sachen angepasst an cmtt
\newlength{\ttquantwd}
\setlength{\ttquantwd}{1ex}
\newlength{\ttquantht}
\setlength{\ttquantht}{6.75pt}
\def\plall{%
  \tikz[line width=0.67pt,line cap=round,line join=round,baseline=(B),alcolor] {
    \draw (-0.5\ttquantwd,\ttquantht) -- node[coordinate,pos=0.4] (lll){} (-0.25pt,-0.0pt) -- (0.25pt,-0.0pt) -- node[coordinate,pos=0.6] (rrr){} (0.5\ttquantwd,\ttquantht);
    \draw (lll) -- (rrr);
    \coordinate (B) at (0,-0.35pt);
  }%
}
\def\plexist{%
  \tikz[line width=0.67pt,line cap=round,line join=round,baseline=(B),alcolor] {
    \draw (-0.9\ttquantwd,\ttquantht) -- (0,\ttquantht) -- node[coordinate,pos=0.5] (mmm){} (0,0) --  (-0.9\ttquantwd,0);
    \draw (mmm) -- ++(-0.75\ttquantwd,0);
    \coordinate (B) at (0,-0.35pt);
  }\ensuremath{\,}%
}
\let\plexists=\plexist
\newcommand{\NT}[1]{\ensuremath{\langle\mathrm{#1} \rangle}}

\newcommand{\CPL}{\text{\itshape Const}_{PL}}
\newcommand{\FPL}{\text{\itshape Fun}_{PL}}
\newcommand{\RPL}{\text{\itshape Rel}_{PL}}
\newcommand{\VPL}{\text{\itshape Var}_{PL}}
\newcommand{\ATer}{A_{\text{\itshape Ter}}}
\newcommand{\ARel}{A_{\text{\itshape Rel}}}
\newcommand{\AFor}{A_{\text{\itshape For}}}
\newcommand{\LTer}{L_{\text{\itshape Ter}}}
\newcommand{\LRel}{L_{\text{\itshape Rel}}}
\newcommand{\LFor}{L_{\text{\itshape For}}}
\newcommand{\NTer}{N_{\text{\itshape Ter}}}
\newcommand{\NRel}{N_{\text{\itshape Rel}}}
\newcommand{\NFor}{N_{\text{\itshape For}}}
\newcommand{\PTer}{P_{\text{\itshape Ter}}}
\newcommand{\PRel}{P_{\text{\itshape Rel}}}
\newcommand{\PFor}{P_{\text{\itshape For}}}

\newcommand{\plka}{\alka}
\newcommand{\plkz}{\alkz}
%\newcommand{\plka}{\plfoo{(}}
%\newcommand{\plkz}{\plfoo{)}}
\newcommand{\plcomma}{\hbox{\texttt{\color{alcolor},}}}
\newcommand{\pleq}{{\color{alcolor}\,\dot=\,}}

% MODIFIED (DJ)
% previously: \newcommand{\plfoo}[1]{\mathtt{\color{alcolor}#1}}
\newcommand{\plfoo}[1]{\texttt{\color{alcolor}#1}}

\newcommand{\plc}{\plfoo{c}}
\newcommand{\pld}{\plfoo{d}}
\newcommand{\plf}{\plfoo{f}}
\newcommand{\plg}{\plfoo{g}}
\newcommand{\plh}{\plfoo{h}}
\newcommand{\plx}{\plfoo{x}}
\newcommand{\ply}{\plfoo{y}}
\newcommand{\plz}{\plfoo{z}}
\newcommand{\plR}{\plfoo{R}}
\newcommand{\plS}{\plfoo{S}}

\newcommand{\bv}{\mathrm{bv}}
\newcommand{\fv}{\mathrm{fv}}

%\newcommand{\AxAL}{\hbox{\textit{Ax}}_{AL}}
%\newcommand{\AALV}{\hbox{\textit{Var}}_{AL}}

%\renewcommand{\#}[1]{\literal{#1}}
\newcommand{\A}{\mathcal{A}}
\newcommand{\Adr}{\text{Adr}}
\newcommand{\ar}{\mathrm{ar}}
\newcommand{\ascii}[1]{\literal{\char#1}}
%\newcommand{\assert}[1]{\text{/\!\!/\ } #1}
\newcommand{\assert}[1]{\colorbox{black!7!white}{\ensuremath{\{\;#1\;\}}}}
\newcommand{\Assert}[1]{$\langle$\textit{#1}$\rangle$}
\newcommand{\B}{\mathcal{B}}
\newcommand{\bfmod}{\mathbin{\kw{ mod }}}
\newcommand{\bb}{{\text{bb}}}
\def\bottom{\hbox{\small$\pmb{\bot}$}}
\newcommand{\card}[1]{|#1|}
%\newcommand{\cod}{\mathop{\text{cod}}}  % ist in thwmathabbrevs
\newcommand{\Conf}{\mathcal{C}}
\newcommand{\define}[1]{\emph{#1}}
%\renewcommand{\dh}{d.\,h.\@\xspace}
%\newcommand{\Dh}{D.\,h.\@\xspace}
%\newcommand{\engl}[1]{engl.\xspace\emph{#1}}
\newcommand{\eps}{\varepsilon}
%\newcommand{\evtl}{evtl.\@\xspace}
\newcommand{\fbin}{\text{bin}}
\newcommand{\finv}{\text{inv}}
\newcommand{\fnum}{\text{num}}
\newcommand{\fNum}{{\text{Num}}}
\newcommand{\frepr}{\text{repr}}
\newcommand{\fRepr}{\text{Repr}}
\newcommand{\fZkpl}{\text{Zkpl}}
\newcommand{\fLen}{\text{Len}}
\newcommand{\fsem}{\text{sem}}
\providecommand{\fspace}{\mathord{\text{space}}}
\providecommand{\fSpace}{\mathord{\text{Space}}}
\providecommand{\ftime}{\mathord{\text{time}}}
\providecommand{\fTime}{\mathord{\text{Time}}}
\newcommand{\fTrans}{\text{Trans}}
\newcommand{\fVal}{\text{Val}}

% MODIFIED (DJ)
\newcommand{\Val}{\text{Val}}

%\def\G{\mathbb{Z}}
\newcommand{\HT}[1]{\normalfont\textsc{HT-#1}}
\newcommand{\htr}[3]{\{#1\}\;#2\; \{#3\}}
\newcommand{\Id}{\text{I}}
%\newcommand{\ie}{i.\,e.\@\xspace}
\newcommand{\instr}[2]{\texttt{#1}\ \textit{#2}}
\newcommand{\Instr}[2]{\texttt{#1}\ \textrm{#2}}
\newcommand{\instrr}[3]{\texttt{#1}\ \textit{#2}\texttt{(#3)}}
\newcommand{\Instrr}[3]{\texttt{#1}\ \textrm{#2}\texttt{(#3)}}
\newcommand{\io}{\!\mid\!}
\usepackage{KITcolors}
\newcommand{\literal}[1]{\hbox{\textcolor{blue!95!white}{\textup{\texttt{\scalebox{1.11}{#1}}}}}}
%\newcommand{\literal}[1]{\hbox{\textcolor{KITblue!80!black}{\textup{\texttt{#1}}}}}
\def\kasten#1{\leavevmode\literal{\setlength{\fboxsep}{1pt}\fbox{\vrule  width 0pt height 1.5ex depth 0.5ex #1}}}
\newcommand{\kw}[1]{\ensuremath{\mathbf{#1}}}
\newcommand{\lang}[1]{\ensuremath{\langle#1\rangle}}
%\newcommand{\maw}{m.\,a.\,w.\@\xspace}
%\newcommand{\MaW}{M.\,a.\,w.\@\xspace}
\newcommand{\mdefine}[2][FOOBAR]{\define{#2}\def\foobar{FOOBAR}\def\optarg{#1}\ifx\foobar\optarg\def\optarg{#2}\fi\graffito{\optarg}}
\newcommand{\meins}{\rotatebox[origin=c]{180}{1}}
\newcommand{\Mem}{\text{Mem}}
\newcommand{\memread}{\text{memread}}
\newcommand{\memwrite}{\text{memwrite}}
\providecommand{\meta}[1]{\ensuremath{\langle}\textit{#1}\ensuremath{\rangle}}
%\newcommand{\N}{\mathbb{N}}
\newcommand{\NP}{\mathbf{NP}}
\newcommand{\Nadd}{N_{\text{add}}}
\newcommand{\Nmult}{N_{\text{mult}}}
% MODIFIED (DJ): added \!, mathcal{O}
\newcommand{\Oh}[1]{\mathcal{O}\!\left(#1\right)}
\newcommand{\Om}[1]{\Omega\!\left(#1\right)}
\newcommand{\personname}[1]{\textsc{#1}}
\newcommand{\regname}[1]{\texttt{#1}}
\newcommand{\mima}{\textsc{Mima}\xspace}
\newcommand{\mimax}{\textsc{Mima-X}\xspace}

\def\Pclass{\text{\bfseries P}}
\def\PSPACE{\text{\bfseries PSPACE}}

\newcommand{\SPush}{\text{push}}
\newcommand{\SPop}{\text{pop}}
\newcommand{\SPeek}{\text{peek}}
\newcommand{\STop}{\text{top}}
\newcommand{\STos}{\text{\itshape tos}}
\newcommand{\SBos}{\text{\itshape bos}}

%\newcommand{\R}{\mathbb{R}}
\newcommand{\Rnullplus}{\R_0^{+}}
\newcommand{\Rplus}{\R_{+}}
\newcommand{\resp}{resp.\@\xspace}
\newcommand{\Sem}{\text{Sem}}
\newcommand{\sgn}{\mathop{\text{sgn}}}
\newcommand{\sqbox}{\mathop{\raisebox{-6.2pt}{\hbox{\hbox to 0pt{$^{^{\sqcap}}$\hss}$^{^{\sqcup}}$}}}}
\newcommand{\sqleq}{\sqsubseteq}
\newcommand{\sqgeq}{\sqsupseteq}
% MODIFIED (DJ): added \!
\newcommand{\Th}[1]{\Theta\!\left(#1\right)}
%\newcommand{\usw}{usw.\@\xspace}
\newcommand{\V}[1]{\hbox{\textit{#1}}}
\newcommand{\x}{\times}
\newcommand{\ZK}{\mathbb{K}}
%\newcommand{\Z}{\mathbb{Z}}
\newcommand{\zB}{z.\,B.\@\xspace}
\newcommand{\ZB}{Z.\,B.\@\xspace}
% \newcommand{\bb}{{\text{bb}}}
% \def\##1{\hbox{\textcolor{darkblue}{\texttt{#1}}}}
% \def\A{\mathcal{A}}
% \newcommand{\0}{\#0}
% \newcommand{\1}{\#1}
% \newcommand{\Obj}{\text{Obj}}
% \newcommand{\start}{\mathop{\text{start}}}
% \newcommand{\compactlist}{\addtolength{\itemsep}{-\parskip}}
% \newcommand{\fval}{\text{val}}
% \newcommand{\lang}[1]{\ensuremath{\langle#1\rangle}}
% \newcommand{\io}{\!\mid\!}
% \def\sqbox{\mathop{\raisebox{-6.2pt}{\hbox{\hbox to 0pt{$^{^{\sqcap}}$\hss}$^{^{\sqcup}}$}}}}
% \def\sqleq{\sqsubseteq}
% \def\sqgeq{\sqsupseteq}
\def\Td{T_{\overline{d}}}
% \newcommand{\csym}[1]{\ensuremath{\#{c}_{\#{\hbox{\scriptsize #1}}}}}
% \newcommand{\F}{\ensuremath{\mathcal{F}}}
% \newcommand{\fsym}[2]{\ensuremath{\#{f}^{\#{\hbox{\scriptsize #1}}}_{\#{\hbox{\scriptsize #2}}}}}
% \newcommand{\rsym}[2]{\ensuremath{\#{R}^{\#{\hbox{\scriptsize #1}}}_{\#{\hbox{\scriptsize #2}}}}}
% \newcommand{\xsym}[1]{\ensuremath{\#{x}_{\#{\hbox{\scriptsize #1}}}}}
% \newcommand{\I}{\mathcal{I}}
% ********************************************************************

\usepackage[blue]{../framework/thwregex}
\usepackage{environ}
\usepackage{bm}
\usepackage{calc}
\usepackage{varwidth}
\usepackage{wasysym}
\usepackage{mathtools}


% Das ist der KIT-Stil
%\usepackage{../TutTexbib/beamerthemekit}
\usepackage[deutsch,titlepage0]{../framework/KIT/beamerthemeKITmod}
\TitleImage[width=\titleimagewd]{../figures/titlepage.jpg}
%\usetheme[deutsch,titlepage0]{KIT}

% Include PDFs
\usepackage{pdfpages}

% Libertine font (Original GBI font)
\usepackage{libertine}
%\renewcommand*\familydefault{\sfdefault}  %% Only if the base font of the document is to be sans serif

% Nicer math symbols
\usepackage{eulervm}
%\usepackage{mathpazo}
\renewcommand\ttdefault{cmtt} % Computer Modern typewriter font, see lecture slides.

\usepackage{csquotes}

%%%%%%

%% Schönere Schriften
\usepackage[TS1,T1]{fontenc}

%% Bibliothek für Graphiken
\usepackage{graphicx}

%% der wird sowieso in jeder Datei gesetzt
\graphicspath{{../figures/}}

%% Anzeigetiefe für Inhaltsverzeichnis: 1 Stufe
\setcounter{tocdepth}{1}

%% Hyperlinks
\usepackage{hyperref}
% I don't know why, but this works and only includes sections and NOT subsections in the pdf-bookmarks.
\hypersetup{bookmarksdepth=subsection} 

%\usepackage{lmodern}
\usepackage{colortbl}
\usepackage[absolute,overlay]{textpos}
\usepackage{listings}
\usepackage{forloop}
%\usepackage{algorithmic} % PseudoCode package 

\usepackage{tikz}
\usetikzlibrary{matrix}
\usetikzlibrary{arrows.meta}
\usetikzlibrary{automata}
\usetikzlibrary{tikzmark}

% Needed for gbi-macros
\usepackage{xspace}

%%%%%%

%% Verbatim
\usepackage{moreverb}

%%%%%%%%%%%%%%%%%%%%%%%%%%%%%%%%%%%% Copy end

%% Tabellen
\usepackage{array}
\usepackage{multicol}
\usepackage{hhline}

%% Bibliotheken für viele mathematische Symbole
\usepackage{amsmath, amsfonts, amssymb}

%% Deutsche Silbentrennung und Beschriftungen
\usepackage[ngerman]{babel}

\usepackage{kbordermatrix}

% kbordermatrix settings
\renewcommand{\kbldelim}{(} % Left delimiter
\renewcommand{\kbrdelim}{)} % Right delimiter

\input{../config.tex}



% define custom \handout command flag if handout mode is toggled  #DirtyAsHellButWell...
\only<beamer:0>{\def\handout{}} %beamer:0 == handout mode

\newcommand{\R}{\mathbb{R}}
\newcommand{\N}{\mathbb{N}}
\newcommand{\Z}{\mathbb{Z}}
\newcommand{\Q}{\mathbb{Q}}
\newcommand{\BB}{\mathbb{B}}
\newcommand{\C}{\mathbb{C}}
\newcommand{\K}{\mathbb{K}}
\newcommand{\G}{\mathbb{G}}
\newcommand{\nullel}{\mathcal{O}}
\newcommand{\einsel}{\mathds{1}}
\newcommand{\Pot}{\mathcal{P}}
\renewcommand{\O}{\text{O}}

\def\word#1{\hbox{\textcolor{blue}{\texttt{#1}}}}
\let\literal\word
\def\mword#1{\hbox{\textcolor{blue}{$\mathtt{#1}$}}}  % math word
\def\sp{\scalebox{1}[.5]{\textvisiblespace}}
\def\wordsp{\word{\sp}}

%\newcommand{\literal}[1]{\textcolor{blue}{\texttt{#1}}}
\newcommand{\realTilde}{\textasciitilde \ }
\newcommand{\setsize}[1]{\ensuremath{\left\lvert #1 \right\rvert}}
\newcommand{\size}[1]{\setsize{#1}}  % Shame on you, TeXStudio...
\newcommand{\set}[1]{\left\{#1\right\}}
\newcommand{\tuple}[1]{\left(#1\right)}
\newcommand{\normalvar}[1]{\text{$#1$}}

% Modified by DJ
\let\oldemptyset\emptyset
\let\emptyset\varnothing % proper emptyset

\newcommand{\boder}{\ensuremath{\mathbin{\textcolor{blue}{\vee}}}\xspace}
\newcommand{\bund}{\ensuremath{\mathbin{\textcolor{blue}{\wedge}}}\xspace}
\newcommand{\bimp}{\ensuremath{\mathrel{\textcolor{blue}{\to}}}\xspace}
\newcommand{\bgdw}{\ensuremath{\mathrel{\textcolor{blue}{\leftrightarrow}}}\xspace}
\newcommand{\bnot}{\ensuremath{\textcolor{blue}{\neg}}\xspace}
\newcommand{\bone}{\ensuremath{\textcolor{blue}{1}}\text{}}
\newcommand{\bzero}{\ensuremath{\textcolor{blue}{0}}\text{}}
\newcommand{\bleftBr}{\ensuremath{\textcolor{blue}{\texttt{(}}}\text{}}
\newcommand{\brightBr}{\ensuremath{\textcolor{blue}{\texttt{)}}}\text{}}

% Fix of \b... commands:

\renewcommand{\boder}{\alor}
\renewcommand{\bund}{\aland}
\renewcommand{\bimp}{\alimpl}
\renewcommand{\bgdw}{\aleqv}
\renewcommand{\bnot}{\alnot}
\renewcommand{\bleftBr}{\alka}
\renewcommand{\brightBr}{\alkz}
\newcommand{\alA}{\word A}
\newcommand{\alB}{\word B}
\newcommand{\alC}{\word C}

\newcommand{\plB}{\plfoo{B}}
\newcommand{\plE}{\plfoo{E}}

\newcommand{\summe}[2]{\sum\limits_{#1}^{#2}}
\newcommand{\limes}[1]{\lim\limits_{#1}}

%\newcommand{\numpp}{\advance \value{weeknum} by -2 \theweeknum \advance \value{weeknum} by 2}
%\newcommand{\nump}{\advance \value{weeknum} by -1 \theweeknum \advance \value{weeknum} by 1}

\newcommand{\mycomment}[1]{}
\newcommand{\Comment}[1]{}

%% DISCLAIMER START 
% It is INSANELY IMPORTANT NOT TO DO THIS OUTSIDE BEAMER CLASS! IN ARTCILE DOCUMENTS, THIS IS VERY LIKELY TO BUG AROUND!
\makeatletter%
\@ifclassloaded{beamer}%
{
	% TODO 
	% no time...
	% redefine section to ignore multiple \section calls with the same title
}%
{
	\errmessage{ERROR: section command redefinition outside of beamer class document! Please contact the author of this code.}
}%
\makeatother%
%% DISCLAIMER END

\newcounter{abc}
\newenvironment{alist}{
  \begin{list}{(\alph{abc})}{
      \usecounter{abc}\setlength{\leftmargin}{8mm}\setlength{\labelsep}{2mm}
    }
}{\end{list}}


\newcommand{\stdarraystretch}{1.20}
\renewcommand{\arraystretch}{\stdarraystretch}  % for proper row spacing in tables

\newcommand{\morescalingdelimiters}{   % for proper \left( \right) typography
	\delimitershortfall=-1pt  
	\delimiterfactor=1
}

\newcommand{\centered}[1]{\vspace{-\baselineskip}\begin{center}#1\end{center}\vspace{-\baselineskip}}

% for \implitem and \item[bla] stuff to look right:
\setbeamercolor*{itemize item}{fg=black}
\setbeamercolor*{itemize subitem}{fg=black}
\setbeamercolor*{itemize subsubitem}{fg=black}

\setbeamercolor*{description item}{fg=black}
\setbeamercolor*{description subitem}{fg=black}
\setbeamercolor*{description subsubitem}{fg=black}

\renewcommand{\qedsymbol}{\textcolor{black}{\openbox}}

\renewcommand{\mod}{\mathop{\textbf{mod}}}
\renewcommand{\div}{\mathop{\textbf{div}}}

\newcommand{\ceil}[1]{\left\lceil#1\right\rceil}
\newcommand{\floor}[1]{\left\lfloor#1\right\rfloor}
\newcommand{\abs}[1]{\left\lvert #1 \right\rvert}
\newcommand{\Matrix}[1]{\begin{pmatrix} #1 \end{pmatrix}}
\newcommand{\braced}[1]{\left\lbrace #1 \right\rbrace}

% "something" placeholder. Useful for repairing spacing of operator sections, like `\sth = 42`.
\def\sth{\vphantom{.}}

\def\fract#1/#2 {\frac{#1}{#2}} % ! Trailing space is crucial!
\def\dfract#1/#2 {\dfrac{#1}{#2}} % ! Trailing space is crucial!

\newcommand{\Mid}{\;\middle|\;}

\let\after\circ



\def\·{\cdot}
\def\*{\cdot}
\def\?>{\ensuremath{\rightsquigarrow}}  % Fuck you, Latex
\def\~~>{\ensuremath{\rightsquigarrow}}  

\newcommand{\tight}[1]{{\renewcommand{\arraystretch}{0.76} #1}}
\newcommand{\stackedtight}[1]{\renewcommand{\arraystretch}{0.76} \begin{matrix} #1 \end{matrix} }
\newcommand{\stacked}[1]{\begin{matrix} #1 \end{matrix} }
\newcommand{\casesl}[1]{\delimitershortfall=0pt  \left\lbrace\hspace{-.3\baselineskip}\begin{array}{ll} #1 \end{array}\right.}
\newcommand{\casesr}[1]{\delimitershortfall=0pt  \left.\begin{array}{ll} #1 \end{array}\hspace{-.3\baselineskip}\right\rbrace}
\newcommand{\caseslr}[1]{\delimitershortfall=0pt  \left\lbrace\hspace{-.3\baselineskip}\begin{array}{ll} #1 \end{array}\hspace{-.3\baselineskip}\right\rbrace}

\def\q#1uad{\ifnum#1=0\relax\else\quad\q{\the\numexpr#1-1\relax}uad\fi}
% e.g. \q1uad = \quad, \q2uad = \qquad etc.

\newcommand{\qqquad}{\q3uad}
\newcommand{\minusquad}{\hspace{-1em}}

%% Placeholder utils
% \§{#1}   Saves #1 as placeholder and prints it
% \.       Prints an \hphantom with the size of the recalled placeholder.
\def\indentstring{}
\def\§#1{\def\indentstring{#1}#1}
\def\.{{$\hphantom{\text{\indentstring}}$}}
%% Placeholder utils end

\newcommand{\impl}{\ifmmode\ensuremath{\mskip\thinmuskip\Rightarrow\mskip\thinmuskip}\else$\Rightarrow$\fi\xspace}
\newcommand{\Impl}{\ifmmode\implies\else$\Longrightarrow$\fi\xspace}

\newcommand{\derives}{\Rightarrow}

\newcommand{\gdw}{\ifmmode\mskip\thickmuskip\Leftrightarrow\mskip\thickmuskip\else$\Leftrightarrow$\fi\xspace}
\newcommand{\Gdw}{\ifmmode\iff\else$\Longleftrightarrow$\fi\xspace}

% Legacy code from the algo tutorial slides. Perhaps useful. Try with care.
\mycomment{
	\newcommand{\impl}{\ifmmode\ensuremath{\mskip\thinmuskip\Rightarrow\mskip\thinmuskip}\else$\Rightarrow$\xspace\fi}  
	\newcommand{\Impl}{\ifmmode\implies\else$\Longrightarrow$\xspace\fi}
	
	\newcommand{\gdw}{\ifmmode\mskip\thickmuskip\Leftrightarrow\mskip\thickmuskip\else$\Leftrightarrow$\xspace\fi}
	\newcommand{\Gdw}{\ifmmode\iff\else$\Longleftrightarrow$\xspace\fi}
}
	
\newcommand{\gdwdef}{\ifmmode\mskip\thickmuskip:\Leftrightarrow\mskip\thickmuskip\else:$\Leftrightarrow$\xspace\fi}
\newcommand{\Gdwdef}{\ifmmode\mskip\thickmuskip:\Longleftrightarrow\mskip\thickmuskip\else:$\Longleftrightarrow$\xspace\fi}

\newcommand{\symbitemnegoffset}{\hspace{-.5\baselineskip}}
\newcommand{\implitem}{\item[\impl\symbitemnegoffset]}
\newcommand{\Implitem}{\item[\Impl\symbitemnegoffset]}


\newcommand{\forcenewline}{\mbox{}\\}

\newcommand{\bfalert}[1]{\textbf{\alert{#1}}}
\let\elem\in   % I'm a Haskell freak. Don't judge me. :P


\def\|#1|{\text{\normalfont #1}}  % | steht für senkrecht (anstatt kursiv wie sonst im math mode)


% proper math typography
\newcommand{\functionto}{\longrightarrow}
\renewcommand{\geq}{\geqslant}
\renewcommand{\leq}{\leqslant}
\let\oldsubset\subset
\renewcommand{\subset}{\subseteq} % for all idiots out there using subset

\newenvironment{threealign}{%
	\[
	\begin{array}{r@{\ }c@{\ }l}
}{%
	\end{array}	
	\]
}

\newcommand{\concludes}{ \\ \hline  }
\newcommand{\deduction}[1]{
	\begin{varwidth}{.8\linewidth}
		\begin{tabular}{>{$}c<{$}}
			#1
		\end{tabular}
	\end{varwidth}	
}

\definecolor{hoareorange}{rgb}{1,.85,.6}
\newcommand{\hoareassert}[1]{\setlength{\fboxsep}{1pt}\setlength{\fboxrule}{-1.4pt}\fcolorbox{white}{hoareorange}{\ensuremath{\{\;#1\;\}}}\setlength\fboxrule{\defaultfboxrule}\setlength{\fboxsep}{3pt}}

\newcommand{\mailto}[1]{\href{mailto:#1}{{\textcolor{blue}{\underline{#1}}}}}
\newcommand{\urlnamed}[2]{\href{#2}{\textcolor{blue}{\underline{#1}}}}
\renewcommand{\url}[1]{\urlnamed{#1}{#1}}

\newcommand{\hanging}{\hangindent=0.7cm}
\newcommand{\indented}{\hanging}


% \hstretchto prints #2 left-aligned into a box of the width of #1
\def\hstretchto#1#2{%
	\mbox{}\vphantom{#2}\rlap{#2}\hphantom{#1}%
}

\def\vstretchto#1#2{%
	\mbox{}\hphantom{#2}\smash{#2}\vphantom{#1}%
}


%requires \thisyear to be defined (s. config.tex)!
\edef\nextyear{\the\numexpr\thisyear+1\relax}


% --- \frameheight constant ---
\newlength\fullframeheight
\newlength\framewithtitleheight
\setlength\fullframeheight{.92\textheight}
\setlength\framewithtitleheight{.86\textheight}

\newlength\frameheight
\setlength\frameheight{\fullframeheight}

\let\frametitleentry\relax
\let\oldframetitle\frametitle
\def\newframetitle#1{\global\def\frametitleentry{#1}\if\relax\frametitleentry\relax\else\setlength\frameheight{\framewithtitleheight}\fi\oldframetitle{#1}}
\let\frametitle\newframetitle

\def\newframetitleoff{\let\frametitle\oldframetitle}
\def\newframetitleon{\let\frametitle\newframetitle}
% --- \frameheight constant end ---

\newcommand{\fakeframetitle}[1]{%
	\vspace{-2.05\baselineskip}%
	{\Large \textbf{#1}} \\%
	\smallskip
}



\newenvironment{headframe}{\Huge THIS IS AN ERROR. PLEASE CONTACT THE ADMIN OF THIS TEX CODE. (headframe env def failed)}{}
\RenewEnviron{headframe}[1][]{
	\begin{frame}\frametitle{\ }
		\centering
		\Huge\textbf{\textsc{\BODY} \\
		}
		\Large {#1}
		\frametitle{\ }
	\end{frame}
}


\makeatletter
% Provides color if undefined.
\newcommand{\colorprovide}[2]{%
	\@ifundefinedcolor{#1}{\colorlet{#1}{#2}}{}}
\makeatother


\colorprovide{lightred}{red!30}
\colorprovide{lightgreen}{green!40}
\colorprovide{lightyellow}{yellow!50}
\colorprovide{lightblue}{blue!10}
\colorprovide{beamerlightred}{lightred}
\colorprovide{beamerlightgreen}{lightgreen}
\colorprovide{beamerlightyellow}{lightyellow}
\colorprovide{beamerlightblue}{lightblue}
\colorprovide{fullred}{red!60}
\colorprovide{fullgreen}{green}
\definecolor{darkred}{RGB}{115,48,38}
\definecolor{darkgreen}{RGB}{48,115,38}
\definecolor{darkyellow}{RGB}{100,100,0}

\only<handout:0>{\colorlet{adaptinglightred}{beamerlightred}}
\only<handout:0>{\colorlet{adaptinglightgreen}{beamerlightgreen}}
\only<handout:0>{\colorlet{adaptinglightyellow}{beamerlightyellow}}
\only<handout:0>{\colorlet{adaptinglightblue}{beamerlightblue}}
\only<beamer:0>{\colorlet{adaptinglightred}{lightred}}
\only<beamer:0>{\colorlet{adaptinglightgreen}{lightgreen}}
\only<beamer:0>{\colorlet{adaptinglightyellow}{lightyellow}}
\only<beamer:0>{\colorlet{adaptinglightblue}{lightblue}}
\only<handout:0>{\colorlet{adaptingred}{lightred}}
\only<beamer:0>{\colorlet{adaptingred}{fullred}}
\only<handout:0>{\colorlet{adaptinggreen}{lightgreen}}
\only<beamer:0>{\colorlet{adaptinggreen}{fullgreen}}



\newcommand{\TrueQuestion}[1]{
	\TrueQuestionE{#1}{}
}

\newcommand{\YesQuestion}[1]{
	\YesQuestionE{#1}{}
}

\newcommand{\FalseQuestion}[1]{
	\FalseQuestionE{#1}{}
}

\newcommand{\NoQuestion}[1]{
	\NoQuestionE{#1}{}
}

\newcommand{\DependsQuestion}[1]{
	\DependsQuestionE{#1}{}
}

\newcommand{\QuestionVspace}{\vspace{4pt}}
\newcommand{\QuestionParbox}[1]{\begin{varwidth}{.85\linewidth}#1\end{varwidth}}
\newcommand{\ExplanationParbox}[1]{\begin{varwidth}{.97\linewidth}#1\end{varwidth}}
\colorlet{questionlightgray}{gray!23}
\let\defaultfboxrule\fboxrule

% #1: bg color
% #2: fg color short answer
% #3: short answer text
% #4: question
% #5: explanation
\newcommand{\GenericQuestion}[5]{
	\setlength\fboxrule{2pt}
	\only<+|handout:0>{\hspace{-2pt}\fcolorbox{white}{questionlightgray}{\QuestionParbox{#4} \quad\textbf{?}}}
	\visible<+->{\hspace{-2pt}\fcolorbox{white}{#1}{\QuestionParbox{#4} \quad\textbf{\textcolor{#2}{#3}}} \if\relax#5\relax\else\ExplanationParbox{#5}\fi} \\
	\setlength\fboxrule{\defaultfboxrule}
}

% #1: Q text
% #2: Explanation
\newcommand{\TrueQuestionE}[2]{
	\GenericQuestion{adaptinglightgreen}{darkgreen}{Wahr.}{#1}{#2}
}

% #1: Q text
% #2: Explanation
\newcommand{\YesQuestionE}[2]{
	\GenericQuestion{adaptinglightgreen}{darkgreen}{Ja.}{#1}{#2}
}

% #1: Q text
% #2: Explanation
\newcommand{\FalseQuestionE}[2]{
	\GenericQuestion{adaptinglightred}{darkred}{Falsch.}{#1}{#2}
}

% #1: Q text
% #2: Explanation
\newcommand{\NoQuestionE}[2]{
	\GenericQuestion{adaptinglightred}{darkred}{Nein.}{#1}{#2}
}

% #1: Q text
% #2: Explanation
\newcommand{\DependsQuestionE}[2]{
	\GenericQuestion{adaptinglightyellow}{darkyellow}{Je nachdem!}{#1}{#2}
}

% #1: Q text
% #2: Answer
\newcommand{\ContentQuestion}[2]{
	\GenericQuestion{adaptinglightblue}{black}{\minusquad}{#1}{#2}
}

\ifnum\thisyear=2018 \else \errmessage{Old ILIAS link inside preamble. Please update.} \fi

\newcommand{\ILIAS}{\urlnamed{ILIAS}{https://ilias.studium.kit.edu/ilias.php?ref\_id=855240\&cmdClass=ilrepositorygui\&cmdNode=5r\&baseClass=ilrepositorygui}\xspace}

\newcommand{\Socrative}{\ifdefined\mysocrativeroom \only<handout:0>{socrative.com $\quad \~~> \quad $ Student login \\ Raumname:  \mysocrativeroom\\ \medskip}\else\fi}

\newcommand{\thasse}[1]{
	\ifdefined\ThassesTut #1\xspace \else\fi
}
\newcommand{\daniel}[1]{
	\ifdefined\DanielsTut #1\xspace \else\fi
}
\newcommand{\thassedaniel}[2]{\ifdefined\ThassesTut #1\else\ifdefined\DanielsTut #2\fi\fi\xspace}

\ifdefined\ThassesTut \ifdefined\DanielsTut \errmessage{ERROR: Both ThassesTut and DanielsTut flags are set. This is most likely an error. Please check your config.tex file.} \else \fi \else \ifdefined\DanielsTut \else \errmessage{ERROR: Neither ThassesTut  nor DanielsTut flags are set. This is most likely an error. Please check your config.tex file.} \fi\fi

%\newcommand{\sgn}{\text{sgn}}

%%%%%%%%%%%% INHALT %%%%%%%%%%%%%%%%

%% Wochennummer
\newcounter{weeknum}

%% Titelinformationen
\title[GBI-Tutorium \mytutnumber, Woche \theweeknum]{Grundbegriffe der Informatik \\ Tutorium \mytutnumber}

\subtitle{Woche \theweeknum \ | \mydate{\theweeknum} \\ \myname \ \  \normalfont (\mailto{\mymail})}
\author[\myname]{\myname}
\institute{KIT -- Karlsruher Institut für Technologie}
\date{\mydate{\theweeknum}\ }

% Modified, DJ (better safe than sorry)
\AuthorTitleSep{ – }

%% Titel einfügen
\newcommand{\titleframe}{\frame{\titlepage}}

%% Alles starten mit \starttut{X}
\newcommand{\starttut}[1]{\setcounter{weeknum}{#1}\pdfinfo{
		/Author (\myname)
		/Title  (GBI-Tutorium \mytutnumber, Woche \theweeknum)
	}\titleframe\frame{\frametitle{Inhalt}\tableofcontents} \AtBeginSection[]{%
		\begin{frame}{Wo sind wir gerade?}
		\tableofcontents[currentsection]
	\end{frame}\addtocounter{framenumber}{-1}}}


\newcommand{\framePrevEpisode}{
\begin{headframe}
	\mylasttimestext
\end{headframe}
}

\newcommand{\lastframetitled}[6]{
	\frame{\frametitle{#6}
		\vspace{-#2\baselineskip}
		\begin{figure}[H]
			\centering
			\LARGE \textbf{\textsc{#5}} \\
			\vspace{.2\baselineskip}
			\includegraphics[#1]{#3}
			\vspace{-6pt}
			\begin{center}
				\small \url{#4} 
			\end{center}
		\end{figure} 
	}
}

% #1 number
% #2 title 
% #3 vspace (positive) without unit (\baselineskip)
\newcommand{\xkcdframe}[3]{
	\lastframetitled{width=.96\textwidth}{#3}{xkcd/#1}{http://xkcd.com/#1}{}{#2}
}

\newcommand{\xkcdframevert}[3]
{
	\lastframetitled{height=.96\frameheight}{#3}{xkcd/#1}{http://xkcd.com/#1}{}{#2}
}

% #1 number
% #2 title 
% #3 vspace (positive) without unit (\baselineskip)
% #4 \includegraphics[] optional parameters
\newcommand{\xkcdframecustom}[4]
{
	\lastframetitled{#4}{#3}{xkcd/#1}{http://xkcd.com/#1}{}{#2}
}

\newcommand{\slideThanks}{
	\begin{frame}
	\frametitle{Credits}
	\begin{block}{}
		An der Erstellung des Foliensatzes haben mitgewirkt:\\[1em]
		Daniel Jungkind \\
		Thassilo Helmold \\
		Philipp Basler \\
		Nils Braun \\
		Dominik Doerner \\
		Ou Yue \\
	\end{block}
\end{frame}
}

%% Wörter DEPRECATED! DO NOT USE
\newcommand{\code}[1]{$\mathbf{#1}$}

\morescalingdelimiters

\begin{document}
\starttut{5}

% Zum Aufwärmen: PEBA Tutorenprogramm Aktivierung Code merken (PDF im gleichen Ordner zu finden)

 \framePrevEpisode


 
\begin{frame}{Rückblick}
	Auf formale Sprachen können wir \textbf{ähnliche} Operationen anwenden wie auf Wörtern:
	\begin{itemize}
		\item $L_1 \cdot L_2 = \{w_1 w_2 \mid w_1 \in L_1 \text{ und } w_2 \in L_2 \}$\\
		Jeweils ein Wort aus $L_1$ konkateniert mit einem Wort aus $L_2$.
		\pause
		\item $L^0 = \{\varepsilon \}, \qquad L^{i+1} = L^i \cdot L$\\
		Alle Wörter, die aus $i = 0,1,2...$ Wörtern der Sprache zusammengesetzt wurden
		\pause
		\item $L^+ = \bigcup \limits_{i=1}^\infty L^i \qquad L^* = L^+ \cup L^0$\\
		Alle Wörter, die sich aus den Wörtern der Sprache bilden lassen \\ 
		(ohne/mit zusätzlichem $\eps$ als Würze).
	\end{itemize}
	Ein Alphabet selbst ist \textbf{auch} ne formale Sprache, nämlich mit Wörtern der Länge 1.
\end{frame}

\begin{frame}[t]{Wahr oder falsch?}
	\FalseQuestionE{Jede Sprache enthält Wörter.}{ $\emptyset$ ist auch eine gültige Sprache.}
	\FalseQuestionE{$\word{0}^* = \{\eps, \word{0}, \word{00}, \word{000}, ...\}$}{$\word{0}^*$ gibt es nicht, denn $\word{0} \neq \{\word{0}\}$.}
	\TrueQuestionE{Es gibt Sprachen $L$, für die gilt $\eps \in L^+$.}{\ZB $L = \set{\eps, \word{aaa}}$.}
	\FalseQuestionE{$L^+ = L^* \setminus L^0$.}{Gilt nicht, wenn $\varepsilon \in L$.}
	\TrueQuestionE{$\{\}^* \neq \{\} $.}{ $\{\}^* = \{\varepsilon\}$.}	
\end{frame}

\section{Von der Darstellung zur Zahl}

\subsection{Definitionen}
\begin{frame}{Numerischer Wert}
	\begin{Definition}
		Zu einer Zahlenbasis $b$ definiere 
		$$ \text{Num}_b(\varepsilon) = 0  $$  
		$$ \text{Num}_b(wx) = b\cdot \text{Num}_b(w) + \text{num}_b(x) \text{ für alle } w\in Z_b^\ast, x\in Z_b $$ 
	\end{Definition}

	\pause
	Beachte: $Num_b : Z_b^* \to \Z$ ist Abbildung, die einem Wort (Zahlendarstellung) eine Zahl (Wert) zuordnet. Wir müssen diesen Wert aber natürlich wieder in eine  Darstellung umwandeln, um ihn aufschreiben zu können.
\end{frame}
\begin{frame}{Aufgabe}
	Berechnet die Zahlenwerte von $ 11_2, 321_4, B2_{16}$.
	\begin{align*} 
	\visible<1->{\text{Num}_2(11) &= 2\cdot \text{Num}_2(1) + \text{num}_2(1) \\
	&= 2\cdot 1 + 1 \\
	&= 3  \\}
	\visible<3->{\text{Num}_4(321) &=} \visible<4->{ 4\cdot \text{Num}_4(32) + \text{num}_4(1) \\
	&= 4\cdot \left( 4\cdot \text{Num}_4(3) + \text{num}_4(2) \right) + \text{num}_4(1) \\
	&= 4^2\cdot \text{num}_4(3) + 4 \cdot \text{num}_4(2) + \text{num}_4(1) \\
	&= 57 \\}
	\visible<5->{\text{Num}_{16}(B2) &=} \visible<6->{ 16 \cdot \text{Num}_{16}(B) + \text{num}_{16}(2) \\
	&= 16\cdot 11 + 2 \\
	&= 178}
	\end{align*}

\end{frame}

\begin{frame}{Wohldefiniertheit}
	\emph{Behauptung}: Die Definition 
		$$ \text{Num}_b(\varepsilon) = 0  $$  
		$$ \text{Num}_b(wx) = b\cdot \text{Num}_b(w) + \text{num}_b(x) \text{ für alle } w\in Z_b^\ast, x\in Z_b $$ 
		ist wohldefiniert und weist jedem Wort eine eindeutige Bedeutung zu, die dem Zahlenwert entspricht.
\end{frame}
\begin{frame}{Beweis}
	\begin{block}{Beweis durch vollständige Induktion über $n=\vert w \vert $}
	\begin{itemize}
		\only<1-2>{\item<1->[\emph{IA}] $n = 0 = \vert w \vert \implies w = \varepsilon $. \\
		Für $w = \varepsilon $ ist $\text{Num}_b$ wohldefiniert und sinnvoll (nämlich $Num_b(\varepsilon) = 0$).
		\item<2->[\emph{IV}] Für ein beliebig aber festes $n\in\N_0$ sei $Num_b(w) für alle w mit \setsize{w} = n wohldefiniert und entspreche dem Zahlenwert.$ }
		\only<3->{\item<3->[\emph{IS}] Wähle $w'$ mit $\vert w' \vert = n+1 $, dann gibt es ein $w\in Z_b^n, x\in Z_b$, so dass $ w' = wx $ \\
		Mit der Definition gilt nun $$ \text{Num}_b(w') = b\cdot \underbrace{\text{Num}_b(w)}_{IV} + \text{num}_b(x) $$
		Die Summe ist laut $IV$ wohldefiniert. Auch ist laut $IV$ $\text{Num}_b(w)$ der Zahlenwert von $w$ und damit auch $\text{Num}_b(w')$.}
	\end{itemize}
	\end{block}

\end{frame}

%\subsection{Aufgabe}
%\begin{frame}{Aufgabe. WS 2010 }
%Es bezeichne $\Z$ die Menge der ganzen Zahlen. Gegeben sei eine Ziffernmenge $Z_{-2} = \{N, E\}$ mit der Festlegung $num_2 (N) = 0$ und $num_2 (E) = 1$. Wir definieren eine Abbildung $Num_{-2} : Z_{-2}^\ast \to \Z$ wie folgt:
%	$$Num_{-2} (\varepsilon) = 0$$
%	$$\forall \ w \in Z_{-2}^\ast \ \forall \ x \in Z_{-2} : Num_{-2} (wx) = -2 \cdot Num_{-2} (w) + num_2 ( x )$$
%
%	\begin{itemize}	
%		\item Geben Sie für $w \in \{E, EN, EE, ENE, EEN, EEE\}$ jeweils $Num_{-2} (w)$ an.
%		\item Für welche Zahlen $x \in \Z$ gibt es ein $w \in Z_{-2}^\ast$ mit $Num_{-2} (w) = x$?
%	\end{itemize}
%\end{frame}
%
%\begin{frame}{Lösung}
%\textit{Geben Sie für $w \in \{E, EN, EE, ENE, EEN, EEE\}$ jeweils $Num_{-2} (w)$ an.} \pause
%	\begin{table}[h!]	
%		\begin{tabular}{>{$}l<{$}>{$}l<{$}}
%			Num_{-2} (E)\pause & = 1 \\ \pause 
%			Num_{-2} (EN)\pause & = -2 \\ \pause
%			Num_{-2} (EE)\pause & = -1 \\ \pause
%			Num_{-2} (ENE)\pause & = 5 \\ \pause
%			Num_{-2} (EEN)\pause & = 2 \\ \pause
%			Num_{-2} (EEE)\pause & = 3
%	\end{tabular}
%	\end{table}
%	\pause
%	\textit{Für welche Zahlen $x \in \Z$ gibt es ein $w \in Z_{-2}^\ast$ mit $Num_{-2} (w) = x$?} \\[1em]\pause
%	Für alle!
%\end{frame}

\section{Von der Zahl zur Darstellung}
\begin{frame}{Division und Modulo}
	\begin{block}{Definition}
		$ x \div y$ ist die ganzzahlige Division von x durch y.\\
		$ x \mod y$ liefert den Rest dieser Division
	\end{block} 
	\pause
	
	\begin{block}{Beobachtung}
		$ x\div y \in \N_0 \qquad x\mod y \in \{0,\dots, y-1\} $
	\end{block}
	\pause
	
	\begin{block}{Lemma}
		$$ x = y \cdot (x \div y ) + \left( x \mod y \right)$$ 
	\end{block}
	
\end{frame}

\begin{frame}
	\begin{block}{Beispiel}
		\begin{tabular}{ccc}
			& $x\div y$ & $x\mod y$ \\
			$x=2,y=3$ \pause &  0 & 2 \\  \pause
			$x=5 ,y=2$ \pause & 2 & 1 \\	\pause
			$x=8,y=2$ \pause & 4 & 0 \\	
		\end{tabular}
	\end{block}
\end{frame}

\begin{frame}{Beispiel}
	\begin{table}[h!]
		\centering
		\begin{tabular}{c|cccccccccccc}
			$x$ & 0 & 1 & 2 & 3 & 4 & 5 & 6 & 7 & 8 & 9 & 10 & 11 \\ \hline
			$x\div 4 $ & \only<2->{0 & 0 & 0 & 0 & 1 & 1 & 1 & 1 & 2 & 2 & 2 & 2 } \only<1|handout:0>{&&&&&&&&&&&} \\
			$4\left( x\div 4\right) $ & \only<3->{0&0&0&0&4&4&4&4&8&8&8&8} \only<1-2|handout:0>{&&&&&&&&&&}  \\
			$x\mod 4$ & \only<4>{0&1&2&3&0&1&2&3&0&1&2&3} \only<1-3|handout:0>{&} 
		\end{tabular}
	\end{table}
\end{frame}

\begin{frame}{Repräsentation}
	\begin{block}{Definition}
		$Repr_k(n)$ ist das kürzeste Wort $w\in Z_k^\ast$ mit $Num_k(w)=n$, also 
		$$ Num_k\left( Repr_k(n)\right) = n $$ 
	\end{block}
	\pause
	\emph{Anmerkung}:
	Im Allgemeinen $$ Repr_k\left(Num_k(w)\right) \neq w $$ da überflüssige Nullen wegfallen. 
\end{frame}

\begin{frame}{Repräsentation}
	Wir definieren
	\begin{align*}
		Repr_k : \; &\N_0 \to Z_k  \\
		n&\mapsto \begin{cases} repr_k(n) & n<k \\ Repr_k\left( n\div k \right) \cdot repr_k\left( n \mod k \right) & n\geq k 
		\end{cases} 
	\end{align*}
	
	\begin{block}{Aufgabe}
		Berechne folgende Darstellungen:\\
		$Repr_2(42) = \pause 101010$ \\
		$Repr_4(42) = \pause 222$ \\
		$Repr_8(42) = \pause 52$ \\
		$Repr_{16}(42) = \pause 2A$
	\end{block}
\end{frame}

\begin{frame}{Beispiel: Lösung}
	\begin{align*}
		Repr_8(42) &= Repr_8(42 \div 8) \cdot repr_8(42 \mod 8) \\
		&= Repr_8(5) \cdot repr_8(2)\\
		&= repr_8(5) \cdot 2\\
		&= 5 \cdot 2\\
		&= 52_8
	\end{align*}
	
\end{frame}


\mycomment{
	\begin{frame}{Rückblick: Zahlendarstellung}
		\begin{itemize}
			\item Zahlen sind Objekte, die einen festen numerischen Wert haben.\\
			Um eine Zahl aufzuschreiben, benötigen wir aber eine Darstellung. \\
			Die gleiche Zahl kann viele verschiedenen Darstellungen annehmen.
			\item Mit der Auswertungsfunktion $Num_b(\cdot)$ berechnen wir von einer Zahlendarstellung den numerischen Wert.
			\item Mit der Funktion $Repr_b(\cdot)$ können wir eine Zahl in einer beliebigen Darstellung angeben.
		\end{itemize}
		
		Bis jetzt haben wir das alles nur für positive Zahlen gesehen!
	\end{frame}
}

\begin{frame}{Aufgabe: Zahlendarstellung}
	Berechnet die folgenden Zahlenwerte:
	\begin{align*}
		Num_2(\word 1) &= \visible<2->{1} \\
		Num_2(\word{11}) &= \visible<3->{3} \\
		Num_2(\word{111}) &= \visible<4->{7} \\
		Num_2(\word{1111}) &= \visible<5->{15}
	\end{align*}
	
	Gibt es ein allgemeines Muster? \\ \pause[6]
	\delimitershortfall=0pt
	Ja, es gilt $$Num_2(\word 1^\ell) = 2^\ell - 1$$ und allgemein \pause[7] $$Num_b\left({\underbrace{(b-1)}_{\text{\footnotesize $\in Z_b$ }}}^\ell\right) = b^\ell - 1$$
\end{frame}

\section{Zweierkomplement}

\begin{frame}{Ein asymetrischer Zahlenbereich}
	\[
	\nK_{\ell} = \{ x\in \nZ \mid -2^{\ell-1} \leq x \leq 2^{\ell-1} -1 \} \;.
	\]
	\\[0.2cm]
	
	\begin{figure}
		\centering
		\includegraphics[scale=0.45]{ZK_K4}
	\end{figure}
	
\end{frame}

\begin{frame}{Zweierkomplement}

	Das Zweierkomplement ist eine Möglichkeit, negative Zahlen binär darzustellen. Im Vergleich zu anderen Darstellungsarten ist es besonders vorteilhaft bei arithmetischen Rechnungen mit Hardware \textit{(mehr dazu in Technischer Informatik)}.

	\begin{block}{Definition}
		$$Zkpl_l(x) = \begin{cases} 0 bin_{l-1}(x) & \text{falls } x \geq 0 \\ 1 bin_{l-1}(2^{l-1}+x) & \text{falls } x < 0\end{cases}$$
		
		Äquivalent:
		$$Zkpl_l(x) = \begin{cases} bin_{l}(x) & \text{falls } x \geq 0 \\ bin_{l}(2^{l}+x) & \text{falls } x < 0\end{cases}$$
	\end{block}
\end{frame}

\begin{frame}{ZK-Beispiel}
	\begin{block}{Beispiele}
		\begin{align*}
			&Zkpl_5(0) \only<2->{= 00000} \\
			&Zkpl_5(2) \only<3->{= 00010} \\
			&Zkpl_5(15) \only<4->{= 01111} \\
			&Zkpl_5(-1) \only<5->{= 11111} \\
			&Zkpl_5(-6) \only<6->{= 11010} \\
			&Zkpl_5(-16) \only<7->{= 10000}
		\end{align*}
	\end{block}
\end{frame}

\begin{frame}{ZK: Einfache Berechnung}
	Zum \enquote{intuitiven} Berechnen des Zweierkomplement können wir so vorgehen (für $x < 0$):
	\begin{enumerate}
		\item Binärdarstellung von $\setsize{x}$ berechnen
		\item Mit führenden Nullen auffüllen bis zur Länge $\ell$
		\item Alle binären Ziffern negieren
		\item 1 addieren
	\end{enumerate}

	\begin{Beispiel}
		$$Zkpl_4(-2): 2 \rightarrow 10 \rightarrow 0010 \rightarrow 1101 \rightarrow 1110 $$
	\end{Beispiel}
\end{frame}

\begin{frame}{ZK: Einfache Berechnung}
	Die einzelnen Schritte können wir auch formal angeben:\\
	(Wir operieren jeweils auf Wörtern aus $\{0, 1\}^* = Z_2^*$) \\[0.5em]
	1. Binärdarstellung von $\setsize{x}$ berechnen: $Repr_2(\setsize{\cdot})$ \\
	2. Mit führenden Nullen auffüllen bis zur Länge $\ell$\\ \pause
	\begin{align*}
		Fill_\ell : Z_2^m &\to Z_2^\ell \qquad (m \le \ell) \\ \visible<3-> {
		w &\mapsto \begin{cases}
		0^\ell & w = \varepsilon \\
		Fill_{\ell-1}(w') \cdot \mu & w = w' \cdot \mu, w' \in Z_2^*, \mu \in Z_2
		\end{cases} \\
		&\text{oder deutlich einfacher} \\
		w &\mapsto \begin{cases}
		w & \setsize{w} = \ell \\
		Fill_\ell(0w) & \text{sonst}
		\end{cases} \\
	}
	\end{align*}
	\pause[4] 3. / 4. Analog %TODO
\end{frame}

\begin{frame}{Von einer Zahlendarstellung zur Anderen}
	Eigentlich recht intuitiv:
	$$\fTrans_{3,5} = \fRepr_3 \after \fNum_5$$
	
	\begin{block}{Aufgabe}
		Berechne folgende Darstellungen:\\
		\begin{enumerate}[(1)]
			\item $\fRepr_2(42) = \visible<1->{ \word{101010}_2}$ 
			\item $\fTrans_{4,2}(\word{101010}) = \visible<2->{ \word{222}_4}$ 
			\item $\fTrans_{8,10}(\word{42}) = \visible<3->{ \word{52}_8}$ 
			\item $\fTrans_{16,10}(\word{42}) = \visible<4->{ \word{2A}_{16}}$
		\end{enumerate}
	\end{block}

	Was ist bei allen Wörtern gleich? \only<5->{\impl Die Bedeutung!} \\
	Was macht die Rechnungen (2) -- (4) vergleichsweise einfach? \pause[6] \\
	\impl Wir können die Struktur ausnutzen und zeichenweise vorgehen! ($\fTrans_{8,10}(\word{42}) = \fTrans_{8,2}(\word{101} \cdot \word{010}) = \fTrans_{8,2}(\word{101}) \cdot \fTrans_{8,2}(\word{010})$)
\end{frame}

\section{Codierungen}

\begin{frame}{Übersetzungen}
	\textbf{Übersetzung:} „Bedeutungserhaltende“ Abbildung \\[0.5em] \pause
	\textbf{Codierung:} Injektive Übersetzung \\ \pause
	\begin{itemize}
		\item Injektiv reicht, weil wir jedem $f(w)$ sein erzeugendes $w$ zuordnen können
		\implitem Definiere Bedeutung von $f(w) := \text{Bedeutung von } w$
	\end{itemize}
	
	\pause
	\begin{itemize}
		\item \textbf{Problem}: \emph{Beliebige} Codierungen zu speichern sehr aufwendig
		\item $\abs{\text{Definitionsbereich}} = \infty$ \impl sogar unmöglich!
		\implitem Bringen wir etwas Struktur ins Spiel!
	\end{itemize}
\end{frame}

\begin{frame}{Homomorphismen}
	Ein \textbf{Homomorphismus} ist eine \textit{strukturerhaltende} Abbildung \\ \pause
	\begin{threealign}
		\Phi : A &\functionto& B \\
		\text{ mit } \forall a \in A, b \in B: \quad  \Phi(a \sim b) &=& \Phi(a) \bowtie \Phi(b)
	\end{threealign}
	\medskip \\
	Dabei stehen $\sim$ und $\bowtie$ hier für feste beliebige Operationen, wie z.~B. $+, -, \·, /, \oplus, \barwedge, ...$ \\
	(Muss es auf $A$ bzw. $B$ natürlich geben!)
\end{frame}

\begin{frame}{Homomorphismen}
	In GBI: Homomorphismen auf Wörtern mit „$\·$“ als Operation.\\
	\impl $A, B$ Alphabete, dann ist $h: A^* \functionto B^*$ ein Homomorphismus, wenn
	$$ \forall x, y \in A^* : \quad h(x \cdot y) = h(x) \cdot h(y). $$
	
	\pause
	\impl Brauchen nur Funktionswerte für einzelne Zeichen $a \in A$, dann kennen wir schon das ganze $h$! \\
	\impl Brauchen uns nur eine Tabelle für die Zeichen zu merken 
	\impl $A$ endlich \impl Tabelle endlich \impl gut speicherbar
	
	\pause
	\begin{Beispiel}
		Sei $h$ ein Homomorphismus, für den gilt: $h(\word a) = \word 2, h(\word b) = \word 3$. \\
		Dann gilt $h(\word{aba}) = h(\word a) \cdot h(\word b) \cdot h(\word a) = \word{232}. $ \\[0.5em]
		$Trans_{3,2}$ ist kein Homomorphismus (siehe dazu ÜB WS15/16)
	\end{Beispiel}
\end{frame}

\begin{frame}{Homomorphismen bauen}
	Haben: Abbildung der einzelnen Zeichen \\
	Wollen: Abbildung ganzer Wörter 
	\begin{Definition}
		Sei $f: A \functionto B^*,$ \pause definiere $f^{**}:A^* \functionto B^*$ als
		\begin{align*}
		f^{**}(\eps) &= \eps  \\
		\forall w\in A^*, x\in A: \quad  f^{**}(w \· x) &= f^{**}(w) \· f(x)       
		\end{align*}
	\end{Definition}

	$f^{**}$ heißt der durch $f$ \textbf{induzierte} Homomorphismus.
\end{frame}

\begin{frame}{$\eps$-Freiheit}
	Klar ist: Für jeden Hom. $h$ ist $h(\eps) = \eps$.
	
	\pause
	\begin{Definition}
		Ein Homomorphismus heißt $\eps$-frei, wenn 	$$ \forall x\in A : h(x) \neq \eps. $$
	\end{Definition}

	
	Was ist das Problem mit Homomorphismen, die nicht $\eps$-frei sind? \\ \pause
	\impl Es geht Information verloren.\\
	
	\begin{Beispiel}
		Sei $h$ ein Hom. mit $h(\word c) = \eps, \; h(\word b) = \word 2$. \\
		Haben unbekanntes $w \in \{\word b, \word c\}^*$ und wissen: $h(w) = \word 2$. \\
		\smallskip
		Wie kommen wir von $h(w)$ wieder zu $w$ zurück? \\ \pause
		\impl Gar nicht! (Wissen nicht, wieviel \word c in $w$ drin sind.)
	\end{Beispiel}
\end{frame}

\begin{frame}{Präfixfreiheit}
	\begin{Definition}
		Ein Homomorphismus heißt \textbf{präfixfrei}, wenn für
		\emph{keine} zwei verschiedenen Symbole $x_1,x_2\in A$ gilt: $h(x_1)$
		ist ein Präfix von $h(x_2)$.
	\end{Definition}

	\bigskip
	Was ist das Problem mit Homomorphismen, die nicht präfixfrei sind? \\ \pause
	\impl Es geht Information verloren.\\
	
	\begin{Beispiel}
		Sei $h$ ein Hom. mit $h(\word a) = \word 2, \; h(\word b) = \word 3, \; h(\word c) = \word{23}$. \\ 
		Haben unbek. $w \in \{\word a, \word b, \word c\}^*$ und wissen: $h(w) = \word{23}$. \\
		\smallskip
		Wie kommen wir von $h(w)$ wieder zu $w$ zurück?\\ \pause
		\impl Gar nicht! $w = \word c$ oder $w = \word{ab}$, wir wissen es nicht!
	\end{Beispiel}
	
\end{frame}

\begin{frame}{Zurück zu Codierungen}
	\begin{block}{Beobachtung}
		Präfixfreie Homomorphismen sind $\eps$-frei.
	\end{block}

	\pause
	\begin{block}{Lemma}
		Präfixfreie Homomorphismen sind Codierungen (also injektiv).
	\end{block}

	\pause
	\begin{block}{Beobachtung}
		Präfixfreie Codes kann man \enquote{einfach} decodieren:
		\[
		u(w) = 
		\begin{cases}
		\eps, & \text{ falls } w=\eps\\
		x\·u(w'), & \text{ falls } w=h(x) \· w' \text{ für ein } x\in A \\
		\bot,  & \text{ sonst }\\
		\end{cases}
		\]
	\end{block}
	($\bot$: „undefiniert“, „bottom“.)
	
\end{frame}

\section{Huffman-Codierung}

\begin{frame}
	Kann man mit einer Codierung die benötigte Anzahl der Zeichen für ein Wort reduzieren und trotzdem den Sinn erhalten?\\[0.5em]
	\pause
	Natürlich geht das (manchmal), dieses Verfahren ist überall im Einsatz:\\
	Komprimierung!
\end{frame}

\begin{frame}{Huffman-Codierung}
	Eine Huffman-Codierung ist ein präfixfreier (und demnach \enquote{einfach} zu decodierenden) Homomorphismus, bei der die Codierung eines Zeichens umso länger wird, je seltener das Zeichen vorkommt.
	
	Die Huffman-Codierung für ein Wort ist dabei nicht eindeutig, wie wir gleich im Konstruktionsverfahren sehen werden.
	
	\begin{block}{Lemma}
		Unter allen präfixfreien Codes führen Huffman-Codes zu kürzesten Codierungen
		\textbf{des Wortes, für das die Huffman-Codierung konstruiert wurde.}
	\end{block}
\end{frame}

\begin{frame}{Konstruktionsverfahren}
	Formal: In der Vorlesung\\
	Hier: Vorgehensweise (ausreichend!)
	\begin{enumerate}
		\item Für jedes Zeichen die Häufigkeit ermitteln
		\item Alle Zeichen mit ihrer Häufigkeit als Blätter in die unterste Ebene zeichnen
		\item Jeweils die zwei Knoten (nicht unbedingt Blätter!) mit den geringsten Häufigkeiten \enquote{verbinden}(also einen neuen Knoten darüber anlegen, der die Summe der Häufigkeiten erhält)
		\item Fortfahren, bis der ganze Baum aufgebaut ist.
		\item Die linken Äste mit 0 beschriften, die rechten Äste mit 1.
		\item Codierungen der Zeichen ablesen
		\item Ausgangswort codieren (wenn gefordert, vergesst das nicht!)
	\end{enumerate}
\end{frame}

\begin{frame}
	\frametitle{Beispiel}
	Gegeben : $w= abadcadaac $ (10 Zeichen) \\[1em]
	
	\only<1-3|handout:1>{ \visible<3>{
	\begin{minipage}{0.6\linewidth}
		\begin{figure}[b]
			\centering
			\begin{tikzpicture}
			[level 1/.style={sibling distance=40mm},
			level 2/.style={sibling distance=20mm},
			level 3/.style={sibling distance=15mm}]
			\node {$10$}
			child {
				node{$5$}
				child{
					node{$3$}
					child{
						node{$1,b$}
						edge from parent node[left] {\bzero}
					}
					child {
						node{$2,d$}
						edge from parent node[right] {\bone}
					}
					edge from parent node[left] {\bzero};
				}
				child{
					node{$2,c$}
					edge from parent node[right] {\bone}
				}
				edge from parent node[left] {\bzero};
			}
			child{
				node{$5,a$}
				edge from parent node[right] {\bone}
			};
			\end{tikzpicture}
		\end{figure}  
	\end{minipage}
	}}
	\only<4-|handout:2> {
		\begin{minipage}{0.6\linewidth}
			Wie lang wird das neue Wort ?\\ \visible<5->{ 
			$5*1 + 1*3 + 2*2 + 2*3 = 18$ Zeichen\\ 
			Aber wir wollen doch komprimieren? Was haben wir falsch gemacht?\\
			\visible<6->{Nichts. Zur Codierung eines der Zeichen $a, b, c, d$ benötigen wir mindestens 2 Bit (da 4 Möglichkeiten). Für die Zeichen $0, 1$ brauchen wir aber nur ein Bit. \\
			Also haben wir 18 Bit statt 20 Bit und somit komprimiert!
		}}
		\end{minipage}
	}
	\visible<2-> {
		\begin{minipage}{0.2\linewidth}
			\hfill
			\hfill 
			\vspace*{0.1\linewidth}
			\begin{table}[H]
				\begin{tabular}{c|cccc}
					\hline
					x & a & b & c & d  \\ \hline
					$|w|_x$  & 5 & 1 & 2 & 2 \\ \hline
					h(x) & 1 & 000 & 01 & 001 \\ \hline 
				\end{tabular}
			\end{table}
		\end{minipage}
	}
\end{frame}

%TODO: Make solution frame!
\begin{frame}
		Huffman-Codierungen funktionieren immer nur gut für Wörter, die eine gleiche/ähnliche relative Zeichenhäufigkeit haben wie das Wort, für das der Code erstellt wurde.
		
		\begin{Beispiel}
			$w_1 = badcfehg, w_2 = a^1b^2c^4d^8e^{16}f^{32}g^{64}h^{128}$
			Erstellt eine Huffman-Codierung für jedes der beiden Wörter und Codiert jeweils beide Wörter mit der erstellten Codierung.\\
			Wie verhalten sich die Längen der Codewörter?
		\end{Beispiel}
\end{frame}

\begin{frame}
	\begin{block}{Erweiterung}
		Wir können nicht nur einzelne Buchstaben codieren. \\
		Bei $ w = a^{10}b^{10}c^{10} $ lohnt es sich pro Block gleicher Buchstaben eine Codierung zu haben.
	\end{block}
\end{frame}


\subsection{Aufgaben}
\begin{frame}
	\frametitle{Aufgabe (WS 2008) }
	Das Wort $$w = \mathbf{0000\only<3->{\text{ } }0001\only<3->{\text{ } }0011\only<3->{\text{ } }0001\only<3->{\text{ } }0011\only<3->{\text{ } }0000\only<3->{\text{ } }0000\only<3->{\text{ } }1110\only<3->{\text{ } }0001\only<3->{\text{ } }0000}$$ soll komprimiert werden.
	
	\pause
	\begin{itemize}[<+->]
		\item Zerlegen Sie $w$ in Viererblöcke und bestimmen Sie die Häufigkeiten der vorkommenden Blöcke.
		\item Zur Kompression soll ein Huffman-Code verwendet werden. Benutzen Sie die in Teilaufgabe a) bestimmten Häufigkeiten, um den entsprechenden Baum aufzustellen. Beschriften Sie alle Knoten und Kanten.
		\item Geben Sie die Codierung des Wortes $w$ mit Ihrem Code an.
	\end{itemize}
\end{frame}

\begin{frame}
	\frametitle{Lösung}
	$$w = \mathbf{0000000100110001001100000000111000010000}$$
	\textit{Zerlegen Sie $w$ in Viererblöcke und bestimmen Sie die Häufigkeiten der vorkommenden Blöcke.} \\[1em]
	\pause
	$$w = \mathbf{0000 \ 0001 \ 0011 \ 0001 \ 0011 \ 0000 \ 0000 \ 1110 \ 0001 \ 0000}$$ \pause
	\begin{table}[h!]
		\centering
		\begin{tabular}{l|cccc}	
			& 0000 & 0001 & 0011 & 1110 \\ \hline
			Absolute Häufigkeiten: & 4 & 3 & 2 & 1 \\
			Relative Häufigkeiten:  & 0,4 & 0,3 & 0,2 & 0,1\\
		\end{tabular}
	\end{table}
\end{frame}

\begin{frame}
	\frametitle{Lösung}
	\vspace*{1em}
	\begin{minipage}{0.45\linewidth}
		\textit{Zur Kompression soll ein Huffman-Code verwendet werden. Benutzen Sie die in Teilaufgabe a) bestimmten Häufigkeiten, um den entsprechenden Baum aufzustellen. Beschriften Sie alle Knoten und Kanten.}
		\pause 
		\begin{table}[h!]
			\centering
			\begin{tabular}{cccc}	
				0000 & 0001 & 0011 & 1110 \\ \hline
				4 & 3 & 2 & 1 \\	
			\end{tabular}
		\end{table}
	\end{minipage}
	\hfill
	\begin{minipage}{0.5\linewidth}
		\begin{figure}[h!]
			\centering
			\only<3>{\includegraphics[scale=0.35]{Huffman.pdf}}
		\end{figure}
	\end{minipage}
\end{frame}

\begin{frame}
	\frametitle{Lösung}
	\textit{Geben Sie die Codierung des Wortes $w$ mit Ihrem Code an.} \\[2em] \pause
	$0000000100110001001100000000111000010000$ \\ \hfill $\to 1010010100111000011$
\end{frame}

%\begin{frame}
%	\frametitle{Aufgabe (WS 2010)}
%	Seien $n, k \in \nN_0$ mit $1 \leq k \leq n$. In einem Wort $w \in \{a, b, c\}^\ast$ der Länge $3n$ komme $k$ mal das Zeichen $a$, $n$ mal das Zeichen $b$ und $2n - k$ mal das Zeichen $c$ vor.
%	\begin{itemize}
%		\item Geben Sie den für die Huffman-Codierung benötigten Baum an.
%		\item Geben Sie (in Abhängigkeit von $k$ und $n$) die Länge des zu $w$ gehörenden Huffman-Codes an.
%	\end{itemize}
%\end{frame}
%
%\begin{frame}
%	\frametitle{Lösung}
%	\vspace*{1em}
%	\begin{minipage}{0.45\linewidth}
%		\textit{$\dots$ Länge $3n$ komme $k$ mal das Zeichen $a$, $n$ mal das Zeichen $b$ und $2n - k$ mal das Zeichen $c$ vor. \\[1em] Geben Sie den für die Huffman-Codierung benötigten Baum an.} 
%		\pause
%		\begin{table}[h!]
%			\centering
%			\begin{tabular}{ccc}	
%				$a$ & $b$ & $c$ \\ \hline
%				$k$ & $n$ & $2n-k$ \\	
%			\end{tabular}
%		\end{table}
%		\pause
%		$$k \leq n \leq 2n -k $$ $$ n+k+2n-k = 3n$$
%	\end{minipage}
%	\hfill
%	\begin{minipage}{0.5\linewidth}
%		\begin{figure}[h!]
%			\centering
%			\only<4>{\includegraphics[scale=0.35]{Huffman2.pdf}}
%		\end{figure}
%	\end{minipage}
%\end{frame}
%
%\begin{frame}
%	\frametitle{Lösung}
%	\textit{Geben Sie die Länge des zu $w$ gehörenden Huffman-Codes an.} \\[2em]
%	\pause
%	Jedes $a$ und jedes $b$ wird durch zwei Zeichen codiert, und jedes $c$ wird durch ein Zeichen codiert. Damit erhält man insgesamt $$2k + 2n + 2n - k = 4n + k$$ Zeichen in der Codierung.
%\end{frame}


\begin{frame}{Ausblick}
	Die Huffman-Codierung hat ein Problem: Zum Decodieren muss der Huffman-Baum, der für die Codierung verwendet wurde, bekannt sein. Im wesentlichen gibt es dafür zwei Möglichkeiten:
	\begin{enumerate}
		\item Der Codebaum wird vor dem eigentlichen Codewort angegeben.\\ 
		Problem: Das verlängert das Codewort.
		\item Es wird ein vorher festgelegter Codebaum verwendet.\\
		 Problem: Dieser Codebaum ist nicht an das spezifische Wort angepasst und kann evtl. (bei komplett anderer Zeichenhäufigkeit) zu sehr schlechten Ergebnissen führen.
	\end{enumerate}

	Diese Probleme können durch andere Codierungsverfahren gelöst werden, indem z.B. das Wörterbuch dynamisch während der Decodierung aus dem Codewort aufgebaut wird (z.B. Lempel-Ziv-Welch-Verfahren).
\end{frame}



\begin{frame}	
	\begin{block}{Was ihr nun wissen solltet}
		\begin{itemize}
			\item Wie man Zahlen anders darstellt
			\item Wie man das Zweierkomplement bildet
			\item Übersetzungen und Codierungen
			\item Huffman-Codierung
		\end{itemize}
	\end{block}
	
	\begin{block}{Was nächstes Mal kommt}
		\begin{itemize}
			\item Speicher -- Damit wir nicht alles gleich wieder vergessen
			\item MIMA -- Den Bits beim Arbeiten zuschauen
		\end{itemize}
	\end{block}
\end{frame}

\xkcdframe{571}{Danke für eure Aufmerksamkeit! \smiley}{2.5} % Zweierkomplement
% \lastframe{0.75}{0}{xkcd/tar.png}{https://www.xkcd.com/1168/} % Komprimierung

\slideThanks

\end{document}