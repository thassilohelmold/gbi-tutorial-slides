%beamer

% Comment/uncomment this line to toggle handout mode
% \newcommand{\handout}{}

%% Beamer-Klasse im korrekten Modus
\ifdefined \handout
\documentclass[handout]{beamer} % Handout mode
\else
\documentclass{beamer}
\fi

%% UTF-8-Encoding
\usepackage[utf8]{inputenc}

% % \bigtimes abgeschrieben von http://tex.stackexchange.com/questions/14386/importing-a-single-symbol-from-a-different-font
% \DeclareFontFamily{U}{mathx}{\hyphenchar\font45}
% \DeclareFontShape{U}{mathx}{m}{n}{
%       <5> <6> <7> <8> <9> <10> gen * mathx
%       <10.95> mathx10 <12> <14.4> <17.28> <20.74> <24.88> mathx12
%       }{}
% \DeclareSymbolFont{mathx}{U}{mathx}{m}{n}
% \DeclareMathSymbol{\bigtimes}{\mathop}{mathx}{161}

\RequirePackage{xcolor}

\def\9{\square}
%\def\9{\blank}

% f"ur Aussagenlogik
\colorlet{alcolor}{blue}
\RequirePackage{tikz}
\usetikzlibrary{arrows.meta}
\newcommand{\alimpl}{\mathrel{\tikz[x={(0.1ex,0ex)},y={(0ex,0.1ex)},>={Classical TikZ Rightarrow[]}]{\draw[alcolor,->,line width=0.7pt,line cap=round] (0,0) -- (15,0);\path (0,-6);}}}
\newcommand{\aleqv}{\mathrel{\tikz[x={(0.1ex,0ex)},y={(0ex,0.1ex)},>={Classical TikZ Rightarrow[]}]{\draw[alcolor,<->,line width=0.7pt,line cap=round] (0,0) -- (18,0);\path (0,-6);}}}
\newcommand{\aland}{\mathbin{\raisebox{-0.6pt}{\rotatebox{90}{\texttt{\color{alcolor}\char62}}}}}
\newcommand{\alor}{\mathbin{\raisebox{-0.8pt}{\rotatebox{90}{\texttt{\color{alcolor}\char60}}}}}
%\newcommand{\ali}[1]{_{\mathtt{\color{alcolor}#1}}}
\newcommand{\alv}[1]{\mathtt{\color{alcolor}#1}}
\newcommand{\alnot}{\mathop{\tikz[x={(0.1ex,0ex)},y={(0ex,0.1ex)}]{\draw[alcolor,line width=0.7pt,line cap=round,line join=round] (0,0) -- (10,0) -- (10,-4);\path (0,-8) ;}}}
\newcommand{\alP}{\alv{P}} %ali{#1}}
%\newcommand{\alka}{\negthinspace\hbox{\texttt{\color{alcolor}(}}}
\newcommand{\alka}{\negthinspace\text{\texttt{\color{alcolor}(}}}
%\newcommand{\alkz}{\texttt{\color{alcolor})}}\negthinspace}
\newcommand{\alkz}{\text{\texttt{\color{alcolor})}}\negthinspace}
\newcommand{\AAL}{A_{AL}}
\newcommand{\LAL}{\hbox{\textit{For}}_{AL}}
\newcommand{\AxAL}{\hbox{\textit{Ax}}_{AL}}
\newcommand{\AxEq}{\hbox{\textit{Ax}}_{Eq}}
\newcommand{\AxPL}{\hbox{\textit{Ax}}_{PL}}
\newcommand{\AALV}{\hbox{\textit{Var}}_{AL}}
\newcommand{\MP}{\hbox{\textit{MP}}}
\newcommand{\GEN}{\hbox{\textit{GEN}}}
\newcommand{\W}{\ensuremath{\hbox{\textbf{w}}}\xspace}
\newcommand{\F}{\ensuremath{\hbox{\textbf{f}}}\xspace}
\newcommand{\WF}{\ensuremath{\{\W,\F\}}\xspace}
\newcommand{\val}{\hbox{\textit{val}}}
\newcommand{\valDIb}{\val_{D,I,\beta}}

\newcommand*{\from}{\colon}

% die nachfolgenden Sachen angepasst an cmtt
\newlength{\ttquantwd}
\setlength{\ttquantwd}{1ex}
\newlength{\ttquantht}
\setlength{\ttquantht}{6.75pt}
\def\plall{%
  \tikz[line width=0.67pt,line cap=round,line join=round,baseline=(B),alcolor] {
    \draw (-0.5\ttquantwd,\ttquantht) -- node[coordinate,pos=0.4] (lll){} (-0.25pt,-0.0pt) -- (0.25pt,-0.0pt) -- node[coordinate,pos=0.6] (rrr){} (0.5\ttquantwd,\ttquantht);
    \draw (lll) -- (rrr);
    \coordinate (B) at (0,-0.35pt);
  }%
}
\def\plexist{%
  \tikz[line width=0.67pt,line cap=round,line join=round,baseline=(B),alcolor] {
    \draw (-0.9\ttquantwd,\ttquantht) -- (0,\ttquantht) -- node[coordinate,pos=0.5] (mmm){} (0,0) --  (-0.9\ttquantwd,0);
    \draw (mmm) -- ++(-0.75\ttquantwd,0);
    \coordinate (B) at (0,-0.35pt);
  }\ensuremath{\,}%
}
\let\plexists=\plexist
\newcommand{\NT}[1]{\ensuremath{\langle\mathrm{#1} \rangle}}

\newcommand{\CPL}{\text{\itshape Const}_{PL}}
\newcommand{\FPL}{\text{\itshape Fun}_{PL}}
\newcommand{\RPL}{\text{\itshape Rel}_{PL}}
\newcommand{\VPL}{\text{\itshape Var}_{PL}}
\newcommand{\ATer}{A_{\text{\itshape Ter}}}
\newcommand{\ARel}{A_{\text{\itshape Rel}}}
\newcommand{\AFor}{A_{\text{\itshape For}}}
\newcommand{\LTer}{L_{\text{\itshape Ter}}}
\newcommand{\LRel}{L_{\text{\itshape Rel}}}
\newcommand{\LFor}{L_{\text{\itshape For}}}
\newcommand{\NTer}{N_{\text{\itshape Ter}}}
\newcommand{\NRel}{N_{\text{\itshape Rel}}}
\newcommand{\NFor}{N_{\text{\itshape For}}}
\newcommand{\PTer}{P_{\text{\itshape Ter}}}
\newcommand{\PRel}{P_{\text{\itshape Rel}}}
\newcommand{\PFor}{P_{\text{\itshape For}}}

\newcommand{\plka}{\alka}
\newcommand{\plkz}{\alkz}
%\newcommand{\plka}{\plfoo{(}}
%\newcommand{\plkz}{\plfoo{)}}
\newcommand{\plcomma}{\hbox{\texttt{\color{alcolor},}}}
\newcommand{\pleq}{{\color{alcolor}\,\dot=\,}}

% MODIFIED (DJ)
% previously: \newcommand{\plfoo}[1]{\mathtt{\color{alcolor}#1}}
\newcommand{\plfoo}[1]{\texttt{\color{alcolor}#1}}

\newcommand{\plc}{\plfoo{c}}
\newcommand{\pld}{\plfoo{d}}
\newcommand{\plf}{\plfoo{f}}
\newcommand{\plg}{\plfoo{g}}
\newcommand{\plh}{\plfoo{h}}
\newcommand{\plx}{\plfoo{x}}
\newcommand{\ply}{\plfoo{y}}
\newcommand{\plz}{\plfoo{z}}
\newcommand{\plR}{\plfoo{R}}
\newcommand{\plS}{\plfoo{S}}

\newcommand{\bv}{\mathrm{bv}}
\newcommand{\fv}{\mathrm{fv}}

%\newcommand{\AxAL}{\hbox{\textit{Ax}}_{AL}}
%\newcommand{\AALV}{\hbox{\textit{Var}}_{AL}}

%\renewcommand{\#}[1]{\literal{#1}}
\newcommand{\A}{\mathcal{A}}
\newcommand{\Adr}{\text{Adr}}
\newcommand{\ar}{\mathrm{ar}}
\newcommand{\ascii}[1]{\literal{\char#1}}
%\newcommand{\assert}[1]{\text{/\!\!/\ } #1}
\newcommand{\assert}[1]{\colorbox{black!7!white}{\ensuremath{\{\;#1\;\}}}}
\newcommand{\Assert}[1]{$\langle$\textit{#1}$\rangle$}
\newcommand{\B}{\mathcal{B}}
\newcommand{\bfmod}{\mathbin{\kw{ mod }}}
\newcommand{\bb}{{\text{bb}}}
\def\bottom{\hbox{\small$\pmb{\bot}$}}
\newcommand{\card}[1]{|#1|}
%\newcommand{\cod}{\mathop{\text{cod}}}  % ist in thwmathabbrevs
\newcommand{\Conf}{\mathcal{C}}
\newcommand{\define}[1]{\emph{#1}}
%\renewcommand{\dh}{d.\,h.\@\xspace}
%\newcommand{\Dh}{D.\,h.\@\xspace}
%\newcommand{\engl}[1]{engl.\xspace\emph{#1}}
\newcommand{\eps}{\varepsilon}
%\newcommand{\evtl}{evtl.\@\xspace}
\newcommand{\fbin}{\text{bin}}
\newcommand{\finv}{\text{inv}}
\newcommand{\fnum}{\text{num}}
\newcommand{\fNum}{{\text{Num}}}
\newcommand{\frepr}{\text{repr}}
\newcommand{\fRepr}{\text{Repr}}
\newcommand{\fZkpl}{\text{Zkpl}}
\newcommand{\fLen}{\text{Len}}
\newcommand{\fsem}{\text{sem}}
\providecommand{\fspace}{\mathord{\text{space}}}
\providecommand{\fSpace}{\mathord{\text{Space}}}
\providecommand{\ftime}{\mathord{\text{time}}}
\providecommand{\fTime}{\mathord{\text{Time}}}
\newcommand{\fTrans}{\text{Trans}}
\newcommand{\fVal}{\text{Val}}

% MODIFIED (DJ)
\newcommand{\Val}{\text{Val}}

%\def\G{\mathbb{Z}}
\newcommand{\HT}[1]{\normalfont\textsc{HT-#1}}
\newcommand{\htr}[3]{\{#1\}\;#2\; \{#3\}}
\newcommand{\Id}{\text{I}}
%\newcommand{\ie}{i.\,e.\@\xspace}
\newcommand{\instr}[2]{\texttt{#1}\ \textit{#2}}
\newcommand{\Instr}[2]{\texttt{#1}\ \textrm{#2}}
\newcommand{\instrr}[3]{\texttt{#1}\ \textit{#2}\texttt{(#3)}}
\newcommand{\Instrr}[3]{\texttt{#1}\ \textrm{#2}\texttt{(#3)}}
\newcommand{\io}{\!\mid\!}
\usepackage{KITcolors}
\newcommand{\literal}[1]{\hbox{\textcolor{blue!95!white}{\textup{\texttt{\scalebox{1.11}{#1}}}}}}
%\newcommand{\literal}[1]{\hbox{\textcolor{KITblue!80!black}{\textup{\texttt{#1}}}}}
\def\kasten#1{\leavevmode\literal{\setlength{\fboxsep}{1pt}\fbox{\vrule  width 0pt height 1.5ex depth 0.5ex #1}}}
\newcommand{\kw}[1]{\ensuremath{\mathbf{#1}}}
\newcommand{\lang}[1]{\ensuremath{\langle#1\rangle}}
%\newcommand{\maw}{m.\,a.\,w.\@\xspace}
%\newcommand{\MaW}{M.\,a.\,w.\@\xspace}
\newcommand{\mdefine}[2][FOOBAR]{\define{#2}\def\foobar{FOOBAR}\def\optarg{#1}\ifx\foobar\optarg\def\optarg{#2}\fi\graffito{\optarg}}
\newcommand{\meins}{\rotatebox[origin=c]{180}{1}}
\newcommand{\Mem}{\text{Mem}}
\newcommand{\memread}{\text{memread}}
\newcommand{\memwrite}{\text{memwrite}}
\providecommand{\meta}[1]{\ensuremath{\langle}\textit{#1}\ensuremath{\rangle}}
%\newcommand{\N}{\mathbb{N}}
\newcommand{\NP}{\mathbf{NP}}
\newcommand{\Nadd}{N_{\text{add}}}
\newcommand{\Nmult}{N_{\text{mult}}}
% MODIFIED (DJ): added \!, mathcal{O}
\newcommand{\Oh}[1]{\mathcal{O}\!\left(#1\right)}
\newcommand{\Om}[1]{\Omega\!\left(#1\right)}
\newcommand{\personname}[1]{\textsc{#1}}
\newcommand{\regname}[1]{\texttt{#1}}
\newcommand{\mima}{\textsc{Mima}\xspace}
\newcommand{\mimax}{\textsc{Mima-X}\xspace}

\def\Pclass{\text{\bfseries P}}
\def\PSPACE{\text{\bfseries PSPACE}}

\newcommand{\SPush}{\text{push}}
\newcommand{\SPop}{\text{pop}}
\newcommand{\SPeek}{\text{peek}}
\newcommand{\STop}{\text{top}}
\newcommand{\STos}{\text{\itshape tos}}
\newcommand{\SBos}{\text{\itshape bos}}

%\newcommand{\R}{\mathbb{R}}
\newcommand{\Rnullplus}{\R_0^{+}}
\newcommand{\Rplus}{\R_{+}}
\newcommand{\resp}{resp.\@\xspace}
\newcommand{\Sem}{\text{Sem}}
\newcommand{\sgn}{\mathop{\text{sgn}}}
\newcommand{\sqbox}{\mathop{\raisebox{-6.2pt}{\hbox{\hbox to 0pt{$^{^{\sqcap}}$\hss}$^{^{\sqcup}}$}}}}
\newcommand{\sqleq}{\sqsubseteq}
\newcommand{\sqgeq}{\sqsupseteq}
% MODIFIED (DJ): added \!
\newcommand{\Th}[1]{\Theta\!\left(#1\right)}
%\newcommand{\usw}{usw.\@\xspace}
\newcommand{\V}[1]{\hbox{\textit{#1}}}
\newcommand{\x}{\times}
\newcommand{\ZK}{\mathbb{K}}
%\newcommand{\Z}{\mathbb{Z}}
\newcommand{\zB}{z.\,B.\@\xspace}
\newcommand{\ZB}{Z.\,B.\@\xspace}
% \newcommand{\bb}{{\text{bb}}}
% \def\##1{\hbox{\textcolor{darkblue}{\texttt{#1}}}}
% \def\A{\mathcal{A}}
% \newcommand{\0}{\#0}
% \newcommand{\1}{\#1}
% \newcommand{\Obj}{\text{Obj}}
% \newcommand{\start}{\mathop{\text{start}}}
% \newcommand{\compactlist}{\addtolength{\itemsep}{-\parskip}}
% \newcommand{\fval}{\text{val}}
% \newcommand{\lang}[1]{\ensuremath{\langle#1\rangle}}
% \newcommand{\io}{\!\mid\!}
% \def\sqbox{\mathop{\raisebox{-6.2pt}{\hbox{\hbox to 0pt{$^{^{\sqcap}}$\hss}$^{^{\sqcup}}$}}}}
% \def\sqleq{\sqsubseteq}
% \def\sqgeq{\sqsupseteq}
\def\Td{T_{\overline{d}}}
% \newcommand{\csym}[1]{\ensuremath{\#{c}_{\#{\hbox{\scriptsize #1}}}}}
% \newcommand{\F}{\ensuremath{\mathcal{F}}}
% \newcommand{\fsym}[2]{\ensuremath{\#{f}^{\#{\hbox{\scriptsize #1}}}_{\#{\hbox{\scriptsize #2}}}}}
% \newcommand{\rsym}[2]{\ensuremath{\#{R}^{\#{\hbox{\scriptsize #1}}}_{\#{\hbox{\scriptsize #2}}}}}
% \newcommand{\xsym}[1]{\ensuremath{\#{x}_{\#{\hbox{\scriptsize #1}}}}}
% \newcommand{\I}{\mathcal{I}}
% ********************************************************************

\usepackage[blue]{../framework/thwregex}
\usepackage{environ}
\usepackage{bm}
\usepackage{calc}
\usepackage{varwidth}
\usepackage{wasysym}
\usepackage{mathtools}


% Das ist der KIT-Stil
%\usepackage{../TutTexbib/beamerthemekit}
\usepackage[deutsch,titlepage0]{../framework/KIT/beamerthemeKITmod}
\TitleImage[width=\titleimagewd]{../figures/titlepage.jpg}
%\usetheme[deutsch,titlepage0]{KIT}

% Include PDFs
\usepackage{pdfpages}

% Libertine font (Original GBI font)
\usepackage{libertine}
%\renewcommand*\familydefault{\sfdefault}  %% Only if the base font of the document is to be sans serif

% Nicer math symbols
\usepackage{eulervm}
%\usepackage{mathpazo}
\renewcommand\ttdefault{cmtt} % Computer Modern typewriter font, see lecture slides.

\usepackage{csquotes}

%%%%%%

%% Schönere Schriften
\usepackage[TS1,T1]{fontenc}

%% Bibliothek für Graphiken
\usepackage{graphicx}

%% der wird sowieso in jeder Datei gesetzt
\graphicspath{{../figures/}}

%% Anzeigetiefe für Inhaltsverzeichnis: 1 Stufe
\setcounter{tocdepth}{1}

%% Hyperlinks
\usepackage{hyperref}
% I don't know why, but this works and only includes sections and NOT subsections in the pdf-bookmarks.
\hypersetup{bookmarksdepth=subsection} 

%\usepackage{lmodern}
\usepackage{colortbl}
\usepackage[absolute,overlay]{textpos}
\usepackage{listings}
\usepackage{forloop}
%\usepackage{algorithmic} % PseudoCode package 

\usepackage{tikz}
\usetikzlibrary{matrix}
\usetikzlibrary{arrows.meta}
\usetikzlibrary{automata}
\usetikzlibrary{tikzmark}

% Needed for gbi-macros
\usepackage{xspace}

%%%%%%

%% Verbatim
\usepackage{moreverb}

%%%%%%%%%%%%%%%%%%%%%%%%%%%%%%%%%%%% Copy end

%% Tabellen
\usepackage{array}
\usepackage{multicol}
\usepackage{hhline}

%% Bibliotheken für viele mathematische Symbole
\usepackage{amsmath, amsfonts, amssymb}

%% Deutsche Silbentrennung und Beschriftungen
\usepackage[ngerman]{babel}

\usepackage{kbordermatrix}

% kbordermatrix settings
\renewcommand{\kbldelim}{(} % Left delimiter
\renewcommand{\kbrdelim}{)} % Right delimiter

\input{../config.tex}



% define custom \handout command flag if handout mode is toggled  #DirtyAsHellButWell...
\only<beamer:0>{\def\handout{}} %beamer:0 == handout mode

\newcommand{\R}{\mathbb{R}}
\newcommand{\N}{\mathbb{N}}
\newcommand{\Z}{\mathbb{Z}}
\newcommand{\Q}{\mathbb{Q}}
\newcommand{\BB}{\mathbb{B}}
\newcommand{\C}{\mathbb{C}}
\newcommand{\K}{\mathbb{K}}
\newcommand{\G}{\mathbb{G}}
\newcommand{\nullel}{\mathcal{O}}
\newcommand{\einsel}{\mathds{1}}
\newcommand{\Pot}{\mathcal{P}}
\renewcommand{\O}{\text{O}}

\def\word#1{\hbox{\textcolor{blue}{\texttt{#1}}}}
\let\literal\word
\def\mword#1{\hbox{\textcolor{blue}{$\mathtt{#1}$}}}  % math word
\def\sp{\scalebox{1}[.5]{\textvisiblespace}}
\def\wordsp{\word{\sp}}

%\newcommand{\literal}[1]{\textcolor{blue}{\texttt{#1}}}
\newcommand{\realTilde}{\textasciitilde \ }
\newcommand{\setsize}[1]{\ensuremath{\left\lvert #1 \right\rvert}}
\newcommand{\size}[1]{\setsize{#1}}  % Shame on you, TeXStudio...
\newcommand{\set}[1]{\left\{#1\right\}}
\newcommand{\tuple}[1]{\left(#1\right)}
\newcommand{\normalvar}[1]{\text{$#1$}}

% Modified by DJ
\let\oldemptyset\emptyset
\let\emptyset\varnothing % proper emptyset

\newcommand{\boder}{\ensuremath{\mathbin{\textcolor{blue}{\vee}}}\xspace}
\newcommand{\bund}{\ensuremath{\mathbin{\textcolor{blue}{\wedge}}}\xspace}
\newcommand{\bimp}{\ensuremath{\mathrel{\textcolor{blue}{\to}}}\xspace}
\newcommand{\bgdw}{\ensuremath{\mathrel{\textcolor{blue}{\leftrightarrow}}}\xspace}
\newcommand{\bnot}{\ensuremath{\textcolor{blue}{\neg}}\xspace}
\newcommand{\bone}{\ensuremath{\textcolor{blue}{1}}\text{}}
\newcommand{\bzero}{\ensuremath{\textcolor{blue}{0}}\text{}}
\newcommand{\bleftBr}{\ensuremath{\textcolor{blue}{\texttt{(}}}\text{}}
\newcommand{\brightBr}{\ensuremath{\textcolor{blue}{\texttt{)}}}\text{}}

% Fix of \b... commands:

\renewcommand{\boder}{\alor}
\renewcommand{\bund}{\aland}
\renewcommand{\bimp}{\alimpl}
\renewcommand{\bgdw}{\aleqv}
\renewcommand{\bnot}{\alnot}
\renewcommand{\bleftBr}{\alka}
\renewcommand{\brightBr}{\alkz}
\newcommand{\alA}{\word A}
\newcommand{\alB}{\word B}
\newcommand{\alC}{\word C}

\newcommand{\plB}{\plfoo{B}}
\newcommand{\plE}{\plfoo{E}}

\newcommand{\summe}[2]{\sum\limits_{#1}^{#2}}
\newcommand{\limes}[1]{\lim\limits_{#1}}

%\newcommand{\numpp}{\advance \value{weeknum} by -2 \theweeknum \advance \value{weeknum} by 2}
%\newcommand{\nump}{\advance \value{weeknum} by -1 \theweeknum \advance \value{weeknum} by 1}

\newcommand{\mycomment}[1]{}
\newcommand{\Comment}[1]{}

%% DISCLAIMER START 
% It is INSANELY IMPORTANT NOT TO DO THIS OUTSIDE BEAMER CLASS! IN ARTCILE DOCUMENTS, THIS IS VERY LIKELY TO BUG AROUND!
\makeatletter%
\@ifclassloaded{beamer}%
{
	% TODO 
	% no time...
	% redefine section to ignore multiple \section calls with the same title
}%
{
	\errmessage{ERROR: section command redefinition outside of beamer class document! Please contact the author of this code.}
}%
\makeatother%
%% DISCLAIMER END

\newcounter{abc}
\newenvironment{alist}{
  \begin{list}{(\alph{abc})}{
      \usecounter{abc}\setlength{\leftmargin}{8mm}\setlength{\labelsep}{2mm}
    }
}{\end{list}}


\newcommand{\stdarraystretch}{1.20}
\renewcommand{\arraystretch}{\stdarraystretch}  % for proper row spacing in tables

\newcommand{\morescalingdelimiters}{   % for proper \left( \right) typography
	\delimitershortfall=-1pt  
	\delimiterfactor=1
}

\newcommand{\centered}[1]{\vspace{-\baselineskip}\begin{center}#1\end{center}\vspace{-\baselineskip}}

% for \implitem and \item[bla] stuff to look right:
\setbeamercolor*{itemize item}{fg=black}
\setbeamercolor*{itemize subitem}{fg=black}
\setbeamercolor*{itemize subsubitem}{fg=black}

\setbeamercolor*{description item}{fg=black}
\setbeamercolor*{description subitem}{fg=black}
\setbeamercolor*{description subsubitem}{fg=black}

\renewcommand{\qedsymbol}{\textcolor{black}{\openbox}}

\renewcommand{\mod}{\mathop{\textbf{mod}}}
\renewcommand{\div}{\mathop{\textbf{div}}}

\newcommand{\ceil}[1]{\left\lceil#1\right\rceil}
\newcommand{\floor}[1]{\left\lfloor#1\right\rfloor}
\newcommand{\abs}[1]{\left\lvert #1 \right\rvert}
\newcommand{\Matrix}[1]{\begin{pmatrix} #1 \end{pmatrix}}
\newcommand{\braced}[1]{\left\lbrace #1 \right\rbrace}

% "something" placeholder. Useful for repairing spacing of operator sections, like `\sth = 42`.
\def\sth{\vphantom{.}}

\def\fract#1/#2 {\frac{#1}{#2}} % ! Trailing space is crucial!
\def\dfract#1/#2 {\dfrac{#1}{#2}} % ! Trailing space is crucial!

\newcommand{\Mid}{\;\middle|\;}

\let\after\circ



\def\·{\cdot}
\def\*{\cdot}
\def\?>{\ensuremath{\rightsquigarrow}}  % Fuck you, Latex
\def\~~>{\ensuremath{\rightsquigarrow}}  

\newcommand{\tight}[1]{{\renewcommand{\arraystretch}{0.76} #1}}
\newcommand{\stackedtight}[1]{\renewcommand{\arraystretch}{0.76} \begin{matrix} #1 \end{matrix} }
\newcommand{\stacked}[1]{\begin{matrix} #1 \end{matrix} }
\newcommand{\casesl}[1]{\delimitershortfall=0pt  \left\lbrace\hspace{-.3\baselineskip}\begin{array}{ll} #1 \end{array}\right.}
\newcommand{\casesr}[1]{\delimitershortfall=0pt  \left.\begin{array}{ll} #1 \end{array}\hspace{-.3\baselineskip}\right\rbrace}
\newcommand{\caseslr}[1]{\delimitershortfall=0pt  \left\lbrace\hspace{-.3\baselineskip}\begin{array}{ll} #1 \end{array}\hspace{-.3\baselineskip}\right\rbrace}

\def\q#1uad{\ifnum#1=0\relax\else\quad\q{\the\numexpr#1-1\relax}uad\fi}
% e.g. \q1uad = \quad, \q2uad = \qquad etc.

\newcommand{\qqquad}{\q3uad}
\newcommand{\minusquad}{\hspace{-1em}}

%% Placeholder utils
% \§{#1}   Saves #1 as placeholder and prints it
% \.       Prints an \hphantom with the size of the recalled placeholder.
\def\indentstring{}
\def\§#1{\def\indentstring{#1}#1}
\def\.{{$\hphantom{\text{\indentstring}}$}}
%% Placeholder utils end

\newcommand{\impl}{\ifmmode\ensuremath{\mskip\thinmuskip\Rightarrow\mskip\thinmuskip}\else$\Rightarrow$\fi\xspace}
\newcommand{\Impl}{\ifmmode\implies\else$\Longrightarrow$\fi\xspace}

\newcommand{\derives}{\Rightarrow}

\newcommand{\gdw}{\ifmmode\mskip\thickmuskip\Leftrightarrow\mskip\thickmuskip\else$\Leftrightarrow$\fi\xspace}
\newcommand{\Gdw}{\ifmmode\iff\else$\Longleftrightarrow$\fi\xspace}

% Legacy code from the algo tutorial slides. Perhaps useful. Try with care.
\mycomment{
	\newcommand{\impl}{\ifmmode\ensuremath{\mskip\thinmuskip\Rightarrow\mskip\thinmuskip}\else$\Rightarrow$\xspace\fi}  
	\newcommand{\Impl}{\ifmmode\implies\else$\Longrightarrow$\xspace\fi}
	
	\newcommand{\gdw}{\ifmmode\mskip\thickmuskip\Leftrightarrow\mskip\thickmuskip\else$\Leftrightarrow$\xspace\fi}
	\newcommand{\Gdw}{\ifmmode\iff\else$\Longleftrightarrow$\xspace\fi}
}
	
\newcommand{\gdwdef}{\ifmmode\mskip\thickmuskip:\Leftrightarrow\mskip\thickmuskip\else:$\Leftrightarrow$\xspace\fi}
\newcommand{\Gdwdef}{\ifmmode\mskip\thickmuskip:\Longleftrightarrow\mskip\thickmuskip\else:$\Longleftrightarrow$\xspace\fi}

\newcommand{\symbitemnegoffset}{\hspace{-.5\baselineskip}}
\newcommand{\implitem}{\item[\impl\symbitemnegoffset]}
\newcommand{\Implitem}{\item[\Impl\symbitemnegoffset]}


\newcommand{\forcenewline}{\mbox{}\\}

\newcommand{\bfalert}[1]{\textbf{\alert{#1}}}
\let\elem\in   % I'm a Haskell freak. Don't judge me. :P


\def\|#1|{\text{\normalfont #1}}  % | steht für senkrecht (anstatt kursiv wie sonst im math mode)


% proper math typography
\newcommand{\functionto}{\longrightarrow}
\renewcommand{\geq}{\geqslant}
\renewcommand{\leq}{\leqslant}
\let\oldsubset\subset
\renewcommand{\subset}{\subseteq} % for all idiots out there using subset

\newenvironment{threealign}{%
	\[
	\begin{array}{r@{\ }c@{\ }l}
}{%
	\end{array}	
	\]
}

\newcommand{\concludes}{ \\ \hline  }
\newcommand{\deduction}[1]{
	\begin{varwidth}{.8\linewidth}
		\begin{tabular}{>{$}c<{$}}
			#1
		\end{tabular}
	\end{varwidth}	
}

\definecolor{hoareorange}{rgb}{1,.85,.6}
\newcommand{\hoareassert}[1]{\setlength{\fboxsep}{1pt}\setlength{\fboxrule}{-1.4pt}\fcolorbox{white}{hoareorange}{\ensuremath{\{\;#1\;\}}}\setlength\fboxrule{\defaultfboxrule}\setlength{\fboxsep}{3pt}}

\newcommand{\mailto}[1]{\href{mailto:#1}{{\textcolor{blue}{\underline{#1}}}}}
\newcommand{\urlnamed}[2]{\href{#2}{\textcolor{blue}{\underline{#1}}}}
\renewcommand{\url}[1]{\urlnamed{#1}{#1}}

\newcommand{\hanging}{\hangindent=0.7cm}
\newcommand{\indented}{\hanging}


% \hstretchto prints #2 left-aligned into a box of the width of #1
\def\hstretchto#1#2{%
	\mbox{}\vphantom{#2}\rlap{#2}\hphantom{#1}%
}

\def\vstretchto#1#2{%
	\mbox{}\hphantom{#2}\smash{#2}\vphantom{#1}%
}


%requires \thisyear to be defined (s. config.tex)!
\edef\nextyear{\the\numexpr\thisyear+1\relax}


% --- \frameheight constant ---
\newlength\fullframeheight
\newlength\framewithtitleheight
\setlength\fullframeheight{.92\textheight}
\setlength\framewithtitleheight{.86\textheight}

\newlength\frameheight
\setlength\frameheight{\fullframeheight}

\let\frametitleentry\relax
\let\oldframetitle\frametitle
\def\newframetitle#1{\global\def\frametitleentry{#1}\if\relax\frametitleentry\relax\else\setlength\frameheight{\framewithtitleheight}\fi\oldframetitle{#1}}
\let\frametitle\newframetitle

\def\newframetitleoff{\let\frametitle\oldframetitle}
\def\newframetitleon{\let\frametitle\newframetitle}
% --- \frameheight constant end ---

\newcommand{\fakeframetitle}[1]{%
	\vspace{-2.05\baselineskip}%
	{\Large \textbf{#1}} \\%
	\smallskip
}



\newenvironment{headframe}{\Huge THIS IS AN ERROR. PLEASE CONTACT THE ADMIN OF THIS TEX CODE. (headframe env def failed)}{}
\RenewEnviron{headframe}[1][]{
	\begin{frame}\frametitle{\ }
		\centering
		\Huge\textbf{\textsc{\BODY} \\
		}
		\Large {#1}
		\frametitle{\ }
	\end{frame}
}


\makeatletter
% Provides color if undefined.
\newcommand{\colorprovide}[2]{%
	\@ifundefinedcolor{#1}{\colorlet{#1}{#2}}{}}
\makeatother


\colorprovide{lightred}{red!30}
\colorprovide{lightgreen}{green!40}
\colorprovide{lightyellow}{yellow!50}
\colorprovide{lightblue}{blue!10}
\colorprovide{beamerlightred}{lightred}
\colorprovide{beamerlightgreen}{lightgreen}
\colorprovide{beamerlightyellow}{lightyellow}
\colorprovide{beamerlightblue}{lightblue}
\colorprovide{fullred}{red!60}
\colorprovide{fullgreen}{green}
\definecolor{darkred}{RGB}{115,48,38}
\definecolor{darkgreen}{RGB}{48,115,38}
\definecolor{darkyellow}{RGB}{100,100,0}

\only<handout:0>{\colorlet{adaptinglightred}{beamerlightred}}
\only<handout:0>{\colorlet{adaptinglightgreen}{beamerlightgreen}}
\only<handout:0>{\colorlet{adaptinglightyellow}{beamerlightyellow}}
\only<handout:0>{\colorlet{adaptinglightblue}{beamerlightblue}}
\only<beamer:0>{\colorlet{adaptinglightred}{lightred}}
\only<beamer:0>{\colorlet{adaptinglightgreen}{lightgreen}}
\only<beamer:0>{\colorlet{adaptinglightyellow}{lightyellow}}
\only<beamer:0>{\colorlet{adaptinglightblue}{lightblue}}
\only<handout:0>{\colorlet{adaptingred}{lightred}}
\only<beamer:0>{\colorlet{adaptingred}{fullred}}
\only<handout:0>{\colorlet{adaptinggreen}{lightgreen}}
\only<beamer:0>{\colorlet{adaptinggreen}{fullgreen}}



\newcommand{\TrueQuestion}[1]{
	\TrueQuestionE{#1}{}
}

\newcommand{\YesQuestion}[1]{
	\YesQuestionE{#1}{}
}

\newcommand{\FalseQuestion}[1]{
	\FalseQuestionE{#1}{}
}

\newcommand{\NoQuestion}[1]{
	\NoQuestionE{#1}{}
}

\newcommand{\DependsQuestion}[1]{
	\DependsQuestionE{#1}{}
}

\newcommand{\QuestionVspace}{\vspace{4pt}}
\newcommand{\QuestionParbox}[1]{\begin{varwidth}{.85\linewidth}#1\end{varwidth}}
\newcommand{\ExplanationParbox}[1]{\begin{varwidth}{.97\linewidth}#1\end{varwidth}}
\colorlet{questionlightgray}{gray!23}
\let\defaultfboxrule\fboxrule

% #1: bg color
% #2: fg color short answer
% #3: short answer text
% #4: question
% #5: explanation
\newcommand{\GenericQuestion}[5]{
	\setlength\fboxrule{2pt}
	\only<+|handout:0>{\hspace{-2pt}\fcolorbox{white}{questionlightgray}{\QuestionParbox{#4} \quad\textbf{?}}}
	\visible<+->{\hspace{-2pt}\fcolorbox{white}{#1}{\QuestionParbox{#4} \quad\textbf{\textcolor{#2}{#3}}} \if\relax#5\relax\else\ExplanationParbox{#5}\fi} \\
	\setlength\fboxrule{\defaultfboxrule}
}

% #1: Q text
% #2: Explanation
\newcommand{\TrueQuestionE}[2]{
	\GenericQuestion{adaptinglightgreen}{darkgreen}{Wahr.}{#1}{#2}
}

% #1: Q text
% #2: Explanation
\newcommand{\YesQuestionE}[2]{
	\GenericQuestion{adaptinglightgreen}{darkgreen}{Ja.}{#1}{#2}
}

% #1: Q text
% #2: Explanation
\newcommand{\FalseQuestionE}[2]{
	\GenericQuestion{adaptinglightred}{darkred}{Falsch.}{#1}{#2}
}

% #1: Q text
% #2: Explanation
\newcommand{\NoQuestionE}[2]{
	\GenericQuestion{adaptinglightred}{darkred}{Nein.}{#1}{#2}
}

% #1: Q text
% #2: Explanation
\newcommand{\DependsQuestionE}[2]{
	\GenericQuestion{adaptinglightyellow}{darkyellow}{Je nachdem!}{#1}{#2}
}

% #1: Q text
% #2: Answer
\newcommand{\ContentQuestion}[2]{
	\GenericQuestion{adaptinglightblue}{black}{\minusquad}{#1}{#2}
}

\ifnum\thisyear=2018 \else \errmessage{Old ILIAS link inside preamble. Please update.} \fi

\newcommand{\ILIAS}{\urlnamed{ILIAS}{https://ilias.studium.kit.edu/ilias.php?ref\_id=855240\&cmdClass=ilrepositorygui\&cmdNode=5r\&baseClass=ilrepositorygui}\xspace}

\newcommand{\Socrative}{\ifdefined\mysocrativeroom \only<handout:0>{socrative.com $\quad \~~> \quad $ Student login \\ Raumname:  \mysocrativeroom\\ \medskip}\else\fi}

\newcommand{\thasse}[1]{
	\ifdefined\ThassesTut #1\xspace \else\fi
}
\newcommand{\daniel}[1]{
	\ifdefined\DanielsTut #1\xspace \else\fi
}
\newcommand{\thassedaniel}[2]{\ifdefined\ThassesTut #1\else\ifdefined\DanielsTut #2\fi\fi\xspace}

\ifdefined\ThassesTut \ifdefined\DanielsTut \errmessage{ERROR: Both ThassesTut and DanielsTut flags are set. This is most likely an error. Please check your config.tex file.} \else \fi \else \ifdefined\DanielsTut \else \errmessage{ERROR: Neither ThassesTut  nor DanielsTut flags are set. This is most likely an error. Please check your config.tex file.} \fi\fi

%\newcommand{\sgn}{\text{sgn}}

%%%%%%%%%%%% INHALT %%%%%%%%%%%%%%%%

%% Wochennummer
\newcounter{weeknum}

%% Titelinformationen
\title[GBI-Tutorium \mytutnumber, Woche \theweeknum]{Grundbegriffe der Informatik \\ Tutorium \mytutnumber}

\subtitle{Woche \theweeknum \ | \mydate{\theweeknum} \\ \myname \ \  \normalfont (\mailto{\mymail})}
\author[\myname]{\myname}
\institute{KIT -- Karlsruher Institut für Technologie}
\date{\mydate{\theweeknum}\ }

% Modified, DJ (better safe than sorry)
\AuthorTitleSep{ – }

%% Titel einfügen
\newcommand{\titleframe}{\frame{\titlepage}}

%% Alles starten mit \starttut{X}
\newcommand{\starttut}[1]{\setcounter{weeknum}{#1}\pdfinfo{
		/Author (\myname)
		/Title  (GBI-Tutorium \mytutnumber, Woche \theweeknum)
	}\titleframe\frame{\frametitle{Inhalt}\tableofcontents} \AtBeginSection[]{%
		\begin{frame}{Wo sind wir gerade?}
		\tableofcontents[currentsection]
	\end{frame}\addtocounter{framenumber}{-1}}}


\newcommand{\framePrevEpisode}{
\begin{headframe}
	\mylasttimestext
\end{headframe}
}

\newcommand{\lastframetitled}[6]{
	\frame{\frametitle{#6}
		\vspace{-#2\baselineskip}
		\begin{figure}[H]
			\centering
			\LARGE \textbf{\textsc{#5}} \\
			\vspace{.2\baselineskip}
			\includegraphics[#1]{#3}
			\vspace{-6pt}
			\begin{center}
				\small \url{#4} 
			\end{center}
		\end{figure} 
	}
}

% #1 number
% #2 title 
% #3 vspace (positive) without unit (\baselineskip)
\newcommand{\xkcdframe}[3]{
	\lastframetitled{width=.96\textwidth}{#3}{xkcd/#1}{http://xkcd.com/#1}{}{#2}
}

\newcommand{\xkcdframevert}[3]
{
	\lastframetitled{height=.96\frameheight}{#3}{xkcd/#1}{http://xkcd.com/#1}{}{#2}
}

% #1 number
% #2 title 
% #3 vspace (positive) without unit (\baselineskip)
% #4 \includegraphics[] optional parameters
\newcommand{\xkcdframecustom}[4]
{
	\lastframetitled{#4}{#3}{xkcd/#1}{http://xkcd.com/#1}{}{#2}
}

\newcommand{\slideThanks}{
	\begin{frame}
	\frametitle{Credits}
	\begin{block}{}
		An der Erstellung des Foliensatzes haben mitgewirkt:\\[1em]
		Daniel Jungkind \\
		Thassilo Helmold \\
		Philipp Basler \\
		Nils Braun \\
		Dominik Doerner \\
		Ou Yue \\
	\end{block}
\end{frame}
}

%% Wörter DEPRECATED! DO NOT USE
\newcommand{\code}[1]{$\mathbf{#1}$}

\morescalingdelimiters

\begin{document}
\starttut{7}

% Was hat das mit MIMA zu tun? => Vllt. ans Ende?
% \lastframe{0.40}{25}{xkcd/password_strength.png}{https://www.xkcd.com/936/}

\mycomment{
	\begin{frame}{Zu Blatt \#2}
		Durchschnitt: etwa \thassedaniel{60}{61} \% der Punkte 
		\begin{itemize}		
			\item Vollständige Induktion: Eine Standard-IV lautet \\
			{\enquote{für \textbf{\alert{ein}} $n$ gilt...}}.\\
			Für alle $n$ wollt ihr's ja \textbf{zeigen}!
			\item \textbf{A~2.2}: IV mit „$K=L$“ funktioniert nicht: \\ 
			Das ist ne $\forall$-Aussage wegen der $\bigcup\limits^\infty_{n=0}$-Definition \\
			\impl Besser $K_n = L_n$ zeigen
			\item \textbf{A~2.3}: hat 11 statt 10 Gesamtpunkte (Zenkel hat sich verrechnet...~:P) \\
			\impl Nicht wundern, wenn ihr einen Punkt mehr habt, als ihr solltet
			\item \textbf{A~2.3b}: IA für $n=0 \textbf{ und } 1$, \; IS: $n \?> n+\mathbf{2}$ \quad --- alles andere geht nicht
			\item \textbf{A~2.6}: \emph{Surjektivität} per Induktion geht NICHT!
			\item \textbf{A~2.7}: \emph{Induktive} Definitionen: KEINE Set comprehensions mit Wortlängen-Magie! (Sonst könnt ihr direkt die Def. abschreiben)
		\end{itemize}
		
	\end{frame}
}


\framePrevEpisode

\begin{frame}[t]{Wahr oder falsch?}
	\Socrative
	\FalseQuestionE{$\memwrite(m,a,v) \in \Val$.}{$\memwrite(m,a,v) \in \Val^\Adr$.}
	\FalseQuestionE{$\memread(\memwrite(m,a,v), a) = m(a)$ \quad für alle $m,a,v$.}{$\memread(\memwrite(m,a,v), a) = v$.}
	\FalseQuestionE{Im IR wird die Adresse des aktuellen Befehls gespeichert.}{Das geschieht im IAR – Im IR steht der Befehl selbst.}
	%\TrueQuestionE{Die Befehlsholphase ist jede Ausführungsrunde identisch.}{}
	\TrueQuestionE{Ein MIMA-Programm, welches von $-1$ abwärts zählt, kommt \\ irgendwann bei null raus.}{Unterlauf: Nach $-2^{23}$ kommt $2^{23}-1$.}
\end{frame}

\begin{frame}[t]{Wahr oder falsch?}
	\FalseQuestionE{Der Akku führt bei der MIMA Berechnungen aus.}{Das macht die ALU. Im Akku wird das letzte Ergebnis zwischengespeichert.}
	\FalseQuestionE{LDC $-5$ lädt $-5$ in den Akku.}{LDC funktioniert nicht mit negativen Konstanten!}
	\FalseQuestionE{Mit der MIMA können wir Zufallszahlen erzeugen.}{Die MIMA arbeitet rein deterministisch und damit ohne jeden Zufall. 
		\vspace{-.8\baselineskip}
		\begin{figure}[H]
			\centering
			\includegraphics[scale=0.6]{xkcd/random_number} \\
			%\vspace{-7pt}
			{\url{http://xkcd.com/221}}
		\end{figure}
	}
\end{frame}

%\section{MIMA}

\begin{frame}{MIMA}
	\includegraphics[width=\linewidth]{MIMA_simple.png}\\
	Die \emph{MIMA} ist ein idealisierter Prozessor. 
\end{frame}

\begin{frame}{Eigenschaften}
	\begin{itemize}[<+->]
		\item Adressen sind 20 Bit lang
		\item \enquote{Werte} sind 24 Bit lang
		\item Befehlscodierungen -- zwei Formate:
		\begin{itemize}
			\item[a)] 4 Bit für den OpCode und 20 Bit für einen Parameter (Adresse / Konstante)
			\item[b)] 8 Bit Befehl (Rest irrelevant)\\
			\item[]<+(-2)-> \includegraphics[width=100px]{MIMA_commands.png} 
		\end{itemize} 
	\end{itemize}
\end{frame}


\begin{frame}{Wichtige Register}
	\begin{itemize}[<+->]
		\item \emph{IAR} : InstruktionsAdressRegister : Speichert Adresse des aktuell auszuführenden Befehls.
		\item \emph{IR}: InstruktionsRegister : Speichert den auszuführenden Befehl selbst.
		\item \emph{SAR}: SpeicherAdressRegister : Enthält die Adresse eines Wertes, der aus dem Speicher gelesen werden soll.
		\item \emph{SDR}: SpeicherDatenRegister : Enthält einen Wert, der aus dem Speicher geladen wurde.
		\item \emph{Akku}: enthält Ausgangswerte/Ergebnisse von Berechnungen
	\end{itemize}
\end{frame}

\begin{frame}[t]{Aufbau}
	\begin{figure}
		\centering
		\includegraphics[width=\linewidth]{MIMA.png}
	\end{figure}
\end{frame}

\begin{frame}[t]{Befehlsholphase}
	\begin{figure}
		\centering
		\includegraphics[width=\linewidth]{MIMA.png}
	\end{figure}
	\pause
	\only<2-6>{\begin{itemize}
		\only<2|handout:1> {\item[1.] IAR $\to $ SAR  und IAR $\to$ X 
			\item[] Befehlsadresse dem Speicher übergeben und Zähler zum Erhöhen an ALU geben.}
		\only<3|handout:2>{\item[2.] Eins $\to$ Y  
			\item[] 1-Wert für Erhöhung des Zählers an ALU geben.}
		\only<4|handout:3>{\item[3.] ALU aufaddieren ($Z=X+Y$) 
			\item[] Nächste Befehlsadresse berechnen. }
		\only<5|handout:4>{\item[4.] Z $\to$ IAR 
			\item[] Adresse für nächste Runde speichern. }
		\only<6|handout:5>{\item[5.] SDR $\to$ IR 
			\item[] Wert zur angefragten Adresse erhalten. }
	\end{itemize}}	
\end{frame}

\newcommand{\itemizeconfig}{\setlength{\parsep}{0pt}\setlength{\parskip}{0pt}\setlength{\topsep}{0pt}\setlength{\partopsep}{0pt}}
\newcommand{\explain}[1]{\hfill {\small #1 }}
\begin{frame}{Befehle}
	Die MIMA besitzt einen Befehlssatz mit möglichen Befehlen. %Andere Befehle (oder Varianten) werden NICHT unterstützt und können daher nicht verwendet werden!
	\textbf{Nur die Befehle aus der VL} dürft ihr benutzen! \\ 
	\smallskip
	(Befehle, die eigentlich zwei Operanden brauchen, lesen zusätzl. vom Akku.)
	{
	\begin{itemize}[<+->] \itemizeconfig
		\item Rechenoperationen:
		\begin{itemize} 
			\item ADD adr \explain{Akku $\leftarrow$ Akku + Mem(adr)}
			\item AND, OR, XOR adr \explain{Dasselbe mit bitweisem Und, Oder und eXclusive OR}
			\item NOT, RAR \explain{Akku $\leftarrow$ not(Akku); \quad  Rotate-Akku-Right();}
		\end{itemize}
		\item Datentransport:
		\begin{itemize}
			\item LDC const \explain{Akku $\leftarrow$ const\quad (const ist dabei eine \textbf{20-Bit}-Konstante) }
			\item LDV, STV adr \explain{(Lese/Schreibe in/vom Akku von/an Adresse)}
			\item LDIV, STIV adr \explain{(Lese/Schreibe von/an Adresse, die an der Adresse steht)}
		\end{itemize}
		\item Vergleichsoperation: EQL adr \explain{(liefert $-1$ wenn Akku $=$ Mem(adr), 0 sonst)}
		\item Sprünge:
		\begin{itemize}
			\item JMP adr \explain{Springe zu Befehl in Adresse adr}
			\item JMN adr \explain{Springe, falls Akku negativ \quad „JuMp if Negative“}
		\end{itemize}
		\item HALT \explain{Stoppt die MIMA}
	\end{itemize}}
\end{frame}

\begin{frame}{Bemerkungen}
	\begin{block}{Indirekte Adressierung (LDIV, STIV)}
		\centering
		\includegraphics[width=150px]{MIMA_indirect.png}
	\end{block}
	
	\pause
	\begin{block}{HALT}
		Jedes Programm muss mit HALT enden! Sonst läuft das Programm endlos weiter! (Kostet dann Punkte!)
	\end{block}

	\pause
	\begin{block}{Negative Konstanten}
		Negative Konstanten können \textbf{nicht} mit LDC geladen werden. Warum? \\
		\pause \impl Unser Akku ist 24 Bit breit, aber wir können nur in die \textbf{hinteren 20 Bit} laden!
	\end{block}
\end{frame}

\begin{frame}{Negative Zahlen}
	\begin{block}{Aufgabe}
		Schreibt ein Programm, das von einer an Adresse $a_1$ gegebenen positiven Zahl das Zweierkomplement berechnet und an Adresse $a_2$ ablegt.
	\end{block}

	\visible<2|handout:2>{
		\begin{block}{Lösung}
			LDV $a_1$\\
			NOT\\
			STV $a_2$\\
			LDC 1\\
			ADD $a_2$\\
			STV $a_2$
		\end{block}
	}
\end{frame}

\begin{frame}{Beispiele}
	Beispiele zur Umsetzung von Anweisungen aus Hochsprachen (if, while, for....):\\
	Siehe Übung WS 15/16
\end{frame}

\begin{frame}{Übung: Modulo 2 und Betrag}
	1. Schreibe ein Programm, das eine an Speicheradresse $a_1$ gegebenen Zahl Modulo 2 rechnet und an Adresse $a_2$ ablegt. \\
	\medskip
	2. Schreibe ein Programm, das den Betrag einer an Adresse $a_1$ gegebenen Zahl berechnet und an Adresse $a_2$ ablegt.
\end{frame}

\begin{frame}{Lösung: Modulo 2}
	\begin{tabbing}
		start: \; \= LDC 1 \quad  //$= 000000000000000000000001_2$ \\
				\> AND $a_1$ \\
				\> STV $a_2$ \\
				\> HALT
	\end{tabbing}
\end{frame}

\begin{frame}{Lösung: Betrag}
	\begin{tabbing}
		start: \quad \= LDV $a_1$ \\
					 \> JMN negate \\
		end: 		 \> STV $a_2$ \\
					 \> HALT \\
		negate:		 \> NOT \\
					 \> STV $a_2$ \\
					 \> LDC 1 \\
					 \> ADD $a_2$  \\
					 \> JMP end	 \\		 
	\end{tabbing}
\end{frame}

\begin{frame}{Übung: Modulo 3}
	Schreibe ein Programm, das eine an Speicheradresse $a_1$ gegebenen Zahl Modulo \textbf{3} (drei!) rechnet und an Adresse $a_2$ ablegt.
\end{frame}

\begin{frame}{Lösung: Modulo 3}
  \begin{tabbing}
    start: \quad \= LDC 1 \\
           \> STV One\\
           \> LDC 3\\
           \> NOT\\
           \> ADD One\\
           \> STV MinusThree\\
           \> LDV $a_1$\\
    \medskip
    while: \> ADD MinusThree\\
           \> JMN end\\
           \> JMP while\\
    \medskip
    end:   \> STV $a_1$\\
           \> LDC 3\\
           \> ADD $a_1$\\
           \> STV $a_2$ \\
           \> HALT\\
  \end{tabbing}
\end{frame}

\begin{frame}{Noch offen: Klammerausdrücke}
	A long, long time ago, in a land far away:\\
	\medskip
	Formale Sprachen angeben durch $\set{}, \*, {}^+, {}^*$...\\
	Was ist mit der \textbf{Sprache aller gültigen Klammerausdrücke}? Können wir die auch so angeben?\\[1em]
	\pause
	\impl Jetzt wissen wir: \textbf{Nein}, das geht nicht! (Siehe VL)\\[1em]
	
	\begin{figure}[H]
		\centering
		\includegraphics[scale=0.7]{xkcd/(.png}
		\vspace{-7pt}
		\caption{ \texttt{\url{https://xkcd.com/859/}} }
	\end{figure}
\end{frame}

\section{Kontextfreie Grammatiken}
\begin{frame}{Kontextfreie Grammatiken}
	
	\begin{Definition}
		Eine \textbf{kontextfreie Grammatik} ist ein 4-Tupel $G = (N, T, S ,P)$ mit
		\begin{itemize}
			\item[$N$] Alphabet von Nichtterminalsymbolen
			\item[$T$] Alphabet von Terminalsymbolen ($N \cap T = \emptyset$)
			\item[$S$] Startsymbol ($S \in N$)
			\item[$P$] Produktionsmenge ($P \subseteq N \times (N \cup T)^\ast$)
		\end{itemize}
	\end{Definition}

	\pause
	\begin{Beispiel}
		Sei $A$ das deutsche Alphabet (mit Klein-/Großbuchstaben).\\
		$G_{MI} := \left(\{S, M, I, N\}, \, A \cup \N_+, \, S, \, P\right)$ mit
		\begin{align*}
			P = \{S &\to \text{M\word\sp I\word\sp N}, \\
			M &\to \word{Monkey}, \\
			I &\to \word{Island}, \\
			N &\to \word 1 \mid \word 2 \mid \word 3 \}
		\end{align*}
	\end{Beispiel}
\end{frame}

\begin{frame}{Kontextfreie Grammatiken}
	\begin{block}{Produktionen}
		$=$ Menge von Ersetzungsregeln. \\
		Schema: \\
		\qqquad $\underbrace{L}_{\mathclap{\text{\textbf{genau ein} Nichtterminal}}} \to [\text{rechte Seite aus \word{Terminalen} und/oder Nichtterminalen}]$ \\
		\smallskip
		Heißt: Wir können in \textbf{einem} Ersetzungsschritt \textbf{genau ein} solches Nichtterminal $L$ mit der rechten Seite ersetzen. \textbf{Wenn wir wollen}. \\
		\medskip
		\pause
		Mehrere Möglichkeiten zur Auswahl: \\
		\qquad $L \to [\text{rechte Seite}]_1 \mid [\text{rechte Seite}]_2 \mid [\text{rechte Seite}]_3 \mid ...$ \\
		Heißt: $L$ kann durch die rechte Seite 1, 2 oder 3 ersetzt werden. \textbf{Wie wir wollen.} \\
		\pause
		\begin{Beispiel}
			$S \to \word aB\word x \mid \word{zz}$ \\
			\impl $S$ kann durch $\word aB\word x$ oder \word{zz} ersetzt werden.
		\end{Beispiel}
	\end{block}
\end{frame}

\begin{frame}{Ableitungen von Wörtern}
	\begin{Definition}
		Wort $u = .....X.....$ \textbf{ableitbar} nach $v = ..... w......$ \\
		wenn es eine Produktion $X \to w$ gibt. \; ($X$ ist Nichtterminal.) \\
		Schreibweise: \quad $u \derives v$ \quad oder \quad $u \derives^1 v$ \\
	\end{Definition}
	
	\medskip
	\textbf{\alert{Achtung}}: Aufpassen mit Pfeilen: \quad  $\derives$ vs. $\to$
	
	\pause
	\begin{Beispiel}
		Für $G_{MI}$ gilt: \\
		$S \derives M\word{\sp} I\word{\sp}N$\\
		$ M\word{\sp} I\word{\sp}N \derives \word{Monkey}\word{\sp} I\word{\sp} N \derives \word{Monkey}\word{\sp} I\word{\sp 3}$ \\
		$\word{Monkey}\word{\sp} I\word{\sp} N \derives \word{Monkey}\word{\sp} I\word{\sp 1} \derives \word{Monkey}\word{\sp}\word{Island}\word{\sp 1}$ \\
		Es gibt kein Wort, das aus \word{Monkey\sp Island\sp 1} abgeleitet werden kann.
	\end{Beispiel}
	
\end{frame}

\begin{frame}{Ableitung}	
	\begin{block}{Schreibweisen}
		$u \derives^k v$ \quad heißt: $u$ in $k$ Schritten ableitbar zu $v$ \\
		$u \derives^0 v$ \quad heißt also: in 0 Schritten ableitbar ($u=v$) \\
		\smallskip
		$u \derives^* v$ \quad heißt: $u$ in \emph{beliebig vielen} Schritten ableitbar zu $v$
	\end{block}
	
	\mycomment{
		\pause
		\begin{block}{Beobachtung}
			Die Definitionen stimmen mit den Potenzen der Relation $\derives$ überein.\\
			$\derives^\ast$ ist die reflexiv-transitive Hülle von $\derives$.
		\end{block}
	}
	
	\pause
	\begin{Beispiel}
		Für $G_{MI}$ gilt: $S \derives^3 \word{Monkey\sp Island\sp}N \derives^1 \word{Monkey\sp Island\sp3}$\\
		$S \derives^* \word{Monkey\sp Island\sp2}$
	\end{Beispiel}
\end{frame}


\begin{frame}{Erzeugte Sprache einer Grammatik}
	\begin{Definition}
		Sei $G = (N, T, S, P)$ eine kontextfreie Grammatik. Wir nennen die Sprache $$L(G) := \{w \in T^\ast \mid S \Rightarrow^\ast w \} \subseteq T^*$$ die von $G$ \textbf{erzeugte Sprache}.
	\end{Definition} \pause
	Das sind also alle Wörter aus \emph{Terminalsymbolen}, die vom Startsymbol aus ableitbar sind.\\
	\bigskip
	Achtung: Die erzeugte Sprache kann auch \textbf{leer} sein. \\
	\§{Beispiele:} \pause $L\left(\left(\{X\},\, \{\word a, \word b\},\, X\, ,\{X\to X\}\right)\right) = \emptyset$ \pause \quad oder \\
	\. $L\left(\left(\{X\},\, \{\word a, \word b\},\, X\, ,\emptyset \right)\right) = \emptyset$
\end{frame}

\begin{frame}{Erzeugte Sprache}
	\begin{Beispiel}
		$L(G_{MI}) = \{\word{Monkey\sp Island\sp1}, \word{Monkey\sp Island\sp 2}, \word{Monkey\sp Island\sp 3}\}$\\
		$M\wordsp I\wordsp N \notin L(G_{MI})$ \quad (enthält nämlich  Nichtterminale!)
	\end{Beispiel}
	
	\pause
	\begin{Definition}
		Eine Sprache $L$, für die irgendeine kontextfreie Grammatik G mit $L(G) = L$ existiert, heißt \textbf{kontextfrei}.
	\end{Definition}
	\medskip
	
	Viele Programmiersprachen sind kontextfrei. \\ 
	% Fun fact: Einrücksensitive Sprachen (wie Python – oder Haskell <3) sind dies streng genommen nicht! Das Lexing kann dann nicht von einem Regex-Automaten gemacht werden, der („übermächtige“ Stack-Automaten-)Lexer muss dann den Input vorverarbeiten, der daraus entstehende Tokenstream kann dann kontextfrei geparst werden.  Source: http://trevorjim.com/python-is-not-context-free/
	Eine vereinfachte Variante der englischen Sprache auch. \smiley
% Grob falsch:
%	Viele \enquote{natürlich vorkommende} Sprachen sind kontextfrei.
		
\end{frame}

\subsection{Beispiele}
\begin{frame}{Beispiel}
	$$ G = (\{X\}, \, \{\word a, \word  b\}, \, X, \, \{X \to \word aX\word b \mid \eps\}) $$
	\delimitershortfall=1pt
	\begin{itemize}
		\item Gilt $X \derives \word aX\word b$, \; $X \derives \word{aa}X\word{bb}$, \; $XX \derives \word{a}X\word{ba}X\word{b}$? \\
			  \visible<2-|handout:2->{Ja, Nein, Nein.}
		\item Welche Wörter über $\{\word a, \word  b\}$ lassen sich aus $\word{aa}X\word{bb}$ ableiten? Und aus $XX$? \\
			  \visible<3-|handout:2->{
				  Aus $\word{aa}X\word{bb}$: \quad $\set{\word a^k\word b^k \mid k \geq 2}$ \\
				  Aus $XX$: \quad $\set{\word a^k\word b^k \word a^\ell \word b^\ell \mid k,\ell \in \N_0}$
			  }
		\item Gebt $L(G)$ an! \\
		      \visible<4-|handout:2->{ $L(G) = \set{\word a^k\word b^k\mid k \in \N_0}$ }
	\end{itemize}
	
\end{frame}

\begin{frame}{Klammerausdrücke}
	Gegeben sei die Grammatik $$G = (\{X\}, \{\word{(}, \word)\}, X, \{X \to XX \mid \word(X\word) \mid \eps\})$$
	\begin{itemize}
		\item Wie leitet man \word{(())} ab?
		\item Wie leitet man \word{()()} ab?
		\item Kann man \word{(()(} ableiten? \pause \impl Nein!
		\item Und wie leitet man \word{(())()()} ab?\\ \pause
			$X \derives XX \derives XXX \derives \word(X\word)XX \derives \word(X\word)X\word(X\word) \derives \word(X\word)\word(X\word)\word(X\word)$ \\
			$\quad \derives \word(X\word{)()(}X\word) \derives \word{((}X\word{))()(}X\word) \derives \word{((}X\word{))()()} \derives \word{(())()()}$\\
			Geht das auch übersichtlicher? \impl Ableitungsbäume
	\end{itemize}
\end{frame}


\begin{frame}{Ableitungsbäume}
	\centering
	$G = (\{X\}, \{\word{(}, \word)\}, X, \{X \to XX \mid \word(X\word) \mid \eps\})$
	\medskip
	
	\begin{tikzpicture}
	[level 1/.style={sibling distance=50mm},
	level 2/.style={sibling distance=30mm},
	level 3/.style={sibling distance=10mm}]
	\node {$X$}
	child { node {$X$}
		[level 2/.style={sibling distance=15mm}]
		child {node {$\literal{(}$} }
		child {node {$X$} 
			child {node {$\literal{(}$} }
			child {node {$X$}
				child {node {$\varepsilon$} }
			}
			child {node {$\literal{)}$} }
		}
		child {node {$\literal{)}$} }
	} 
	child { node {$X$} 
		child { node {$X$} 
			child {node {$\literal{(}$} }
			child {node {$X$}
				child {node {$\varepsilon$} }
			}
			child {node {$\literal{)}$} }
		}
		child { node {$X$} 
			child {node {$\literal{(}$} }
			child {node {$X$}
				child {node {$\varepsilon$} }
			}
			child {node {$\literal{)}$} }
		}
	} ;
	\end{tikzpicture}
\end{frame}


\begin{frame}{Klammerausdrücke}
	Gegeben sei die Grammatik $$G = (\{X\}, \{\word{(}, \word)\}, X, \{X \to XX \mid \word(X\word) \mid \eps\})$$
	Was ist $L(G)$? Was kann man also aus $X$ ableiten?\\ \pause 
	Alle \enquote{\textit{wohlgeformten Klammerausdrücke}} \quad ($=$ alle, die Sinn machen).\\[1em]
	
	Was bedeutet wohlgeformt in diesem Kontext?\\ \pause
	$$\forall w \in L(G): N_{\word(}(w) = N_{\word)}(w)$$ 
	Reicht das? \pause \impl Notwendig, aber nicht hinreichend! 

\end{frame}

\begin{frame}{Klammerausdrücke}
	Wir dürfen eine Klammer erst schließen, \textit{nachdem} wir sie geöffnet haben.\\
	Also: Anzahl der schließenden Klammern darf nie größer als Anzahl der öffnenden Klammern sein! \pause \\[1em]
	\impl Für jedes Präfix $v$ von einem Wort $w \in L(G)$ gilt $$N_{\word(} (v) \geq N_{\word)} (v)$$ \pause
	
	\textbf{Achtung}: Grammatiken sind \textbf{nicht eindeutig}! Wir können zur gleichen Sprache mehrere verschiedene erzeugende Grammatiken finden. \\
	Alternative Grammatik für wohlgeformte Klammerausdrücke: \pause $$G = (\{X\}, \, \{\word (, \word )\}, \, X, \, \{X \to \word(X\word)X \mid \varepsilon\})$$
\end{frame}

\begin{frame}{Und jetzt ihr...}
	Gebt jeweils eine Grammatik über dem Alphabet $T = \{\word a, \word b\}$ an, die folgende Sprache erzeugt:
	\begin{itemize}
		\item Alle Wörter, in denen irgendwo das Teilwort $\word{baa}$ vorkommt.\\
		\visible<2-|handout:2>{
			$(\{X,Y\},T,X,P)$ mit $P=\{X \to Y\word{baa}Y, Y \to \word aY \mid \word bY \mid \varepsilon\}$
		}
		
		\item Alle Wörter, in denen $\word{ab}$ als Teilwort vorkommt oder kein $\word a$ enthalten ist. \\
		\visible<3-|handout:2>{
			$G = (\{X, Y\}, \{\word a, \word b\}, X, P)$ mit $P = \{X \to \word bX \mid Y\word{ab}Y \mid \varepsilon, Y \to \word aY \mid \word bY \mid \varepsilon\}$
		}
		
		\item Die Menge aller Wörter $w\in T^*$ mit der Eigenschaft, dass
		für alle Präfixe $v$ von $w$ gilt: $\setsize{N_{\word{a}}(v) - N_{\word{b}}(v)} \leq
		1$.\\
		\emph{Tipp}: Was für eine Struktur haben Wörter der Länge $2$, $4$, \dots? \\
		% $\{ab, ba\}^*$
		\visible<4-|handout:2>{
			$(\{X,Y\},T,X,P)$ mit $P=\{X \to \word{ab}X \mid \word{ba}X \mid \word a \mid \word b \mid \varepsilon\}$
		}
		
	\end{itemize}
\end{frame}


\begin{frame}{Aufgabe: Palindrome}
	\begin{itemize}
		\item Geben Sie eine kontextfreie Grammatik $$G = (N, \{\word a, \word b\}, S, P )$$ an, für die $L(G)$ die Menge aller Palindrome über dem Alphabet $\{\word a, \word b\}$ ist.
		\item Geben Sie eine Ableitung der Wörter \word{baaab} und \word{abaaaba} aus dem Startsymbol Ihrer Grammatik an.
		\item Beweisen Sie, dass Ihre Grammatik jedes Palindrom über dem Alphabet $\{\word a, \word b\}$ erzeugt.\\
		(\emph{Tipp}: Induktion: Wenn's für $n$ und $n+1$ gilt, dann gilt's auch für $n+2$.)
	\end{itemize}
\end{frame}

\begin{frame}{Lösung}
	Die Grammatik $$G = (\{S\}, \{\word a, \word b\}, S, P = \{S \to \word aS\word a \ | \ \word bS\word b \ | \ \word a \ | \ \word b \ | \ \varepsilon \})$$ erzeugt gerade die Menge der Palindrome. \pause Die Ableitungen der Wörter mit dieser Grammatik sind 
	$$S \derives \word bS\word b \derives \word{ba}S\word{ab} \derives \word{baaab}$$
	$$S \derives \word aS\word a \derives \word{ab}S\word{ba} \derives \word{aba}S\word{aba} \derives \word{abaaaba}$$
\end{frame}

\begin{frame}{Lösung}
	Sei $w$ ein Palindrom über $\{\word a,\word b\}$. Wir zeigen durch Induktion über $n = \size{w}$, dass alle Palindrome aus $S$ abgeleitet werden können. \pause
	\begin{block}{Induktionsanfang} \pause
		Für $n = 0$ ist das leere Wort $\varepsilon$ in einem Schritt aus $S$ ableitbar. \\
		Für $n=1$: Die einzigen Wörter aus $\{\word a,\word b\}^\ast$ der Länge 1 sind $\word a$ und $\word b$. Auch diese sind offensichtlich aus $S$ ableitbar.
	\end{block}
 	\pause
	\begin{block}{Induktionsvoraussetzung} \pause
		Für ein festes, aber beliebiges $n \in \N_0$ gilt, dass alle Palindrome der Länge $n$ und alle Palindrome der Länge $n + 1$ aus S abgeleitet werden können.
	\end{block}
\end{frame}

\begin{frame}{Lösung}
		\begin{block}{Induktionsschritt}
			Sei $w$ ein Palindrom der Länge $n + 2$. Das erste (und damit auch das letzte) Zeichen sei oBdA. ein $\word a$. Dann gibt es ein $w' \in \{\word a, \word b\}^\ast$, so dass $w = \word aw'\word a$ ist. Da $w$ ein Palindrom ist, muss auch $w'$ ein Palindrom sein. Weiterhin gilt $\size{w'} = n$. \pause Nach IV gibt es somit eine Ableitung $S \derives^\ast w'$. Somit gibt es die Ableitung $$S \derives \word aS\word a \overset{IV}{\derives^\ast} \word aw'\word a = w$$ und $w \in L(G)$ folgt. \pause Entsprechendes gilt, wenn das erste Zeichen von w ein \word b ist. \\
			Mit der IV haben wir also gezeigt, dass auch Palindrome der Länge $n+1$ und $n+2$ aus $S$ ableitbar sind.
	\end{block}

\end{frame}

\begin{frame}{Gibt es noch mehr?}
	Viele Sprachen in der Informatik sind kontextfrei.\\[1em]
	Was ist mit der Sprache $L_{vv} = \{v\word cv \mid v \in \{\word a,\word b\}^*\}$\\
	\smallskip
	\pause
	In der Vorlesung: Es gibt keine kontextfreie Grammatik, die $L_{vv}$ erzeugt.\\
	\medskip
	\pause
	Können wir die Sprache trotzdem irgendwie \enquote{verarbeiten}?
	
	\begin{block}{}
		\Large
		\centering
		Soon...\\[1em]
	\end{block}
\end{frame}


\section{Relationen}
\begin{frame}{Eigenschaften}
	\begin{Definition}
		Sei $R \subseteq A \times A$ eine (binäre) Relation auf der Menge $A$. Wir nennen $R$
		\begin{itemize}[<+->]
			\item \textbf{reflexiv}, falls gilt $$\forall x \in A: (x,x) \in R$$
			\item \textbf{symmetrisch}, falls gilt $$\forall x,y \in A: (x,y) \in R \implies (y,x) \in R$$
			\item \textbf{transitiv}, falls gilt $$\forall x,y,z \in A: (x,y) \in R \text{ und } (y,z) \in R \implies (x,z) \in R$$
		\end{itemize}
	\end{Definition}
\end{frame}

\begin{frame}{Beispiele}
	\begin{itemize}
		\item Die Relation $=$ ist \pause reflexiv, symmetrisch und transitiv. Man nennt so etwas auch \emph{Äquivalenzrelation}.
		\item \pause Die Relation $<$ ist \pause nicht reflexiv und nicht symmetrisch, aber transitiv.
		\item \pause Die Relation $\leq$ ist \pause reflexiv, nicht symmetrisch, aber transitiv.
	\end{itemize}
\end{frame}

\begin{frame}{Produkt}
	\begin{Definition}
		Das \textbf{Produkt} von zwei Relationen $R \subseteq M \times N, S \subseteq N \times L$ definieren wir als $$S \circ R = \set{(x,z) \in M \times L \Mid \exists y \in N : (x,y) \in R \text{ und } (y,z) \in S }.$$
	\end{Definition}	
	\pause
	
	\begin{Definition}
		Die \textbf{Potenz} einer Relation $R \subseteq M \times M$ definieren wir als
		\begin{align*}
			R^0 &= I_M = \{(x,x) \mid x \in M \} \\
			R^{i+1} &= R^i \circ R
		\end{align*}
	\end{Definition}

	\pause
	\begin{block}{Beobachtung}
		Wenn $f$ und $g$ Funktionen sind (also linkstotale, rechtseindeutige Relationen), entspricht $f \circ g$ der Hintereinanderauswertung von $f$ nach $g$.
	\end{block}
\end{frame}

\begin{frame}{Reflexiv-transitive Hülle}
	\begin{Definition}
		Die \textbf{reflexiv-transitive Hülle} einer Relation $R$ ist
		$$R^\ast = \bigcup \limits_{i=0}^\infty R^i$$
	\end{Definition}

	\pause
	\begin{block}{Satz}
		$R^*$ ist die kleinste Relation, die $R$ umfasst und reflexiv und
		transitiv ist.
	\end{block}

	\pause
	\begin{Beispiel}
		Sei $A = \{a, b, c, d, e\}$ und R = $\{(a, b), (b, c), (c, e)\} \subseteq A \times A$\\ \pause
		%TODO Align
		$R^*=\{(a,a), (b,b), (c,c), (d,d), (e,e),$ \\
		$(a,b), (b,c), (c,e),$ \\
		$(a,c), (b,e),(a,e)\}$
	\end{Beispiel}
	
\end{frame}




\begin{frame}	
	\begin{block}{Was ihr nun wissen solltet}
		\begin{itemize}
			%\item Mehr Eigenschaften von Relationen
			\item Was eine kontextfreie Grammatik ist
			\item Wie man Sprachen mit Grammatiken beschreiben kann
			\item Welche Eigenschaften Relationen haben können
		\end{itemize}
	\end{block}
	
	\begin{block}{Was nächstes Mal kommt}
		\begin{itemize}
			\item Noch mehr Grammatiken
			\item Nachts sind alle Katzen grau -- Prädikatenlogik
			%\item Algorithmen: Kochrezepte der Informatik
		\end{itemize}
	\end{block}
\end{frame}	

\xkcdframe{327}{Danke für eure Aufmerksamkeit! \smiley}{2.5}

\slideThanks

\end{document}