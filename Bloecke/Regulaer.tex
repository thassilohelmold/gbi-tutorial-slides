% TODO: Fix dirty hack in thwregex.sty, where I changed line 42 to print star in math-mode
% 		because otherwise the star was always raised in my config.
%	COmment(Daniel): Reverted your change. No problems whatsoever.
%		Use \rx everywhere

% Fuck you, Latex. In my algo tutorial slides, this isn't necessary. Why here then!?
\begin{frame}[t]
	\fakeframetitle{\only<2->{Grenzen endlicher Akzeptoren}}
	Gibt es einen endlichen Akzeptor $A$ mit $$L(A) = \set{ \word a^k\word b^k \Mid k\in \N_0 }?$$
	\pause
	Nein! Warum nicht? \visible<3->{Endliche Automaten können nicht unendlich weit „zählen“!}\\
	
	Gibt es einen endlichen Akzeptor, der alle gültigen Klammerausdrücke erkennt?\\ \pause
	Nein, aus dem selben Grund.
	\begin{figure}[H]
		\includegraphics[scale=0.5]{xkcd/tags_1144}
		\caption{ \texttt{\url{https://www.xkcd.com/1144/}} }
	\end{figure}
	\pause
	Kontextfreie Grammatiken \enquote{können also mehr} als endliche Akzeptoren.\\
	Wir wollen nun ein \enquote{gleichmächtiges} Konzept zu Akzeptoren.
\end{frame}

\section{Reguläre Ausdrücke}
\subsection{Definition}

\begin{frame}{Disclaimer!}
	\begin{center}
		\Large
			\bfalert{\Huge ACHTUNG!} \\ \medskip
		
		Gemeint sind \textbf{NICHT} sog. \emph{Regular Expressions}, die ihr vllt. aus Programmiersprachen kennt! \\ \bigskip
		
		{\normalsize (Die sind ähnlich, aber eben nicht das gleiche.)}
	\end{center}
\end{frame}

\begin{frame}{Reguläre Ausdrücke}
	Wir können uns reguläre Ausdrücke zusammenbauen aus
	\begin{itemize}
		\item den einzelnen Symbolen $x$ aus $A$ \pause
		\item zwei regulären Ausdrücken $R_1$ und $R_2$ mit $$\rx(R_1 R_2\rx) \qquad \text{ oder } \qquad \rx(R_1\rx|R_2\rx)$$ \pause
		\item einem Stern $R\rx*$ \pause
		\item oder dem leeren Ausdruck $\rx O$ \pause
	\end{itemize} 
	Klammern dürfen nach den Klammerregeln weggelassen werden:\\
	Stern vor „Punkt“ {\small (in diesem Fall unsichtbar)} vor Strich.
\end{frame}

\begin{frame}{Reguläre Ausdrücke}
	\begin{Beispiel}
		Sei $ A = \{ \word a, \word b, \word c\}$. Dann sind gültige reguläre Ausdrücke über $A$:\\
		$\rx{abc}$\\
		$\rx{a|b|c}$\\
		$\rx{(ab)*}$\\
		$\rx{O*}$
	\end{Beispiel}
\end{frame}


\begin{frame}{Sprache eines Ausdruckes}
	Die durch $R$ beschriebene Sprache $\lang{R}$ ist wie folgt definiert:
	\begin{itemize}
		\item $\lang{\rx{O}} = \emptyset$
		\item $\lang{x}=\{x\} \quad (\text{für }x\in A)$
		\item $\lang{R_1 \rx| R_2} = \lang{R_1} \cup \lang{R_2}$
		\item $\lang{R_1 R_2} = \lang{R_1} \cdot \lang{R_2}$
		\item $\lang{R\rx*} = \lang{R}^*$
	\end{itemize} 

	\bigskip
	Eine Sprache, für die es einen beschreibenden regulären Ausdruck gibt, nennt man \textbf{regulär}.
\end{frame}

\begin{frame}{Sprache eines Ausdruckes}
	\begin{Beispiel}
		\begin{itemize}
			\item $\lang{\rx{a}} = \{\word a\}$. \pause
			\item $\lang{\rx{ab}} = \lang{\rx{a}} \cdot \lang{\rx{b}} = \{\word a\word b\}$. \pause
			\item $\lang{\rx{a|b}} = \lang{\rx{a}}\cup\lang{\rx{b}} = \{\word a,\word b\}$. \pause
			\item $\lang{\rx{(a|b)*}} = \lang{\rx{a|b}}^* = \{\word a,\word b\}^*$. \pause
			\item $\lang{\rx{(a*b*)*}} = \lang{\rx{a*b*}}^* = \left(\lang{\rx{a*}}\lang{\rx{b*}}\right)^* 
			= \left(\lang{\word a}^*\lang{\word b}^*\right)^* = \left(\{\word a\}^* \· \{\word b\}^*\right)^*$\\
			$= \{\word a,\word b\}^*$.  
		\end{itemize}
	\end{Beispiel}
	\pause
	\begin{block}{Aufgabe}
		Gebt einen regulären Ausdruck $R$ an mit $\lang{R} = $  
		\begin{itemize}
			\item die Sprache aller binären Zweierpotenzen \\ \visible<+->{}
				  \visible<+-|handout:2>{\impl $R = \rx{0*10*}$}
			\item die Sprache aller geraden Binärzahlen \\
				  \visible<+-|handout:2>{\impl $R = \rx{(0|1)*0}$}
		\end{itemize}
	\end{block}
\end{frame}

\begin{frame}{Aufgabe: Reguläre Ausdrücke}
	In dieser Aufgabe geht es um die formalen Sprachen
	$$L_1 = \set{\word a^k \word b^m \Mid k, m \in \N_0 }, \qquad L_2 = \set{\word b^k \word a^m \Mid k, m \in \N_0 }.$$
	Gebt für jede der folgenden formalen Sprachen $L$ je einen regulären Ausdruck $R$ an mit $ \langle R \rangle = L$.
	\begin{itemize}
		\item $L = L_1 \cup L_2$ \\
			\visible<2-|handout:2>{$ \rx{a*b*|b*a*}$}
		\item $L = L_1 \cap L_2$ \\
			\visible<3-|handout:2>{$ \rx{a*|b*}$}
		\item $L = L_1\cdot L_2$ \\
			\visible<4-|handout:2>{$ \rx{a*b*b*a*}$ oder $\rx{a*b*a*}$}
		\item $L = L_1^*$ \\
			\visible<5-|handout:2>{$ \rx{(a*b*)*}$ oder $\rx{(a|b)*}$}
	\end{itemize}
\end{frame}

\begin{frame}{Aufgabe: Sprachen regulärer Ausdrücke}
	\begin{itemize}
		\item $\lang{\rx{(a|b)*abb(a|b)*}} = \visible<2-|handout:2>{\{\word a, \word b\}^* \cdot \{\word a\word b\word b\} \cdot \{\word a, \word b\}^*}$
		\item $\lang{\rx{a**}} = \visible<3-|handout:2>{\{\word a\}^*}$
		\item $\lang{\visible<4-|handout:2>{R\rx{(}R\rx{)*}}} = \lang{R}^+$ \quad (für bel. reg. Ausdruck $R$)
		\item $\lang{\visible<5-|handout:2>{\rx{O*}}} = \{\eps\}$
		\item $\lang{\visible<6-|handout:2>{\rx{a*ba*ba*b(a|b)*}}} = \set{ w \in \{\word a, \word b\}^* \Mid \size{w}_{\word b} > 2 } $
		\item $\lang{\visible<7-|handout:2>{\rx{b*a*}}} =$ Sprache aller Wörter über $\set{\word a, \word b}$, in denen das Teilwort \word{ab} nicht vorkommt.
	\end{itemize}
\end{frame}
