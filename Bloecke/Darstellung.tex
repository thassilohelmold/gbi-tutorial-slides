\section{Von der Darstellung zur Zahl}

\subsection{Definitionen}
\begin{frame}{Numerischer Wert}
	\begin{Definition}
		Zu einer Zahlenbasis $b$ definiere 
		$$ \text{Num}_b(\varepsilon) = 0  $$  
		$$ \text{Num}_b(wx) = b\cdot \text{Num}_b(w) + \text{num}_b(x) \text{ für alle } w\in Z_b^\ast, x\in Z_b $$ 
	\end{Definition}

	\pause
	Beachte: $Num_b : Z_b^* \to \Z$ ist Abbildung, die einem Wort (Zahlendarstellung) eine Zahl (Wert) zuordnet. Wir müssen diesen Wert aber natürlich wieder in eine  Darstellung umwandeln, um ihn aufschreiben zu können.
\end{frame}
\begin{frame}{Aufgabe}
	Berechnet die Zahlenwerte von $ 11_2, 321_4, B2_{16}$.
	\begin{align*} 
	\visible<1->{\text{Num}_2(11) &= 2\cdot \text{Num}_2(1) + \text{num}_2(1) \\
	&= 2\cdot 1 + 1 \\
	&= 3  \\}
	\visible<3->{\text{Num}_4(321) &=} \visible<4->{ 4\cdot \text{Num}_4(32) + \text{num}_4(1) \\
	&= 4\cdot \left( 4\cdot \text{Num}_4(3) + \text{num}_4(2) \right) + \text{num}_4(1) \\
	&= 4^2\cdot \text{num}_4(3) + 4 \cdot \text{num}_4(2) + \text{num}_4(1) \\
	&= 57 \\}
	\visible<5->{\text{Num}_{16}(B2) &=} \visible<6->{ 16 \cdot \text{Num}_{16}(B) + \text{num}_{16}(2) \\
	&= 16\cdot 11 + 2 \\
	&= 178}
	\end{align*}

\end{frame}

\begin{frame}{Wohldefiniertheit}
	\emph{Behauptung}: Die Definition 
		$$ \text{Num}_b(\varepsilon) = 0  $$  
		$$ \text{Num}_b(wx) = b\cdot \text{Num}_b(w) + \text{num}_b(x) \text{ für alle } w\in Z_b^\ast, x\in Z_b $$ 
		ist wohldefiniert und weist jedem Wort eine eindeutige Bedeutung zu, die dem Zahlenwert entspricht.
\end{frame}
\begin{frame}{Beweis}
	\begin{block}{Beweis durch vollständige Induktion über $n=\vert w \vert $}
	\begin{itemize}
		\only<1-2>{\item<1->[\emph{IA}] $n = 0 = \vert w \vert \implies w = \varepsilon $. \\
		Für $w = \varepsilon $ ist $\text{Num}_b$ wohldefiniert und sinnvoll (nämlich $Num_b(\varepsilon) = 0$).
		\item<2->[\emph{IV}] Für ein beliebig aber festes $n\in\N_0$ sei $Num_b(w) für alle w mit \setsize{w} = n wohldefiniert und entspreche dem Zahlenwert.$ }
		\only<3->{\item<3->[\emph{IS}] Wähle $w'$ mit $\vert w' \vert = n+1 $, dann gibt es ein $w\in Z_b^n, x\in Z_b$, so dass $ w' = wx $ \\
		Mit der Definition gilt nun $$ \text{Num}_b(w') = b\cdot \underbrace{\text{Num}_b(w)}_{IV} + \text{num}_b(x) $$
		Die Summe ist laut $IV$ wohldefiniert. Auch ist laut $IV$ $\text{Num}_b(w)$ der Zahlenwert von $w$ und damit auch $\text{Num}_b(w')$.}
	\end{itemize}
	\end{block}

\end{frame}

%\subsection{Aufgabe}
%\begin{frame}{Aufgabe. WS 2010 }
%Es bezeichne $\Z$ die Menge der ganzen Zahlen. Gegeben sei eine Ziffernmenge $Z_{-2} = \{N, E\}$ mit der Festlegung $num_2 (N) = 0$ und $num_2 (E) = 1$. Wir definieren eine Abbildung $Num_{-2} : Z_{-2}^\ast \to \Z$ wie folgt:
%	$$Num_{-2} (\varepsilon) = 0$$
%	$$\forall \ w \in Z_{-2}^\ast \ \forall \ x \in Z_{-2} : Num_{-2} (wx) = -2 \cdot Num_{-2} (w) + num_2 ( x )$$
%
%	\begin{itemize}	
%		\item Geben Sie für $w \in \{E, EN, EE, ENE, EEN, EEE\}$ jeweils $Num_{-2} (w)$ an.
%		\item Für welche Zahlen $x \in \Z$ gibt es ein $w \in Z_{-2}^\ast$ mit $Num_{-2} (w) = x$?
%	\end{itemize}
%\end{frame}
%
%\begin{frame}{Lösung}
%\textit{Geben Sie für $w \in \{E, EN, EE, ENE, EEN, EEE\}$ jeweils $Num_{-2} (w)$ an.} \pause
%	\begin{table}[h!]	
%		\begin{tabular}{>{$}l<{$}>{$}l<{$}}
%			Num_{-2} (E)\pause & = 1 \\ \pause 
%			Num_{-2} (EN)\pause & = -2 \\ \pause
%			Num_{-2} (EE)\pause & = -1 \\ \pause
%			Num_{-2} (ENE)\pause & = 5 \\ \pause
%			Num_{-2} (EEN)\pause & = 2 \\ \pause
%			Num_{-2} (EEE)\pause & = 3
%	\end{tabular}
%	\end{table}
%	\pause
%	\textit{Für welche Zahlen $x \in \Z$ gibt es ein $w \in Z_{-2}^\ast$ mit $Num_{-2} (w) = x$?} \\[1em]\pause
%	Für alle!
%\end{frame}

\section{Von der Zahl zur Darstellung}
\begin{frame}{Division und Modulo}
	\begin{block}{Definition}
		$ x \div y$ ist die ganzzahlige Division von x durch y.\\
		$ x \mod y$ liefert den Rest dieser Division
	\end{block} 
	\pause
	
	\begin{block}{Beobachtung}
		$ x\div y \in \N_0 \qquad x\mod y \in \{0,\dots, y-1\} $
	\end{block}
	\pause
	
	\begin{block}{Lemma}
		$$ x = y \cdot (x \div y ) + \left( x \mod y \right)$$ 
	\end{block}
	
\end{frame}

\begin{frame}
	\begin{block}{Beispiel}
		\begin{tabular}{ccc}
			& $x\div y$ & $x\mod y$ \\
			$x=2,y=3$ \pause &  0 & 2 \\  \pause
			$x=5 ,y=2$ \pause & 2 & 1 \\	\pause
			$x=8,y=2$ \pause & 4 & 0 \\	
		\end{tabular}
	\end{block}
\end{frame}

\begin{frame}{Beispiel}
	\begin{table}[h!]
		\centering
		\begin{tabular}{c|cccccccccccc}
			$x$ & 0 & 1 & 2 & 3 & 4 & 5 & 6 & 7 & 8 & 9 & 10 & 11 \\ \hline
			$x\div 4 $ & \only<2->{0 & 0 & 0 & 0 & 1 & 1 & 1 & 1 & 2 & 2 & 2 & 2 } \only<1|handout:0>{&&&&&&&&&&&} \\
			$4\left( x\div 4\right) $ & \only<3->{0&0&0&0&4&4&4&4&8&8&8&8} \only<1-2|handout:0>{&&&&&&&&&&}  \\
			$x\mod 4$ & \only<4>{0&1&2&3&0&1&2&3&0&1&2&3} \only<1-3|handout:0>{&} 
		\end{tabular}
	\end{table}
\end{frame}

\begin{frame}{Repräsentation}
	\begin{block}{Definition}
		$Repr_k(n)$ ist das kürzeste Wort $w\in Z_k^\ast$ mit $Num_k(w)=n$, also 
		$$ Num_k\left( Repr_k(n)\right) = n $$ 
	\end{block}
	\pause
	\emph{Anmerkung}:
	Im Allgemeinen $$ Repr_k\left(Num_k(w)\right) \neq w $$ da überflüssige Nullen wegfallen. 
\end{frame}

\begin{frame}{Repräsentation}
	Wir definieren
	\begin{align*}
		Repr_k : \; &\N_0 \to Z_k  \\
		n&\mapsto \begin{cases} repr_k(n) & n<k \\ Repr_k\left( n\div k \right) \cdot repr_k\left( n \mod k \right) & n\geq k 
		\end{cases} 
	\end{align*}
	
	\begin{block}{Aufgabe}
		Berechne folgende Darstellungen:\\
		$Repr_2(42) = \pause 101010$ \\
		$Repr_4(42) = \pause 222$ \\
		$Repr_8(42) = \pause 52$ \\
		$Repr_{16}(42) = \pause 2A$
	\end{block}
\end{frame}

\begin{frame}{Beispiel: Lösung}
	\begin{align*}
		Repr_8(42) &= Repr_8(42 \div 8) \cdot repr_8(42 \mod 8) \\
		&= Repr_8(5) \cdot repr_8(2)\\
		&= repr_8(5) \cdot 2\\
		&= 5 \cdot 2\\
		&= 52_8
	\end{align*}
	
\end{frame}
