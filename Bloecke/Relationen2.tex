\section{Relationen}
\begin{frame}{Eigenschaften}
	\begin{Definition}
		Sei $R \subseteq A \times A$ eine (binäre) Relation auf der Menge $A$. Wir nennen $R$
		\begin{itemize}[<+->]
			\item \textbf{reflexiv}, falls gilt $$\forall x \in A: (x,x) \in R$$
			\item \textbf{symmetrisch}, falls gilt $$\forall x,y \in A: (x,y) \in R \implies (y,x) \in R$$
			\item \textbf{transitiv}, falls gilt $$\forall x,y,z \in A: (x,y) \in R \text{ und } (y,z) \in R \implies (x,z) \in R$$
		\end{itemize}
	\end{Definition}
\end{frame}

\begin{frame}{Beispiele}
	\begin{itemize}
		\item Die Relation $=$ ist \pause reflexiv, symmetrisch und transitiv. Man nennt so etwas auch \emph{Äquivalenzrelation}.
		\item \pause Die Relation $<$ ist \pause nicht reflexiv und nicht symmetrisch, aber transitiv.
		\item \pause Die Relation $\leq$ ist \pause reflexiv, nicht symmetrisch, aber transitiv.
	\end{itemize}
\end{frame}

\begin{frame}{Produkt}
	\begin{Definition}
		Das \textbf{Produkt} von zwei Relationen $R \subseteq M \times N, S \subseteq N \times L$ definieren wir als $$S \circ R = \set{(x,z) \in M \times L \Mid \exists y \in N : (x,y) \in R \text{ und } (y,z) \in S }.$$
	\end{Definition}	
	\pause
	
	\begin{Definition}
		Die \textbf{Potenz} einer Relation $R \subseteq M \times M$ definieren wir als
		\begin{align*}
			R^0 &= I_M = \{(x,x) \mid x \in M \} \\
			R^{i+1} &= R^i \circ R
		\end{align*}
	\end{Definition}

	\pause
	\begin{block}{Beobachtung}
		Wenn $f$ und $g$ Funktionen sind (also linkstotale, rechtseindeutige Relationen), entspricht $f \circ g$ der Hintereinanderauswertung von $f$ nach $g$.
	\end{block}
\end{frame}

\begin{frame}{Reflexiv-transitive Hülle}
	\begin{Definition}
		Die \textbf{reflexiv-transitive Hülle} einer Relation $R$ ist
		$$R^\ast = \bigcup \limits_{i=0}^\infty R^i$$
	\end{Definition}

	\pause
	\begin{block}{Satz}
		$R^*$ ist die kleinste Relation, die $R$ umfasst und reflexiv und
		transitiv ist.
	\end{block}

	\pause
	\begin{Beispiel}
		Sei $A = \{a, b, c, d, e\}$ und R = $\{(a, b), (b, c), (c, e)\} \subseteq A \times A$\\ \pause
		%TODO Align
		$R^*=\{(a,a), (b,b), (c,c), (d,d), (e,e),$ \\
		$(a,b), (b,c), (c,e),$ \\
		$(a,c), (b,e),(a,e)\}$
	\end{Beispiel}
	
\end{frame}


