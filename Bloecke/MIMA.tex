\section{MIMA}

\begin{frame}{MIMA}
	\includegraphics[width=\linewidth]{MIMA_simple.png}\\
	Die \emph{MIMA} ist ein idealisierter Prozessor. 
\end{frame}

\begin{frame}{Eigenschaften}
	\begin{itemize}[<+->]
		\item Adressen sind 20 Bit lang
		\item \enquote{Werte} sind 24 Bit lang
		\item Befehlscodierungen -- zwei Formate:
		\begin{itemize}
			\item 4 Bit für den OpCode und 20 Bit für einen Parameter (Adresse / Konstante)
			\item 8 Bit Befehl (Rest irrelevant)\\
			\item[]<+(-2)-> \includegraphics[width=100px]{MIMA_commands.png} 
		\end{itemize} 
	\end{itemize}
\end{frame}


\begin{frame}{Wichtige Register}
	\begin{itemize}[<+->]
		\item \emph{IAR} : InstruktionsAdressRegister : Speichert Adresse des aktuell auszuführenden Befehls.
		\item \emph{IR}: InstruktionsRegister : Speichert den auszuführenden Befehl.
		\item \emph{SAR}: SpeicherAdressRegister : Enthält die Adresse eines Wertes, der aus dem Speicher gelesen werden soll.
		\item \emph{SDR}: SpeicherDatenRegister : Enthält einen Wert, der aus dem Speicher geladen wurde.
	\end{itemize}
\end{frame}

\begin{frame}{Aufbau}
	\begin{figure}
		\centering
		\includegraphics[width=\linewidth]{MIMA.png}
	\end{figure}
\end{frame}

\begin{frame}{Befehlsholphase}
	\begin{figure}
		\centering
		\includegraphics[width=\linewidth]{MIMA.png}
	\end{figure}
	\pause
	\only<2-6>{\begin{itemize}
		\only<2|handout:1> {\item[1.] IAR $\to $ SAR  und IAR $\to$ X 
			\item[] Befehlsadresse dem Speicher übergeben und Zähler zum Erhöhen an ALU geben.}
		\only<3|handout:2>{\item[2.] Eins $\to$ Y  
			\item[] 1-Wert für Erhöhung des Zählers an ALU geben.}
		\only<4|handout:3>{\item[3.] ALU aufaddieren ($Z=X+Y$) 
			\item[] Nächste Befehlsadresse berechnen. }
		\only<5|handout:4>{\item[4.] Z $\to$ IAR 
			\item[] Adresse für nächste Runde speichern. }
		\only<6|handout:5>{\item[5.] SDR $\to$ IR 
			\item[] Wert zur angefragten Adresse erhalten. }
	\end{itemize}}	
\end{frame}

\newcommand{\itemizeconfig}{\setlength{\parsep}{0pt}\setlength{\parskip}{0pt}\setlength{\topsep}{0pt}\setlength{\partopsep}{0pt}}
\begin{frame}{Befehle}
	Die MIMA besitzt einen Befehlssatz mit möglichen Befehlen. %Andere Befehle (oder Varianten) werden NICHT unterstützt und können daher nicht verwendet werden!
	\textbf{Nur die Befehle aus der VL} dürft ihr benutzen! \\ 
	\smallskip
	(Befehle, die eigentlich zwei Operanden brauchen, lesen zusätzl. vom Akku.)
	{
	\begin{itemize}[<+->] \itemizeconfig
		\item Rechenoperationen:
		\begin{itemize} 
			\item ADD adr 
			\item AND, OR, XOR adr
			\item NOT, RAR \quad (keine Parameter \impl Akku) \quad „Rotate Akku Right“
		\end{itemize}
		\item Datentransport:
		\begin{itemize}
			\item LDC const \quad (const ist dabei eine \textbf{20-Bit}-Konstante)
			\item LDV, STV adr \quad (Lese/Schreibe von/an Adresse)
			\item LDIV, STIV adr \quad (Lese/Schreibe von/an Adresse, die an der Adresse steht)
		\end{itemize}
		\item Vergleichsoperation: EQL adr \quad ($-1$ wenn gleich, 0 sonst)
		\item Sprünge:
		\begin{itemize}
			\item JMP adr
			\item JMN adr \quad „JuMp if Negative“
		\end{itemize}
		\item HALT
	\end{itemize}}
\end{frame}

\begin{frame}{Bemerkungen}
	\begin{block}{Indirekte Adressierung (LDIV, STIV)}
		\centering
		\includegraphics[width=150px]{MIMA_indirect.png}
	\end{block}
	
	\pause
	\begin{block}{HALT}
		Jedes Programm muss mit HALT enden! Sonst läuft das Programm endlos weiter! (Kostet Punkte!)
	\end{block}

	\pause
	\begin{block}{Negative Konstanten}
		Negative Konstanten können \textbf{nicht} mit LDC geladen werden. Warum? \\
		\pause \impl Unser Akku ist 24 Bit breit, aber wir können nur in die \textbf{hinteren 20 Bit} laden!
	\end{block}
\end{frame}

\begin{frame}{Negative Zahlen}
	\begin{block}{Aufgabe}
		Schreibt ein Programm, das von einer an Adresse $a_1$ gegebenen positiven Zahl das Zweierkomplement berechnet und an Adresse $a_2$ ablegt.
	\end{block}

	\visible<2|handout:2>{
		\begin{block}{Lösung}
			LDV $a_1$\\
			NOT\\
			STV $a_2$\\
			LDC 1\\
			ADD $a_2$\\
			STV $a_2$
		\end{block}
	}
\end{frame}

\begin{frame}{Beispiele}
	Beispiele zur Umsetzung von Anweisungen aus Hochsprachen:\\
	Siehe Übung WS 15/16
\end{frame}

\begin{frame}{Übung: Modulo}
	Schreibe ein Programm, das eine an Speicheradresse $a_1$ gegebenen Zahl Modulo R rechnet und an Adresse $a_2$ ablegt.
	\begin{itemize}
		\item Modulo 2
		\item Modulo 3
	\end{itemize}
\end{frame}

\begin{frame}{Lösung: Modulo 2}
	\begin{tabbing}
		start:  \= LDC 1 \quad  //$=$ 000000000000000000000001 \\
				\> AND $a_1$ \\
				\> STV $a_1$ \\
				\> HALT
	\end{tabbing}
	
\end{frame}

\begin{frame}{Lösung: Modulo 3}
  \begin{tabbing}
    start:\; \= LDC 1 \\
           \> STV One\\
           \> LDC 3\\
           \> NOT\\
           \> ADD One\\
           \> STV MThree\\
    \medskip
    while: \> LDV MThree\\
           \> ADD $a_1$\\
           \> JMN ende\\
           \> STV $a_1$\\
           \> JMP while\\
    \medskip
    ende:  \> LDC 3\\
           \> ADD $a_1$\\
           \> HALT\\
  \end{tabbing}
\end{frame}