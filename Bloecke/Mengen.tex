\section{Mengen}

%TODO: Handout mode
\mycomment{
\begin{frame}
	\frametitle{Problem}
	
	\includegraphics[width=\linewidth]{moviesActors.jpg}
\end{frame}
}

\begin{frame}
	\frametitle{Problem}
	Wir haben \enquote{ein Universum} an Elementen (Filme, Serien, Schauspieler):\\[0.5em]
	
	$U$ enthält Sherlock, Benedict Cumberbatch, Lea Thompson, Martin Freeman, The Imitation Game, Mark Gatiss, Christopher Lloyd, Crispin Glover, Zurück in die Zukunft, Michael Fox, Keira Knightley, \dots  
\end{frame}

\begin{frame}
	\frametitle{Mengen}
	Mengen sind eines der Grundelemente der Mathematik. \\[1em]
	Sammelt bitte 5 Minuten lang \\(alleine / mit eurem Nebensitzer / in Kleingruppen)\\ alles, was ihr über Mengen bereits wisst.
\end{frame}

\begin{frame}
	\frametitle{Mengen}

	\begin{block}{Definition}
		Eine \textbf{Menge} $M$ ist eine Ansammlung verschiedener Objekte. Ein Objekt aus der Menge nennt man ein \textbf{Element} der Menge. Man schreibt
		$$m \in M \qquad M = \{m_1, m_2, m_3 \} \qquad M = \{m \mid m \geq 0\} \qquad M = \emptyset = \{\} $$
		Die Reihenfolge der Aufzählung ist dabei irrelevant, Elemente kommen nicht doppelt vor. Die leere Menge $\emptyset$ enthält keine Elemente.
	\end{block}
	\pause
	
	\begin{block}{Beispiel}
		Wichtige Mengen sind
		$$\N, \Z, \Q, \R, \C\qquad \N_+, \N_0$$
		Es gilt: $$-5 \in \Z\qquad -5 \notin \N\qquad \{2, 1, 3, 1, 4\} = \{1, 2, 3, 4\}$$
	\end{block}
	
\end{frame}

\begin{frame}
	\frametitle{Teilmengen}
	
	\begin{block}{Definition}
		Die Anzahl der Elemente in einer endlichen Menge (Kardinalität) bezeichnet man mit $\setsize{M}$. Es gilt $$ \setsize{M \cup N} \ = \ \setsize{M} + \setsize{N} - \setsize{M \cap N} $$
	\end{block} 
	\pause

	\begin{block}{Definition}
		Eine Menge $N$ ist eine \textbf{Teilmenge} von $M$, wenn jedes Element aus N auch in M enthalten ist.
		$$ N \subseteq M \iff \forall n \in N : n \in M$$
	\end{block}
	\pause
	
	\begin{block}{Definition}
		Zwei Mengen $N$ und $M$ sind \textbf{gleich}, wenn sie die gleichen Elemente enthalten.
		$$ N = M \iff N \subseteq M \ \text{und} \ M \subseteq N$$
	\end{block}
	
\end{frame}

\begin{frame}
	\frametitle{Teilmengen}
	
	\begin{block}{Beispiel}
		
	\begin{align*}
		\setsize{\{1,2,3,2,1\}} &= \only<2->{3} \\
		\setsize{ \emptyset } &= \only<3->{0} \\
		\{1,2\} \only<1-3|handout:0>{\ &\mathrel{?} \ } \only<4->{&\subseteq}  \{1,2,3\} \\
		\{1,2\} \only<1-4|handout:0>{\ &\mathrel{?} \ } \only<5->{&\nsubseteq} \{\text{Hund}, \text{Katze}, \text{Maus}\}
	\end{align*}
	\end{block} 

	\only<6->{
		\begin{block}{Lemma}
			Es gilt: $$ N \subseteq M \iff N \setminus M = \emptyset $$
			
		\end{block} 

	}
\end{frame}

\begin{frame}{Mengengleichheit: Beispiel}
	\begin{itemize}
		\item<1-> Sei $ A $ und $M$ beliebige Mengen. Zeigen Sie, dass gilt 
		\begin{align*}
		A &=  \underbrace{ \left(A \setminus M \right)}_{T_1} \cup  \underbrace{ \left(A \cap M \right)}_{T_2}  
		\end{align*}		 
		\item<2-5>
		Richtung: $ A \subseteq T_1 \cup T_2 $ 
		\only<3-5>{\item<3-5> Wähle $x\in A$ und wende Fallunterscheidung an
			\item<4-5> Fall 1: Ist $x\in M$ so gilt $x\in A $ und $x\in M$, und damit $x\in A\cap M  = T_2 $
			\item<5-5> Fall 2 : Ist $x\notin M$ so gilt $x\in T_1$ da $T_1 = \left\{ x \in A \text{ und } x\notin M \right\} $  }
		\only<6->{\item<6-> Richtung : $T_1 \cup T_2 \subseteq A $ 
			\item<7-> Wähle $x\in T_1 \cup T_2$. Dies bedeutet $x\in T_1 $ oder $x\in T_2$. 
			\item<8-> Fall 1: $x\in T_1$. Aus Definition folgt $x\in A$
			\item<9-> Fall 2: $x\in T_2$. Somit $x\in A$ und $x\in M$.}
	\end{itemize}
\end{frame}

\begin{frame}
	\frametitle{Zurück zu unserem Problem}
	$U = \{ $ Sherlock, Benedict Cumberbatch, Lea Thompson, Martin Freeman, The Imitation Game, Mark Gatiss, Christopher Lloyd, Crispin Glover, Zurück in die Zukunft, Michael Fox, Keira Knightley, \dots $ \} $  \\[0.5em]
	
	\pause
	Ordnen wir diese in eine Teilmenge $M$ für Filme/Serien und jeweils eine Teilmenge $A_m$ für die Schauspieler eines Films $m \in M$. \\[0.5em]
	
	\pause
	$M = \{$ The Imitation Game, Sherlock, Zurück in die Zunkunft $\}$ \\[0.3em]
	$A_{Sherlock} = \{$ Benedict Cumberbatch, Martin Freeman, Mark Gatiss $\}$ \\
	$A_{Imitation Game} = \{$ Benedict Cumberbatch, Keira Knightley $\}$ \\
	$A_{BTTF} = \{$ Michael J. Fox, Christopher Lloyd, Lea Thompson, Crispin Glover $\}$
	
\end{frame}

\begin{frame}
	\frametitle{Schnitt und Vereinigung}
	
	\begin{block}{Definition}
		Sind $M$ und $N$ zwei Mengen, so definiert man
		$$M \cap N = \{x \mid x \in M \text{ und } x \in N\} \qquad M \cup N = \{x \mid x \in M \text{ oder } x \in N\} $$
		als den \textbf{Durchschnitt} und die \textbf{Vereinigung}.\\[1em] 
		\pause
		Zwei Mengen $M$, $N$ heißen \textbf{disjunkt}, wenn ihr Durchschnitt leer ist, sie also keine gemeinsamen Elemente besitzen. $$M \cap N = \emptyset$$
	\end{block}

	\pause
	\begin{block}{Beispiel}
		$$ \{1,2\} \cup \{2,3\} = \{1,2,3\} \qquad \{1,2\} \cap \{2,3\} = \{2\} $$
	\end{block}

\end{frame}

\begin{frame}
	\frametitle{Schnitt und Vereinigung}
	\begin{block}{Definition}
		Seien $A$ und $B$ zwei beliebige Mengen, so gilt $$ A\setminus B = \left\{ x\in A \text{ und } x\notin  B  \right\} $$ 
	\end{block}
	
	\pause

	\begin{block}{Weitere Beispiele}
		\begin{align*}
			A \cup \emptyset \only<3->{ &= A }  \\
			A \cap \emptyset \only<4->{ &= \emptyset }\\
			\N_+ \cup \{0\} \only<5->{ &= \N_0} \\
		\end{align*}
	\end{block}

\end{frame}

\begin{frame}
	\frametitle{Eine Menge Mengen...}
	\begin{block}{Aufgabe}
		Es seien $A = \{1, 2\}, B = \{3\}, C = \{1, 3\}  \subseteq M = \{1, 2, 3\}$ Mengen.\\
		Man berechne folgende Mengen:
		\begin{align*}
		A \cup B &= \only<2->{ \{1, 2, 3\} }  \\
		A \cap C &= \only<3->{ \{1\} }\\
		A \setminus C &= \only<4->{ \{2\} }\\
		B \setminus A &= \only<5->{ \{3\} }\\
		A \cup (B \setminus C) &= \only<6->{ \{1, 2\} }\\
		C &= \{1, 3\} \\
		(A \setminus C) \cup B &= \only<7->{ \{2, 3\} }\\
		A \cap B &= \only<8->{ \emptyset }
		\end{align*}
	\end{block}
\end{frame}

\section{Potenzmengen}
\begin{frame}
	\frametitle{... in einer Menge! (Potenzmengen)}
	\begin{block}{Definition}
		Die \textbf{Potenzmenge} $2^M$ oder auch $\Pot (M)$ ist die Menge aller möglicher Teilmengen von $M$. Es gilt also 
		\begin{align*}
			2^M = \{N \mid N \subseteq M\}
		\end{align*}
	\end{block}
	\pause
	
	\begin{block}{Beispiel}
		Betrachten wir nun   $M = \left\{ 1,2,0 \right\} $. \\ \pause
		
		Dann gilt 
		\begin{align*}
		2^M &= \left\{ \emptyset, \left\{ 0 \right\}, \left\{ 1 \right\}, \left\{ 2 \right\}, \left\{ 0,1 \right\} , \left\{ 0,2 \right\}, \left\{ 1,2 \right\}, \left\{ 0,1,2 \right\} \right\}
		\end{align*}
		
		Beachte: Es gilt immer $M \in 2^M \ \text{und} \ \emptyset \in 2^M$
	\end{block}
	
\end{frame}

\begin{frame}
	\frametitle{Potenzmengen}
	\begin{block}{Aufgabe}
		Wie viele Elemente enthält $2^M$? \\[0.5em]
		
		\pause
		$2^{\mid M \mid }$
	\end{block}
	
	\pause
	\begin{block}{Aufgabe}
		Geben Sie eine Abbildung $\phi \colon 2^{M} \longrightarrow 2^{M}$ so an,
		dass für jedes $L \in 2^{M}$ und für jedes $w \in M$ gilt:
		\begin{equation*}
			w \in L \text{ genau dann, wenn } w \notin \phi(L).
		\end{equation*}
		
		\pause
		\begin{align*}
			\phi \colon 2^{M} &\longrightarrow 2^{M},\\
			L &\mapsto M \setminus L.
		\end{align*}
	\end{block}

\end{frame}

\begin{frame}
	\frametitle{Zurück zu unserem Problem}
	Ordnen wir diese in eine Teilmenge $M$ für Filme/Serien und jeweils eine Teilmenge $A_m$ für die Schauspieler eines Films $m \in M$. \\[0.5em]
	$M = \{$ The Imitation Game, Sherlock, Zurück in die Zunkunft $\}$ \\[0.3em]
	$A_{Sherlock} = \{$ Benedict Cumberbatch, Martin Freeman, Mark Gatiss $\}$ \\
	$A_{Imitation Game} = \{$ Benedict Cumberbatch, Keira Knightley $\}$ \\
	$A_{BTTF} = \{$ Michael J. Fox, Christopher Lloyd, Lea Thompson, Crispin Glover $\}$ \\[2em]
	
	\pause
	$A_{Sherlock} \cap A_{Imitation Game} = \{ \text{Benedict Cumberbatch} \}$
	
\end{frame}

\section{Paare}
\begin{frame}
	\frametitle{Paare}
	\begin{block}{Definition}
		Seien $A$ und $B$ zwei Mengen und $a \in A$, $b \in B$.\\
		$$(a, b)$$ heißt \textbf{Paar} mit der ersten Komponente $a$ und der zweiten Komponente $b$.\\[1em]
		\pause
		In Paaren können Elemente mehrfach vorkommen, und die Reihenfolge der Elemente ist wichtig.\\
	\end{block}

	\pause
	\begin{block}{Beispiel}
		$$ (KI, T) = (KI, T) \qquad (KI, T) \neq (T, KI) \qquad (1, 1) $$
	\end{block}
\end{frame}

\begin{frame}
	\frametitle{Paare}
	Das Konzept der Paare lässt sich auf das Konzept der Mengen zurückführen.
	
	\begin{block}{Aufgabe}
		Gegeben sei die Menge $M = \{m_1, m_2\}$.\\
		Wie kann man die Paare $(m_1, m_2)$ und $(m_2, m_1)$ eindeutig darstellen, nur unter Verwendung von Mengen und $m_1, m_2$?
	\end{block}
	\pause
	\begin{block}{Lösung}
		Wir definieren: \\
		$(m_1, m_2) := \{m_1, m_2, \{m_1\}\}$ und $(m_2, m_1) := \{m_1, m_2, \{m_2\}\}$
	\end{block}
\end{frame}