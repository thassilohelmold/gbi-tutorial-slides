\section{Wörter}

\begin{frame}
	\frametitle{Wörter}
	\begin{block}{Definition}
		\begin{itemize}
			\item Ein \textbf{Alphabet} ist eine endliche Menge von Zeichen. \pause
			\item Ein \textbf{Wort} $w$  über einem Alphabet A ist ein \textbf{endliche Folge von Zeichen} aus A \\ \pause
				\emph{Formal}: Eine surjektive Abbildung $f : \nZ_n \to A$\\[0.5em]
				Zur Erinnerung: $ \nZ_n = \{i\in \nN_0 \mid 0 \leq i < n \} $ 
		\end{itemize}
	\end{block}	

	\pause
	\begin{block}{Definition}
		Ist $A$ ein Alphabet, dann ist $A^*$ die \textbf{Menge aller Wörter}, die nur Zeichen aus $A$ enthalten, also:\\
		\pause 
		$A^*$ ist die Menge aller Abbildungen $w: \nZ_n \to B$ mit $n \in \nN_0$ und $B \subseteq A$. \\
	\end{block}

	\pause
	\begin{block}{Beispiel}
		Sei $ A = \{ a,b\} $ ein Alphabet. 
		Dann sind $ w_1 = aabbabab$ und $w_2 = ab $ zwei mögliche Wörter.
		Es gilt also $ w_1 \in A^*, w_2 \in A^*$
	\end{block}

\end{frame}

\begin{frame}
	\frametitle{Das Leere Wort}
	\begin{block}{Definition}
		Wir definieren das \textbf{leere Wort} als $$ \varepsilon := \nZ_0  \to A \qquad \varepsilon := \{\} \to \{\}$$ \pause
		Es gilt $ \varepsilon \circ w \circ \varepsilon = w \;$ (Beweis: VL) \\[1em] \pause
		
		Wichtig: Das leere Wort ist auch ein \enquote{echtes, gleichberechtigtes} Wort. Die Null ist bei den natürlichen Zahlen ja auch nicht einfach \enquote{nichts}. \\[1em] \pause
		
		Ist $\varepsilon := \{\} \to \{\}$ eine Relation? Und eine Funktion? Ist es surjektiv? \\ \pause
		Bemerkung: In der formalen Definition fordern wir die Surjektivität der Funktion, damit das leere Wort eindeutig ist!
	\end{block}
	
\end{frame}


\begin{frame}
	\frametitle{Konkatenation}
	\begin{block}{}
		Sei $w_1 = $ Schrank , $w_2 = $ Schlüssel \\
		Dann gilt $ w_1 \circ w_2 = \text{SchrankSchlüssel} \neq w_2 \circ w_1 = \text{SchlüsselSchrank}$\\ \pause
		Konkatenation ist also \textbf{nicht kommutativ!} \\
		Ist sie \textbf{assoziativ}? \pause Ja!		
	\end{block}

	\begin{block}{Beobachtung}
		Falls $w=w_1\circ w_2 $ und $w_1 \in A^* , w_2 \in B^* $, dann gilt
		$ w\in (A\cup B)^* $ 
	\end{block}

	\begin{block}{Mehrfachkonkatenation}
		\begin{align*}
			w^0 &= \varepsilon \\
			w^k &= \underbrace{w\circ w\circ \cdots \circ w}_{k-mal}
		\end{align*}

	\end{block}

\end{frame}

\begin{frame}
	\frametitle{Länge von Wörtern}
	\begin{block}{Definition}
		Unter der \textbf{Länge} eines Wortes versteht man die Anzahl der Zeichen, aus der das Wort besteht.
	\end{block}

	
	\begin{block}{Beispiel}
		$ \vert \text{hallo} \vert = \pause 5 $ \\
		$ \vert \varepsilon \vert = \pause 0 $
	\end{block}

	\pause
	\begin{block}{Lemma}
		$$ \vert a\circ b \vert = \vert a \vert + \vert b \vert $$
	\end{block}

	\pause
	\begin{block}{Lemma}
		$$ \vert w^k \vert = k \vert w \vert $$
	\end{block}
\end{frame}

\begin{frame}
	\frametitle{Wörter}
	$A^n$: \emph{Menge aller Wörter der Länge $n$} über dem Alphabet $A$.\\
	Wie kann man damit $A^*$ ausdrücken? \\
	\pause
	\[ A^\ast = \bigcup \limits_{i = 0}^\infty A^i \]
	
	\pause
	\begin{block}{Rückblick}
		\begin{align*}
		\bigcup_{i\in I} M_i &= \{ x \mid \text{es gibt ein } i\in I \text{ so, dass } x\in  M_i \}  \;
		\end{align*}
	\end{block}
\end{frame}

\begin{frame}
	\frametitle{Aufgabe}
	\begin{itemize}
		\item Welche Wörter lassen sich aus dem Alphabet $A = \{a , b \}$ bilden? Was enthält die Menge $A^*$?
		\item Ist das Wort $w =$ \code{aabb}$\,\cdot\,$\code{ba} ein Element der Menge $A^5$?
		\item Was ist $A^2 \times A^2$? \\
			Wir definieren die Abbildung $f : A^* \times A^* \to A^*, \; (w_1, w_2) \mapsto w_1 \cdot w_2$ \\
			Was ist $f(A^2 \times A^2)$?
	\end{itemize}
\end{frame}

\begin{frame}
	\frametitle{Lösung}
	\textit{Welche Wörter lassen sich aus dem Alphabet $A = \{a , b \}$ bilden? Was enthält die Menge $A^*$?} \\[2em] \pause
	
	Aus $A$ lassen sich z.B. die Wörter $$\text{a, b, aa, bb, ab, ba, aaa, bbb, }\dots$$ bilden. Die Menge $A^*$ enthält gerade diese Wörter.\\
	\pause
	Beachte: Auch $\varepsilon$ ist in $A^*$!
\end{frame}

\begin{frame}
	\frametitle{Lösung}
	\textit{Ist das Wort $w =$ \code{aabb}$\,\cdot\,$\code{ba} ein Element der Menge $A^5$?} \\[2em] \pause
	
	Nein. Es gilt $w = \textbf{aabbba}$. Das Wort besteht zwar aus Symbolen, die alle in $A$ liegen, ist aber 6 Zeichen lang.
\end{frame}

\begin{frame}
	\frametitle{Lösung}
	\textit{Was ist $A^2 \times A^2$? \\
		Wir definieren die Abbildung $f : A^* \times A^* \to A^*, \; (w_1, w_2) \mapsto w_1 \cdot w_2$ \\
		Was ist $f(A^2 \times A^2)$?} \\[2em]  \pause
	
	 $$ A^2 \times A^2 = \{\mathbf{(aa,aa),(aa,bb),(aa,ab),(aa,ba),(bb,aa),} \dots \}$$
	 \pause
	 $$ f(A^2 \times A^2) = \{\mathbf{aaaa, aabb, aaab, aaba, bbaa,} \dots \} = A^4 $$
\end{frame}