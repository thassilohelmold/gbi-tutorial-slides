%% Collection of stuff that might be deprecated or just came out of use. Saved here for later (possible) revival.

%%%%%%%%%%%%%%%%%%

%%%%%%%%%%%%%%%%%%

%%%%%%%%%%%%%%%%%%

%%%%%%%%%%%%%%%%%%

%%%%%%%%%%%%%%%%%%

\daniel{
	\begin{frame}{Achtung, Tut 36: Feiertag!}
		\begin{itemize}
			\item Nächsten Donnerstag (01. November) ist Feiertag
			\item Daher: \textbf{Tut 36 fällt aus}!
			\item Aber: Themen werden \textbf{NICHT} nachgeholt (Folien im ILIAS)!
			\item Bitte anderes Tutorium besuchen 
		\end{itemize}		 
	\end{frame}
}

%%%%%%%%%%%%%%%%%%

% From gbitut01:

\thasse{ % No time.
	\begin{frame}{Grundbegriffe?}
		\begin{itemize}
			\item Cloud
			\item Künstliche Intelligenz
			\item Compilerbau
			\item Big Data
			\item Apps
			\item Roboter
			\item Industrie~4.0
		\end{itemize}
		
		\pause
		Kommt alles noch... Aber nicht in GBI.
	\end{frame}
	
	\begin{frame}{Grundbegriffe!}
		\begin{itemize}
			\item Mengen, Relationen, Funktionen
			\item Wörter, Sprachen, Grammatiken
			\item Graphen
			\item Algorithmen
			\item Turingmaschinen
			\item Logik
			\item Prozessoren
			\item ...
		\end{itemize}
		\impl You have been warned. :P
	\end{frame}
}

%%%%%%%%%%%%%%%%%%

\begin{frame}{Zu Blatt \#4}
	Durchschnitt: \quad etwa \thassedaniel{86 (58)}{85 (66)}~\% der Punkte (s. Bonuspunkte!)
	\begin{itemize}
		\item \textbf{A4.3}: Ja, LDC $x$ für Speicherstelle $x$ geht (und ist notwendig)!
		\item \textbf{A4.5}: MuLö war falsch (Übung auch!), wurde aktualisiert. \\
			  \textbf{Aufgabe}: \; AL-Fml. $G$ in PL-Fml. $P$ bedeutungserhaltend übersetzen. \\ 
			  \pause
			  \textbf{Falsche MuLö}: \; $I(\word c) = \W$ \\
			  \quad Setze $P := G \pleq \word c$ \qquad Bsp.: \; $G = \plka \word A \aland \alnot \word B \plkz \impl P := \plka \word A \aland \alnot \word B \plkz \pleq \word c$ \\ 
				  \pause 
				  \quad \impl Kaputt, weil nur Terme in $\pleq$ eingesetzt werden dürfen! \\
			  \pause
			  \textbf{Möglicher Fix}: \; Ersetze jede AL-Variable durch Vergleich mit \word c: \\
			  \quad $G = \word A \aland \alnot \word B \; \?> \; P := \plka \word A \pleq \word c\plkz \aland \alnot \plka \word B \pleq \word c\plkz$ \\
			  Alternativ: S. aktualisierte MuLö.
	\end{itemize}
\end{frame}

%%%%%%%%%%%%%%%%%%

\begin{frame}{Zu Blatt \#5}
	\textbf{Durchschnitt}: \quad etwa \thassedaniel{14}{16}~/~25~P \quad ($\hat{=}$ \thassedaniel{56}{64}~\%) \\
	\begin{itemize}
		\item \textbf{A5.1}: $\plall \plx \plka \word{Vogel(x)} \aland \alnot \word{schwimmfähig(x)}\plkz$... \bfalert{falsch}, \\
		Richtig: $\plall \plx \plka \word{Vogel(x)} \underline{\alimpl} \alnot \word{schwimmfähig(x)}\plkz$...
		\item \textbf{A5.2}: Einfache Interpretationen sind hervorragend! (z.~B. \textbf{leere} Relationen!) Grade in der Klausur! 
		\item \textbf{A5.5b)}: $N = \set{L_i \Mid i \elem \N_0}$ geht \bfalert{NICHT!} $N$ muss endlich sein! \\
		(War sowieso unlösbar...)
	\end{itemize}
\end{frame}

%%%%%%%%%%%%%%%%%%

Wenn ihr eine Formel aufstellt, dann \textbf{prüft das Ergebnis für kleine Werte}.\\[0.2em]
90~\% der falschen Antworten in 1.6.c) haben schon für $n \in \{0, 1\}$ nicht gestimmt.\\[0.5em]

%------ Weihnachtsgebäck ------
\mycomment{ % Kam letztes Jahr nicht gut an, ist wohl etwas "albern". - TH
	\begin{frame}{Zum Aufwärmen: Weihnachtsgebäck}
		Weihnachten steht vor der Tür, aber Geschenke sind von Jahr zu Jahr teurer geworden! Um den Staatshaushalt des Nordpols zu schonen, hat sich der Weihnachtsmann etwas einfallen lassen: Dieses Jahr bringt er nur den Haushalten Geschenke, bei denen das für ihn bereitgestellte Gebäck seinen speziellen Qualitätsanforderungen genügt.\\[1em]
		
		Dazu hat der Weihnachtsmann drei Gebäckarten (Lebkuchen $L$, Zimtstern $Z$, Dominostein $D$) normiert und eine kontextfreie Grammatik aufgestellt, die gültige Kombinationen beschreibt.
	\end{frame}
	
	\begin{frame}{Zum Aufwärmen: Weihnachtsgebäck}
		\begin{align*}
		G &= (\{S, A, B, X, Y\}, \{L, Z, D\}, S, \{ \\
		S &\to A \mid B\\
		A &\to LLLLLLLLLLX, \qquad X \to LX \mid e\\
		B &\to ZYZ \mid DYD, \qquad Y \to Z \mid D \mid L\\
		\})
		\end{align*}
		\begin{enumerate}
			\item Welche Kombinationen sind gültig?\\[0.5em]
			\item Der Weihnachtsmann muss auf seine Figur achten. Deswegen möchte er für nächstes Jahr eine neue, kalorienarme Kombination von Gebäck:\\
			3 Folgen mit 42, 17 und 41 x ein Zimtstern oder ein Dominostein. Die einzelnen Folgen sollen durch Lebkuchen getrennt werden.\\
			Stelle eine Kontextfreie Grammatik dafür auf.
		\end{enumerate}
	\end{frame}
	
	\begin{frame}{Zum Aufwärmen: Weihnachtsgebäck}
		\begin{align*}
		G &= (\{S, A, B, X, Y\}, \{L, Z, D\}, S, \{ \\
		S &\to A \mid B\\
		A &\to LLLLLLLLLLX, \qquad X \to LX \mid e\\
		B &\to ZYZ \mid DYD, \qquad Y \to Z \mid D \mid L\\
		\})
		\end{align*}
		
		Welche Kombinationen sind gültig?\\[0.5em] \pause
		$\{ L^n \mid n \geq 10\} \cup \{w \mid \text{w ist Palindrom und } |w|_L \leq 1\}$
	\end{frame}
	
	\begin{frame}{Zum Aufwärmen: Weihnachtsgebäck}
		Der Weihnachtsmann muss auf seine Figur achten. Deswegen möchte er für nächstes Jahr eine neue, kalorienarme Kombination von Gebäck:\\
		3 Folgen mit 42, 17 und 41 x ein Zimtstern oder ein Dominostein. Die einzelnen Folgen sollen durch Lebkuchen getrennt werden.\\
		Stelle eine Kontextfreie Grammatik dafür auf.
		
		\pause
		\begin{align*}
		G &= (\{S\} \cup \{F_i \mid 0 \leq i \leq 42 \}, \{L, Z, D\}, S, P)\\
		P &= \{S \to F_{42}LF_{17}LF_{41} \}\\
		&\cup \{F_i \to ZF_{i-1}, F_i \to DF_{i-1} \mid 1 \leq i \leq 42 \} \\
		&\cup \{F_0 \to \eps \}
		\end{align*}
	\end{frame}
}

%------------------------------

