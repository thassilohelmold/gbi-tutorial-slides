%beamer

% Comment/uncomment this line to toggle handout mode
%\newcommand{\handout}{}

%% Beamer-Klasse im korrekten Modus
\ifdefined \handout
\documentclass[handout]{beamer} % Handout mode
\else
\documentclass{beamer}
\fi

%% UTF-8-Encoding
\usepackage[utf8]{inputenc}

% % \bigtimes abgeschrieben von http://tex.stackexchange.com/questions/14386/importing-a-single-symbol-from-a-different-font
% \DeclareFontFamily{U}{mathx}{\hyphenchar\font45}
% \DeclareFontShape{U}{mathx}{m}{n}{
%       <5> <6> <7> <8> <9> <10> gen * mathx
%       <10.95> mathx10 <12> <14.4> <17.28> <20.74> <24.88> mathx12
%       }{}
% \DeclareSymbolFont{mathx}{U}{mathx}{m}{n}
% \DeclareMathSymbol{\bigtimes}{\mathop}{mathx}{161}

\RequirePackage{xcolor}

\def\9{\square}
%\def\9{\blank}

% f"ur Aussagenlogik
\colorlet{alcolor}{blue}
\RequirePackage{tikz}
\usetikzlibrary{arrows.meta}
\newcommand{\alimpl}{\mathrel{\tikz[x={(0.1ex,0ex)},y={(0ex,0.1ex)},>={Classical TikZ Rightarrow[]}]{\draw[alcolor,->,line width=0.7pt,line cap=round] (0,0) -- (15,0);\path (0,-6);}}}
\newcommand{\aleqv}{\mathrel{\tikz[x={(0.1ex,0ex)},y={(0ex,0.1ex)},>={Classical TikZ Rightarrow[]}]{\draw[alcolor,<->,line width=0.7pt,line cap=round] (0,0) -- (18,0);\path (0,-6);}}}
\newcommand{\aland}{\mathbin{\raisebox{-0.6pt}{\rotatebox{90}{\texttt{\color{alcolor}\char62}}}}}
\newcommand{\alor}{\mathbin{\raisebox{-0.8pt}{\rotatebox{90}{\texttt{\color{alcolor}\char60}}}}}
%\newcommand{\ali}[1]{_{\mathtt{\color{alcolor}#1}}}
\newcommand{\alv}[1]{\mathtt{\color{alcolor}#1}}
\newcommand{\alnot}{\mathop{\tikz[x={(0.1ex,0ex)},y={(0ex,0.1ex)}]{\draw[alcolor,line width=0.7pt,line cap=round,line join=round] (0,0) -- (10,0) -- (10,-4);\path (0,-8) ;}}}
\newcommand{\alP}{\alv{P}} %ali{#1}}
%\newcommand{\alka}{\negthinspace\hbox{\texttt{\color{alcolor}(}}}
\newcommand{\alka}{\negthinspace\text{\texttt{\color{alcolor}(}}}
%\newcommand{\alkz}{\texttt{\color{alcolor})}}\negthinspace}
\newcommand{\alkz}{\text{\texttt{\color{alcolor})}}\negthinspace}
\newcommand{\AAL}{A_{AL}}
\newcommand{\LAL}{\hbox{\textit{For}}_{AL}}
\newcommand{\AxAL}{\hbox{\textit{Ax}}_{AL}}
\newcommand{\AxEq}{\hbox{\textit{Ax}}_{Eq}}
\newcommand{\AxPL}{\hbox{\textit{Ax}}_{PL}}
\newcommand{\AALV}{\hbox{\textit{Var}}_{AL}}
\newcommand{\MP}{\hbox{\textit{MP}}}
\newcommand{\GEN}{\hbox{\textit{GEN}}}
\newcommand{\W}{\ensuremath{\hbox{\textbf{w}}}\xspace}
\newcommand{\F}{\ensuremath{\hbox{\textbf{f}}}\xspace}
\newcommand{\WF}{\ensuremath{\{\W,\F\}}\xspace}
\newcommand{\val}{\hbox{\textit{val}}}
\newcommand{\valDIb}{\val_{D,I,\beta}}

\newcommand*{\from}{\colon}

% die nachfolgenden Sachen angepasst an cmtt
\newlength{\ttquantwd}
\setlength{\ttquantwd}{1ex}
\newlength{\ttquantht}
\setlength{\ttquantht}{6.75pt}
\def\plall{%
  \tikz[line width=0.67pt,line cap=round,line join=round,baseline=(B),alcolor] {
    \draw (-0.5\ttquantwd,\ttquantht) -- node[coordinate,pos=0.4] (lll){} (-0.25pt,-0.0pt) -- (0.25pt,-0.0pt) -- node[coordinate,pos=0.6] (rrr){} (0.5\ttquantwd,\ttquantht);
    \draw (lll) -- (rrr);
    \coordinate (B) at (0,-0.35pt);
  }%
}
\def\plexist{%
  \tikz[line width=0.67pt,line cap=round,line join=round,baseline=(B),alcolor] {
    \draw (-0.9\ttquantwd,\ttquantht) -- (0,\ttquantht) -- node[coordinate,pos=0.5] (mmm){} (0,0) --  (-0.9\ttquantwd,0);
    \draw (mmm) -- ++(-0.75\ttquantwd,0);
    \coordinate (B) at (0,-0.35pt);
  }\ensuremath{\,}%
}
\let\plexists=\plexist
\newcommand{\NT}[1]{\ensuremath{\langle\mathrm{#1} \rangle}}

\newcommand{\CPL}{\text{\itshape Const}_{PL}}
\newcommand{\FPL}{\text{\itshape Fun}_{PL}}
\newcommand{\RPL}{\text{\itshape Rel}_{PL}}
\newcommand{\VPL}{\text{\itshape Var}_{PL}}
\newcommand{\ATer}{A_{\text{\itshape Ter}}}
\newcommand{\ARel}{A_{\text{\itshape Rel}}}
\newcommand{\AFor}{A_{\text{\itshape For}}}
\newcommand{\LTer}{L_{\text{\itshape Ter}}}
\newcommand{\LRel}{L_{\text{\itshape Rel}}}
\newcommand{\LFor}{L_{\text{\itshape For}}}
\newcommand{\NTer}{N_{\text{\itshape Ter}}}
\newcommand{\NRel}{N_{\text{\itshape Rel}}}
\newcommand{\NFor}{N_{\text{\itshape For}}}
\newcommand{\PTer}{P_{\text{\itshape Ter}}}
\newcommand{\PRel}{P_{\text{\itshape Rel}}}
\newcommand{\PFor}{P_{\text{\itshape For}}}

\newcommand{\plka}{\alka}
\newcommand{\plkz}{\alkz}
%\newcommand{\plka}{\plfoo{(}}
%\newcommand{\plkz}{\plfoo{)}}
\newcommand{\plcomma}{\hbox{\texttt{\color{alcolor},}}}
\newcommand{\pleq}{{\color{alcolor}\,\dot=\,}}

% MODIFIED (DJ)
% previously: \newcommand{\plfoo}[1]{\mathtt{\color{alcolor}#1}}
\newcommand{\plfoo}[1]{\texttt{\color{alcolor}#1}}

\newcommand{\plc}{\plfoo{c}}
\newcommand{\pld}{\plfoo{d}}
\newcommand{\plf}{\plfoo{f}}
\newcommand{\plg}{\plfoo{g}}
\newcommand{\plh}{\plfoo{h}}
\newcommand{\plx}{\plfoo{x}}
\newcommand{\ply}{\plfoo{y}}
\newcommand{\plz}{\plfoo{z}}
\newcommand{\plR}{\plfoo{R}}
\newcommand{\plS}{\plfoo{S}}

\newcommand{\bv}{\mathrm{bv}}
\newcommand{\fv}{\mathrm{fv}}

%\newcommand{\AxAL}{\hbox{\textit{Ax}}_{AL}}
%\newcommand{\AALV}{\hbox{\textit{Var}}_{AL}}

%\renewcommand{\#}[1]{\literal{#1}}
\newcommand{\A}{\mathcal{A}}
\newcommand{\Adr}{\text{Adr}}
\newcommand{\ar}{\mathrm{ar}}
\newcommand{\ascii}[1]{\literal{\char#1}}
%\newcommand{\assert}[1]{\text{/\!\!/\ } #1}
\newcommand{\assert}[1]{\colorbox{black!7!white}{\ensuremath{\{\;#1\;\}}}}
\newcommand{\Assert}[1]{$\langle$\textit{#1}$\rangle$}
\newcommand{\B}{\mathcal{B}}
\newcommand{\bfmod}{\mathbin{\kw{ mod }}}
\newcommand{\bb}{{\text{bb}}}
\def\bottom{\hbox{\small$\pmb{\bot}$}}
\newcommand{\card}[1]{|#1|}
%\newcommand{\cod}{\mathop{\text{cod}}}  % ist in thwmathabbrevs
\newcommand{\Conf}{\mathcal{C}}
\newcommand{\define}[1]{\emph{#1}}
%\renewcommand{\dh}{d.\,h.\@\xspace}
%\newcommand{\Dh}{D.\,h.\@\xspace}
%\newcommand{\engl}[1]{engl.\xspace\emph{#1}}
\newcommand{\eps}{\varepsilon}
%\newcommand{\evtl}{evtl.\@\xspace}
\newcommand{\fbin}{\text{bin}}
\newcommand{\finv}{\text{inv}}
\newcommand{\fnum}{\text{num}}
\newcommand{\fNum}{{\text{Num}}}
\newcommand{\frepr}{\text{repr}}
\newcommand{\fRepr}{\text{Repr}}
\newcommand{\fZkpl}{\text{Zkpl}}
\newcommand{\fLen}{\text{Len}}
\newcommand{\fsem}{\text{sem}}
\providecommand{\fspace}{\mathord{\text{space}}}
\providecommand{\fSpace}{\mathord{\text{Space}}}
\providecommand{\ftime}{\mathord{\text{time}}}
\providecommand{\fTime}{\mathord{\text{Time}}}
\newcommand{\fTrans}{\text{Trans}}
\newcommand{\fVal}{\text{Val}}

% MODIFIED (DJ)
\newcommand{\Val}{\text{Val}}

%\def\G{\mathbb{Z}}
\newcommand{\HT}[1]{\normalfont\textsc{HT-#1}}
\newcommand{\htr}[3]{\{#1\}\;#2\; \{#3\}}
\newcommand{\Id}{\text{I}}
%\newcommand{\ie}{i.\,e.\@\xspace}
\newcommand{\instr}[2]{\texttt{#1}\ \textit{#2}}
\newcommand{\Instr}[2]{\texttt{#1}\ \textrm{#2}}
\newcommand{\instrr}[3]{\texttt{#1}\ \textit{#2}\texttt{(#3)}}
\newcommand{\Instrr}[3]{\texttt{#1}\ \textrm{#2}\texttt{(#3)}}
\newcommand{\io}{\!\mid\!}
\usepackage{KITcolors}
\newcommand{\literal}[1]{\hbox{\textcolor{blue!95!white}{\textup{\texttt{\scalebox{1.11}{#1}}}}}}
%\newcommand{\literal}[1]{\hbox{\textcolor{KITblue!80!black}{\textup{\texttt{#1}}}}}
\def\kasten#1{\leavevmode\literal{\setlength{\fboxsep}{1pt}\fbox{\vrule  width 0pt height 1.5ex depth 0.5ex #1}}}
\newcommand{\kw}[1]{\ensuremath{\mathbf{#1}}}
\newcommand{\lang}[1]{\ensuremath{\langle#1\rangle}}
%\newcommand{\maw}{m.\,a.\,w.\@\xspace}
%\newcommand{\MaW}{M.\,a.\,w.\@\xspace}
\newcommand{\mdefine}[2][FOOBAR]{\define{#2}\def\foobar{FOOBAR}\def\optarg{#1}\ifx\foobar\optarg\def\optarg{#2}\fi\graffito{\optarg}}
\newcommand{\meins}{\rotatebox[origin=c]{180}{1}}
\newcommand{\Mem}{\text{Mem}}
\newcommand{\memread}{\text{memread}}
\newcommand{\memwrite}{\text{memwrite}}
\providecommand{\meta}[1]{\ensuremath{\langle}\textit{#1}\ensuremath{\rangle}}
%\newcommand{\N}{\mathbb{N}}
\newcommand{\NP}{\mathbf{NP}}
\newcommand{\Nadd}{N_{\text{add}}}
\newcommand{\Nmult}{N_{\text{mult}}}
% MODIFIED (DJ): added \!, mathcal{O}
\newcommand{\Oh}[1]{\mathcal{O}\!\left(#1\right)}
\newcommand{\Om}[1]{\Omega\!\left(#1\right)}
\newcommand{\personname}[1]{\textsc{#1}}
\newcommand{\regname}[1]{\texttt{#1}}
\newcommand{\mima}{\textsc{Mima}\xspace}
\newcommand{\mimax}{\textsc{Mima-X}\xspace}

\def\Pclass{\text{\bfseries P}}
\def\PSPACE{\text{\bfseries PSPACE}}

\newcommand{\SPush}{\text{push}}
\newcommand{\SPop}{\text{pop}}
\newcommand{\SPeek}{\text{peek}}
\newcommand{\STop}{\text{top}}
\newcommand{\STos}{\text{\itshape tos}}
\newcommand{\SBos}{\text{\itshape bos}}

%\newcommand{\R}{\mathbb{R}}
\newcommand{\Rnullplus}{\R_0^{+}}
\newcommand{\Rplus}{\R_{+}}
\newcommand{\resp}{resp.\@\xspace}
\newcommand{\Sem}{\text{Sem}}
\newcommand{\sgn}{\mathop{\text{sgn}}}
\newcommand{\sqbox}{\mathop{\raisebox{-6.2pt}{\hbox{\hbox to 0pt{$^{^{\sqcap}}$\hss}$^{^{\sqcup}}$}}}}
\newcommand{\sqleq}{\sqsubseteq}
\newcommand{\sqgeq}{\sqsupseteq}
% MODIFIED (DJ): added \!
\newcommand{\Th}[1]{\Theta\!\left(#1\right)}
%\newcommand{\usw}{usw.\@\xspace}
\newcommand{\V}[1]{\hbox{\textit{#1}}}
\newcommand{\x}{\times}
\newcommand{\ZK}{\mathbb{K}}
%\newcommand{\Z}{\mathbb{Z}}
\newcommand{\zB}{z.\,B.\@\xspace}
\newcommand{\ZB}{Z.\,B.\@\xspace}
% \newcommand{\bb}{{\text{bb}}}
% \def\##1{\hbox{\textcolor{darkblue}{\texttt{#1}}}}
% \def\A{\mathcal{A}}
% \newcommand{\0}{\#0}
% \newcommand{\1}{\#1}
% \newcommand{\Obj}{\text{Obj}}
% \newcommand{\start}{\mathop{\text{start}}}
% \newcommand{\compactlist}{\addtolength{\itemsep}{-\parskip}}
% \newcommand{\fval}{\text{val}}
% \newcommand{\lang}[1]{\ensuremath{\langle#1\rangle}}
% \newcommand{\io}{\!\mid\!}
% \def\sqbox{\mathop{\raisebox{-6.2pt}{\hbox{\hbox to 0pt{$^{^{\sqcap}}$\hss}$^{^{\sqcup}}$}}}}
% \def\sqleq{\sqsubseteq}
% \def\sqgeq{\sqsupseteq}
\def\Td{T_{\overline{d}}}
% \newcommand{\csym}[1]{\ensuremath{\#{c}_{\#{\hbox{\scriptsize #1}}}}}
% \newcommand{\F}{\ensuremath{\mathcal{F}}}
% \newcommand{\fsym}[2]{\ensuremath{\#{f}^{\#{\hbox{\scriptsize #1}}}_{\#{\hbox{\scriptsize #2}}}}}
% \newcommand{\rsym}[2]{\ensuremath{\#{R}^{\#{\hbox{\scriptsize #1}}}_{\#{\hbox{\scriptsize #2}}}}}
% \newcommand{\xsym}[1]{\ensuremath{\#{x}_{\#{\hbox{\scriptsize #1}}}}}
% \newcommand{\I}{\mathcal{I}}
% ********************************************************************

\usepackage[blue]{../framework/thwregex}
\usepackage{environ}
\usepackage{bm}
\usepackage{calc}
\usepackage{varwidth}
\usepackage{wasysym}
\usepackage{mathtools}


% Das ist der KIT-Stil
%\usepackage{../TutTexbib/beamerthemekit}
\usepackage[deutsch,titlepage0]{../framework/KIT/beamerthemeKITmod}
\TitleImage[width=\titleimagewd]{../figures/titlepage.jpg}
%\usetheme[deutsch,titlepage0]{KIT}

% Include PDFs
\usepackage{pdfpages}

% Libertine font (Original GBI font)
\usepackage{libertine}
%\renewcommand*\familydefault{\sfdefault}  %% Only if the base font of the document is to be sans serif

% Nicer math symbols
\usepackage{eulervm}
%\usepackage{mathpazo}
\renewcommand\ttdefault{cmtt} % Computer Modern typewriter font, see lecture slides.

\usepackage{csquotes}

%%%%%%

%% Schönere Schriften
\usepackage[TS1,T1]{fontenc}

%% Bibliothek für Graphiken
\usepackage{graphicx}

%% der wird sowieso in jeder Datei gesetzt
\graphicspath{{../figures/}}

%% Anzeigetiefe für Inhaltsverzeichnis: 1 Stufe
\setcounter{tocdepth}{1}

%% Hyperlinks
\usepackage{hyperref}
% I don't know why, but this works and only includes sections and NOT subsections in the pdf-bookmarks.
\hypersetup{bookmarksdepth=subsection} 

%\usepackage{lmodern}
\usepackage{colortbl}
\usepackage[absolute,overlay]{textpos}
\usepackage{listings}
\usepackage{forloop}
%\usepackage{algorithmic} % PseudoCode package 

\usepackage{tikz}
\usetikzlibrary{matrix}
\usetikzlibrary{arrows.meta}
\usetikzlibrary{automata}
\usetikzlibrary{tikzmark}

% Needed for gbi-macros
\usepackage{xspace}

%%%%%%

%% Verbatim
\usepackage{moreverb}

%%%%%%%%%%%%%%%%%%%%%%%%%%%%%%%%%%%% Copy end

%% Tabellen
\usepackage{array}
\usepackage{multicol}
\usepackage{hhline}

%% Bibliotheken für viele mathematische Symbole
\usepackage{amsmath, amsfonts, amssymb}

%% Deutsche Silbentrennung und Beschriftungen
\usepackage[ngerman]{babel}

\usepackage{kbordermatrix}

% kbordermatrix settings
\renewcommand{\kbldelim}{(} % Left delimiter
\renewcommand{\kbrdelim}{)} % Right delimiter

\input{../config.tex}



% define custom \handout command flag if handout mode is toggled  #DirtyAsHellButWell...
\only<beamer:0>{\def\handout{}} %beamer:0 == handout mode

\newcommand{\R}{\mathbb{R}}
\newcommand{\N}{\mathbb{N}}
\newcommand{\Z}{\mathbb{Z}}
\newcommand{\Q}{\mathbb{Q}}
\newcommand{\BB}{\mathbb{B}}
\newcommand{\C}{\mathbb{C}}
\newcommand{\K}{\mathbb{K}}
\newcommand{\G}{\mathbb{G}}
\newcommand{\nullel}{\mathcal{O}}
\newcommand{\einsel}{\mathds{1}}
\newcommand{\Pot}{\mathcal{P}}
\renewcommand{\O}{\text{O}}

\def\word#1{\hbox{\textcolor{blue}{\texttt{#1}}}}
\let\literal\word
\def\mword#1{\hbox{\textcolor{blue}{$\mathtt{#1}$}}}  % math word
\def\sp{\scalebox{1}[.5]{\textvisiblespace}}
\def\wordsp{\word{\sp}}

%\newcommand{\literal}[1]{\textcolor{blue}{\texttt{#1}}}
\newcommand{\realTilde}{\textasciitilde \ }
\newcommand{\setsize}[1]{\ensuremath{\left\lvert #1 \right\rvert}}
\newcommand{\size}[1]{\setsize{#1}}  % Shame on you, TeXStudio...
\newcommand{\set}[1]{\left\{#1\right\}}
\newcommand{\tuple}[1]{\left(#1\right)}
\newcommand{\normalvar}[1]{\text{$#1$}}

% Modified by DJ
\let\oldemptyset\emptyset
\let\emptyset\varnothing % proper emptyset

\newcommand{\boder}{\ensuremath{\mathbin{\textcolor{blue}{\vee}}}\xspace}
\newcommand{\bund}{\ensuremath{\mathbin{\textcolor{blue}{\wedge}}}\xspace}
\newcommand{\bimp}{\ensuremath{\mathrel{\textcolor{blue}{\to}}}\xspace}
\newcommand{\bgdw}{\ensuremath{\mathrel{\textcolor{blue}{\leftrightarrow}}}\xspace}
\newcommand{\bnot}{\ensuremath{\textcolor{blue}{\neg}}\xspace}
\newcommand{\bone}{\ensuremath{\textcolor{blue}{1}}\text{}}
\newcommand{\bzero}{\ensuremath{\textcolor{blue}{0}}\text{}}
\newcommand{\bleftBr}{\ensuremath{\textcolor{blue}{\texttt{(}}}\text{}}
\newcommand{\brightBr}{\ensuremath{\textcolor{blue}{\texttt{)}}}\text{}}

% Fix of \b... commands:

\renewcommand{\boder}{\alor}
\renewcommand{\bund}{\aland}
\renewcommand{\bimp}{\alimpl}
\renewcommand{\bgdw}{\aleqv}
\renewcommand{\bnot}{\alnot}
\renewcommand{\bleftBr}{\alka}
\renewcommand{\brightBr}{\alkz}
\newcommand{\alA}{\word A}
\newcommand{\alB}{\word B}
\newcommand{\alC}{\word C}

\newcommand{\plB}{\plfoo{B}}
\newcommand{\plE}{\plfoo{E}}

\newcommand{\summe}[2]{\sum\limits_{#1}^{#2}}
\newcommand{\limes}[1]{\lim\limits_{#1}}

%\newcommand{\numpp}{\advance \value{weeknum} by -2 \theweeknum \advance \value{weeknum} by 2}
%\newcommand{\nump}{\advance \value{weeknum} by -1 \theweeknum \advance \value{weeknum} by 1}

\newcommand{\mycomment}[1]{}
\newcommand{\Comment}[1]{}

%% DISCLAIMER START 
% It is INSANELY IMPORTANT NOT TO DO THIS OUTSIDE BEAMER CLASS! IN ARTCILE DOCUMENTS, THIS IS VERY LIKELY TO BUG AROUND!
\makeatletter%
\@ifclassloaded{beamer}%
{
	% TODO 
	% no time...
	% redefine section to ignore multiple \section calls with the same title
}%
{
	\errmessage{ERROR: section command redefinition outside of beamer class document! Please contact the author of this code.}
}%
\makeatother%
%% DISCLAIMER END

\newcounter{abc}
\newenvironment{alist}{
  \begin{list}{(\alph{abc})}{
      \usecounter{abc}\setlength{\leftmargin}{8mm}\setlength{\labelsep}{2mm}
    }
}{\end{list}}


\newcommand{\stdarraystretch}{1.20}
\renewcommand{\arraystretch}{\stdarraystretch}  % for proper row spacing in tables

\newcommand{\morescalingdelimiters}{   % for proper \left( \right) typography
	\delimitershortfall=-1pt  
	\delimiterfactor=1
}

\newcommand{\centered}[1]{\vspace{-\baselineskip}\begin{center}#1\end{center}\vspace{-\baselineskip}}

% for \implitem and \item[bla] stuff to look right:
\setbeamercolor*{itemize item}{fg=black}
\setbeamercolor*{itemize subitem}{fg=black}
\setbeamercolor*{itemize subsubitem}{fg=black}

\setbeamercolor*{description item}{fg=black}
\setbeamercolor*{description subitem}{fg=black}
\setbeamercolor*{description subsubitem}{fg=black}

\renewcommand{\qedsymbol}{\textcolor{black}{\openbox}}

\renewcommand{\mod}{\mathop{\textbf{mod}}}
\renewcommand{\div}{\mathop{\textbf{div}}}

\newcommand{\ceil}[1]{\left\lceil#1\right\rceil}
\newcommand{\floor}[1]{\left\lfloor#1\right\rfloor}
\newcommand{\abs}[1]{\left\lvert #1 \right\rvert}
\newcommand{\Matrix}[1]{\begin{pmatrix} #1 \end{pmatrix}}
\newcommand{\braced}[1]{\left\lbrace #1 \right\rbrace}

% "something" placeholder. Useful for repairing spacing of operator sections, like `\sth = 42`.
\def\sth{\vphantom{.}}

\def\fract#1/#2 {\frac{#1}{#2}} % ! Trailing space is crucial!
\def\dfract#1/#2 {\dfrac{#1}{#2}} % ! Trailing space is crucial!

\newcommand{\Mid}{\;\middle|\;}

\let\after\circ



\def\·{\cdot}
\def\*{\cdot}
\def\?>{\ensuremath{\rightsquigarrow}}  % Fuck you, Latex
\def\~~>{\ensuremath{\rightsquigarrow}}  

\newcommand{\tight}[1]{{\renewcommand{\arraystretch}{0.76} #1}}
\newcommand{\stackedtight}[1]{\renewcommand{\arraystretch}{0.76} \begin{matrix} #1 \end{matrix} }
\newcommand{\stacked}[1]{\begin{matrix} #1 \end{matrix} }
\newcommand{\casesl}[1]{\delimitershortfall=0pt  \left\lbrace\hspace{-.3\baselineskip}\begin{array}{ll} #1 \end{array}\right.}
\newcommand{\casesr}[1]{\delimitershortfall=0pt  \left.\begin{array}{ll} #1 \end{array}\hspace{-.3\baselineskip}\right\rbrace}
\newcommand{\caseslr}[1]{\delimitershortfall=0pt  \left\lbrace\hspace{-.3\baselineskip}\begin{array}{ll} #1 \end{array}\hspace{-.3\baselineskip}\right\rbrace}

\def\q#1uad{\ifnum#1=0\relax\else\quad\q{\the\numexpr#1-1\relax}uad\fi}
% e.g. \q1uad = \quad, \q2uad = \qquad etc.

\newcommand{\qqquad}{\q3uad}
\newcommand{\minusquad}{\hspace{-1em}}

%% Placeholder utils
% \§{#1}   Saves #1 as placeholder and prints it
% \.       Prints an \hphantom with the size of the recalled placeholder.
\def\indentstring{}
\def\§#1{\def\indentstring{#1}#1}
\def\.{{$\hphantom{\text{\indentstring}}$}}
%% Placeholder utils end

\newcommand{\impl}{\ifmmode\ensuremath{\mskip\thinmuskip\Rightarrow\mskip\thinmuskip}\else$\Rightarrow$\fi\xspace}
\newcommand{\Impl}{\ifmmode\implies\else$\Longrightarrow$\fi\xspace}

\newcommand{\derives}{\Rightarrow}

\newcommand{\gdw}{\ifmmode\mskip\thickmuskip\Leftrightarrow\mskip\thickmuskip\else$\Leftrightarrow$\fi\xspace}
\newcommand{\Gdw}{\ifmmode\iff\else$\Longleftrightarrow$\fi\xspace}

% Legacy code from the algo tutorial slides. Perhaps useful. Try with care.
\mycomment{
	\newcommand{\impl}{\ifmmode\ensuremath{\mskip\thinmuskip\Rightarrow\mskip\thinmuskip}\else$\Rightarrow$\xspace\fi}  
	\newcommand{\Impl}{\ifmmode\implies\else$\Longrightarrow$\xspace\fi}
	
	\newcommand{\gdw}{\ifmmode\mskip\thickmuskip\Leftrightarrow\mskip\thickmuskip\else$\Leftrightarrow$\xspace\fi}
	\newcommand{\Gdw}{\ifmmode\iff\else$\Longleftrightarrow$\xspace\fi}
}
	
\newcommand{\gdwdef}{\ifmmode\mskip\thickmuskip:\Leftrightarrow\mskip\thickmuskip\else:$\Leftrightarrow$\xspace\fi}
\newcommand{\Gdwdef}{\ifmmode\mskip\thickmuskip:\Longleftrightarrow\mskip\thickmuskip\else:$\Longleftrightarrow$\xspace\fi}

\newcommand{\symbitemnegoffset}{\hspace{-.5\baselineskip}}
\newcommand{\implitem}{\item[\impl\symbitemnegoffset]}
\newcommand{\Implitem}{\item[\Impl\symbitemnegoffset]}


\newcommand{\forcenewline}{\mbox{}\\}

\newcommand{\bfalert}[1]{\textbf{\alert{#1}}}
\let\elem\in   % I'm a Haskell freak. Don't judge me. :P


\def\|#1|{\text{\normalfont #1}}  % | steht für senkrecht (anstatt kursiv wie sonst im math mode)


% proper math typography
\newcommand{\functionto}{\longrightarrow}
\renewcommand{\geq}{\geqslant}
\renewcommand{\leq}{\leqslant}
\let\oldsubset\subset
\renewcommand{\subset}{\subseteq} % for all idiots out there using subset

\newenvironment{threealign}{%
	\[
	\begin{array}{r@{\ }c@{\ }l}
}{%
	\end{array}	
	\]
}

\newcommand{\concludes}{ \\ \hline  }
\newcommand{\deduction}[1]{
	\begin{varwidth}{.8\linewidth}
		\begin{tabular}{>{$}c<{$}}
			#1
		\end{tabular}
	\end{varwidth}	
}

\definecolor{hoareorange}{rgb}{1,.85,.6}
\newcommand{\hoareassert}[1]{\setlength{\fboxsep}{1pt}\setlength{\fboxrule}{-1.4pt}\fcolorbox{white}{hoareorange}{\ensuremath{\{\;#1\;\}}}\setlength\fboxrule{\defaultfboxrule}\setlength{\fboxsep}{3pt}}

\newcommand{\mailto}[1]{\href{mailto:#1}{{\textcolor{blue}{\underline{#1}}}}}
\newcommand{\urlnamed}[2]{\href{#2}{\textcolor{blue}{\underline{#1}}}}
\renewcommand{\url}[1]{\urlnamed{#1}{#1}}

\newcommand{\hanging}{\hangindent=0.7cm}
\newcommand{\indented}{\hanging}


% \hstretchto prints #2 left-aligned into a box of the width of #1
\def\hstretchto#1#2{%
	\mbox{}\vphantom{#2}\rlap{#2}\hphantom{#1}%
}

\def\vstretchto#1#2{%
	\mbox{}\hphantom{#2}\smash{#2}\vphantom{#1}%
}


%requires \thisyear to be defined (s. config.tex)!
\edef\nextyear{\the\numexpr\thisyear+1\relax}


% --- \frameheight constant ---
\newlength\fullframeheight
\newlength\framewithtitleheight
\setlength\fullframeheight{.92\textheight}
\setlength\framewithtitleheight{.86\textheight}

\newlength\frameheight
\setlength\frameheight{\fullframeheight}

\let\frametitleentry\relax
\let\oldframetitle\frametitle
\def\newframetitle#1{\global\def\frametitleentry{#1}\if\relax\frametitleentry\relax\else\setlength\frameheight{\framewithtitleheight}\fi\oldframetitle{#1}}
\let\frametitle\newframetitle

\def\newframetitleoff{\let\frametitle\oldframetitle}
\def\newframetitleon{\let\frametitle\newframetitle}
% --- \frameheight constant end ---

\newcommand{\fakeframetitle}[1]{%
	\vspace{-2.05\baselineskip}%
	{\Large \textbf{#1}} \\%
	\smallskip
}



\newenvironment{headframe}{\Huge THIS IS AN ERROR. PLEASE CONTACT THE ADMIN OF THIS TEX CODE. (headframe env def failed)}{}
\RenewEnviron{headframe}[1][]{
	\begin{frame}\frametitle{\ }
		\centering
		\Huge\textbf{\textsc{\BODY} \\
		}
		\Large {#1}
		\frametitle{\ }
	\end{frame}
}


\makeatletter
% Provides color if undefined.
\newcommand{\colorprovide}[2]{%
	\@ifundefinedcolor{#1}{\colorlet{#1}{#2}}{}}
\makeatother


\colorprovide{lightred}{red!30}
\colorprovide{lightgreen}{green!40}
\colorprovide{lightyellow}{yellow!50}
\colorprovide{lightblue}{blue!10}
\colorprovide{beamerlightred}{lightred}
\colorprovide{beamerlightgreen}{lightgreen}
\colorprovide{beamerlightyellow}{lightyellow}
\colorprovide{beamerlightblue}{lightblue}
\colorprovide{fullred}{red!60}
\colorprovide{fullgreen}{green}
\definecolor{darkred}{RGB}{115,48,38}
\definecolor{darkgreen}{RGB}{48,115,38}
\definecolor{darkyellow}{RGB}{100,100,0}

\only<handout:0>{\colorlet{adaptinglightred}{beamerlightred}}
\only<handout:0>{\colorlet{adaptinglightgreen}{beamerlightgreen}}
\only<handout:0>{\colorlet{adaptinglightyellow}{beamerlightyellow}}
\only<handout:0>{\colorlet{adaptinglightblue}{beamerlightblue}}
\only<beamer:0>{\colorlet{adaptinglightred}{lightred}}
\only<beamer:0>{\colorlet{adaptinglightgreen}{lightgreen}}
\only<beamer:0>{\colorlet{adaptinglightyellow}{lightyellow}}
\only<beamer:0>{\colorlet{adaptinglightblue}{lightblue}}
\only<handout:0>{\colorlet{adaptingred}{lightred}}
\only<beamer:0>{\colorlet{adaptingred}{fullred}}
\only<handout:0>{\colorlet{adaptinggreen}{lightgreen}}
\only<beamer:0>{\colorlet{adaptinggreen}{fullgreen}}



\newcommand{\TrueQuestion}[1]{
	\TrueQuestionE{#1}{}
}

\newcommand{\YesQuestion}[1]{
	\YesQuestionE{#1}{}
}

\newcommand{\FalseQuestion}[1]{
	\FalseQuestionE{#1}{}
}

\newcommand{\NoQuestion}[1]{
	\NoQuestionE{#1}{}
}

\newcommand{\DependsQuestion}[1]{
	\DependsQuestionE{#1}{}
}

\newcommand{\QuestionVspace}{\vspace{4pt}}
\newcommand{\QuestionParbox}[1]{\begin{varwidth}{.85\linewidth}#1\end{varwidth}}
\newcommand{\ExplanationParbox}[1]{\begin{varwidth}{.97\linewidth}#1\end{varwidth}}
\colorlet{questionlightgray}{gray!23}
\let\defaultfboxrule\fboxrule

% #1: bg color
% #2: fg color short answer
% #3: short answer text
% #4: question
% #5: explanation
\newcommand{\GenericQuestion}[5]{
	\setlength\fboxrule{2pt}
	\only<+|handout:0>{\hspace{-2pt}\fcolorbox{white}{questionlightgray}{\QuestionParbox{#4} \quad\textbf{?}}}
	\visible<+->{\hspace{-2pt}\fcolorbox{white}{#1}{\QuestionParbox{#4} \quad\textbf{\textcolor{#2}{#3}}} \if\relax#5\relax\else\ExplanationParbox{#5}\fi} \\
	\setlength\fboxrule{\defaultfboxrule}
}

% #1: Q text
% #2: Explanation
\newcommand{\TrueQuestionE}[2]{
	\GenericQuestion{adaptinglightgreen}{darkgreen}{Wahr.}{#1}{#2}
}

% #1: Q text
% #2: Explanation
\newcommand{\YesQuestionE}[2]{
	\GenericQuestion{adaptinglightgreen}{darkgreen}{Ja.}{#1}{#2}
}

% #1: Q text
% #2: Explanation
\newcommand{\FalseQuestionE}[2]{
	\GenericQuestion{adaptinglightred}{darkred}{Falsch.}{#1}{#2}
}

% #1: Q text
% #2: Explanation
\newcommand{\NoQuestionE}[2]{
	\GenericQuestion{adaptinglightred}{darkred}{Nein.}{#1}{#2}
}

% #1: Q text
% #2: Explanation
\newcommand{\DependsQuestionE}[2]{
	\GenericQuestion{adaptinglightyellow}{darkyellow}{Je nachdem!}{#1}{#2}
}

% #1: Q text
% #2: Answer
\newcommand{\ContentQuestion}[2]{
	\GenericQuestion{adaptinglightblue}{black}{\minusquad}{#1}{#2}
}

\ifnum\thisyear=2018 \else \errmessage{Old ILIAS link inside preamble. Please update.} \fi

\newcommand{\ILIAS}{\urlnamed{ILIAS}{https://ilias.studium.kit.edu/ilias.php?ref\_id=855240\&cmdClass=ilrepositorygui\&cmdNode=5r\&baseClass=ilrepositorygui}\xspace}

\newcommand{\Socrative}{\ifdefined\mysocrativeroom \only<handout:0>{socrative.com $\quad \~~> \quad $ Student login \\ Raumname:  \mysocrativeroom\\ \medskip}\else\fi}

\newcommand{\thasse}[1]{
	\ifdefined\ThassesTut #1\xspace \else\fi
}
\newcommand{\daniel}[1]{
	\ifdefined\DanielsTut #1\xspace \else\fi
}
\newcommand{\thassedaniel}[2]{\ifdefined\ThassesTut #1\else\ifdefined\DanielsTut #2\fi\fi\xspace}

\ifdefined\ThassesTut \ifdefined\DanielsTut \errmessage{ERROR: Both ThassesTut and DanielsTut flags are set. This is most likely an error. Please check your config.tex file.} \else \fi \else \ifdefined\DanielsTut \else \errmessage{ERROR: Neither ThassesTut  nor DanielsTut flags are set. This is most likely an error. Please check your config.tex file.} \fi\fi

%\newcommand{\sgn}{\text{sgn}}

%%%%%%%%%%%% INHALT %%%%%%%%%%%%%%%%

%% Wochennummer
\newcounter{weeknum}

%% Titelinformationen
\title[GBI-Tutorium \mytutnumber, Woche \theweeknum]{Grundbegriffe der Informatik \\ Tutorium \mytutnumber}

\subtitle{Woche \theweeknum \ | \mydate{\theweeknum} \\ \myname \ \  \normalfont (\mailto{\mymail})}
\author[\myname]{\myname}
\institute{KIT -- Karlsruher Institut für Technologie}
\date{\mydate{\theweeknum}\ }

% Modified, DJ (better safe than sorry)
\AuthorTitleSep{ – }

%% Titel einfügen
\newcommand{\titleframe}{\frame{\titlepage}}

%% Alles starten mit \starttut{X}
\newcommand{\starttut}[1]{\setcounter{weeknum}{#1}\pdfinfo{
		/Author (\myname)
		/Title  (GBI-Tutorium \mytutnumber, Woche \theweeknum)
	}\titleframe\frame{\frametitle{Inhalt}\tableofcontents} \AtBeginSection[]{%
		\begin{frame}{Wo sind wir gerade?}
		\tableofcontents[currentsection]
	\end{frame}\addtocounter{framenumber}{-1}}}


\newcommand{\framePrevEpisode}{
\begin{headframe}
	\mylasttimestext
\end{headframe}
}

\newcommand{\lastframetitled}[6]{
	\frame{\frametitle{#6}
		\vspace{-#2\baselineskip}
		\begin{figure}[H]
			\centering
			\LARGE \textbf{\textsc{#5}} \\
			\vspace{.2\baselineskip}
			\includegraphics[#1]{#3}
			\vspace{-6pt}
			\begin{center}
				\small \url{#4} 
			\end{center}
		\end{figure} 
	}
}

% #1 number
% #2 title 
% #3 vspace (positive) without unit (\baselineskip)
\newcommand{\xkcdframe}[3]{
	\lastframetitled{width=.96\textwidth}{#3}{xkcd/#1}{http://xkcd.com/#1}{}{#2}
}

\newcommand{\xkcdframevert}[3]
{
	\lastframetitled{height=.96\frameheight}{#3}{xkcd/#1}{http://xkcd.com/#1}{}{#2}
}

% #1 number
% #2 title 
% #3 vspace (positive) without unit (\baselineskip)
% #4 \includegraphics[] optional parameters
\newcommand{\xkcdframecustom}[4]
{
	\lastframetitled{#4}{#3}{xkcd/#1}{http://xkcd.com/#1}{}{#2}
}

\newcommand{\slideThanks}{
	\begin{frame}
	\frametitle{Credits}
	\begin{block}{}
		An der Erstellung des Foliensatzes haben mitgewirkt:\\[1em]
		Daniel Jungkind \\
		Thassilo Helmold \\
		Philipp Basler \\
		Nils Braun \\
		Dominik Doerner \\
		Ou Yue \\
	\end{block}
\end{frame}
}

%% Wörter DEPRECATED! DO NOT USE
\newcommand{\code}[1]{$\mathbf{#1}$}

\morescalingdelimiters

\begin{document}
\starttut{6}


%\thasse{\lastframe{0.65}{25}{xkcd/advent_calendar.png}{https://www.xkcd.com/994/}}

\begin{frame}{Übungsschein-/Klausuranmeldung}
	\centering \Large Übungsschein- und Klausuranmeldung sind freigeschaltet! \\
	\bigskip
	\Impl \textbf{ANMELDEN}! \textbf{Rechtzeitig}! Am besten \textbf{sofort}, falls man jetzt schon weiß, dass man die Klausur schreibt. \\
	\bigskip
	(Übungsschein sowieso anmelden, weil nix zu verlieren!)
\end{frame}

\framePrevEpisode

\begin{frame}[t]{Wahr oder falsch?} 
		% Socrative: https://b.socrative.com/teacher/#import-quiz/31589970
		
		\Socrative
		\TrueQuestionE{$\forall x \in \Z: \quad x \mod 2 = 0 \iff x \text{ ist gerade}$.}{ $\text{Und auch: \quad} x \mod 2 = 1 \iff \text{x ist ungerade}$.}
		\FalseQuestionE{$4 \cdot (x \div 4) = x$.}{ $ 4 \cdot (x \div 4) + (x \mod 4) = x$}
		% \TrueQuestionE{$\left(\{\word a\}^\ast\cdot \{\word b,\eps\}\right)^\ast = \{\word a,\word b\}^\ast$}{}  % not fitting here
		% \FalseQuestionE{Jede Abbildung besitzt eine Umkehrabbildung.}{ Nur bijektive Abbildungen besitzen eine Umkehrabbildung.}  % ditto
		% \TrueQuestionE{Für jede Abbildung können wir das Urbild angeben.}{}  % ditto
		\TrueQuestionE{Das Zweierkomplement ist gut zum Rechnen.}{(Für den Rechner schon!)}
		\FalseQuestionE{Für jede Codierung gilt $h(\eps) = \eps$.}{Das muss nur für Homomorphismen gelten.}
		\FalseQuestion{Jeder Homomorphismus ist präfixfrei.}
		\TrueQuestion{Jede Huffman-Codierung ist präfixfrei.}
		\FalseQuestionE{Jeder $\eps$-freie Homomorphismus ist präfixfrei.}{ Aber: Jeder präfixfreie Homomorphismus ist $\varepsilon$-frei.}		
		\TrueQuestionE{Präfixfreie Codes sind einfach zu decodieren.}{}
\end{frame}



\mycomment{  % Not necessary here
	\begin{frame}{Rückblick: Codierungen}
		\begin{itemize}
			\item Codierungen: Injektive Abbildungen (meist Homomorphismen)
			\item $\eps$-Freiheit, Präfixfreiheit
			\item Einfaches Decodieren für präfixfreie Homomorphismen
			\item Huffman-Codierungen: Präfixfreie Codes mit optimaler Länge
		\end{itemize}
	\end{frame}
}



%\section{Huffman-Codierung}

\begin{frame}
	Kann man mit einer Codierung die benötigte Anzahl der Zeichen für ein Wort reduzieren und trotzdem den Sinn erhalten?\\[0.5em]
	\pause
	Natürlich geht das (manchmal), dieses Verfahren ist überall im Einsatz:\\
	Komprimierung!
\end{frame}

\begin{frame}{Huffman-Codierung}
	Eine Huffman-Codierung ist ein präfixfreier (und demnach \enquote{einfach} zu decodierenden) Homomorphismus, bei der die Codierung eines Zeichens umso länger wird, je seltener das Zeichen vorkommt.
	
	Die Huffman-Codierung für ein Wort ist dabei nicht eindeutig, wie wir gleich im Konstruktionsverfahren sehen werden.
	
	\begin{block}{Lemma}
		Unter allen präfixfreien Codes führen Huffman-Codes zu kürzesten Codierungen
		\textbf{des Wortes, für das die Huffman-Codierung konstruiert wurde.}
	\end{block}
\end{frame}

\begin{frame}{Konstruktionsverfahren}
	Formal: In der Vorlesung\\
	Hier: Vorgehensweise (ausreichend!)
	\begin{enumerate}
		\item Für jedes Zeichen die Häufigkeit ermitteln
		\item Alle Zeichen mit ihrer Häufigkeit als Blätter in die unterste Ebene zeichnen
		\item Jeweils die zwei Knoten (nicht unbedingt Blätter!) mit den geringsten Häufigkeiten \enquote{verbinden}(also einen neuen Knoten darüber anlegen, der die Summe der Häufigkeiten erhält)
		\item Fortfahren, bis der ganze Baum aufgebaut ist.
		\item Die linken Äste mit 0 beschriften, die rechten Äste mit 1.
		\item Codierungen der Zeichen ablesen
		\item Ausgangswort codieren (wenn gefordert, vergesst das nicht!)
	\end{enumerate}
\end{frame}

\begin{frame}
	\frametitle{Beispiel}
	Gegeben : $w= abadcadaac $ (10 Zeichen) \\[1em]
	
	\only<1-3|handout:1>{ \visible<3>{
	\begin{minipage}{0.6\linewidth}
		\begin{figure}[b]
			\centering
			\begin{tikzpicture}
			[level 1/.style={sibling distance=40mm},
			level 2/.style={sibling distance=20mm},
			level 3/.style={sibling distance=15mm}]
			\node {$10$}
			child {
				node{$5$}
				child{
					node{$3$}
					child{
						node{$1,b$}
						edge from parent node[left] {\bzero}
					}
					child {
						node{$2,d$}
						edge from parent node[right] {\bone}
					}
					edge from parent node[left] {\bzero};
				}
				child{
					node{$2,c$}
					edge from parent node[right] {\bone}
				}
				edge from parent node[left] {\bzero};
			}
			child{
				node{$5,a$}
				edge from parent node[right] {\bone}
			};
			\end{tikzpicture}
		\end{figure}  
	\end{minipage}
	}}
	\only<4-|handout:2> {
		\begin{minipage}{0.6\linewidth}
			Wie lang wird das neue Wort ?\\ \visible<5->{ 
			$5*1 + 1*3 + 2*2 + 2*3 = 18$ Zeichen\\ 
			Aber wir wollen doch komprimieren? Was haben wir falsch gemacht?\\
			\visible<6->{Nichts. Zur Codierung eines der Zeichen $a, b, c, d$ benötigen wir mindestens 2 Bit (da 4 Möglichkeiten). Für die Zeichen $0, 1$ brauchen wir aber nur ein Bit. \\
			Also haben wir 18 Bit statt 20 Bit und somit komprimiert!
		}}
		\end{minipage}
	}
	\visible<2-> {
		\begin{minipage}{0.2\linewidth}
			\hfill
			\hfill 
			\vspace*{0.1\linewidth}
			\begin{table}[H]
				\begin{tabular}{c|cccc}
					\hline
					x & a & b & c & d  \\ \hline
					$|w|_x$  & 5 & 1 & 2 & 2 \\ \hline
					h(x) & 1 & 000 & 01 & 001 \\ \hline 
				\end{tabular}
			\end{table}
		\end{minipage}
	}
\end{frame}

%TODO: Make solution frame!
\begin{frame}
		Huffman-Codierungen funktionieren immer nur gut für Wörter, die eine gleiche/ähnliche relative Zeichenhäufigkeit haben wie das Wort, für das der Code erstellt wurde.
		
		\begin{Beispiel}
			$w_1 = badcfehg, w_2 = a^1b^2c^4d^8e^{16}f^{32}g^{64}h^{128}$
			Erstellt eine Huffman-Codierung für jedes der beiden Wörter und Codiert jeweils beide Wörter mit der erstellten Codierung.\\
			Wie verhalten sich die Längen der Codewörter?
		\end{Beispiel}
\end{frame}

\begin{frame}
	\begin{block}{Erweiterung}
		Wir können nicht nur einzelne Buchstaben codieren. \\
		Bei $ w = a^{10}b^{10}c^{10} $ lohnt es sich pro Block gleicher Buchstaben eine Codierung zu haben.
	\end{block}
\end{frame}


\subsection{Aufgaben}
\begin{frame}
	\frametitle{Aufgabe (WS 2008) }
	Das Wort $$w = \mathbf{0000\only<3->{\text{ } }0001\only<3->{\text{ } }0011\only<3->{\text{ } }0001\only<3->{\text{ } }0011\only<3->{\text{ } }0000\only<3->{\text{ } }0000\only<3->{\text{ } }1110\only<3->{\text{ } }0001\only<3->{\text{ } }0000}$$ soll komprimiert werden.
	
	\pause
	\begin{itemize}[<+->]
		\item Zerlegen Sie $w$ in Viererblöcke und bestimmen Sie die Häufigkeiten der vorkommenden Blöcke.
		\item Zur Kompression soll ein Huffman-Code verwendet werden. Benutzen Sie die in Teilaufgabe a) bestimmten Häufigkeiten, um den entsprechenden Baum aufzustellen. Beschriften Sie alle Knoten und Kanten.
		\item Geben Sie die Codierung des Wortes $w$ mit Ihrem Code an.
	\end{itemize}
\end{frame}

\begin{frame}
	\frametitle{Lösung}
	$$w = \mathbf{0000000100110001001100000000111000010000}$$
	\textit{Zerlegen Sie $w$ in Viererblöcke und bestimmen Sie die Häufigkeiten der vorkommenden Blöcke.} \\[1em]
	\pause
	$$w = \mathbf{0000 \ 0001 \ 0011 \ 0001 \ 0011 \ 0000 \ 0000 \ 1110 \ 0001 \ 0000}$$ \pause
	\begin{table}[h!]
		\centering
		\begin{tabular}{l|cccc}	
			& 0000 & 0001 & 0011 & 1110 \\ \hline
			Absolute Häufigkeiten: & 4 & 3 & 2 & 1 \\
			Relative Häufigkeiten:  & 0,4 & 0,3 & 0,2 & 0,1\\
		\end{tabular}
	\end{table}
\end{frame}

\begin{frame}
	\frametitle{Lösung}
	\vspace*{1em}
	\begin{minipage}{0.45\linewidth}
		\textit{Zur Kompression soll ein Huffman-Code verwendet werden. Benutzen Sie die in Teilaufgabe a) bestimmten Häufigkeiten, um den entsprechenden Baum aufzustellen. Beschriften Sie alle Knoten und Kanten.}
		\pause 
		\begin{table}[h!]
			\centering
			\begin{tabular}{cccc}	
				0000 & 0001 & 0011 & 1110 \\ \hline
				4 & 3 & 2 & 1 \\	
			\end{tabular}
		\end{table}
	\end{minipage}
	\hfill
	\begin{minipage}{0.5\linewidth}
		\begin{figure}[h!]
			\centering
			\only<3>{\includegraphics[scale=0.35]{Huffman.pdf}}
		\end{figure}
	\end{minipage}
\end{frame}

\begin{frame}
	\frametitle{Lösung}
	\textit{Geben Sie die Codierung des Wortes $w$ mit Ihrem Code an.} \\[2em] \pause
	$0000000100110001001100000000111000010000$ \\ \hfill $\to 1010010100111000011$
\end{frame}

%\begin{frame}
%	\frametitle{Aufgabe (WS 2010)}
%	Seien $n, k \in \nN_0$ mit $1 \leq k \leq n$. In einem Wort $w \in \{a, b, c\}^\ast$ der Länge $3n$ komme $k$ mal das Zeichen $a$, $n$ mal das Zeichen $b$ und $2n - k$ mal das Zeichen $c$ vor.
%	\begin{itemize}
%		\item Geben Sie den für die Huffman-Codierung benötigten Baum an.
%		\item Geben Sie (in Abhängigkeit von $k$ und $n$) die Länge des zu $w$ gehörenden Huffman-Codes an.
%	\end{itemize}
%\end{frame}
%
%\begin{frame}
%	\frametitle{Lösung}
%	\vspace*{1em}
%	\begin{minipage}{0.45\linewidth}
%		\textit{$\dots$ Länge $3n$ komme $k$ mal das Zeichen $a$, $n$ mal das Zeichen $b$ und $2n - k$ mal das Zeichen $c$ vor. \\[1em] Geben Sie den für die Huffman-Codierung benötigten Baum an.} 
%		\pause
%		\begin{table}[h!]
%			\centering
%			\begin{tabular}{ccc}	
%				$a$ & $b$ & $c$ \\ \hline
%				$k$ & $n$ & $2n-k$ \\	
%			\end{tabular}
%		\end{table}
%		\pause
%		$$k \leq n \leq 2n -k $$ $$ n+k+2n-k = 3n$$
%	\end{minipage}
%	\hfill
%	\begin{minipage}{0.5\linewidth}
%		\begin{figure}[h!]
%			\centering
%			\only<4>{\includegraphics[scale=0.35]{Huffman2.pdf}}
%		\end{figure}
%	\end{minipage}
%\end{frame}
%
%\begin{frame}
%	\frametitle{Lösung}
%	\textit{Geben Sie die Länge des zu $w$ gehörenden Huffman-Codes an.} \\[2em]
%	\pause
%	Jedes $a$ und jedes $b$ wird durch zwei Zeichen codiert, und jedes $c$ wird durch ein Zeichen codiert. Damit erhält man insgesamt $$2k + 2n + 2n - k = 4n + k$$ Zeichen in der Codierung.
%\end{frame}


\begin{frame}{Ausblick}
	Die Huffman-Codierung hat ein Problem: Zum Decodieren muss der Huffman-Baum, der für die Codierung verwendet wurde, bekannt sein. Im wesentlichen gibt es dafür zwei Möglichkeiten:
	\begin{enumerate}
		\item Der Codebaum wird vor dem eigentlichen Codewort angegeben.\\ 
		Problem: Das verlängert das Codewort.
		\item Es wird ein vorher festgelegter Codebaum verwendet.\\
		 Problem: Dieser Codebaum ist nicht an das spezifische Wort angepasst und kann evtl. (bei komplett anderer Zeichenhäufigkeit) zu sehr schlechten Ergebnissen führen.
	\end{enumerate}

	Diese Probleme können durch andere Codierungsverfahren gelöst werden, indem z.B. das Wörterbuch dynamisch während der Decodierung aus dem Codewort aufgebaut wird (z.B. Lempel-Ziv-Welch-Verfahren).
\end{frame}



%% Inhalt

\section{Funktionen von Funktionen}

\begin{frame}{Funktionen}
	\begin{Definition}
		Seien A und B Mengen. Dann ist $$B^A := \set{f \Mid f \from A \functionto B }$$ die Menge aller Abbildungen von $A$ nach $B$.
	\end{Definition}

	Abbildungen kann man sich auch als Tabellen vorstellen: \\
	Wir wählen für jedes $a \in A$ ein $b \in B$.
	
	\begin{block}{Beobachtung}
		Für endliche Mengen A und B gilt:
		\delimitershortfall=0pt
		$$\setsize{B^A} = \setsize B^{\setsize A}$$
	\end{block}
\end{frame}

\newcommand{\filter}{filter_{\only<2-3|handout:1>{even}\only<4|handout:2>{odd}\only<5|handout:3>{Uglg}}}

\begin{frame}{Problem: Ziffern filtern}
	Wir wollen aus einer Ziffernfolge (Wort aus $Z^*_{10}$) Ziffern herausfiltern, die eine bestimmte Eigenschaft haben:\\
	\visible<2->{\impl\ }\only<2-3|handout:1>{Gerade}\only<4|handout:2>{Ungerade}\only<5|handout:3>{Löst die Ungleichung $x^2 - 3 < 20$}
	
	\visible<3-> {
		\begin{threealign}
		\filter \from Z^*_{10} &\functionto& Z^*_{10} \\
		\eps &\mapsto& \eps \\
		z \cdot v &\mapsto& \begin{cases}
		z \cdot \filter(v) &\text{falls } \only<2-3|handout:1>{z \in \{\word 2,\word 4,\word 6,\word 8,\word 0\}} \only<4|handout:2>{z \in \{\word 1,\word 3,\word 5,\word 7,\word 9\}} \only<5|handout:3>{\fnum_{10}(z)^2 - 3 < 20 }\\
		\filter(v) &\text{sonst}
		\end{cases}\\
		& \multicolumn{2}{l}{\text{wobei } z \in Z_{10} \text{ und } v \in Z_{10}^*.}
		\end{threealign}
	}
\end{frame}

\begin{frame}{Problem: Ziffern filtern}
	\begin{block}{Aufgabe}
		Schreibt die Funktionen zum Filtern aller Ziffern...
		\begin{itemize}
			\item $> 5$
			\item $< 5$
			\item $<2$ oder $>7$
			\item ... die perfekt sind
			\item ... die sich ausmalen lassen
		\end{itemize}
	\end{block}

	\pause
	\impl Blödsinn!\\
	Alle Funktionen unterscheiden sich nur \textbf{an einer Stelle}: \pause \\
	Der \textbf{Bedingung}, die überprüft wird.
\end{frame}

\begin{frame}{Filter}
	Definiere stattdessen eine \textbf{universelle} Filter-Funktion:
	
	\begin{threealign}
		filter_p : Z^*_{10} &\functionto& Z^*_{10} \\
		\eps &\mapsto& \eps \\
		z \cdot v &\mapsto& \begin{cases}
		z \cdot filter_p(v) &\text{falls } p(z) = \W\\
		filter_p(v) &\text{sonst}
		\end{cases}\\
		&\multicolumn{2}{l}{\text{wobei } z \in Z_{10} \text{ und } v \in Z_{10}^*}
	\end{threealign}
	\medskip
	und zwar für alle \quad $p: Z_{10} \functionto \BB$.
\end{frame}

\begin{frame}{Filter}
	\begin{threealign}
		filter_p : Z^*_{10} &\functionto& Z^*_{10} \\
		\eps &\mapsto& \eps \\
		z \cdot v &\mapsto& \begin{cases}
		z \cdot filter_p(v) &\text{falls } p(z) = \W\\
		filter_p(v) &\text{sonst}
		\end{cases}\\
		&\multicolumn{2}{l}{\text{wobei } z \in Z_{10} \text{ und } v \in Z_{10}^*}
	\end{threealign}
	
	Definiere beispielsweise nun:
	$$ even:  Z_{10} \functionto \BB, \; z \mapsto \begin{cases}
	\textbf{w} &\text{falls } z \in \{\word 2,\word  4,\word  6,\word  8,\word  0\}\\
	\textbf{f} &\text{sonst }
	\end{cases}$$
	
	\pause
	Dann ist $filter_{even}(\word{123456}) = \word{246}$.
	
	\smallskip
	\impl Brauchen bloß $p$-Funktionen definieren anstatt jedes Mal ganze Filter-Funktion!
	
\end{frame}

% TOO MUCH FOR THE FIRST TIME, I GUESS...
\mycomment{
	\begin{frame}{Charakteristische Funktion}
		Sei $M$ eine abzählbar unendliche Menge und $L \subseteq M$.
		
		\begin{Definition}
			Die \textbf{Charakteristische Funktion} einer (Teil-)Menge $L$ ist die Funktion 
			\begin{threealign}
			C_L : M &\functionto& \{0, 1\}\\
			x &\mapsto& \begin{cases}
			1 &x \in L \\
			0 &x \notin L
			\end{cases}.
			\end{threealign}
		\end{Definition}
		
		Also ist $C_L \in \{0, 1\}^M$.
		\smallskip
		
		Klar ist außerdem: Es gibt eine Bijektion $C$ zwischen Mengen und Charakteristischen Funktionen: $$C : 2^M \to \{0, 1\}^M, L \mapsto C_L$$
	\end{frame}
	
	\begin{frame}{Charakteristische Funktion}
		Jetzt können wir auf diesen Funktionen äquivalente Operationen wie auf Mengen definieren...
		
		Vereinigung: $V\colon \{0,1\}^M \times \{0,1\}^M\to \{0,1\}^M$\\[1em]
		
		\only<2|handout:1>{
			Beispielbild für $L_1=\{a,c,d\}$ und $L_2=\{b,c\}$
			
			\begin{tabular}{*{4}{>{$}c<{$}}}
				& L_1 & L_2 & L_1\cup L_2 \\
				x & f_1(x) & f_2(x) & V(f_1,f_2) \\
				a & 1 & 0 & 1 \\
				b & 0 & 1 & 1 \\
				c & 1 & 1 & 1 \\
				d & 1 & 0 & 1 \\
				e & 0 & 0 & 0 \\
			\end{tabular}
		}
		
		\only<3-|handout:2> {
			Wie definiert man $V(f_1,f_2)$? Zum Beispiel so:
			\begin{align*}
			V \colon \{0,1\}^M \times  \{0,1\}^M &\to \{0,1\}^M \\
			(f_1,f_2) &\mapsto (x \mapsto \max(f_1(x),f_2(x))) \\
			\end{align*}
		} \only<4|handout:2> {
		Oder so: $V(f_1,f_2) (x) = \max(f_1(x),f_2(x))$
	}
	\end{frame}
}

\newcommand{\yellow}[1]{\fcolorbox{white}{yellow!60}{\ensuremath{#1}}}
\begin{frame}{Funktionen auf LSD: Filter}
	\impl Idee: Mache $p$ zu einem Argument von $filter$!
	\medskip
	\begin{threealign}
		filter \from \yellow{\BB^{Z_{10}} \times } Z^*_{10} &\functionto& Z^*_{10} \\
		(\yellow{p}, \eps) &\mapsto& \eps \\
		(\yellow{p}, z \cdot v) &\mapsto& \begin{cases}
		z \cdot filter(\yellow{p, }v) &\text{falls } p(z) = \W\\
		filter(\yellow{p, }v) &\text{sonst}
		\end{cases}\\
		&\multicolumn{2}{l}{\text{wobei } z \in Z_{10} \text{ und } v \in Z_{10}^*.}
	\end{threealign} \\
	\pause
	
	\impl $filter(even, \word{123456}) = \word{246}$. \\
	\impl $filter(odd, \word{123456}) = \word{135}$.
	\medskip
	
	\impl $filter$ ist jetzt eine Funktion, die \textbf{eine Funktion als Argument} nimmt.
\end{frame}

\newcommand{\fconst}{\text{const}}
\begin{frame}{Funktionen auf LSD: const}
	Schnell: Gebt sieben verschiedene konstante Funktionen an. \\
	\pause
	\impl Zu aufwendig? \visible<6->{\impl Nicht mit \fconst: $\set{\fconst(1),\fconst(2),\fconst(3),...}$}
	\pause
	
	\begin{threealign}
		\fconst \from \R &\functionto& \R^\R \\
		\fconst(a) &:=& \underbrace{\big( x \mapsto a \big)}_{\text{eine Funktion!}}
	\end{threealign} \\
	\pause
	\impl Definiere $f := \fconst(42)$. \quad Dann ist $f(5) = f(1000) = f(-2.3) = 42$. \\
	\smallskip
	\pause
	Oder auch: $\left(\fconst(42)\right)(5) = 42$. \\
	\medskip
	\pause[7]
	
	\impl $\fconst$ ist eine Funktion, die \textbf{eine Funktion als Ergebnis} zurückliefert.
\end{frame}

\begin{frame}
	Wir haben gesehen:
	\begin{itemize}
		\item Eine Abbildung, die eine Funktion auf einen Wert abbildet
		\item Eine Abbildung, die einen Wert auf eine Funktion abbildet
	\end{itemize}
	Gleich werden wir das kombinieren! \\
	\medskip
	(Hinweis: Abbildung $=$ Funktion, gell? \smiley)
\end{frame}

\section{Speicher}

\begin{frame}{Bit vs. Byte}
	\begin{Definition}
		Eine \textbf{Bit} ist ein Zeichen des Alphabets $\{\word 0, \word 1\}$. \\
		Ein Wort aus 8 \emph{Bits} wird \textbf{Byte} genannt. 
	\end{Definition}
	\pause	
	
	\begin{Definition}
		Ein \textbf{Speicher} $$m \from \Adr \functionto \Val$$ bildet Adressen ($\Adr$) auf Werte ($\Val$) ab.\\
		Wir schreiben $\Mem$ für $\Val^\Adr$. \\
		Also: $m \in \Val^{\Adr}$ \quad bzw. \quad $m \in \Mem$.
	\end{Definition}
	\impl Betrachten abstrakte Funktionsweise anstatt konkrete Realisierung 
	%Hier interessiert uns nicht die konkrete Realisierung der Speicherung, sondern nur die abstrakte Funktionsweise.
	\pause
	
	\begin{block}{Hinweis}
		Häufig: Adressen und Werte als Binärzahlen. \\
		%Im Umgang mit Speichern benutzen wir hier nur Zahlen im Binärsystem. \\
		% Zu detailliert:                     und was ist mit negatiuven Zahlen?
		%$\Adr = \set{0, ..., 2^k-1}, \Val = {0,...,2^l-1} \quad \text{für } k,l \in \N_+$.
	\end{block}
\end{frame}

\begin{frame}{Speicher}

	Operationen: \\
	\begin{itemize}
		\item $\memread(m, adr)$, um eine Speicherzelle zu lesen 
		\pause
		\item $\memwrite(m, adr, val)$, um in eine Speicherzelle zu „schreiben“.  
	\end{itemize}
	
	\pause
	\begin{threealign}
		\memread \from \Val^{\Adr} \times \Adr &\functionto& \Val \\
		(m,a) &\mapsto& m(a)  \medskip  
		\visible<4->{ \\
			\memwrite \from \Val^{\Adr} \times \Adr \times \Val &\functionto& \Val^{\Adr} \\
			(m,a',v') &\mapsto& m' , \quad \text{wobei} 
		} 
		\visible<5->{ \\ 
				m'(a) &=&
				\begin{cases}
				v' & \text{ falls } a=a' \\
				m(a) & \text{ falls } a\not=a' \\
				\end{cases}	
		}
	\end{threealign}	
	\pause[6]
	$\memwrite$ gibt die neue Speichertabelle $m'$ zurück!
\end{frame}

\begin{frame}{Speicher}
	\begin{block}{Beispiel}
		
	$\memread(m, \word{01}) = \only<2->{\word{00000111}}$\\
	\visible<2-|handout:2>{$m' := \memwrite(m, \word{01}, \word{11111100})$\\}
	\visible<3-|handout:2>{$\memread\big(\underbrace{\memwrite(m, \word{01}, \word{11111100})}_{m'}, \word{01}\big) = \word{11111100}$\\}
	\bigskip
	
	\begin{table}
		\begin{tabular}{r}
			\\ \\ \\ \\  \\
			\visible<4-|handout:2>{Realworld-Analogie: }
		\end{tabular}
		\begin{tabular}{|c|c|}
			\hline
			\multicolumn{2}{|c|}{Speicher $m$} \\
			\hline
			\word{00} & \word{00101000} \\
			\word{01} & \word{00000111} \\
			\word{10} & \word{10010110} \\
			\word{11} & \word{00100101} \\ 
			\hline 
			\multicolumn{2}{c}{\visible<4-|handout:2>{Zustand bei $t=0$}}
		\end{tabular} 
		\visible<2-|handout:2>{
			\begin{tabular}{|c|c|}
				\hline
				\multicolumn{2}{|c|}{Speicher $m'$} \\
				\hline
				\word{00} & \word{00101000} \\
				\underline{\word{01}} & \underline{\word{11111100}} \\
				\word{10} & \word{10010110} \\
				\word{11} & \word{00100101} \\
				\hline
				\multicolumn{2}{c}{\visible<4-|handout:2>{Zustand bei $t=1$}}
			\end{tabular} 
		}
	\end{table}

	\end{block}
\end{frame}

\mycomment{
	\only<handout:0>{\begin{frame}{Aufgaben}
		
	\end{frame}}
}

\section{MIMA}

\begin{frame}{MIMA}
	\includegraphics[width=\linewidth]{MIMA_simple.png}\\
	Die \emph{MIMA} ist ein idealisierter Prozessor. 
\end{frame}

\begin{frame}{Eigenschaften}
	\begin{itemize}[<+->]
		\item Adressen sind 20 Bit lang
		\item \enquote{Werte} sind 24 Bit lang
		\item Befehlscodierungen -- zwei Formate:
		\begin{itemize}
			\item[a)] 4 Bit für den OpCode und 20 Bit für einen Parameter (Adresse / Konstante)
			\item[b)] 8 Bit Befehl (Rest irrelevant)\\
			\item[]<+(-2)-> \includegraphics[width=100px]{MIMA_commands.png} 
		\end{itemize} 
	\end{itemize}
\end{frame}


\begin{frame}{Wichtige Register}
	\begin{itemize}[<+->]
		\item \emph{IAR} : InstruktionsAdressRegister : Speichert Adresse des aktuell auszuführenden Befehls.
		\item \emph{IR}: InstruktionsRegister : Speichert den auszuführenden Befehl selbst.
		\item \emph{SAR}: SpeicherAdressRegister : Enthält die Adresse eines Wertes, der aus dem Speicher gelesen werden soll.
		\item \emph{SDR}: SpeicherDatenRegister : Enthält einen Wert, der aus dem Speicher geladen wurde.
		\item \emph{Akku}: enthält Ausgangswerte/Ergebnisse von Berechnungen
	\end{itemize}
\end{frame}

\begin{frame}[t]{Aufbau}
	\begin{figure}
		\centering
		\includegraphics[width=\linewidth]{MIMA.png}
	\end{figure}
\end{frame}

\begin{frame}[t]{Befehlsholphase}
	\begin{figure}
		\centering
		\includegraphics[width=\linewidth]{MIMA.png}
	\end{figure}
	\pause
	\only<2-6>{\begin{itemize}
		\only<2|handout:1> {\item[1.] IAR $\to $ SAR  und IAR $\to$ X 
			\item[] Befehlsadresse dem Speicher übergeben und Zähler zum Erhöhen an ALU geben.}
		\only<3|handout:2>{\item[2.] Eins $\to$ Y  
			\item[] 1-Wert für Erhöhung des Zählers an ALU geben.}
		\only<4|handout:3>{\item[3.] ALU aufaddieren ($Z=X+Y$) 
			\item[] Nächste Befehlsadresse berechnen. }
		\only<5|handout:4>{\item[4.] Z $\to$ IAR 
			\item[] Adresse für nächste Runde speichern. }
		\only<6|handout:5>{\item[5.] SDR $\to$ IR 
			\item[] Wert zur angefragten Adresse erhalten. }
	\end{itemize}}	
\end{frame}

\newcommand{\itemizeconfig}{\setlength{\parsep}{0pt}\setlength{\parskip}{0pt}\setlength{\topsep}{0pt}\setlength{\partopsep}{0pt}}
\newcommand{\explain}[1]{\hfill {\small #1 }}
\begin{frame}{Befehle}
	Die MIMA besitzt einen Befehlssatz mit möglichen Befehlen. %Andere Befehle (oder Varianten) werden NICHT unterstützt und können daher nicht verwendet werden!
	\textbf{Nur die Befehle aus der VL} dürft ihr benutzen! \\ 
	\smallskip
	(Befehle, die eigentlich zwei Operanden brauchen, lesen zusätzl. vom Akku.)
	{
	\begin{itemize}[<+->] \itemizeconfig
		\item Rechenoperationen:
		\begin{itemize} 
			\item ADD adr \explain{Akku $\leftarrow$ Akku + Mem(adr)}
			\item AND, OR, XOR adr \explain{Dasselbe mit bitweisem Und, Oder und eXclusive OR}
			\item NOT, RAR \explain{Akku $\leftarrow$ not(Akku); \quad  Rotate-Akku-Right();}
		\end{itemize}
		\item Datentransport:
		\begin{itemize}
			\item LDC const \explain{Akku $\leftarrow$ const\quad (const ist dabei eine \textbf{20-Bit}-Konstante) }
			\item LDV, STV adr \explain{(Lese/Schreibe in/vom Akku von/an Adresse)}
			\item LDIV, STIV adr \explain{(Lese/Schreibe von/an Adresse, die an der Adresse steht)}
		\end{itemize}
		\item Vergleichsoperation: EQL adr \explain{(liefert $-1$ wenn Akku $=$ Mem(adr), 0 sonst)}
		\item Sprünge:
		\begin{itemize}
			\item JMP adr \explain{Springe zu Befehl in Adresse adr}
			\item JMN adr \explain{Springe, falls Akku negativ \quad „JuMp if Negative“}
		\end{itemize}
		\item HALT \explain{Stoppt die MIMA}
	\end{itemize}}
\end{frame}

\begin{frame}{Bemerkungen}
	\begin{block}{Indirekte Adressierung (LDIV, STIV)}
		\centering
		\includegraphics[width=150px]{MIMA_indirect.png}
	\end{block}
	
	\pause
	\begin{block}{HALT}
		Jedes Programm muss mit HALT enden! Sonst läuft das Programm endlos weiter! (Kostet dann Punkte!)
	\end{block}

	\pause
	\begin{block}{Negative Konstanten}
		Negative Konstanten können \textbf{nicht} mit LDC geladen werden. Warum? \\
		\pause \impl Unser Akku ist 24 Bit breit, aber wir können nur in die \textbf{hinteren 20 Bit} laden!
	\end{block}
\end{frame}

\begin{frame}{Negative Zahlen}
	\begin{block}{Aufgabe}
		Schreibt ein Programm, das von einer an Adresse $a_1$ gegebenen positiven Zahl das Zweierkomplement berechnet und an Adresse $a_2$ ablegt.
	\end{block}

	\visible<2|handout:2>{
		\begin{block}{Lösung}
			LDV $a_1$\\
			NOT\\
			STV $a_2$\\
			LDC 1\\
			ADD $a_2$\\
			STV $a_2$
		\end{block}
	}
\end{frame}

\begin{frame}{Beispiele}
	Beispiele zur Umsetzung von Anweisungen aus Hochsprachen (if, while, for....):\\
	Siehe Übung WS 15/16
\end{frame}

\begin{frame}{Übung: Modulo 2 und Betrag}
	1. Schreibe ein Programm, das eine an Speicheradresse $a_1$ gegebenen Zahl Modulo 2 rechnet und an Adresse $a_2$ ablegt. \\
	\medskip
	2. Schreibe ein Programm, das den Betrag einer an Adresse $a_1$ gegebenen Zahl berechnet und an Adresse $a_2$ ablegt.
\end{frame}

\begin{frame}{Lösung: Modulo 2}
	\begin{tabbing}
		start: \; \= LDC 1 \quad  //$= 000000000000000000000001_2$ \\
				\> AND $a_1$ \\
				\> STV $a_2$ \\
				\> HALT
	\end{tabbing}
\end{frame}

\begin{frame}{Lösung: Betrag}
	\begin{tabbing}
		start: \quad \= LDV $a_1$ \\
					 \> JMN negate \\
		end: 		 \> STV $a_2$ \\
					 \> HALT \\
		negate:		 \> NOT \\
					 \> STV $a_2$ \\
					 \> LDC 1 \\
					 \> ADD $a_2$  \\
					 \> JMP end	 \\		 
	\end{tabbing}
\end{frame}

\begin{frame}{Übung: Modulo 3}
	Schreibe ein Programm, das eine an Speicheradresse $a_1$ gegebenen Zahl Modulo \textbf{3} (drei!) rechnet und an Adresse $a_2$ ablegt.
\end{frame}

\begin{frame}{Lösung: Modulo 3}
  \begin{tabbing}
    start: \quad \= LDC 1 \\
           \> STV One\\
           \> LDC 3\\
           \> NOT\\
           \> ADD One\\
           \> STV MinusThree\\
           \> LDV $a_1$\\
    \medskip
    while: \> ADD MinusThree\\
           \> JMN end\\
           \> JMP while\\
    \medskip
    end:   \> STV $a_1$\\
           \> LDC 3\\
           \> ADD $a_1$\\
           \> STV $a_2$ \\
           \> HALT\\
  \end{tabbing}
\end{frame}


\begin{frame}	
	\begin{block}{Was ihr nun wissen solltet}
		\begin{itemize}
			%\item Wie die Huffman-Codierung funktioniert
			\item Speicher
			\item Abbildungen, die Abbildungen auf Abbildungen abbilden!
			\item Wie ein einfacher Prozessor aufgebaut ist
			\item Wie man die MIMA für einfache Aufgaben nutzt
		\end{itemize}
	\end{block}
	
	\begin{block}{Was nächstes Mal kommt}
		\begin{itemize}
			%\item MIMA – Den Bits beim Arbeiten zuschauen
			\item Grammatiken – mehr als Subjekt, Objekt, Prädikat
		\end{itemize}
	\end{block}
\end{frame}

\xkcdframevert{676}{Danke für eure Aufmerksamkeit! \smiley}{2.5}
\slideThanks

\end{document}